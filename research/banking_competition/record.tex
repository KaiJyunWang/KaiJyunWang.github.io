\documentclass[a4paper, 12pt]{article}
\input{prefix_large_small_bank.tex}%
\input{symbols_large_small_bank.tex}

\title{Banking Competiotion Research Record}
\author{Kai-Jyun Wang\thanks{National Taiwan University, Department of Economics.}}

\begin{document} 
\setstretch{1.15}
\maketitle

\section{Literature}
\subsection{Bank Run and Banking Competition}
\begin{itemize}
    \item \cite{diamond1983}.
    \item \cite{goldstein2005}.
    \item \cite{egan2017}. 
    \item \cite{gertler2015}. 
    \item \cite{wang2022}. 
    \item \cite{egan2025}. 
\end{itemize}

\section{Monopolistic Bank with Liquidity Motive} 
\subsection{Household}
Time is continuous, starting from $t = 0$ and going on forever. 
Following \cite{moreira2017}, we model the households with 
liquidity needs. There is a unit measure of identical households 
solving the following probelm 
\begin{equation}\label{eq:hh_value}
    V_t = \max\E_t\sbrc{\int_t^\infty e^{-\rho t}\log\pth{W_t\pth{\psi dC_t + dc_t}}}.
\end{equation}
where $C_t$ is the consumption fraction in liquidity event, $\psi > 1$ 
is the marginal utility of liquidity event consumption and 
$c_t$ is the consumption fraction outside of the liquidity event. 

A liquidity event comes with rate $h$ with size $d\overline{C}_t$. 
That is, $dC_t \leq d\overline{C}_t$. Conditional on the 
occurence of a liquidity event, the size follows $\text{Exp}(\eta)$. 

There are two type of securities issued by the bank, money $m$ 
and equity $e$. The money can be used to finance both consumptions 
while the equity can only finance $c_t$. The constraints are 
\begin{align}
    \frac{dW_t}{W_t} &= m_tdr^m_t + e_tdr^e_t - dC_t - dc_t \\ 
    dC_t &\leq \min\set{m_t, d\overline{C}_t} \label{eq:liquidity_constraint}\\ 
    m_t + e_t &= 1.
\end{align}

Recursify \cref{eq:hh_value}, 
\begin{equation}
    \rho V_t(W_t)dt = \log(W_t) + \max_{m_t, e_t, dC_t, dc_t} \E_t\sbrc{\log(\psi dC_t + dc_t)} + \E_t\sbrc{dV_t}. 
\end{equation}
where 
\begin{equation}
    dV_t = V_tdt + V_WdW_t + \frac{1}{2}V_{WW}(dW_t)^2.
\end{equation}
Given $m_t, e_t$, conditional on the liquidity event, the household decides 
$(dC_t, dc_t)$: 
\begin{equation}
    \begin{split}
        &\max_{dC_t, dc_t} \log(\psi dC_t + dc_t) + \theta_t(\min\set{m_t, d\overline{C}_t} - dC_t) \\
        &\quad + V_tdt + V_W\E\sbrc{m_tdr^m_t + e_tdr^e_t - dC_t - dc_t} + \frac{1}{2}V_{WW}\E\sbrc{W_t^2(m_tdr^m_t + e_tdr^e_t - dC_t - dc_t)^2}
    \end{split}
\end{equation}
The first order condition with complementary slackness is 
\begin{align}
    \frac{\psi}{\psi dC_t + dc_t} - \theta_t - V_WW_t - V_{WW}\E\sbrc{W_t^2(m_tdr^m_t + e_tdr^e_t - dC_t -dc_t)} &= 0 \\ 
    \frac{1}{\psi dC_t + dc_t} - V_WW_t - V_{WW}\E\sbrc{W_t^2(m_tdr^m_t + e_tdr^e_t - dC_t -dc_t)} &= 0 \\ 
    \theta_t(\min\set{m_t, d\overline{C}_t} - dC_t) &= 0. 
\end{align}
Thus 
\begin{equation*}
    \theta_t = \frac{\psi - 1}{\psi dC_t + dc_t} > 0
\end{equation*}
and the liquidity constraint always binds. 
\begin{equation*}
    dC_t = \min\set{m_t, d\overline{C}_t}.
\end{equation*}
Now, 
\begin{equation*}
    dc_t
\end{equation*}

\section{Monopolistic Bank with Information Frictions}
The bank can choose to keep operation or exit and taking $E_0$. 
In the continuation region, the bank solves the HJB 
\begin{equation}\label{eq:bank_hjb}
    \begin{split}
        \rho_BE(k,a) &= \max_{r} - rD(r,a) + E_arD(r,a) + \phi(\alpha k + E((1-\alpha)k,a) - E(k,a)) \\
        &\quad + E_k\mu k_t + \eta \int\max\set{E((1-\kappa)k,a) - E(k,a),E_0 - E(k,a)}dF_{\kappa} \\
        &\quad + \lambda \int \max_{k\leq k'\leq \min\set{k+\bar{I},D}}E(k',a) - E(k,a)dF_I(\bar{I}). 
    \end{split}
\end{equation}
$E_k> 0$ so $k' = \min\set{k+\bar{I},D}$. 
\begin{equation}\label{eq:bank_hjb2}
    \begin{split}
        (\rho_B + \phi + \lambda + \eta)E(k,a) &= \max_{r} - rD(r,a) + E_arD(r,a) + \phi(\alpha k + E((1-\alpha)k,a))\\
        &\quad + E_k\mu k_t + \eta \int\max\set{E((1-\kappa)k,a),E_0}dF_\kappa \\
        &\quad + \lambda ((1 - F_I(D-k))E(D,a) + \int_0^{D-k}E(k+\bar{I},a)dF_I(\bar{I})). 
    \end{split}
\end{equation}
Given household asset holdings $a$, the unit measure of households solve 
\begin{equation}\label{eq:hh_prob}
    \max_{\theta_t\in[0,1]} \E_t\sbrc{\frac{da_t}{a_t}} - \frac{\gamma}{2}\Var_t\sbrc{\frac{da_t}{a_t}}
\end{equation}
subject to $da_t = \theta_ta_td\tilde{r}_t$. The solution is given by 
\begin{equation}
    \theta_t = \frac{\E_t\sbrc{d\tilde{r}_t}}{\gamma\Var_t\sbrc{d\tilde{r}_t}}, 
    \quad\Rightarrow\quad 
    D(r,a) = \frac{\E\sbrc{d\tilde{r}_t}}{\gamma\Var\sbrc{d\tilde{r}_t}}a.
\end{equation}
Let $k^*(a)$ be the continuation boundary. We have $E(k^*(a),a) = E_0$. 
Suppose that the when the bank default, the risky asset is illiquid and 
all the deposit disappears. Hence 
\begin{equation}
    da_t = \theta_ta_trdt - \theta_{t^-}a_{t^-}dN_t
\end{equation}
where the rate of $N_t$ is $\eta P((1-\kappa)k\leq k^*(a)) = \eta (1 - F_\kappa(1 - k^*/k)) = \eta\delta$. 
\begin{equation}
    \theta_t = \begin{cases}
        1, & \frac{1}{\gamma}\sbrc{\frac{r}{\eta\delta}-1} > 1 \\
        \frac{1}{\gamma}\sbrc{\frac{r}{\eta\delta}-1}, & \text{o.w.} \\
        0, & \frac{1}{\gamma}\sbrc{\frac{r}{\eta\delta}-1} < 0.
    \end{cases}
\end{equation}
Take the parametric assumption that $F_I$ is $Exp(\beta)$ and $F_\kappa$ is 
$U(0,1)$. The optimal deposit rate solves 
\begin{equation}
    \begin{split}
       &(E_a - 1)(D + rD_r) - \lambda\beta\exp(-\beta(D-k))D_rE(D,a) \\
       &\quad + \lambda\exp(-\beta(D-k))E_k(D,a)D_r + E(D,a)\beta\exp(-\beta(D-k)) = 0.     
    \end{split}
\end{equation} 
We solve the interior case and truncate when needed. 


\bibliographystyle{apacite}
\bibliography{large_small_bank}

\end{document}