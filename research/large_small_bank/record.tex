\documentclass[a4paper, 12pt]{article}
\input{prefix_large_small_bank.tex}%
\input{symbols_large_small_bank.tex}

\title{Research Record for the Project "Large vs Small Banks"}
\author{Kai-Jyun Wang\thanks{National Taiwan University, Department of Economics.}}
\date{Fall 2024}

\begin{document} 
\setstretch{1.15}
\maketitle

\section{Main Idea} 
This project aims to study the relation between the size of a 
bank in terms of its balance sheet and the probability of the 
bank run occurring. And, if the larger the bank, the more can 
a bank immune itself from the bank run, then what is the 
implication of this result? 

Suppose that the banking market is oligopolistic, then there 
is a dead weight loss in the market (under some kind of oligopolistic 
competition), and the larger the bank, the more oligopolistic the 
market is. However, if the size also prevents the bank from the 
crisis, then the limitation of the size of the bank can increase 
the financial fragility of the banking system. This is the 
trade-off between the oligopoly and the financial fragility. 
Consider the limitation of the size of the bank and the federal 
deposit insurance. 

\section{To Do}
\begin{itemize}
    \item See how to incorporate the bank run from \cite{diamond1983}.  
    \item Evidence of the bank size-bank run relation.
\end{itemize}

\section{Model} 
\subsection{Household}
The time is continuous, $t\in[0,\infty)$. There is a unit mass of households. 
Each household solves the following problem:
\begin{equation}
    \rho Vdt = \max_{c,\theta,e} u(c)dt + \E_t\sbrc{dV}
\end{equation}
subject to 
\begin{align}
    da &= \sbrc{a\pth{\sum_{n=1}^N\theta_ndr_n + e_ndr^e_n}} - cdt \\
    a &\geq 0
\end{align}
where $\theta_0$ is the asset proportion holding on hand, $\theta_n$ is the asset 
proportion as deposits and $e_n$ is the proportion as equity. 

Derive deposit supply. 

\textcolor{red}{Think about the information shocks.}

\subsection{Bank}
There are $N$ banks, competing in the deposit market in Cournot's fashion. 

What makes the depositors choosing different banks?

The banks have two type of assets. One is a risk-free bond $B_t$ with fixed 
return rate $r^f$. The others are risky assets $k_t$ that require costs to build. 
A risky asset cannot be held by different banks so the single bank can 
only hold $k_t\in\set{0,1,\ldots}$. The risky assets arrive at rate $\lambda_k$, 
requiring cost $C$, and pays off $R$ stochastically at rate $\lambda_R$. After 
$R$ arrives, the risky asset is destructed. Total asset value is 
\begin{equation}
    A_t = q_{B,t}B_t + q_{k,t}k_t. 
\end{equation}

$d$ and $e$ are the proportion of asset purchasing financed by deposit 
and equity. 
\begin{equation}
    \begin{split}
        \mu_eEdt &= \max_{d,e,B,k} r^fBdt + \E\sbrc{dA} - \E\sbrc{Ad\cdot dr + (Ae - E)dr^e} \\
        &+ \lambda_k\int \one\set{C\leq q_BB}\max\set{E(k+1,q_BB-C)-E, 0}dF(C) \\ 
        &+ \lambda_Rk(E(k-1, q_BB+R) - E)
    \end{split}
\end{equation}

Bank run and competition. 

A bank falls when its net worth becomes negative. When a bank falls, the rest 
of the banks can buy the failed bank's asset and the depositors can take their 
deposits back according to the proportion. 

A new bank enters the market with rate $\psi$. 



\bibliographystyle{apacite}
\bibliography{large_small_bank}

\end{document}