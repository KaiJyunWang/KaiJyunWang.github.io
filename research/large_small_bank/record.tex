\documentclass[a4paper, 12pt]{article}
% For fonts
% \usepackage{palatino}
% \usepackage[T1]{fontenc}
% \usepackage{mathpazo}
% \usepackage{textcomp}
% \usepackage{fouriernc}
% \usepackage{mathastext}
\usepackage[scaled = 0.9]{scholax}



\usepackage{setspace}
\usepackage{amsmath}%
\usepackage{amssymb}%
\usepackage{mathtools}
\usepackage{dsfont}
\usepackage[table]{xcolor}%

\setlength{\marginparwidth}{6cm}
\usepackage{todonotes}
\usepackage[in]{fullpage}%
%\usepackage{enumitem}

\usepackage{amsthm}

\usepackage{titlesec}%
\titlelabel{\thetitle.\hspace{0.5em}}%
\usepackage{xcolor}%
\usepackage{mleftright}
\usepackage{xspace}%
\usepackage{graphicx}
\usepackage{hyperref}%
\usepackage{etoolbox}
\usepackage{lipsum}
\usepackage{appendix}
\usepackage{tikz}
\usepackage{mathrsfs}
\usepackage{listings}
\usepackage{enumitem}
% \usepackage[normalem]{ulem}
\usepackage[nameinlink,noabbrev]{cleveref}
% Math font, must be loaded after amsmath
\usepackage[scaled = 0.95,varbb]{newtxmath}
\usepackage[natbibapa]{apacite}

\usepackage{hyperref}%
\hypersetup{%
unicode,
breaklinks,%
colorlinks=true,%
urlcolor=blue,%
linkcolor=[rgb]{0.5,0.0,0.0},%
citecolor=[rgb]{0,0.3,0.445},%
filecolor=[rgb]{0,0,0.4},
anchorcolor=[rgb]={0.0,0.1,0.2}%
}

\newenvironment{thmenum}{%
\begin{enumerate}[label=\small(\alph*\small),topsep=4pt,itemsep=4pt,partopsep=0pt, parsep=0pt]
}{%
\end{enumerate}
}

\newtheoremstyle{break}
  {\topsep}{\topsep}%
  {\itshape}{}%
  {\bfseries}{}%
  {\newline}{}%
\theoremstyle{break}
\newtheorem{theorem}{Theorem}[section]

\theoremstyle{break}
\newtheorem{lemma}[theorem]{Lemma}

\theoremstyle{break}
\newtheorem{corollary}[theorem]{Corollary}

\theoremstyle{break}
\newtheorem{proposition}[theorem]{Proposition}

\theoremstyle{break}
\newtheorem{definition}[theorem]{Definition}

\theoremstyle{break}
\newtheorem*{remark}{Remark}

\theoremstyle{break}
\newtheorem*{example}{Example}

\newtheorem{innercustomthm}{Exercise}
\newenvironment{exercise}[1]{\renewcommand\theinnercustomthm{#1}\innercustomthm}{\endinnercustomthm}

\newenvironment{solution}{\begin{proof}[Solution]}{\end{proof}}

\renewcommand\qedsymbol{\rule{2mm}{2mm}}
%
\newcommand{\Set}[2]{\left\{ #1 \;\middle\vert\; #2 \right\}}

\newcommand{\pth}[1]{\mleft(#1\mright)}%

\newcommand{\E}{{\operatorname{E}}}
\newcommand{\diam}[1]{\operatorname{diam}\mleft(#1\mright)}
\newcommand{\osc}[2]{\operatorname{osc}\mleft(#1,\, #2\mright)}
\newcommand{\supp}[1]{\operatorname{supp}\mleft(#1\mright)}
\newcommand{\indicator}{\mathbf{1}}
\newcommand{\argmin}{\operatorname{argmin}}
\newcommand{\argmax}{\operatorname{argmax}}

\newcommand{\one}{\mathbf{1}}
\newcommand{\Ex}[1]{\E\mleft[ #1 \mright]}

\newcommand{\ceil}[1]{\mleft\lceil {#1} \mright\rceil}
\newcommand{\floor}[1]{\mleft\lfloor {#1} \mright\rfloor}
\newcommand{\sgn}{\operatorname{sgn}}
\newcommand{\card}{\operatorname{card}}

\newcommand{\brc}[1]{\left\{ {#1} \right\}}
\newcommand{\sbrc}[1]{\left[ {#1} \right]}
\newcommand{\set}[1]{\brc{#1}}%

\newcommand{\abs}[1]{\left\lvert {#1} \right\rvert}%
\newcommand{\norm}[1]{\left\lVert {#1} \right\rVert}
\newcommand{\spanby}{\operatorname{span}}
\newcommand{\esssup}{\operatorname{ess\,sup}}
\newcommand{\inp}[2]{\left\langle {#1},\, {#2} \right\rangle}

\newcommand{\mto}{\overset{m}{\to}}

\newcommand{\ds}{\displaystyle}%

\newcommand{\R}{\mathbb{R}}%
\newcommand{\N}{\mathbb{N}}
\newcommand{\Q}{\mathbb{Q}}
\newcommand{\Z}{\mathbb{Z}}
\newcommand{\C}{\mathbb{C}}
\newcommand{\F}{\mathcal{F}}
\newcommand{\A}{\mathcal{A}}
\newcommand{\B}{\mathcal{B}}
\newcommand{\T}{\mathcal{T}}
\newcommand{\M}{\mathcal{M}}
\renewcommand{\L}{\mathcal{L}}
\renewcommand{\S}{\mathcal{S}}
\newcommand{\I}{\mathcal{I}}
\renewcommand{\H}{\mathcal{H}}
\newcommand{\D}{\mathcal{D}}

\newcommand{\exist}{\exists\:}
\newcommand{\forany}{\forall\:}
\renewcommand{\ae}{\quad\text{a.e.}}


\newcommand{\restrict}[1]{\left.\hspace{-0.2em}\right|_{#1}}
\newcommand{\eval}[3]{\left.{#1}\right|_{#2}^{#3}}

% derivatives
\makeatletter
\renewcommand\d[1]{\mspace{6mu}\mathrm{d}#1\@ifnextchar\d{\mspace{-3mu}}{}}
\makeatother
\def\at{
  \left.
  \vphantom{\int}
  \right|
}

\newcommand{\od}[2]{\frac{d{#1}}{d{#2}}}
\newcommand{\odd}[2]{\frac{d^2{#1}}{d{#2}^2}}
\newcommand{\pd}[2]{\frac{\partial{#1}}{\partial{#2}}}
\newcommand{\pdd}[2]{\frac{\partial^2{#1}}{\partial{#2}^2}}
\newcommand{\pdpd}[3]{\frac{\partial^2{#1}}{\partial{#2}\partial{#3}}}

\makeatletter
\newcommand*{\rom}[1]{\expandafter\@slowromancap\romannumeral #1@}
\makeatother

\title{Research Record for the Project "Large vs Small Banks"}
\author{Kai-Jyun Wang\thanks{National Taiwan University, Department of Economics.}}
\date{Fall 2024}

\begin{document} 
\setstretch{1.15}
\maketitle

\section{Model} 
\subsection{Household}
The time is continuous, $t\in[0,\infty)$. There is a unit mass of households. 
Each household solves the following problem:
\begin{equation}
    \rho Vdt = \max_{c,\theta,e} u(c)dt + \E_t\sbrc{dV}
\end{equation}
subject to 
\begin{align}
    da &= \sbrc{a\pth{\int_0^N\theta_ndr_n + e_ndr^e_ndn}} - cdt \\
    a &\geq 0
\end{align}
where $\theta_0$ is the asset proportion holding on hand, $\theta_n$ is the asset 
proportion as deposits and $e_n$ is the proportion as equity. 

Derive deposit and equity demand. 

$\chi = (\chi_1,\ldots,\chi_N)$ are the believes about 
banks' risky asset holding proportion. At rate $\lambda$, 
a bank's portfolio is going to be revealed. Let $\tilde{\chi}_t$ 
be the underlying portfolio, 
\begin{equation*}
    \chi_t = \E\sbrc{\tilde{\chi}_t|\F_t}. 
\end{equation*}
Now let $d\zeta_t$ be the new signal. 
\begin{equation*}
    \begin{split}
        d\chi_t &= \E\sbrc{\tilde{\chi}_{t+dt}|\F_t,d\zeta_t} - \E\sbrc{\tilde{\chi}_t|\F_t}
        = \E\sbrc{\tilde{\chi}_t + d\tilde{\chi}_t|\F_t,d\zeta_t} - \E\sbrc{\tilde{\chi}_t|\F_t} \\ 
        &= \E\sbrc{d\tilde{\chi}_t}
    \end{split}
\end{equation*}

\textcolor{red}{Think about the information shocks.}

\subsection{Bank}
Banks are indexed in $[0,N_t]$, competing in deposit market in monopolistic 
competition fashion. The banks have two type of assets. One is a risk-free bond 
$B_t$ with fixed return rate $r^f$. The others are risky assets $k_t$ that require 
costs to build. A risky asset cannot be held by different banks so the single bank 
can only hold $k_t\in\set{0,1,\ldots}$. The risky assets arrive at rate $\lambda_k$, 
requiring cost $C$, and pays off $R$ stochastically at rate $\lambda_R$. 
\begin{equation*}
    dR_t = d(idiosyncratic) + d(systematic).
\end{equation*}

After $R$ arrives, the risky asset is destructed. Total asset value is 
\begin{equation}
    A_t = q_{B,t}B_t + q_{k,t}k_t. 
\end{equation}

$d$ and $e$ are the proportion of asset holding financed by deposit 
and equity. 
\begin{equation}
    \begin{split}
        \mu_eEdt &= \max_{dr,dr^e,B,k} r^fBdt + d(q_BB) + d(q_kk) - \E\sbrc{a\theta_n(dr)dr + ae_n(dr^e)dr^e} \\
        &+ \lambda_k\int\one\set{x\leq q_BB}\max\set{q_k(k+1)+q_BB - C - E, 0}dF_C(x)\\ 
        &+ \lambda_Rk\int q_k(k-1) + q_BB + x - EdF_R(x)
    \end{split}
\end{equation}

There are $N$ banks, competing in the security market in monopolistic 
competition manner. The bank $n$ can issue two types of security,  
deposit $d$ and equity $e$. The returns satisfying that 
\begin{equation*}
    dr^d = \mu^ddt
    \quad\text{and}\quad 
    dr^e = \mu^edt + dZ_t^n, 
\end{equation*}
where $dZ_t^n$ is a compensated Poisson process with rate 
$\lambda_n$. Since the banks are risk neutral, they will 
fund their investments through issuing as many deposits as 
they can.


\bibliographystyle{apacite}
\bibliography{large_small_bank}

\end{document}