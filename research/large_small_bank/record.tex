\documentclass[a4paper, 12pt]{article}
\input{prefix_large_small_bank.tex}%
\input{symbols_large_small_bank.tex}

\title{Research Record for the Project "Large vs Small Banks"}
\author{Kai-Jyun Wang\thanks{National Taiwan University, Department of Economics.}}
\date{Fall 2024}

\begin{document} 
\setstretch{1.15}
\maketitle

\section{Model} 
\subsection{Household}
The time is continuous, $t\in[0,\infty)$. There is a unit mass of households. 
Each household solves the following problem:
\begin{equation}
    \rho Vdt = \max_{c,\theta,e} u(c)dt + \E_t\sbrc{dV}
\end{equation}
subject to 
\begin{align}
    da &= \sbrc{a\pth{\int_0^N\theta_ndr_n + e_ndr^e_ndn}} - cdt \\
    a &\geq 0
\end{align}
where $\theta_0$ is the asset proportion holding on hand, $\theta_n$ is the asset 
proportion as deposits and $e_n$ is the proportion as equity. 

Derive deposit and equity demand. 

$\chi = (\chi_1,\ldots,\chi_N)$ are the believes about 
banks' risky asset holding proportion. At rate $\lambda$, 
a bank's portfolio is going to be revealed. Let $\tilde{\chi}_t$ 
be the underlying portfolio, 
\begin{equation*}
    \chi_t = \E\sbrc{\tilde{\chi}_t|\F_t}. 
\end{equation*}
Now let $d\zeta_t$ be the new signal. 
\begin{equation*}
    \begin{split}
        d\chi_t &= \E\sbrc{\tilde{\chi}_{t+dt}|\F_t,d\zeta_t} - \E\sbrc{\tilde{\chi}_t|\F_t}
        = \E\sbrc{\tilde{\chi}_t + d\tilde{\chi}_t|\F_t,d\zeta_t} - \E\sbrc{\tilde{\chi}_t|\F_t} \\ 
        &= \E\sbrc{d\tilde{\chi}_t}
    \end{split}
\end{equation*}

\textcolor{red}{Think about the information shocks.}

\subsection{Bank}
Banks are indexed in $[0,N_t]$, competing in deposit market in monopolistic 
competition fashion. The banks have two type of assets. One is a risk-free bond 
$B_t$ with fixed return rate $r^f$. The others are risky assets $k_t$ that require 
costs to build. A risky asset cannot be held by different banks so the single bank 
can only hold $k_t\in\set{0,1,\ldots}$. The risky assets arrive at rate $\lambda_k$, 
requiring cost $C$, and pays off $R$ stochastically at rate $\lambda_R$. 
\begin{equation*}
    dR_t = d(idiosyncratic) + d(systematic).
\end{equation*}

After $R$ arrives, the risky asset is destructed. Total asset value is 
\begin{equation}
    A_t = q_{B,t}B_t + q_{k,t}k_t. 
\end{equation}

$d$ and $e$ are the proportion of asset holding financed by deposit 
and equity. 
\begin{equation}
    \begin{split}
        \mu_eEdt &= \max_{dr,dr^e,B,k} r^fBdt + d(q_BB) + d(q_kk) - \E\sbrc{a\theta_n(dr)dr + ae_n(dr^e)dr^e} \\
        &+ \lambda_k\int\one\set{x\leq q_BB}\max\set{q_k(k+1)+q_BB - C - E, 0}dF_C(x)\\ 
        &+ \lambda_Rk\int q_k(k-1) + q_BB + x - EdF_R(x)
    \end{split}
\end{equation}

There are $N$ banks, competing in the security market in monopolistic 
competition manner. The bank $n$ can issue two types of security,  
deposit $d$ and equity $e$. The returns satisfying that 
\begin{equation*}
    dr^d = \mu^ddt
    \quad\text{and}\quad 
    dr^e = \mu^edt + dZ_t^n, 
\end{equation*}
where $dZ_t^n$ is a compensated Poisson process with rate 
$\lambda_n$. Since the banks are risk neutral, they will 
fund their investments through issuing as many deposits as 
they can.


\bibliographystyle{apacite}
\bibliography{large_small_bank}

\end{document}