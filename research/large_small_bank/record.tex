\documentclass[a4paper, 12pt]{article}
% For fonts
% \usepackage{palatino}
% \usepackage[T1]{fontenc}
% \usepackage{mathpazo}
% \usepackage{textcomp}
% \usepackage{fouriernc}
% \usepackage{mathastext}
\usepackage[scaled = 0.9]{scholax}



\usepackage{setspace}
\usepackage{amsmath}%
\usepackage{amssymb}%
\usepackage{mathtools}
\usepackage{dsfont}
\usepackage[table]{xcolor}%

\setlength{\marginparwidth}{6cm}
\usepackage{todonotes}
\usepackage[in]{fullpage}%
%\usepackage{enumitem}

\usepackage{amsthm}

\usepackage{titlesec}%
\titlelabel{\thetitle.\hspace{0.5em}}%
\usepackage{xcolor}%
\usepackage{mleftright}
\usepackage{xspace}%
\usepackage{graphicx}
\usepackage{hyperref}%
\usepackage{etoolbox}
\usepackage{lipsum}
\usepackage{appendix}
\usepackage{tikz}
\usepackage{mathrsfs}
\usepackage{listings}
\usepackage{enumitem}
% \usepackage[normalem]{ulem}
\usepackage[nameinlink,noabbrev]{cleveref}
% Math font, must be loaded after amsmath
\usepackage[scaled = 0.95,varbb]{newtxmath}
\usepackage[natbibapa]{apacite}

\usepackage{hyperref}%
\hypersetup{%
unicode,
breaklinks,%
colorlinks=true,%
urlcolor=blue,%
linkcolor=[rgb]{0.5,0.0,0.0},%
citecolor=[rgb]{0,0.3,0.445},%
filecolor=[rgb]{0,0,0.4},
anchorcolor=[rgb]={0.0,0.1,0.2}%
}

\newenvironment{thmenum}{%
\begin{enumerate}[label=\small(\alph*\small),topsep=4pt,itemsep=4pt,partopsep=0pt, parsep=0pt]
}{%
\end{enumerate}
}

\newtheoremstyle{break}
  {\topsep}{\topsep}%
  {\itshape}{}%
  {\bfseries}{}%
  {\newline}{}%
\theoremstyle{break}
\newtheorem{theorem}{Theorem}[section]

\theoremstyle{break}
\newtheorem{lemma}[theorem]{Lemma}

\theoremstyle{break}
\newtheorem{corollary}[theorem]{Corollary}

\theoremstyle{break}
\newtheorem{proposition}[theorem]{Proposition}

\theoremstyle{break}
\newtheorem{definition}[theorem]{Definition}

\theoremstyle{break}
\newtheorem*{remark}{Remark}

\theoremstyle{break}
\newtheorem*{example}{Example}

\newtheorem{innercustomthm}{Exercise}
\newenvironment{exercise}[1]{\renewcommand\theinnercustomthm{#1}\innercustomthm}{\endinnercustomthm}

\newenvironment{solution}{\begin{proof}[Solution]}{\end{proof}}

\renewcommand\qedsymbol{\rule{2mm}{2mm}}
%
\newcommand{\Set}[2]{\left\{ #1 \;\middle\vert\; #2 \right\}}

\newcommand{\pth}[1]{\mleft(#1\mright)}%

\newcommand{\E}{{\operatorname{E}}}
\newcommand{\diam}[1]{\operatorname{diam}\mleft(#1\mright)}
\newcommand{\osc}[2]{\operatorname{osc}\mleft(#1,\, #2\mright)}
\newcommand{\supp}[1]{\operatorname{supp}\mleft(#1\mright)}
\newcommand{\indicator}{\mathbf{1}}
\newcommand{\argmin}{\operatorname{argmin}}
\newcommand{\argmax}{\operatorname{argmax}}

\newcommand{\one}{\mathbf{1}}
\newcommand{\Ex}[1]{\E\mleft[ #1 \mright]}

\newcommand{\ceil}[1]{\mleft\lceil {#1} \mright\rceil}
\newcommand{\floor}[1]{\mleft\lfloor {#1} \mright\rfloor}
\newcommand{\sgn}{\operatorname{sgn}}
\newcommand{\card}{\operatorname{card}}

\newcommand{\brc}[1]{\left\{ {#1} \right\}}
\newcommand{\sbrc}[1]{\left[ {#1} \right]}
\newcommand{\set}[1]{\brc{#1}}%

\newcommand{\abs}[1]{\left\lvert {#1} \right\rvert}%
\newcommand{\norm}[1]{\left\lVert {#1} \right\rVert}
\newcommand{\spanby}{\operatorname{span}}
\newcommand{\esssup}{\operatorname{ess\,sup}}
\newcommand{\inp}[2]{\left\langle {#1},\, {#2} \right\rangle}

\newcommand{\mto}{\overset{m}{\to}}

\newcommand{\ds}{\displaystyle}%

\newcommand{\R}{\mathbb{R}}%
\newcommand{\N}{\mathbb{N}}
\newcommand{\Q}{\mathbb{Q}}
\newcommand{\Z}{\mathbb{Z}}
\newcommand{\C}{\mathbb{C}}
\newcommand{\F}{\mathcal{F}}
\newcommand{\A}{\mathcal{A}}
\newcommand{\B}{\mathcal{B}}
\newcommand{\T}{\mathcal{T}}
\newcommand{\M}{\mathcal{M}}
\renewcommand{\L}{\mathcal{L}}
\renewcommand{\S}{\mathcal{S}}
\newcommand{\I}{\mathcal{I}}
\renewcommand{\H}{\mathcal{H}}
\newcommand{\D}{\mathcal{D}}

\newcommand{\exist}{\exists\:}
\newcommand{\forany}{\forall\:}
\renewcommand{\ae}{\quad\text{a.e.}}


\newcommand{\restrict}[1]{\left.\hspace{-0.2em}\right|_{#1}}
\newcommand{\eval}[3]{\left.{#1}\right|_{#2}^{#3}}

% derivatives
\makeatletter
\renewcommand\d[1]{\mspace{6mu}\mathrm{d}#1\@ifnextchar\d{\mspace{-3mu}}{}}
\makeatother
\def\at{
  \left.
  \vphantom{\int}
  \right|
}

\newcommand{\od}[2]{\frac{d{#1}}{d{#2}}}
\newcommand{\odd}[2]{\frac{d^2{#1}}{d{#2}^2}}
\newcommand{\pd}[2]{\frac{\partial{#1}}{\partial{#2}}}
\newcommand{\pdd}[2]{\frac{\partial^2{#1}}{\partial{#2}^2}}
\newcommand{\pdpd}[3]{\frac{\partial^2{#1}}{\partial{#2}\partial{#3}}}

\makeatletter
\newcommand*{\rom}[1]{\expandafter\@slowromancap\romannumeral #1@}
\makeatother

\title{Research Record for the Project "Large vs Small Banks"}
\author{Kai-Jyun Wang\thanks{National Taiwan University, Department of Economics.}}
\date{Fall 2024}

\begin{document} 
\setstretch{1.15}
\maketitle

\section{Main Idea} 
This project aims to study the relation between the size of a 
bank in terms of its balance sheet and the probability of the 
bank run occurring. And, if the larger the bank, the more can 
a bank immune itself from the bank run, then what is the 
implication of this result? 

Suppose that the banking market is oligopolistic, then there 
is a dead weight loss in the market (under some kind of oligopolistic 
competition), and the larger the bank, the more oligopolistic the 
market is. However, if the size also prevents the bank from the 
crisis, then the limitation of the size of the bank can increase 
the financial fragility of the banking system. This is the 
trade-off between the oligopoly and the financial fragility. 
Consider the limitation of the size of the bank and the federal 
deposit insurance. 

\section{To Do}
\begin{itemize}
    \item See how to incorporate the bank run from \cite{diamond1983}.  
    \item How to write a model having the feature that the bank size affects 
    the probability of a bank run. 
    \item Evidence of the bank size-bank run relation.
\end{itemize}

\section{Model} 
\subsection{Household}
The time is continuous, $t\in[0,\infty)$. There is a unit mass of households. 
Each household solves the following problem:
\begin{equation}
    \max_{c_t, \theta_{n,t}} \E\sbrc{\int_0^\infty e^{-\rho t}u(c_t)dt}
\end{equation}
subject to
\begin{align}
    da_t &= \sbrc{r_t\sum_{n=1}^N\theta_{n,t}a_t - c_t}dt - \sum_{n=1}^N a_{t^-}\theta_{n,t^-}dJ_{n,t}, \\
    a_t &\geq 0, 
\end{align}
where $a_t$ is the total asset of the household at time $t$, 
$\theta_{n,t}$ is the fraction of asset in bank $n$ savings with interest 
rate $r_t$. The bank savings suffer from the risk of default, 
with rate $\lambda_{n,t}$, so that $J_{n,t}$ is a Poisson process with intensity 
$\lambda_{n,t}$. The process $\lambda_{n,t}$ is determined in a similar manner 
as \cite{moreira2017}, equation (10).

\textcolor{red}{Think about the information shocks.}

\subsection{Bank}
There are $N$ banks, competing in the deposit market in Cournot's fashion. 

What makes the depositors choosing different banks?

The banks can choose to invest in a risky asset 
$k$, which yields $y_t = Ak_t$ and follows the process 
\begin{equation*}
    dk_t = k_t\phi(\iota_t)dt - \kappa k_tdZ_t, 
\end{equation*} 
where $dZ_t$ is a compensated Poisson process with intensity $\lambda_t$.
Let $p_t$ be the price of the risky asset, $E_t$ be the value of existing 
equity and $D_t$ be the total deposits. 


A bank falls when its net worth becomes negative. When a bank falls, the rest 
of the banks can buy the failed bank's asset and the depositors can take their 
deposits back according to the proportion. 

A new bank enters the market with rate $\psi$. 



\bibliographystyle{apacite}
\bibliography{large_small_bank}

\end{document}