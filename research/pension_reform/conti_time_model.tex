\documentclass[a4paper, 12pt]{article}
% For fonts
% \usepackage{palatino}
% \usepackage[T1]{fontenc}
% \usepackage{mathpazo}
% \usepackage{textcomp}
% \usepackage{fouriernc}
% \usepackage{mathastext}
\usepackage[scaled = 0.9]{scholax}



\usepackage{setspace}
\usepackage{amsmath}%
\usepackage{amssymb}%
\usepackage{mathtools}
\usepackage{dsfont}
\usepackage[table]{xcolor}%

\setlength{\marginparwidth}{6cm}
\usepackage{todonotes}
\usepackage[in]{fullpage}%
%\usepackage{enumitem}

\usepackage{amsthm}

\usepackage{titlesec}%
\titlelabel{\thetitle.\hspace{0.5em}}%
\usepackage{xcolor}%
\usepackage{mleftright}
\usepackage{xspace}%
\usepackage{graphicx}
\usepackage{hyperref}%
\usepackage{etoolbox}
\usepackage{lipsum}
\usepackage{appendix}
\usepackage{tikz}
\usepackage{mathrsfs}
\usepackage{listings}
\usepackage{enumitem}
% \usepackage[normalem]{ulem}
\usepackage[nameinlink,noabbrev]{cleveref}
% Math font, must be loaded after amsmath
\usepackage[scaled = 0.95,varbb]{newtxmath}
\usepackage[natbibapa]{apacite}

\usepackage{hyperref}%
\hypersetup{%
unicode,
breaklinks,%
colorlinks=true,%
urlcolor=blue,%
linkcolor=[rgb]{0.5,0.0,0.0},%
citecolor=[rgb]{0,0.3,0.445},%
filecolor=[rgb]{0,0,0.4},
anchorcolor=[rgb]={0.0,0.1,0.2}%
}

\newenvironment{thmenum}{%
\begin{enumerate}[label=\small(\alph*\small),topsep=4pt,itemsep=4pt,partopsep=0pt, parsep=0pt]
}{%
\end{enumerate}
}

\newtheoremstyle{break}
  {\topsep}{\topsep}%
  {\itshape}{}%
  {\bfseries}{}%
  {\newline}{}%
\theoremstyle{break}
\newtheorem{theorem}{Theorem}[section]

\theoremstyle{break}
\newtheorem{lemma}[theorem]{Lemma}

\theoremstyle{break}
\newtheorem{corollary}[theorem]{Corollary}

\theoremstyle{break}
\newtheorem{proposition}[theorem]{Proposition}

\theoremstyle{break}
\newtheorem{definition}[theorem]{Definition}

\theoremstyle{break}
\newtheorem*{remark}{Remark}

\theoremstyle{break}
\newtheorem*{example}{Example}

\newtheorem{innercustomthm}{Exercise}
\newenvironment{exercise}[1]{\renewcommand\theinnercustomthm{#1}\innercustomthm}{\endinnercustomthm}

\newenvironment{solution}{\begin{proof}[Solution]}{\end{proof}}

\renewcommand\qedsymbol{\rule{2mm}{2mm}}
%
\newcommand{\Set}[2]{\left\{ #1 \;\middle\vert\; #2 \right\}}

\newcommand{\pth}[1]{\mleft(#1\mright)}%

\newcommand{\E}{{\mathbb{E}}}
\newcommand{\diam}[1]{\operatorname{diam}\mleft(#1\mright)}
\newcommand{\osc}[2]{\operatorname{osc}\mleft(#1,\, #2\mright)}
\newcommand{\supp}[1]{\operatorname{supp}\mleft(#1\mright)}
\newcommand{\indicator}{\mathbf{1}}
\newcommand{\argmin}{\operatorname{argmin}}
\newcommand{\argmax}{\operatorname{argmax}}


\newcommand{\Ex}[1]{\E\mleft[ #1 \mright]}

\newcommand{\ceil}[1]{\mleft\lceil {#1} \mright\rceil}
\newcommand{\floor}[1]{\mleft\lfloor {#1} \mright\rfloor}
\newcommand{\sgn}{\operatorname{sgn}}
\newcommand{\card}{\operatorname{card}}

\newcommand{\brc}[1]{\left\{ {#1} \right\}}
\newcommand{\sbrc}[1]{\left[ {#1} \right]}
\newcommand{\set}[1]{\brc{#1}}%

\newcommand{\abs}[1]{\left\lvert {#1} \right\rvert}%
\newcommand{\norm}[1]{\left\lVert {#1} \right\rVert}
\newcommand{\spanby}{\operatorname{span}}
\newcommand{\esssup}{\operatorname{ess\,sup}}
\newcommand{\inp}[2]{\left\langle {#1},\, {#2} \right\rangle}

\newcommand{\mto}{\overset{m}{\to}}

\newcommand{\ds}{\displaystyle}%

\newcommand{\R}{\mathbb{R}}%
\newcommand{\N}{\mathbb{N}}
\newcommand{\Q}{\mathbb{Q}}
\newcommand{\Z}{\mathbb{Z}}
\newcommand{\C}{\mathbb{C}}
\newcommand{\F}{\mathcal{F}}
\newcommand{\A}{\mathcal{A}}
\newcommand{\B}{\mathcal{B}}
\newcommand{\T}{\mathcal{T}}
\newcommand{\M}{\mathcal{M}}
\renewcommand{\L}{\mathcal{L}}
\renewcommand{\S}{\mathcal{S}}
\newcommand{\I}{\mathcal{I}}
\renewcommand{\H}{\mathcal{H}}
\newcommand{\D}{\mathcal{D}}

\newcommand{\exist}{\exists\:}
\newcommand{\forany}{\forall\:}
\renewcommand{\ae}{\quad\text{a.e.}}


\newcommand{\restrict}[1]{\left.\hspace{-0.2em}\right|_{#1}}
\newcommand{\eval}[3]{\left.{#1}\right|_{#2}^{#3}}

% derivatives
\makeatletter
\renewcommand\d[1]{\mspace{6mu}\mathrm{d}#1\@ifnextchar\d{\mspace{-3mu}}{}}
\makeatother
\def\at{
  \left.
  \vphantom{\int}
  \right|
}

\newcommand{\od}[2]{\frac{d{#1}}{d{#2}}}
\newcommand{\odd}[2]{\frac{d^2{#1}}{d{#2}^2}}
\newcommand{\pd}[2]{\frac{\partial{#1}}{\partial{#2}}}
\newcommand{\pdd}[2]{\frac{\partial^2{#1}}{\partial{#2}^2}}
\newcommand{\pdpd}[3]{\frac{\partial^2{#1}}{\partial{#2}\partial{#3}}}

\makeatletter
\newcommand*{\rom}[1]{\expandafter\@slowromancap\romannumeral #1@}
\makeatother 

\title{Continuous Time Model for Pension Reform}
\author{Kai-Jyun Wang\thanks{National Taiwan University, Department of Economics.}}
\date{Fall 2024}

\begin{document} 
\setstretch{1.15}
\maketitle

\section{Model}

Time is continuous and starts from $t = 0$ to $T$. The model consists of four states: asset 
$a_t$, income level $y_t$, labor supply $\set{e,u}$ in extensive margin. The utility flow is 
given by $u^e(c) = \frac{\alpha_e}{1-\gamma}c^{1-\gamma}$ for employed agents and 
$u^u(c) = \frac{1}{1-\gamma}c^{1-\gamma}$ for unemployed. The agents die at rate $h_t$, when 
the death occurs, the agents receive utility from bequest $b(a) = \frac{\alpha_b}{1-\gamma}
(\kappa + a)^{1-\gamma}$.

The asset states and the income level are governed by the following laws of 
motions: 
\begin{align}
    da_t &= \sbrc{ra_t + y_t - c_t}dt, \\
    dy_t &= y_t(\exp(\mu_t + \sigma e_t) - 1)dJ_t, 
\end{align}
where $J_t$ is the Poisson process. $y_t$ is modelled as a compound Poisson 
process with intensity $\lambda_y$ and $e_t$ drawn from $N(0,1)$. If the agent 
is unemployed, then $y_t = y_f$ is fixed. 

Now assume that the quit opportunity comes at rate $\lambda^e$ and the job offer 
arrives at rate $\lambda^u$ with the income level drawn from $F$. We can write the 
value function recursively as
\begin{equation}\label{eq:value}
    \begin{split}
        &(\rho + h_t)\begin{bmatrix}
        V^e(t, a_t, y_t) \\
        V^u(t, a_t) \\
        \end{bmatrix} \\
        &= \begin{bmatrix}
            \max_{c_t^e} u^e(c_t^e) + h_tb(a_t) + \lambda_y\int V^e(t, a_t, y_t\exp(\mu_t + \sigma e)) - V^e(t, a_t, y_t) d\Phi(e) \\
            \qquad+ \lambda^e\max\set{V^u(t, a_t) - V^e(t, a_t, y_t), 0} + V_t^e + V_a^e(ra_t + y_t - c_t^e) \\
            \max_{c_t^u} u^u(c_t^u) + h_tb(a_t) + \lambda^u\int \max\set{V^e(t, a_t, y) - V^u(t, a_t), 0}dF(y) + V_t^u + V_a^u(ra_t + y_f - c_t^u) \\
        \end{bmatrix}.
    \end{split}
\end{equation}

Solving the maximization problem in \cref{eq:value} gives us the optimal consumption 
when the asset is not binding, i.e., $a>0$. The first-order conditions yield that 
\begin{equation}
    V_a^e = \alpha_e c^{-\gamma}, \quad V_a^u = c^{-\gamma} 
    \quad \Rightarrow\quad 
    c^e = \pth{\frac{\alpha_e}{V^e_a}}^{1/\gamma}, \quad 
    c^u = \pth{\frac{1}{V^u_a}}^{1/\gamma}.
\end{equation}
Hence, 
\begin{align}
    (\rho + h_t + \lambda_y + \lambda^e)V^e &= \frac{\alpha_e}{1-\gamma}\pth{\frac{\alpha_e}{V^e_a}}^{1/\gamma - 1} + h_tb(a) + \lambda_y\int V^e(t,a,y\exp(\mu_t + \sigma e))d\Phi(e) \nonumber \\
    &\qquad + \lambda^e\max\set{V^u, V^e} + V^e_t + V^e_a\pth{ra + y - \pth{\frac{\alpha_e}{V^e_a}}^{1/\gamma}}, \label{eq:value_e}\\
    (\rho + h_t + \lambda^u)V^u &= \frac{1}{1-\gamma}\pth{\frac{1}{V^u_a}}^{1/\gamma - 1} + h_tb(a) + \lambda^u\int \max\set{V^e(t,a,y), V^u(t,a)}dF(y) \nonumber \\
    &\qquad + V^u_t + V^u_a\pth{ra + y_f - \pth{\frac{1}{V^u_a}}^{1/\gamma}}. \label{eq:value_u}
\end{align}
The terminal conditions are given by the bequest. 
\begin{equation}
    \lim_{t\to T} V^e(t,a,y) = b(a), \quad 
    \lim_{t\to T} V^u(t,a) = b(a) \text{ for all } a,y.
\end{equation}
Also, the asset constraint leads to that 
\begin{equation}
    V^e_a(t,0,y) \geq (u^e)'(y) = \alpha_e y^{-\gamma}, \quad
    V^u_a(t,0) \geq (u^u)'(y_f) = y_f^{-\gamma} \text{ for all } t,y. 
\end{equation}

In practice, the terminal condition at infinity is imposed by setting a sufficiently 
large $T$. We use the following terminal condition at $T$: 
\begin{equation}
    (\rho + h_T)V^e(T, a, y) = \frac{\alpha_e}{1-\gamma}(c^e)^{1-\gamma} + h_Tb(a), \quad 
    (\rho + h_T)V^u(T, a) = \frac{1}{1-\gamma}(c^u)^{1-\gamma} + h_Tb(a).
\end{equation}

\section{Solution Method}
We are going to solve the model using PINN (Physics-Informed Neural Network). 
Since the model is a time-inhomogeneous PDE, we will assume that for sufficiently 
large $T$, the value function is closed enough to the bequest utility $b(a)$. Also, 
to avoid the local minimum, we first train the neural network on the domain closed 
to the terminal time and then gradually expand the domain to the whole time interval. 

The PINN loss function is composed of three parts, 
\begin{align}
    \begin{split}
        \L_{DE} = \frac{1}{N_{DE}}\sum_{i=1}^{N_{DE}} E_1(x_i) + E_2(x_i), 
    \end{split}\\
    \begin{split}
        \L_{a} &= \frac{1}{N_a}\sum_{i=1}^{N_a} \abs{V^e_a(x^{a}_i) - \alpha_e (c^e(x^{a}_i))^{-\gamma}} + \abs{V^u_a(x^{a}_i) - (c^u(x^{a}_i))^{-\gamma}} \\
        &\qquad + \max\set{0, c^e(x^{a}_i) - y_i}^2 + \max\set{0, c^u(x^{a}_i) - y_f}^2, \\
    \end{split}\\
    \begin{split}
        \L_{t} &= \frac{1}{N_t}\sum_{i=1}^{N_t} \abs{V^e(x^t_i) - b(a_i)}^2 + \abs{V^u(x^t_i) - b(a_i)}^2 \\
        &\qquad + \abs{V^e_a(x^t_i) - \alpha_e (c^e(x^t_i))^{-\gamma}} + \abs{V^u_a(x^{t}_i) - (c^u(x^{t}_i))^{-\gamma}} \\
        &\qquad + \abs{b'(a_i) - \alpha_e (c^e(x^{a}_i))^{-\gamma}}^2 + \abs{b'(a_i) - (c^u(x^{a}_i))^{-\gamma}}^2,  
    \end{split}
\end{align}
with $E_1$ and $E_2$ defined as the residuals of the PDEs \cref{eq:value_e,eq:value_u}. To avoid the issue 
that the approximator may have negative derivatives so that the residuals are not properly 
defined, we use the consumption combined with the first order conditions to compute the 
residuals and add loss terms to ensure that the first order conditions are satisfied.
\footnote{Which loss to use is somewhat debatable. In general the $L^2$ loss is preferrable, 
while it is possible to use the $L^1$ loss shrinking some parts in the loss function, for 
example, the first order condition parts.} 

\subsection{Hard-Constraint Boundary}
Sometimes the neural network may struggle to learn the boundary conditions. In such
case, we use the hard-constraint method. By taking some transformation of our neural
network, we may enforce the boundary conditions exactly. In our case, we take the 
following: 
\begin{align}
    V^e(t,a,y) &= f^e(t,a,y) - af^e_a(t,0,y) + a(\alpha\exp(-\gamma y) + \exp(f^e_a(t,0,y))), \\
    V^u(t,a) &= f^u(t,a) - af^u_a(t,0) + a(\alpha\exp(-\gamma y_f) + \exp(f^u_a(t,0))).
\end{align}
One can easily verify that the asset boundary conditions are satisfied, and 
\begin{align}
    V^e_t &= f^e_t - af^e_{ta}(t,0,y) + a\exp(f^e_a(t,0,y))f^e_{ta}(t,0,y), \\
    V^u_t &= f^u_t - af^u_{ta}(t,0) + a\exp(f^u_a(t,0))f^u_{ta}(t,0).
\end{align}
To deal with the second order derivatives, we use the FO-PINN method proposed by 
\cite{gladstone2023}. 

\subsection{Quasi-Finite Difference Physics-Informed Neural Network}
In this subsection, we introduce a new approach for solving the PDE with terminal 
conditions, the Quasi-Finite Difference Physics-Informed Neural Network (QFD-PINN). 
The key idea is to combine the finite difference method with the neural network to 
conquer the curse of dimensionality. 

Given a small, prespecified length of time, $dt$, we rewrite our PDE into the following 
semi-discretized form. To avoid the issue of negative derivatives, we replace $V^e_a$ 
and $V^u_a$ with the consumption policy and the first order conditions. The 
consumption policies are transformed to ensure that they are always positive. 
\begin{align}
    V^e(t+dt) &= V^e + dt\sbrc{(\rho + h_t + \lambda_y + \lambda^e)V^e - \frac{\alpha_e}{1-\gamma}(c^e)^{1 - \gamma} - h_tb(a) - \lambda_y\int V^e(t,a,y\exp(\mu_t + \sigma e))d\Phi(e)} \nonumber \\
    &\qquad - dt\sbrc{\lambda^e\max\set{V^u, V^e} + \alpha_e(c^e)^{-\gamma}\pth{ra + y - c^e}}, \label{eq:implicit_qfd_e}\\
    V^u(t+dt) &= V^u + dt\sbrc{(\rho + h_t + \lambda^u)V^u - \frac{1}{1-\gamma}(c^u)^{1 - \gamma} - h_tb(a) - \lambda^u\int \max\set{V^e(t,a,y), V^u(t,a)}dF(y)} \nonumber \\
    &\qquad - dt\sbrc{(c^u)^{-\gamma}\pth{ra + y_f - c^u}}, \label{eq:implicit_qfd_u} \\
    V_a^e &= \alpha_e (c^e)^{-\gamma}, \label{eq:implicit_foc_e} \\ 
    V_a^u &= (c^u)^{-\gamma}. \label{eq:implicit_foc_u} 
\end{align}
We use the auto-differentiation to compute the derivative in the state space. 

Alternatively, we can use the explicit scheme as well, which gives 
\begin{align}
    V^e(t) &= V^e - dt\sbrc{(\rho + h_t + \lambda_y + \lambda^e)V^e - \frac{\alpha_e}{1-\gamma}(c^e)^{1 - \gamma} - h_tb(a) - \lambda_y\int V^e(t+dt,a,y\exp(\mu_t + \sigma e))d\Phi(e)} \nonumber \\
    &\qquad + dt\sbrc{\lambda^e\max\set{V^u, V^e} + \alpha_e(c^e)^{-\gamma}\pth{ra + y - c^e}}, \label{eq:explicit_qfd_e}\\
    V^u(t) &= V^u - dt\sbrc{(\rho + h_t + \lambda^u)V^u - \frac{1}{1-\gamma}(c^u)^{1 - \gamma} - h_tb(a) - \lambda^u\int \max\set{V^e(t+dt,a,y), V^u(t+dt,a)}dF(y)} \nonumber \\
    &\qquad + dt\sbrc{(c^u)^{-\gamma}\pth{ra + y_f - c^u}}, \label{eq:explicit_qfd_u} \\
    V_a^e(t) &= \alpha_e (c^e(t))^{-\gamma}, \label{eq:explicit_foc_e} \\ 
    V_a^u(t) &= (c^u(t))^{-\gamma}, \label{eq:explicit_foc_u} 
\end{align}
where the right hand sides are all evaluated at time $t+dt$.

\bibliographystyle{apacite}
\bibliography{conti_time}

\end{document}