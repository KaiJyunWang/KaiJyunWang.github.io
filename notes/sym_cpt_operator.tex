\begin{definition}
    $\D(A)\subset\H$ is dense in $\H$. A linear operator $A:\D(A)\to\H$ is 
    said to be \textbf{symmetric} if $\inp{Ax}{y} = \inp{x}{Ay}$ for all 
    $x,y\in\D(A)$.
\end{definition}
\begin{remark}
    Note that the domain $\D(A)$ is dense in $\H$. It follows that by density, 
    the domain can often be extended to $\H$. For simplicity, we consider 
    the domain to be $\H$, but the domain can be any dense subset of $\H$.
\end{remark}

\begin{definition}
    $\lambda\in\mathbb{F}$ is an \textbf{eigenvalue} of a linear operator 
    $A:\H\to\H$ if there exists a non-zero vector $x\in\H$ such that 
    $Ax = \lambda x$. The vector $x$ is called the \textbf{eigenvector} 
    corresponding to the eigenvalue $\lambda$.
\end{definition} 

\begin{proposition}\label{prop:sym_eval}
    Let $A:\H\to\H$ be a symmetric operator. The followings are true. 
    \begin{thmenum}
        \item $\inp{Ax}{x}\in\R$ for all $x\in\H$.
        \item If $\lambda\in\mathbb{F}$ is an eigenvalue of $A$, then $\lambda\in\R$. 
        \item If $\lambda_1,\lambda_2\in\mathbb{F}$ are two distinct eigenvalues 
        with respect to eigenvectors $x_1,x_2\in\H$, then $\inp{x_1}{x_2} = 0$. 
        \item Suppose $\set{x_\alpha}$ is an orthonormal basis of $\H$ with 
        the property that each $x_\alpha$ is an eigenvector of $A$ corresponding 
        to the eigenvalue $\lambda_\alpha$. Then if $\mu\in\mathbb{F}$ is also 
        an eigenvalue of $A$, then $\mu = \lambda_\alpha$ for some $\alpha$.
    \end{thmenum}
\end{proposition}
\begin{proof}
    For (a), $\inp{Ax}{x} = \inp{x}{Ax} = \overline{\inp{Ax}{x}}$. Then 
    $\Im\pth{\inp{Ax}{x}} = 0$ and $\inp{Ax}{x}\in\R$.

    For (b), let $x\in\H$ be the corresponding eigenvector to $\lambda$. Then 
    \begin{equation*}
        \inp{Ax}{x} = \inp{\lambda x}{x} = \lambda\inp{x}{x} = \lambda\norm{x}^2
        \implies \lambda = \frac{\inp{Ax}{x}}{\norm{x}^2}\in\R.
    \end{equation*}

    For (c), by symmetry and (b), we have 
    \begin{equation*}
        \lambda_1\inp{x_1}{x_2} = \inp{Ax_1}{x_2} = \inp{x_1}{Ax_2} 
        = \overline{\lambda_2}\inp{x_1}{x_2} = \lambda_2\inp{x_1}{x_2}.
    \end{equation*}
    Since $\lambda_1\neq\lambda_2$, $\inp{x_1}{x_2} = 0$. 

    For (d), let $\mu\in\mathbb{F}$ be an eigenvalue of $A$ with eigenvector 
    $y\in\H$, $y\neq 0$. We claim that $\mu = \lambda_\alpha$ for some $\alpha$. 
    Suppose not. Then write $y = \sum_{j} c_jx_{\alpha_j}$, where $c_j\in\mathbb{F}$. 
    We see that
    \begin{equation*}
        \norm{y}^2 = \lim_{M\to\infty}\inp{y}{\sum_{j=1}^{M} c_jx_{\alpha_j}} 
        = \lim_{M\to\infty} \overline{c_j}\inp{y}{x_{\alpha_j}} = 0
    \end{equation*}
    by (c), but this is a contradiction since $y\neq 0$. Thus $\mu = \lambda_\alpha$ 
    for some $\alpha$.
\end{proof}

\begin{definition}
    A linear operator $A:\H\to\H$ is called \textbf{bounded} if 
    \begin{equation*}
        \norm{A} = \sup_{\norm{x}=1}\norm{Ax} < \infty.
    \end{equation*}
\end{definition}

\begin{proposition}
    $A:\H\to\H$ is a symmetric bounded linear operator. Then 
    \begin{equation*}
        \norm{A} = \sup_{\norm{x} = 1} \abs{\inp{Ax}{x}}.
    \end{equation*}
\end{proposition}
\begin{proof}
    Assume $\norm{x} = 1$. By Cauchy-Schwarz inequality, $\abs{\inp{Ax}{x}} 
    \leq \norm{Ax}\norm{x} = \norm{Ax}$. Taking supremum, 
    \begin{equation*}
        \sup_{\norm{x} = 1}\abs{\inp{Ax}{x}} \leq \sup_{\norm{x} = 1}\norm{Ax} 
        = \norm{A}.
    \end{equation*}
    To see the reverse inequality, note that $\norm{Ax}^2 = \inp{Ax}{Ax} 
    = \inp{A^2x}{x}$. For any nonzero $\lambda\in\R$, define $x^+ 
    = \lambda x + \frac{1}{\lambda}Ax$ and $x^- = \lambda x - \frac{1}{\lambda}Ax$. 
    Then $x = \frac{1}{2\lambda}(x^++x^-)$ and $Ax = \frac{\lambda}{2}(x^+-x^-)$. 
    Now, 
    \begin{equation*}
        \begin{split}
            \inp{A^2x}{x} &= \inp{A\pth{\frac{\lambda}{2}(x^+-x^-)}}{\frac{1}{2\lambda}(x^++x^-)} \\
            &= \frac{1}{4}\inp{Ax^+-Ax^-}{x^++x^-} \\ 
            &= \frac{1}{4}\pth{\inp{Ax^+}{x^+} + \inp{Ax^+}{x^-} - \inp{Ax^-}{x^+} - \inp{Ax^-}{x^-}} 
        \end{split}
    \end{equation*}
    Notice that $\inp{A^2x}{x}$, $\inp{Ax^+}{x^+}$ and $\inp{Ax^-}{x^-}$ are 
    real numbers by \cref{prop:sym_eval}; hence 
    $\Im\pth{\inp{Ax^+}{x^-} - \inp{Ax^-}{x^+}} = 0$. Also, $\inp{Ax^-}{x^+} 
    = \overline{\inp{x^+}{Ax^-}} = \overline{\inp{Ax^+}{x^-}}$. We have 
    $\Im\pth{\inp{Ax^+}{x^-}} = 0$ and $\inp{Ax^+}{x^-} - \inp{Ax^-}{x^+} = 0$. 
    Thus, letting $C = \sup_{\norm{x} = 1}\abs{\inp{Ax}{x}}$, 
    \begin{equation*}
        \begin{split}
            \inp{A^2x}{x} &= \frac{1}{4}\pth{\inp{Ax^+}{x^+} + \inp{Ax^-}{x^-}} \\
            &\leq \frac{1}{4}C\pth{\norm{x^+}^2 + \norm{x^-}^2} \\
            &= \frac{1}{4}C\pth{\inp{x^+}{x^+} + \inp{x^-}{x^-}} \\
            &= \frac{1}{4}C\pth{2\lambda^2\norm{x^2} + \frac{2}{\lambda^2}\norm{Ax}^2}
            = \frac{1}{2}C\pth{\lambda^2\norm{x}^2 + \frac{1}{\lambda^2}\norm{Ax}^2}.
        \end{split}
    \end{equation*}
    Notice that for $a,b\in\R$, $(a-b)^2\geq 0$ and thus $a^2 + b^2\geq 2ab$. Hence 
    \begin{equation*}
        \lambda^2\norm{x}^2 + \frac{1}{\lambda^2}\norm{Ax}^2 \geq 2\lambda\norm{x}\frac{1}{\lambda}\norm{Ax} 
        = 2\norm{Ax}\norm{x}.
    \end{equation*}
    We see that 
    \begin{equation*}
        \norm{Ax}^2 = \inp{A^2x}{x} \leq \frac{1}{2}C\inf_{\lambda\neq 0} \lambda^2\norm{x}^2 + \frac{1}{\lambda^2}\norm{Ax}^2 
        \leq C\norm{Ax}\norm{x}.
    \end{equation*}
    Clearly if $Ax=0$ the inequality holds. Suppose $\norm{Ax} \neq 0$. Then 
    deviding both sides by $\norm{Ax}$ and taking supremum gives 
    \begin{equation*}
        \norm{A} = \sup_{\norm{x} = 1}\norm{Ax} \leq C\sup_{\norm{x} = 1}\norm{x} 
        = C = \sup_{\norm{x} = 1}\abs{\inp{Ax}{x}}.
    \end{equation*}
    We conclude that $\norm{A} = \sup_{\norm{x} = 1} \abs{\inp{Ax}{x}}$.
\end{proof}

\begin{definition}
    $X$ and $Y$ are normed spaces. $M\subset X$. $A:M\to Y$ is an operator. We 
    say that $A$ is \textbf{compact} if $A$ is continuous and for every bounded 
    sequence $x_n\in M$, the sequence $Ax_n\in Y$ has a convergent subsequence.
\end{definition}
\begin{remark}
    A compact operator $A$ transfers bounded sets in $X$ to relatively compact 
    sets in $Y$.
\end{remark}

\begin{example}
    Consider the integral operator $A:C([0,1])\to C([0,1])$ equipped with the 
    supremum norms. Define 
    \begin{equation*}
        Au(x) = \int_0^1 K(x,y)f(y)dy,
    \end{equation*}
    where $K\in C([0,1]^2)$. We verify that $A$ is well-defined, i.e., 
    $Au\in C([0,1])$. Let $x_n\to x\in [0,1]$. 
    \begin{equation*}
        \begin{split}
            \abs{Au(x_n) - Au(x)} &= \abs{\int_0^1 K(x_n,y)u(y)dy - \int_0^1 K(x,y)u(y)dy} \\
            &\leq \int_0^1 \abs{K(x_n,y) - K(x,y)}\abs{u(y)}dy 
            \leq \norm{K(x_n,\cdot) - K(x,\cdot)}_\infty\norm{u}_\infty.
        \end{split}
    \end{equation*}
    Since $K$ is continuous, $(x_n,y)\to (x,y)$ implies $K(x_n,y)\to K(x,y)$. 
    It follows that $Au(x_n)\to Au(x)$. Hence $Au\in C([0,1])$.

    We claim that $A$ is compact. Let $\set{u_n}$ be 
    a bounded sequence in $C([0,1])$. By Arzel\`a-Ascoli theorem, it suffices 
    to show that $\set{Au_n}$ is bounded and uniformly equicontinuous. To see 
    the boundedness, note that by assumption we have $\norm{u_n}_\infty\leq M$ 
    for all $n$. Also, since $K$ is continuous on a compact set, we have 
    $K(x,y)\leq C$ for all $x,y\in [0,1]$. Then 
    \begin{equation*}
        \begin{split}
            \norm{Au_n}_\infty &= \sup_{x\in [0,1]}\abs{\int_0^1 K(x,y)u_n(y)dy} \\
            &\leq \sup_{x\in [0,1]}\int_0^1 \abs{K(x,y)}\abs{u_n(y)}dy \\
            &\leq \sup_{x\in [0,1]}\int_0^1 C\norm{u_n}_\infty dy = C\norm{u_n}_\infty \leq CM.
        \end{split}
    \end{equation*} 
    Thus $\set{Au_n}$ is bounded. To see the uniform equicontinuity, let 
    $\epsilon>0$ be given. By continuity of $K$, we can find $\delta>0$ such that 
    whenever $\abs{x-z}<\delta$, $\abs{K(x,y) - K(z,y)}<\epsilon/M$ for all $y\in [0,1]$. 
    Then for each $n\in\N$, 
    \begin{equation*}
        \begin{split}
            \abs{Au_n(x) - Au_n(z)} &= \abs{\int_0^1 K(x,y)u_n(y)dy - \int_0^1 K(z,y)u_n(y)dy} \\
            &\leq \int_0^1 \abs{K(x,y) - K(z,y)}\abs{u_n(y)}dy 
            \leq \frac{\epsilon}{M}\norm{u_n}_\infty \leq \epsilon.
        \end{split}
    \end{equation*}
    Thus $\set{Au_n}$ is uniformly equicontinuous. By Arzel\`a-Ascoli theorem, 
    $\set{Au_n}$ has a convergent subsequence, i.e., $A$ is compact.

\end{example}