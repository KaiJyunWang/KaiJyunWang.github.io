\begin{definition}
    $\D(A)\subset\H$ is dense in $\H$. A linear operator $A:\D(A)\to\H$ is 
    said to be \textbf{symmetric} if $\inp{Ax}{y} = \inp{x}{Ay}$ for all 
    $x,y\in\D(A)$.
\end{definition}
\begin{remark}
    Note that the domain $\D(A)$ is dense in $\H$. It follows that by density, 
    the domain can often be extended to $\H$. For simplicity, we consider 
    the domain to be $\H$, but the domain can be any dense subset of $\H$.
\end{remark}

\begin{definition}
    $\lambda\in\mathbb{F}$ is an \textbf{eigenvalue} of a linear operator 
    $A:\H\to\H$ if there exists a non-zero vector $x\in\H$ such that 
    $Ax = \lambda x$. The vector $x$ is called the \textbf{eigenvector} 
    corresponding to the eigenvalue $\lambda$.
\end{definition} 

\begin{proposition}
    Let $A:\H\to\H$ be a symmetric operator. The followings are true. 
    \begin{thmenum}
        \item $\inp{Ax}{x}\in\R$ for all $x\in\H$.
        \item If $\lambda\in\mathbb{F}$ is an eigenvalue of $A$, then $\lambda\in\R$. 
        \item If $\lambda_1,\lambda_2\in\mathbb{F}$ are two distinct eigenvalues 
        with respect to eigenvectors $x_1,x_2\in\H$, then $\inp{x_1}{x_2} = 0$. 
        \item Suppose $\set{x_\alpha}$ is an orthonormal basis of $\H$ with 
        the property that each $x_\alpha$ is an eigenvector of $A$ corresponding 
        to the eigenvalue $\lambda_\alpha$. Then if $\mu\in\mathbb{F}$ is also 
        an eigenvalue of $A$, then $\mu = \lambda_\alpha$ for some $\alpha$.
    \end{thmenum}
\end{proposition}
\begin{proof}
    For (a), $\inp{Ax}{x} = \inp{x}{Ax} = \overline{\inp{Ax}{x}}$. Then 
    $\Im\pth{\inp{Ax}{x}} = 0$ and $\inp{Ax}{x}\in\R$.

    For (b), let $x\in\H$ be the corresponding eigenvector to $\lambda$. Then 
    \begin{equation*}
        \inp{Ax}{x} = \inp{\lambda x}{x} = \lambda\inp{x}{x} = \lambda\norm{x}^2
        \implies \lambda = \frac{\inp{Ax}{x}}{\norm{x}^2}\in\R.
    \end{equation*}

    For (c), by symmetry and (b), we have 
    \begin{equation*}
        \lambda_1\inp{x_1}{x_2} = \inp{Ax_1}{x_2} = \inp{x_1}{Ax_2} 
        = \overline{\lambda_2}\inp{x_1}{x_2} = \lambda_2\inp{x_1}{x_2}.
    \end{equation*}
    Since $\lambda_1\neq\lambda_2$, $\inp{x_1}{x_2} = 0$. 

    For (d), let $\mu\in\mathbb{F}$ be an eigenvalue of $A$ with eigenvector 
    $y\in\H$, $y\neq 0$. We claim that $\mu = \lambda_\alpha$ for some $\alpha$. 
    Suppose not. Then write $y = \sum_{j} c_jx_{\alpha_j}$, where $c_j\in\mathbb{F}$. 
    We see that
    \begin{equation*}
        \norm{y}^2 = \lim_{M\to\infty}\inp{y}{\sum_{j=1}^{M} c_jx_{\alpha_j}} 
        = \lim_{M\to\infty} \overline{c_j}\inp{y}{x_{\alpha_j}} = 0
    \end{equation*}
    by (c), but this is a contradiction since $y\neq 0$. Thus $\mu = \lambda_\alpha$ 
    for some $\alpha$.
\end{proof}

\begin{definition}
    A linear operator $A:\H\to\H$ is called \textbf{bounded} if 
    \begin{equation*}
        \norm{A} = \sup_{\norm{x}=1}\norm{Ax} < \infty.
    \end{equation*}
\end{definition}