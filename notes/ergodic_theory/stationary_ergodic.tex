\begin{definition}
    Let $(S,\S,\mu)$ be a measure space. A measurable mapping
    $T:S\to S$ \textbf{preserves} $\mu$ or \textbf{$\mu$-preserving} 
    if $\mu(T^{-1}(A)) = \mu(A)$ for all $A\in\S$. 
\end{definition}

\begin{remark}
    Let $\xi$ be a random variable on $S$. Then $T$ preserves $\mu$ 
    if and only if $\xi\circ T\deq\xi$.
\end{remark} 

\begin{definition}
    Consider the left shift mapping $\theta$ defined on $S^\infty$ by 
    $\theta(x_1,x_2,\ldots) = \theta(x_2,x_3,\ldots)$. A random variable 
    $\xi$ on $S^\infty$ is called \textbf{stationary} if $\theta\xi\deq\xi$.
\end{definition}

\begin{lemma}[Stationary and Invariance]\label{lem:stationary_invariance}
    For any random variable $\xi:S\to E$ and a measurable mapping 
    $T:S\to S$, 
    \begin{thmenum}
        \item $\xi T \deq \xi$ if and only if the sequence $\set{\xi T^n}$ is 
        stationary. 
        \item If $\xi T \deq \xi$, then for every measurable function 
        $f$ defined on $E$, the sequence defined by $\set{f(\xi T^n)}$ is stationary.
        \item Any stationary random sequence $\eta\in S^\infty$ can be represented as 
        $\eta_n = f(\eta T^n)$.
    \end{thmenum}
\end{lemma}
\begin{proof}
    Suppose first that $\xi T\deq\xi$. Then 
    \begin{equation*}
        \theta(\set{f(\xi T^n)}) = \set{f\xi T^{n+1}} = \set{f\xi T^n\circ T} \deq \set{f(\xi T^n)},
    \end{equation*}
    where $\theta$ is the left shift. This proves (b) and the suffiency 
    of (a) by taking $f$ to be the identity function. Conversely, if 
    $\eta = \set{\eta_n}$ is a stationary sequence, then $\set{\eta_n} \deq 
    \theta(\set{\eta_n}) = \set{\eta_{n+1}}$. Let $\eta_n = \pi_0(\theta^n\eta)$, 
    where $\pi_0$ is the projection onto the first coordinate. This 
    proves (c). And, taking $\eta = \set{\xi T^n}$, we have 
    \begin{equation*}
        \xi = \pi_0(\set{\xi T^n}) \deq \pi_0(\theta(\set{\xi T^n})) = \pi_0(\set{\xi T^{n+1}}) = \xi T.
    \end{equation*}
    This proves the necessity of (a).
\end{proof}

\begin{definition}
    We extend the definition of stationarity to a sequence indexed by 
    $\Z$ by requiring that $\theta\xi\deq\xi$, where $\theta$ is the 
    left shift on $S^\Z$.
\end{definition}
\begin{remark}
    The advantage of such definition is that the left shift on $S^\Z$ 
    is now invertible and hence forms a group instead of a semigroup. 
    Observe that \cref{lem:stationary_invariance} does not depend on 
    the choice of our index set and hence still applies.
\end{remark} 

\begin{definition}
    Let $(S,\S)$ be a measurable space and $\mu$ be a measure on $\S$. 
    $\mu$ is said to be \textbf{complete} if $A\subset B$ and $\mu(B) = 0$ 
    implies $A\in\S$.
\end{definition}
\begin{remark}
    The Lebesgue measure on $\R$ is complete, but the Borel measure is not. 
    In fact, the completion of the Borel measure is the Lebesgue measure.
\end{remark}

\begin{definition}
    Let $(S,\S,\mu)$ be a measure space and $T:S\to S$ be a measurable 
    mapping. A set $I\subset S$ is said to be \textbf{invariant} under $T$ 
    if $T^{-1}(I) = I$ and \textbf{almost invariant} if 
    $\mu(T^{-1}(I)\triangle I) = 0$.
\end{definition}

\begin{definition}
    Let $(S,\S,\mu)$ be a measure space, $T:S\to S$ be a measurable 
    mapping, and $\S^\mu$ be the completion of $\S$ with respect to $\mu$. 
    The \textbf{invariant $\sigma$-algebra} of $T$ is the collection 
    $\I\subset\S$ of all sets that are invariant under $T$. Similarly, 
    the \textbf{almost invariant $\sigma$-algebra} of $T$ is the collection 
    $\I'\subset\S^\mu$ of all sets that are almost invariant under $T$.
\end{definition}

\begin{definition}
    A measurable function $f:S\to E$ is said to be \textbf{invariant} if  
    $f\circ T = f$ and \textbf{almost invariant} if $f\circ T = f$ $\mu$-a.e.
\end{definition}