\begin{proposition}\label{prop:bd_functional_on_h}
    $\mathbb{F} = \R$ or $\C$. $T:\H\to\mathbb{F}$ is a nonzero bounded 
    linear functional. 
    \begin{thmenum}
        \item $\H = \spanby(\set{w})\oplus\ker(T)$ for $w\notin \ker(T)$.
        \item If $S,T$ are bounded linear functionals and $\ker(S) = \ker(T)$, 
        then there exists $c\in\mathbb{F}$ such that $S = cT$. 
        \item $\ker(T)$ is closed.
    \end{thmenum}
\end{proposition}
\begin{proof}
    For (a), since $T$ is nonzero, there is some $w$ such that $Tw\neq 0$. 
    For $x\in\H$, set $\alpha = Tx/Tw$ and $u = x - \alpha w$. Then 
    $x = \alpha w + u$ and 
    \begin{equation*}
        Tu = Tx - \frac{Tx}{Tw}Tw = 0.
    \end{equation*}
    Hence $u\in \ker(T)$. Also, if $v\in \spanby(\set{w})\bigcap\ker(T)$, 
    $v = cw$ and $Tv = 0$. Then $cTw = Tv = 0$; $c = 0$ and thus $v = 0$. 
    Therefore, $\spanby(\set{w})\bigcap\ker(T) = \set{0}$ and 
    $\H = \spanby(\set{w})\oplus\ker(T)$. 

    To see (b), note that if $S = 0$, $\H = \ker(S) = \ker(T)$. 
    Thus $T = 0$. If $S\neq 0$, by (a) we can write 
    $\H = \spanby(\set{w})\oplus\ker(S) = \spanby(\set{w})\oplus\ker(T)$. 
    Then for every $x\in\H$, $x = \alpha w + u$ for some $\alpha\in\mathbb{F}$ 
    and $u\in\ker(T)=\ker(S)$. Then $Tw\neq 0$ and
    \begin{equation*}
        Sx = S(\alpha w + u) = \alpha Sw = \alpha Tw\frac{Sw}{Tw} = \frac{Sw}{Tw}T(\alpha w + u) = \frac{Sw}{Tw}Tx.
    \end{equation*}
    Taking $c = Sw/Tw$ gives $S = cT$. 

    For (c), let $x_n\in\ker(T)$ be a sequence such that $x_n\to x\in\H$. 
    Since $T$ is continuous, 
    \begin{equation*}
        Tx = \lim_{n\to\infty} Tx_n = 0. 
    \end{equation*}
    Hence $x\in\ker(T)$ and $\ker(T)$ is closed.
\end{proof}

\begin{theorem}[Riesz Representation on $\H$]
    $\mathbb{F} = \R$ or $\C$. $T:\H\to\mathbb{F}$ is a bounded linear 
    functional. Then there exists $x^*\in\H$ such that $Ty = \inp{y}{x^*}$ 
    for all $y\in\H$.
\end{theorem}
\begin{proof}
    If $T = 0$, pick $x^* = 0$ then $Ty = 0 = \inp{y}{0}$. If $T\neq 0$, 
    there is some $w\in\H$ such that $Tw\neq 0$. By 
    \cref{prop:bd_functional_on_h}, we can write $\H = \spanby(\set{w})\oplus\ker(T)$ 
    with $\ker(T)$ closed. Also, $\H = \ker(T)\oplus\ker(T)^{\perp}$ by 
    \cref{prop:orthogonal_complement}. We claim that $\ker(T)^{\perp} = \spanby(\set{w})$. 
    First note that $\ker(T)^{\perp}\neq \set{0}$ or we would have 
    $\H = \ker(T)$ and $T = 0$, contradicting to our assumption. Now if 
    $z_1,z_2\in\ker(T)^\perp$, write $z_1 = \alpha_1 w + u_1$ and 
    $z_2 = \alpha_2 w + u_2$ for some $\alpha_1,\alpha_2\in\mathbb{F}$ and 
    $u_1,u_2\in\ker(T)$. Then $\alpha_2z_1 - \alpha_1z_2 
    = \alpha_2u_1 - \alpha_1u_2\in\ker(T)$ and $\alpha_2z_1 - \alpha_1z_2
    \in\ker(T)^\perp$. Hence $\alpha_2z_1 - \alpha_1z_2 = 0$ and $z_1,z_2$ 
    are linearly dependent. Now define $S:\H\to\mathbb{F}$ by $Sx = \inp{x}{w}$. 
    Then $S$ is a bounded linear functional and $\ker(S) = \Set{x\in\H}{\inp{x}{w} = 0} 
    = (\ker(T)^\perp)^\perp = \ker(T)$ by \cref{prop:orthogonal_complement}. 
    Applying (b) of \cref{prop:bd_functional_on_h} gives $cS = T$ for some 
    $c\in\mathbb{F}$. Then $Tx = cSx = c\inp{x}{w} = \inp{x}{\overline{c}w}$. 
    Set $x^* = \overline{c}w$ proves the theorem.
\end{proof}

\begin{definition}
    Let $X,Y$ be vector spaces. $T:X\to Y$ is called \textbf{skew-linear} if 
    $T(cx + y) = \overline{c}Tx + Ty$ for all $x,y\in X$ and $c\in\mathbb{F}$.
\end{definition}

\begin{definition}
    $\mathbb{F} = \R$ or $\C$. $B:\H\times\H\to\mathbb{F}$ is called a 
    \textbf{bilinear form} if 
    \begin{thmenum}
        \item $B(\cdot,x)$ is linear for all $x\in\H$. 
        \item $B(x,\cdot)$ is skew-linear for all $x\in\H$.
    \end{thmenum}
\end{definition}

\begin{definition}
    A bilinear form $B:\H\times\H\to\mathbb{F}$ is called \textbf{bounded} if 
    there exists $M\geq 0$ such that $\abs{B(x,y)}\leq M\norm{x}\norm{y}$ for 
    all $x,y\in\H$.
\end{definition}

\begin{definition}
    A bilinear form $B:\H\times\H\to\mathbb{F}$ is called \textbf{coercive} if 
    there exists $c>0$ such that $B(x,x)\geq c\norm{x}^2$ for all $x\in\H$.
\end{definition}