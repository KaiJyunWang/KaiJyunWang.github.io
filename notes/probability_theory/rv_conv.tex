\begin{definition}
    A sequence of probability measures $\P_n$ on $(\R^n, \B(\R^n))$ 
    are  \textbf{consistent} if 
    \begin{equation*}
        \P_{n+1}((a_1, b_1]\times\cdots\times(a_n, b_n]\times\R 
        = \P_n((a_1, b_1]\times\cdots\times(a_n, b_n])
    \end{equation*}
    for every $n$. 
\end{definition}

\begin{theorem}[Kolmogorov Extension]
    Suppose that a sequence of probability measures $\P_n$ on $(\R^n, \B(\R^n))$ 
    are consistent. Then there is a unique probability measure $\P$ on $\R^\N$ 
    satisfying that 
    \begin{equation*}
        \P(\Set{\omega\in\R^\N}{\omega_i\in(a_i, b_i], 1\leq i\leq n}) = \P_n((a_1, b_1]\times\cdots\times(a_n, b_n]). 
    \end{equation*}
\end{theorem}
\begin{proof}
    Let 
    \begin{equation*}
        \S = \Set{(a_1, b_1]\times\cdots\times(a_n, b_n]\times\R\times\cdots}{n\in\N}. 
    \end{equation*}
    Define $\P$ on $\S$ to be 
    \begin{equation*}
        \P((a_1, b_1]\times\cdots\times(a_n, b_n]\times\R\times\cdots) = \P_n((a_1, b_1]\times\cdots\times(a_n, b_n])
    \end{equation*}
    Clearly, $\S$ forms a semi-algebra. From the Carath\'eodory 
    extension theorem, it suffices to show that $\P$ is finitely 
    additive, $\sigma$-additive on $\S$ and $\P(\varnothing) = 0$. 
    Note that $\P(\varnothing) = \P(\varnothing\times\R\times\cdots) 
    = \P_1(\varnothing) = 0$. We verify the first two conditions. 

    First, if $A,B\in\S$ are disjoint, $m\leq n$,   
    \begin{equation*}
        A = \Set{\omega\in\R^\N}{\omega_i\in (a_i, b_i], 1\leq i\leq m}\quad\text{and}\quad 
        B = \Set{\omega\in\R^\N}{\omega_i\in (c_i, d_i], 1\leq i\leq n}, 
    \end{equation*}
    then 
    \begin{equation*}
        \P(A\cup B) = \P_n((\pi_n A)\cup(\pi_n B)) = \P_n(\pi_n A) + \P_n(\pi_n B) 
        = \P(A) + \P(B), 
    \end{equation*}
    where $\pi_n:\omega\to(\omega_1,\ldots,\omega_n)$ is the projection onto 
    the first $n$ components. Hence $\P$ is finitely additive. 

    Next, suppose $A_1,\ldots\in \R^\N$ are countably many disjoint measurable sets. 
    Put $A = \cup_i A_i$. We can consider the algebra $\bar{\S} 
    = \set{\text{finite disjoint union of sets in $\S$}}$ generated by $\S$. 
    $B_n = \cup_{i=1}^n A_i\in\bar{\S}$. Thus 
    \begin{equation*}
        \P(A) = \P(B_n) + \sum_{i=1}^n\P(A_n) 
    \end{equation*} 
    by the previous result. It now suffices to show that 
    $\P(B_n)\to 0$ for any $B_n\searrow \varnothing$. 
\end{proof}