\begin{definition}
    Let $\Omega$ be a set. A collection of subsets $\F$ forms 
    a \textbf{$\sigma$-algebra} if 
    \begin{thmenum}
        \item $\varnothing\in\F$. 
        \item $A\in\F$ implies $A^c\in\F$. 
        \item If $A_i\in\F$ are countably many sets, $\cup_iA_i\in\F$. 
    \end{thmenum}
    The dual $(\Omega, \F)$ is called a \textbf{measurable space} and 
    the sets falling in $\F$ are said to be \textbf{measurable}. 
\end{definition}

\begin{definition}
    Let $(\Omega, \F)$ be a measurable space. A set function $\mu:\F\to[0,\infty]$ 
    is a \textbf{measure} if 
    \begin{thmenum}
        \item $\mu(\varnothing) = 0$. 
        \item For countably many disjoint $A_i\in\F$, $\mu(\cup_i A_i) = \sum_i \mu(A_i)$. 
    \end{thmenum}
    The triple $(\Omega, \F, \mu)$ is called a \textbf{measure space}.
\end{definition}

\begin{definition}
    A \textbf{probability space} is a measure space $(\Omega, \F, \P)$ 
    such that $\P(\Omega) = 1$. 
\end{definition}

\begin{lemma}
    Let $\S$ be a collection of sets. Then there exists the smallest $\sigma$-algebra 
    containing $\S$. 
\end{lemma}
\begin{proof}
    Let $\F$ be the intersection of all $\sigma$-algebra containing $\S$. $\F$ 
    is non-empty since the power set is a $\sigma$-algebra containing $\S$. Now 
    it is clear that $\varnothing\in\F$ since $\varnothing\in\A$ for every $\sigma$-algebra 
    $\A$ contiaining $\S$. If $A\in\F$, $A\in\A$ for all $\A$ containing $\S$ and 
    $A^c\in\A$ for all $\A$. Thus $A^c\in\F$. Finally, if $A_i\in\F$ are 
    countably many sets, then each $A_i$ lies in every $\A$ containing $\S$; 
    so does $\cup_i A_i$ and thus $\cup_i A_i\in\F$. The minimality follows by 
    the construction of $\F$. 
\end{proof}

\begin{definition}
    For any collection of sets $\S$, the smallest $\sigma$-algebra is denoted 
    as $\sigma(\S)$. 
\end{definition}

\begin{theorem}\label{thm:monotone_conv_measure}
    Let $(\Omega, \F, \P)$ be a probability space. Then 
    \begin{thmenum}
        \item If $A, B\in\F$ and $A\subset B$, then $\P(A)\leq\P(B)$. 
        \item For countably many $A_i\in\F$, $\P(\cup_i A_i)\leq \sum_i \P(A_i)$. 
        \item If $A_i\nearrow A$, $\P(A_i)\to \P(A)$. 
        \item If $A_i\searrow A$, $\P(A_i)\to \P(A)$. 
    \end{thmenum}
\end{theorem}
\begin{proof}
    (a) and (b) are clear. For (c), write $E_i = A_i - A_{i-1}$ and $A_0 = \varnothing$. 
    Then since $E_i$ are disjoint and $A_n = \cup_{i=1}^n E_i$, 
    \begin{equation*}
        \P(A_n) = \P\pth{\cup_{i=1}^n E_i} = \sum_{i=1}^n \P(E_i) 
        \to \sum_i \P(E_i) = \P(\cup_i E_i) = \P(A)
    \end{equation*}  
    as $n\to\infty$. 
    
    For (d), note that $A_i^c\nearrow A^c$. Thus $1 - \P(A_i) = \P(A_i^c)\to\P(A^c) = 1 - \P(A)$. 
    Thus $\P(A_i)\to\P(A)$.  
\end{proof}

\begin{definition}
    The \textbf{Borel $\sigma$-algebra} is the $\sigma$-algebra generated by all 
    open sets. 
\end{definition}

\begin{definition}
    Let $\P$ be a probability measure on $(\R, \B)$. The \textbf{distribution function} 
    $F$ is defined as 
    \begin{equation*}
        F(x) = \P((-\infty, x])
    \end{equation*}
    for $x\in\R$. 
\end{definition}

\begin{proposition}
    The distribution function in $(\R, \B)$ satisfies that 
    \begin{thmenum}
        \item $F(x)\leq F(y)$ for all $x\leq y$. 
        \item $F(x)\to F(y)$ as $x\to y^+$. 
        \item $F(-\infty) = 0$ and $F(\infty) = 1$. 
    \end{thmenum}
\end{proposition}
\begin{proof}
    For (a), note that $(-\infty, x]\subset(-\infty, y]$ and 
    \begin{equation*}
        F(x) = \P((-\infty, x]) \leq \P((-\infty, y]) = F(y). 
    \end{equation*}

    For (b), notice that for $x_n\to y^+$, $(-\infty, x_n]\searrow(-\infty, y]$. 
    Hence 
    \begin{equation*}
        F(x_n) = \P((-\infty, x_n]) \to \P((-\infty, y]) = F(y). 
    \end{equation*}
    Similarly, taking $x_n\to\pm\infty$ gives (c). 
\end{proof}

\begin{definition}
    A collection $\S$ of sets is called an \textbf{algebra} if 
    \begin{thmenum}
        \item $\varnothing\in\S$. 
        \item If $A\in\S$, then $A^c\in\S$. 
        \item If $A, B\in\S$, then $A\cup B\in\S$. 
    \end{thmenum}
\end{definition}
\begin{remark}
    An algebra is closed under finite unions. It is also 
    clear that a $\sigma$-algebra is an algebra, while the 
    converse is not true. An example is the collection of 
    all finite unions of intervals in $\R$. 
\end{remark}

\begin{definition}
    A collection $\S$ of sets is called a \textbf{semi-algebra} if 
    \begin{thmenum} 
        \item If $A, B\in\S$, then $A\cap B\in\S$.
        \item If $A\in\S$, then $A^c$ can be written as a finite disjoint union of sets in $\S$.
    \end{thmenum}
\end{definition}
\begin{remark}
    A semi-algebra must contain $\varnothing$ since for any 
    $A\in\S$, $A^c = \cup_i A_i$, where $A_i\in\S$ are disjoint. 
    Then $A\cap A_1 = \varnothing\in\S$.
\end{remark}
\begin{remark}
    An example of being a semi-algebra but not an algebra is the 
    collection of all intervals of the form $(a_i, b_i]$ for $-\infty\leq a_i<b_i\leq\infty$ 
    with the empty set.
\end{remark}

\begin{lemma}\label{lem:semi-algebra}
    If $\S$ is a semi-algebra, then $\overline{S} = \set{\text{finite disjoint unions of sets in } \S}$ 
    forms an algebra.
\end{lemma}
\begin{proof}
    It has been shown that $\varnothing\in\S$. For $A,B\in\overline{\S}$, 
    write $A = \cup_{i=1}^n A_i$ and $B = \cup_{j=1}^m B_j$ for disjoint $A_i, B_j\in\S$, 
    respectively. Then $A\cap B = \cup_{i,j} (A_i\cap B_j)\in\overline{\S}$. 
    Thus $\overline{\S}$ is closed under intersection. Now if $A\in\overline{\S}$, 
    $A = \cup_{i=1}^n A_i$ for disjoint $A_i\in\S$. Then 
    $A^c = \cap_{i=1}^n A_i^c$. By the definition of semi-algebra, 
    $A_i^c$ can be written as a finite disjoint union of sets in $\S$ 
    and thus $A_i^c\in\overline{\S}$. Since $\overline{\S}$ is closed 
    under finite intersection, $A^c = \cap_{i=1}^n A_i^c\in\overline{\S}$. 
    Finally, for $A, B\in\overline{\S}$, $A\cup B = (A^c\cap B^c)^c\in\overline{\S}$.
    We conclude that $\overline{\S}$ is indeed an algebra.
\end{proof}

\begin{definition}
    Suppose $\S$ is a semi-algebra. $\overline{\S} = 
    \set{\text{finite disjoint unions of sets in } \S}$ is called the 
    \textbf{algebra generated by $\S$}.
\end{definition}

\begin{definition}
    Let $\S$ be an algebra. A set function $\mu_0:\S\to[0,\infty]$ 
    is called a \textbf{premeasure} if
    \begin{thmenum}
        \item $\mu_0(\varnothing) = 0$. 
        \item For countable disjoint $A_i\in\S$ such that $\cup_i A_i\in\S$, 
        \begin{equation*}
            \mu_0\pth{\cup_i A_i} = \sum_i \mu_0(A_i).
        \end{equation*}
    \end{thmenum}
\end{definition}

\begin{theorem}\label{thm:premeasure}
    Let $\nu$ be a set function on a semi-algebra $\S$ such that 
    $\nu(\varnothing) = 0$. Suppose that 
    \begin{thmenum}
        \item if $A\in\S$ and $A = \cup_{i=1}^n A_i$ for disjoint $A_i\in\S$, 
        then $\nu(A) = \sum_{i=1}^n \nu(A_i)$;
        \item if $A_i\in\S$ are countably many sets and $A = \cup_i A_i\in\S$, 
        then $\nu(A) \leq \sum_i \nu(A_i)$.
    \end{thmenum}
    Then $\nu$ can be extended to a unique premeasure $\mu_0$ on the 
    algebra generated by $\S$.
\end{theorem}
\begin{proof}
    We first show the existence. From \cref{lem:semi-algebra} we 
    know that $\S$ generates an algebra $\A = 
    \set{\text{finite disjoint union of sets in }\S}$. Define 
    our candidate $\mu_0$ by $\mu_0(A) = \sum_i\nu(A_i)$ 
    for $A = \cup_i A_i$ where $A_i\in\S$ are disjoint. To see that 
    $\mu_0$ is well-defined, suppose $A = \cup_i B_i$ 
    for disjoint $B_i\in\S$. Observe that 
    \begin{equation*}
        A_i = \cup_j (A_i\cap B_j) \quad\text{and}\quad
        B_j = \cup_i (A_i\cap B_j)
    \end{equation*}
    are finite disjoint unions. Then 
    \begin{equation*}
        \sum_i \nu(A_i) = \sum_i\sum_j\nu(A_i\cap B_j) 
        = \sum_j\sum_i\nu(A_i\cap B_j) = \sum_j \nu(B_j) 
    \end{equation*}
    by (a). Thus $\mu_0$ is well-defined. 
    
    Now we check that $\mu_0$ is a premeasure. Clearly 
    $\mu_0(\varnothing) = 0$. For finitely many disjoint 
    $A_i\in\A$ such that $\cup_i A_i\in\A$, we can write 
    $A_i = \cup_j B_{ij}$ for disjoint $B_{ij}\in\S$. Then 
    (a) implies that 
    \begin{equation*}
        \mu_0(\cup_i A_i) = \mu_0(\cup_{i,j} B_{ij})
        = \sum_{i,j} \nu(B_{ij}) = \sum_i\sum_j \mu_0(B_{ij})
        = \sum_i \mu_0(A_i).
    \end{equation*}

    Next, for countably many disjoint $A_i\in\A$ such that 
    $A = \cup_i A_i\in\A$, write $A_i = \cup_j B_{ij}$, where 
    $B_{ij}\in\S$ are finite disjoint for each $i$. Then 
    $\mu_0(A_i) = \sum_j \nu(B_{ij})$ and
    \begin{equation*}
        \sum_i \mu_0(A_i) = \sum_i\sum_j \nu(B_{ij}).
    \end{equation*}
    Without loss of generality, we may choose $A_i$ to be those in $\S$
    since otherwise we can replace $A_i$ by $B_{ij}$. We assume that $A_i\in\S$ 
    from now on. Since $A\in\A$, $A = \cup_i C_i$ for finite disjoint 
    $C_i\in\S$. $C_i = \cup_j (C_i\cap A_j)$. Thus (b) gives that 
    \begin{equation*}
        \nu(C_i) \leq \sum_j \nu(C_i\cap A_j).
    \end{equation*}
    Then 
    \begin{equation*}
        \mu_0(A) = \sum_i \nu(C_i) \leq \sum_i\sum_j \nu(C_i\cap A_j) 
        = \sum_j\sum_i \nu(C_i\cap A_j) = \sum_j \nu(A_j) = \sum_j \mu_0(A_j).
    \end{equation*}
    For the opposite inequality, set $B_n = \cup_{i=1}^n A_i$ and $C_n = A - B_n$. 
    Since $\A$ is an algebra, $C_n\in\A$ and the finite additivity shows that 
    \begin{equation*}
        \mu_0(A) = \sum_{i=1}^n\mu_0(A_i) + \mu_0(C_n) \geq \sum_{i=1}^n \mu_0(A_i).
    \end{equation*}
    Taking $n\to\infty$ gives the desired inequality and thus $\mu_0$ is $\sigma$-additive 
    on $\A$. 

    Finally, if $\mu_1$ is another premeasure on $\A$ extending $\nu$, then for
    $A = \cup_i A_i$ for disjoint $A_i\in\S$,
    \begin{equation*}
        \mu_1(A) = \sum_i \nu(A_i) = \mu_0(A).
    \end{equation*}
\end{proof}

\begin{definition}
    A collection of sets $\mathcal{P}$ is called a \textbf{$\pi$-system} if
    $A, B\in\mathcal{P}$ implies $A\cap B\in\mathcal{P}$.
\end{definition}

\begin{definition}
    A collection of sets $\L$ is called a \textbf{$\lambda$-system} if 
    \begin{thmenum}
        \item $\Omega\in\L$. 
        \item If $A, B\in\L$ and $A\subset B$, then $B - A\in\L$. 
        \item If $A_i\in\L$ and $A_i\nearrow A$, then $A\in\L$.
    \end{thmenum}
\end{definition}

\begin{theorem}[Sierpi\'nski-Dynkin $\pi$-$\lambda$]
    If $\mathcal{P}$ is a $\pi$-system and $\L$ is a $\lambda$-system
    containing $\mathcal{P}$, then $\sigma(\mathcal{P})\subset\L$.
\end{theorem}
\begin{proof}
    First we show that a collection $\S$ is a $\sigma$-algebra if and only if 
    it is both a $\pi$-system and a $\lambda$-system. Suppose first that $\S$ 
    is a $\pi$-system and a $\lambda$-system. $\varnothing = \Omega - \Omega\in\S$. 
    If $A\in\S$, then $A^c = \Omega - A\in\S$. For $A, B\in\S$, $A\cup B = (A^c\cap B^c)^c\in\S$
    since we have shown that $\S$ is closed under complement and intersection by 
    being a $\pi$-system. Thus $\S$ is also closed under finite unions. If 
    $A_i\in\S$ are countably many sets, let $B_n = \cup_{i=1}^n A_i\in\S$. Then 
    $B_n\nearrow \cup_i A_i$ and thus $\cup_i A_i\in\S$. 

    Conversely, if $\S$ is a $\sigma$-algebra, then for $A, B\in\S$, 
    $A\cap B = (A^c\cup B^c)^c\in\S$. Thus $\S$ is a $\pi$-system. 
    If $A, B\in\S$ and $A\subset B$, then $B - A = B\cap A^c\in\S$. 
    Finally, if $A_i\in\S$ and $A_i\nearrow A$, then $A = \cup_i (A_i - A_{i-1})\in\S$ 
    with $A_0 = \varnothing$. Thus $\S$ is a $\lambda$-system. 

    Now set $\L$ to be the smallest $\lambda$-system containing $\mathcal{P}$. 
    It suffices to show that $\L$ is also a $\pi$-system and thus by the above 
    conclusion, $\L$ is a $\sigma$-algebra containing $\mathcal{P}$; hence 
    $\sigma(\mathcal{P})\subset\L$. 

    To show that $\L$ is a $\pi$-system, let $A,B\in\L$. If $A, B\in\mathcal{P}$, 
    $A\cap B\in\mathcal{P}\subset\L$. To extend the result for general $A, B\in\L$, 
    we first fix $B\in\mathcal{P}$ and define 
    \begin{equation*}
        \L_B = \Set{A}{A\cap B\in\L}.
    \end{equation*}
    We claim that $\L_B$ is a $\lambda$-system containing $\mathcal{P}$. For 
    $A\in\mathcal{P}$, $A\cap B\in\L$. Thus $\mathcal{P}\subset\L_B$.
    Clearly $\Omega\in\L_B$. If $E,F\in\L_B$ and $E\subset F$, then
    \begin{equation*}
        (F-E)\cap B = (F\cap B) - (E\cap B)\in\L.  
    \end{equation*}
    Thus $F-E\in\L_B$. Finally, if $E_i\in\L_B$ and $E_i\nearrow E$, then
    \begin{equation*}
        E\cap B = \cup_i (E_i\cap B)\in\L.
    \end{equation*}
    Hence $E\in\L_B$ and we conclude that $\L_B$ is a $\lambda$-system. 
    Since it is a $\lambda$-system containing $\mathcal{P}$, it also contains 
    the smallest $\lambda$-system $\L$ with the intersection property. Thus 
    $A\cap B\in\L$ whenever $A\in\L$ and $B\in\mathcal{P}$. 

    Next, fix $A\in\L$ and define $\L_A = \Set{B}{A\cap B\in\L}$. Clearly 
    $\L_A$ contains $\L$ and $\Omega\in\L_A$. If $E, F\in\L_A$ and $E\subset F$, then
    \begin{equation*}
        (F-E)\cap A = (F\cap A) - (E\cap A)\in\L. 
    \end{equation*}
    Thus $F-E\in\L_A$. Finally, if $E_i\in\L_A$ and $E_i\nearrow E$, then 
    \begin{equation*}
        E\cap A = \cup_i (E_i\cap A)\in\L. 
    \end{equation*}
    Hence $E\in\L_A$ and we conclude that $\L_A$ is a $\lambda$-system. 
    Since it contains $\mathcal{L}$, $A,B\in\L$ implies $A\cap B\in\L$; 
    in other words, $\L$ is a $\pi$-system and the proof is complete.
\end{proof}

\begin{corollary}\label{cor:measure_agree}
    Let $\mu$ and $\nu$ be two probability measures agreeing on a $\pi$-system 
    $\mathcal{P}$, i.e., $\mu(A) = \nu(A)$ for all $A\in\mathcal{P}$. 
    Then $\mu(A) = \nu(A)$ for all $A\in\sigma(\mathcal{P})$. 
\end{corollary}
\begin{proof}
    Put 
    \begin{equation*}
        \L = \Set{A}{\mu(A) = \nu(A)}.
    \end{equation*}
    We claim that $\L$ is a $\lambda$-system containing $\mathcal{P}$. 
    It is clear that by our assumption, $\mathcal{P}\subset\L$ and 
    $\Omega\in\L$. If $A,B\in\L$ and $A\subset B$, then 
    \begin{equation*}
        \mu(B-A) = \mu(B) - \mu(A) = \nu(B) - \nu(A) = \nu(B-A).
    \end{equation*}
    Thus $B-A\in\L$. Finally, if $A_i\in\L$ and $A_i\nearrow A$, then 
    \begin{equation*}
        \mu(A) = \lim_{i\to\infty} \mu(A_i) = \lim_{i\to\infty} \nu(A_i) = \nu(A).
    \end{equation*}
    Hence $A\in\L$ and we conclude that $\L$ is a $\lambda$-system. 
    By the Sierpi\'nski-Dynkin $\pi$-$\lambda$ theorem, 
    $\sigma(\mathcal{P})\subset\L$; in other words, $\mu$ and $\nu$ agree 
    on $\sigma(\mathcal{P})$.
\end{proof}

\begin{definition}
    A measure $\mu$ on a measurable space $(\Omega, \F)$ is 
    called \textbf{$\sigma$-finite} if there exists countable 
    $A_i\in\F$ such that $\cup_i A_i = \Omega$ and $\mu(A_i)<\infty$.
\end{definition}

\begin{definition}
    A set function $\mu^*: 2^\Omega\to[0,\infty]$ is called an \textbf{outer measure} if 
    \begin{thmenum}
        \item $\mu^*(\varnothing) = 0$. 
        \item If $A\subset B$, then $\mu^*(A)\leq \mu^*(B)$. 
        \item For countably many $A_i\subset\Omega$, $\mu^*(\cup_i A_i) \leq \sum_i \mu^*(A_i)$.
    \end{thmenum}
\end{definition}

\begin{definition}
    Let $\mu^*$ be an outer measure. A set $A\subset\Omega$ is said to be 
    \textbf{Carath\'eodory measurable} or \textbf{$\mu^*$-measurable} if for 
    all $E\subset\Omega$, 
    \begin{equation*}
        \mu^*(E) = \mu^*(E\cap A) + \mu^*(E\cap A^c).
    \end{equation*}
\end{definition}

\begin{lemma}\label{lem:caratheodory}
    Let $\mu^*$ be an outer measure on $\Omega$. Then the collection of all
    $\mu^*$-measurable sets forms a $\sigma$-algebra $\F$ and $\mu^*|_\F$ is 
    a measure.
\end{lemma}
\begin{proof}
    Put 
    \begin{equation*}
        \F = \Set{A\subset\Omega}{\mu^*(E) = \mu^*(E\cap A) + \mu^*(E\cap A^c) \text{ for all } E\subset\Omega}.
    \end{equation*}
    We first show that $\F$ is a $\sigma$-algebra. Clearly $\varnothing\in\F$ and
    if $A\in\F$, then $A^c\in\F$. For $A, B\in\F$, let $C = A\cup B$. The property of
    outer measure gives that $\mu^*(E)\leq \mu^*(E\cap C) + \mu^*(E\cap C^c)$.
    To see the opposite inequality, note that $C = A\cup (B\cap A^c)$ and
    \begin{equation*}
        \begin{split}
            \mu^*(E\cap C) + \mu^*(E\cap C^c) 
            &\leq \mu^*(E\cap A) + \mu^*(E\cap B\cap A^c) + \mu^*(E\cap A^c\cap B^c) \\
            &= \mu^*(E\cap A) + \mu^*(E\cap A^c) = \mu^*(E).
        \end{split}
    \end{equation*}
    Hence $C\in\F$ and $\F$ is closed under finite unions. For countable disjoint 
    $A_i\in\F$ with $A = \cup_i A_i$, let $B_n = \cup_{i=1}^n A_i\in\F$. Then 
    \begin{equation*}
        \mu^*(E\cap A) = \mu^*(E\cap B_n) + \mu^*(E\cap B_n^c)
        \geq \mu^*(E\cap B_n) = \sum_{i=1}^n \mu^*(E\cap A_i).
    \end{equation*} 
    Taking $n\to\infty$ gives that 
    \begin{equation*}
        \mu^*(E\cap A) \geq \sum_i \mu^*(E\cap A_i) 
        \geq \mu^*(E\cap A)
    \end{equation*}
    by the $\sigma$-subadditivity of outer measure. Hence 
    $\mu^*(E\cap A) = \sum_i \mu^*(E\cap A_i)$. Note also 
    that $E\cap A^c\subset E\cap B_n^c$ so 
    $\mu^*(E\cap A^c)\leq \mu^*(E\cap B_n^c)$. 
    Thus 
    \begin{equation*}
        \mu^*(E) = \mu^*(E\cap B_n) + \mu^*(E\cap B_n^c) 
        \geq \sum_{i=1}^n \mu^*(E\cap A_i) + \mu^*(E\cap A^c) 
        \to \mu^*(E\cap A) + \mu^*(E\cap A^c)
        \geq \mu^*(E)
    \end{equation*}
    by the $\sigma$-subadditivity of outer measure. We conclude that 
    $\F$ is a $\sigma$-algebra. 

    Finally, denote $\mu^*|_\F$ by $\mu$. Clearly $\mu(\varnothing) = 0$. 
    For countably many disjoint $A_i\in\F$ such that $A = \cup_i A_i\in\F$, 
    let $B_n = \cup_{i=1}^n A_i\in\F$. Then 
    \begin{equation*}
        \mu(A) = \mu(B_n) + \mu(A\cap B_n^c) 
        \geq \mu(B_n) = \sum_{i=1}^n \mu(A_i) 
        \to \sum_i \mu(A_i) \geq \mu(A).
    \end{equation*}
    Hence $\mu(A) = \sum_i \mu(A_i)$ and $\mu$ is a measure on $\F$. 
\end{proof}

\begin{theorem}[Carath\'eodory Extension]
    Let $\nu$ be a finitely additive, $\sigma$-subadditive 
    set function on a semi-algebra $\S$ such that $\nu(\varnothing) = 0$. 
    Then $\nu$ can be extended to a measure on $\sigma(\S)$. 
\end{theorem}
\begin{proof}
    By \cref{thm:premeasure}, $\nu$ can be extended to a premeasure $\mu_0$ 
    on the algebra $\A$ generated by $\S$. Define the outer measure by 
    \begin{equation*}
        \mu^*(A) = \inf\Set{\sum_i \mu_0(E_i)}{A\subset \cup_i E_i, E_i\in\A}
    \end{equation*}
    for all $A\subset\Omega$ with the convention that $\inf\varnothing = \infty$. 
    We check that $\mu^*$ is indeed an outer measure. Clearly $\mu^*(\varnothing) = 0$. 
    If $A\subset B$, then any cover of $B$ by sets in $\A$ is also a cover of 
    $A$ and hence $\mu^*(A)\leq \mu^*(B)$. For countably many $A_i\subset\Omega$, 
    we can find $\set{E_{ij}}_j$ covering $A_i$ such that 
    \begin{equation*}
        \sum_j \mu_0(E_{ij}) \leq \mu^*(A_i) + 2^{-i}\epsilon
    \end{equation*}
    for some $\epsilon>0$. Then $\cup_{i,j} E_{ij}$ covers $\cup_i A_i$ 
    and 
    \begin{equation*}
        \mu^*(\cup_i A_i)\leq \sum_i\sum_j \mu_0(E_{ij})
        \leq \sum_i \mu^*(A_i) + \epsilon.
    \end{equation*}
    Since $\epsilon$ is arbitrary, we conclude that 
    $\mu^*(\cup_i A_i)\leq \sum_i \mu^*(A_i)$ and $\mu^*$ is 
    indeed an outer measure. 

    It follows from \cref{lem:caratheodory} that the collection
    of all $\mu^*$-measurable sets forms a $\sigma$-algebra $\F$ and
    $\mu^*$ restricted on $\F$ is a measure. It is clear that $\A\subset\F$
    and $\sigma(\S)\subset\sigma(\A)\subset\F$ and $\mu = \mu^*|_{\sigma(\S)}$
    is also a measure. Finally, for $A, A_i\in\S$ where $A_i$ covers 
    $A$, 
    \begin{equation*}
        \mu(A) = \mu^*(A) \leq \nu(A) \leq \sum_i \nu(A\cap A_i) 
        \leq \sum_i \nu(A_i).
    \end{equation*}
    Taking the infimum over all such covers, we get $\nu(A) = \mu^*(A)$ 
    and $\mu$ is indeed an extension of $\nu$. 
\end{proof}
\begin{remark}
    If the measures are probability measures, then we have that the extension 
    is unique by \cref{cor:measure_agree}. 
\end{remark}


\begin{theorem}
    If $F$ is non-decreasing, right-continuous and satisfies that $F(-\infty) = 0$, 
    $F(\infty) = 1$, then there is a unique probability measure such that 
    \begin{equation*}
        \P((-\infty, x]) = F(x). 
    \end{equation*}
\end{theorem}
\begin{proof}
    Define 
    \begin{equation*}
        \S = \Set{(a, b]}{-\infty \leq a < b \leq \infty} \cup \set{\varnothing}.
    \end{equation*}
    It is clear that $\S$ is a semi-algebra. Define the set function $\P:\S\to[0,1]$ 
    by 
    \begin{equation*}
        \P((a,b]) = F(b) - F(a)
    \end{equation*}
    and $\P(\varnothing) = 0$. For disjoint, at most countable $(a_i, b_i]\in\S$, we define
    \begin{equation*}
        \P\pth{\cup_i (a_i, b_i]} = \sum_i \P((a_i, b_i]) 
        = \sum_i F(b_i) - F(a_i).
    \end{equation*}
    It is clear that $\P$ is finitely additive. If $(a,b] = \cup_i (a_i, b_i]$ for disjoint 
    $(a_i, b_i]\in\S$, we may assume without loss of generality that $a = a_1 < b_1 < b_2 < \cdots < b_n = b$ 
    and 
    \begin{equation*}
        \P((a,b]) = F(b) - F(a) = \sum_i F(b_i) - F(a_i) = \sum_i \P((a_i, b_i]).
    \end{equation*}
    Hence $\P$ is $\sigma$-additive. It now follows from the Carath\'eodory extension theorem that 
    $\P$ can be extended uniquely to a probability measure on $\sigma(\S) = \B$.
\end{proof}
\begin{remark}
    This theorem shows that the distribution function completely characterizes the 
    probability measure. In other words, the term ``distribution function'' can 
    refer to either the CDF or the probability measure.
\end{remark}
