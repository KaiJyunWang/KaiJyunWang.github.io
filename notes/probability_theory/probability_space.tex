\begin{definition}
    Let $\Omega$ be a set. A collection of subsets $\F$ forms 
    a \textbf{$\sigma$-algebra} if 
    \begin{thmenum}
        \item $\varnothing\in\F$. 
        \item $A\in\F$ implies $A^c\in\F$. 
        \item If $A_i\in\F$ are countably many sets, $\cup_iA_i\in\F$. 
    \end{thmenum}
    The dual $(\Omega, \F)$ is called a \textbf{measurable space} and 
    the sets falling in $\F$ are said to be \textbf{measurable}. 
\end{definition}

\begin{definition}
    Let $(\Omega, \F)$ be a measurable space. A set function $\mu:\F\to[0,\infty]$ 
    is a \textbf{measure} if 
    \begin{thmenum}
        \item $\mu(\varnothing) = 0$. 
        \item For countably many disjoint $A_i\in\F$, $\mu(\cup_i A_i) = \sum_i \mu(A_i)$. 
    \end{thmenum}
    The triple $(\Omega, \F, \mu)$ is called a \textbf{measure space}.
\end{definition}

\begin{definition}
    A \textbf{probability space} is a measure space $(\Omega, \F, \P)$ 
    such that $\P(\Omega) = 1$. 
\end{definition}

\begin{lemma}
    Let $\S$ be a collection of sets. Then there exists the smallest $\sigma$-algebra 
    containing $\S$. 
\end{lemma}
\begin{proof}
    Let $\F$ be the intersection of all $\sigma$-algebra containing $\S$. $\F$ 
    is non-empty since the power set is a $\sigma$-algebra containing $\S$. Now 
    it is clear that $\varnothing\in\F$ since $\varnothing\in\A$ for every $\sigma$-algebra 
    $\A$ contiaining $\S$. If $A\in\F$, $A\in\A$ for all $\A$ containing $\S$ and 
    $A^c\in\A$ for all $\A$. Thus $A^c\in\F$. Finally, if $A_i\in\F$ are 
    countably many sets, then each $A_i$ lies in every $\A$ containing $\S$; 
    so does $\cup_i A_i$ and thus $\cup_i A_i\in\F$. The minimality follows by 
    the construction of $\F$. 
\end{proof}

\begin{definition}
    For any collection of sets $\S$, the smallest $\sigma$-algebra is denoted 
    as $\sigma(\S)$. 
\end{definition}

\begin{theorem}
    Let $(\Omega, \F, \P)$ be a probability space. Then 
    \begin{thmenum}
        \item If $A, B\in\F$ and $A\subset B$, then $\P(A)\leq\P(B)$. 
        \item For countably many $A_i\in\F$, $\P(\cup_i A_i)\leq \sum_i \P(A_i)$. 
        \item If $A_i\nearrow A$, $\P(A_i)\to \P(A)$. 
        \item If $A_i\searrow A$, $\P(A_i)\to \P(A)$. 
    \end{thmenum}
\end{theorem}
\begin{proof}
    (a) and (b) are clear. For (c), write $E_i = A_i - A_{i-1}$ and $A_0 = \varnothing$. 
    Then since $E_i$ are disjoint and $A_n = \cup_{i=1}^n E_i$, 
    \begin{equation*}
        \P(A_n) = \P\pth{\cup_{i=1}^n E_i} = \sum_{i=1}^n \P(E_i) 
        \to \sum_i \P(E_i) = \P(\cup_i E_i) = \P(A)
    \end{equation*}  
    as $n\to\infty$. 
    
    For (d), note that $A_i^c\nearrow A^c$. Thus $1 - \P(A_i) = \P(A_i^c)\to\P(A^c) = 1 - \P(A)$. 
    Thus $\P(A_i)\to\P(A)$.  
\end{proof}

\begin{definition}
    The \textbf{Borel $\sigma$-algebra} is the $\sigma$-algebra generated by all 
    open sets. 
\end{definition}

\begin{definition}
    Let $\P$ be a probability measure on $(\R, \B)$. The \textbf{distribution function} 
    $F$ is defined as 
    \begin{equation*}
        F(x) = \P((-\infty, x])
    \end{equation*}
    for $x\in\R$. 
\end{definition}

\begin{proposition}
    The distribution function in $(\R, \B)$ satisfies that 
    \begin{thmenum}
        \item $F(x)\leq F(y)$ for all $x\leq y$. 
        \item $F(x)\to F(y)$ as $x\to y^+$. 
        \item $F(-\infty) = 0$ and $F(\infty) = 1$. 
    \end{thmenum}
\end{proposition}
\begin{proof}
    For (a), note that $(-\infty, x]\subset(-\infty, y]$ and 
    \begin{equation*}
        F(x) = \P((-\infty, x]) \leq \P((-\infty, y]) = F(y). 
    \end{equation*}

    For (b), notice that for $x_n\to y^+$, $(-\infty, x_n]\searrow(-\infty, y]$. 
    Hence 
    \begin{equation*}
        F(x_n) = \P((-\infty, x_n]) \to \P((-\infty, y]) = F(y). 
    \end{equation*}
    Similarly, taking $x_n\to\pm\infty$ gives (c). 
\end{proof}

\begin{definition}
    A collection $\S$ of sets is called an \textbf{algebra} if 
    \begin{thmenum}
        \item $\varnothing\in\S$. 
        \item If $A\in\S$, then $A^c\in\S$. 
        \item If $A, B\in\S$, then $A\cup B\in\S$. 
    \end{thmenum}
\end{definition}
\begin{remark}
    An algebra is closed under finite unions. It is also 
    clear that a $\sigma$-algebra is an algebra, while the 
    converse is not true. An example is the collection of 
    all finite unions of intervals in $\R$. 
\end{remark}

\begin{definition}
    A collection $\S$ of sets is called a \textbf{semi-algebra} if 
    \begin{thmenum} 
        \item If $A, B\in\S$, then $A\cap B\in\S$.
        \item If $A\in\S$, then $A^c$ can be written as a finite disjoint union of sets in $\S$.
    \end{thmenum}
\end{definition}
\begin{remark}
    A semi-algebra must contain $\varnothing$ since for any 
    $A\in\S$, $A^c = \cup_i A_i$, where $A_i\in\S$ are disjoint. 
    Then $A\cap A_1 = \varnothing\in\S$.
\end{remark}
\begin{remark}
    An example of being a semi-algebra but not an algebra is the 
    collection of all intervals of the form $(a_i, b_i]$ for $-\infty\leq a_i<b_i\leq\infty$ 
    with the empty set.
\end{remark}

\begin{lemma}\label{lem:semi-algebra}
    If $\S$ is a semi-algebra, then $\overline{S} = \set{\text{finite disjoint unions of sets in } \S}$ 
    forms an algebra.
\end{lemma}
\begin{proof}
    It has been shown that $\varnothing\in\S$. For $A,B\in\overline{\S}$, 
    write $A = \cup_{i=1}^n A_i$ and $B = \cup_{j=1}^m B_j$ for disjoint $A_i, B_j\in\S$, 
    respectively. Then $A\cap B = \cup_{i,j} (A_i\cap B_j)\in\overline{\S}$. 
    Thus $\overline{\S}$ is closed under intersection. Now if $A\in\overline{\S}$, 
    $A = \cup_{i=1}^n A_i$ for disjoint $A_i\in\S$. Then 
    $A^c = \cap_{i=1}^n A_i^c$. By the definition of semi-algebra, 
    $A_i^c$ can be written as a finite disjoint union of sets in $\S$ 
    and thus $A_i^c\in\overline{\S}$. Since $\overline{\S}$ is closed 
    under finite intersection, $A^c = \cap_{i=1}^n A_i^c\in\overline{\S}$. 
    Finally, for $A, B\in\overline{\S}$, $A\cup B = (A^c\cap B^c)^c\in\overline{\S}$.
    We conclude that $\overline{\S}$ is indeed an algebra.
\end{proof}

\begin{definition}
    Suppose $\S$ is a semi-algebra. $\overline{\S} = 
    \set{\text{finite disjoint unions of sets in } \S}$ is called the 
    \textbf{algebra generated by $\S$}.
\end{definition}

\begin{definition}
    Let $\S$ be an algebra. A set function $\mu_0:\S\to[0,\infty]$ 
    is called a \textbf{premeasure} if
    \begin{thmenum}
        \item $\mu_0(\varnothing) = 0$. 
        \item For countable disjoint $A_i\in\S$ such that $\cup_i A_i\in\S$, 
        \begin{equation*}
            \mu_0\pth{\cup_i A_i} = \sum_i \mu_0(A_i).
        \end{equation*}
    \end{thmenum}
\end{definition}

\begin{theorem}
    Let $\nu$ be a set function on a semi-algebra $\S$ such that 
    $\nu(\varnothing) = 0$. Suppose that 
    \begin{thmenum}
        \item if $A\in\S$ and $A = \cup_{i=1}^n A_i$ for disjoint $A_i\in\S$, 
        then $\nu(A) = \sum_{i=1}^n \nu(A_i)$;
        \item if $A_i\in\S$ are countably many sets and $A = \cup_i A_i\in\S$, 
        then $\nu(A) \leq \sum_i \nu(A_i)$.
    \end{thmenum}
    Then $\nu$ can be extended to a unique premeasure $\mu_0$ on the 
    algebra generated by $\S$.
\end{theorem}
\begin{proof}
    We first show the existence. From \cref{lem:semi-algebra} we 
    know that $\S$ generates an algebra $\A = 
    \set{\text{finite disjoint union of sets in }\S}$. Define 
    our candidate $\mu_0$ by $\mu_0(A) = \sum_i\nu(A_i)$ 
    for $A = \cup_i A_i$ where $A_i$ are disjoint. To see that 
    $\mu_0$ is well-defined, suppose $A = \cup_i B_i$ 
    for disjoint $B_i\in\S$. Observe that 
    \begin{equation*}
        A_i = \cup_j (A_i\cap B_j) \quad\text{and}\quad
        B_j = \cup_i (A_i\cap B_j)
    \end{equation*}
    are finite disjoint unions. Then 
    \begin{equation*}
        \sum_i \nu(A_i) = \sum_i\sum_j\nu(A_i\cap B_j) 
        = \sum_j\sum_i\nu(A_i\cap B_j) = \sum_j \nu(B_j) 
    \end{equation*}
    by (a). Thus $\mu_0$ is well-defined. 
    
    Now we check that $\mu_0$ is a premeasure. Clearly 
    $\mu_0(\varnothing) = 0$. For countably many disjoint 
    $A_i\in\A$, if $A = \cup_i A_i\in\A$, 
\end{proof}

\begin{theorem}
    If $F$ is non-decreasing, right-continuous and satisfies that $F(-\infty) = 0$, 
    $F(\infty) = 1$, then there is a probability measure such that 
    \begin{equation*}
        \P((-\infty, x]) = F(x). 
    \end{equation*}
\end{theorem}
