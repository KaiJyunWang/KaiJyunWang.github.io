\begin{definition}
    Let $X_i$ be random variables with $\E\sbrc{X_i^2}<\infty$. They are called 
    \textbf{uncorrelated} if 
    \begin{equation*}
        \E\sbrc{X_iX_j} = \E\sbrc{X_i}\E\sbrc{X_j}. 
    \end{equation*} 
\end{definition}

\begin{theorem}[Weak Law of Large Number \rom{1}]
    Suppose that $X_n$ are uncorrelated random variables with $\Var\sbrc{X_n}\leq C\infty$ 
    and $\E\sbrc{X_n} = \mu$ for all $n$. Let $S_n = \sum_{i=1}^n X_i$. Then 
    \begin{equation*}
        \frac{1}{n}S_n\to \mu
    \end{equation*}
    in $\L^2$ and hence in probability. 
\end{theorem}
\begin{proof}
    Compute that 
    \begin{equation*}
        \E\sbrc{\pth{\frac{1}{n}S_n-\mu}^2} = \frac{1}{n^2}\sum_{i=1}^n\Var(X_i) 
        \leq \frac{C}{n}\to 0. 
    \end{equation*}
    Hence $\frac{1}{n}S_n\to\mu$ in $\L^2$ and thus in probability. 
\end{proof}

\begin{theorem}[Weak Law of Large Number \rom{2}, Khinchin]
    Suppose that $X_i$ is a sequence of independent and identically distributed random variables 
    with $\E\sbrc{\abs{X_1}}<\infty$. Let $S_n = \sum_{i=1}^n X_i$ and $\mu = \E\sbrc{X_1}$. 
    Then 
    \begin{equation*}
        \frac{1}{n}S_n\to\mu
    \end{equation*}
    in $\L^1$ and hence in probability. 
\end{theorem}
\begin{proof}
    By replacing $X_i$ with $X_i-\mu$, we may assume without loss of generality 
    that $\mu = 0$. Now, for $C>0$, 
    \begin{equation*}
        0 = \E\sbrc{X_i} = \E\sbrc{X_i\one\set{\abs{X_i}>C}} + \E\sbrc{X_i\one\set{\abs{X_i}\leq C}}. 
    \end{equation*}
    Also, 
    \begin{equation*}
        \begin{split}
            \frac{1}{n}S_n &= 
            \frac{1}{n}\sum_{i=1}^nX_i\one\set{\abs{X_i}>C}  
            + \frac{1}{n}\sum_{i=1}^nX_i\one\set{\abs{X_i}\leq C} \\ 
            &= \frac{1}{n}\sum_{i=1}^n\pth{X_i\one\set{\abs{X_i}>C} - \E\sbrc{X_i\one\set{\abs{X_i}>C}}} 
            + \frac{1}{n}\sum_{i=1}^n\pth{X_i\one\set{\abs{X_i}\leq C} - \E\sbrc{X_i\one\set{\abs{X_i}\leq C}}}. 
        \end{split}
    \end{equation*}
    Notice that by LDCT,
    \begin{equation*}
        \E\sbrc{\abs{\frac{1}{n}\sum_{i=1}^n\pth{X_i\one\set{\abs{X_i}>C} - \E\sbrc{X_i\one\set{\abs{X_i}>C}}}}} 
        \leq 2\E\sbrc{\abs{X_1}\one\set{\abs{X_1}>C}}\to 0
    \end{equation*}
    as $C\to\infty$ since $\abs{X_1}\one\set{\abs{X_1}>C}\leq\abs{X_1}$ and 
    $\E\sbrc{\abs{X_1}}<\infty$. Also, by H\"older inequality and the independence,  
    \begin{equation*}
        \E\sbrc{\abs{\frac{1}{n}\sum_{i=1}^n\pth{X_i\one\set{\abs{X_i}\leq C} - \E\sbrc{X_i\one\set{\abs{X_i}\leq C}}}}} 
        \leq \sqrt{\frac{1}{n}\Var(X_i\one\set{\abs{X_i}\leq C})} \leq \frac{C}{\sqrt{n}} 
    \end{equation*}
    For any given $\epsilon>0$, there is $C$ such that $2\E\sbrc{\abs{X_1}\one\set{\abs{X_1}>C}} < \epsilon$ and 
    \begin{equation*}
        \begin{split}
            \E\sbrc{\abs{\frac{1}{n}S_n}} &\leq \E\sbrc{\abs{\frac{1}{n}\sum_{i=1}^n\pth{X_i\one\set{\abs{X_i}>C} - \E\sbrc{X_i\one\set{\abs{X_i}>C}}}}} \\
            &\quad + \E\sbrc{\abs{\frac{1}{n}\sum_{i=1}^n\pth{X_i\one\set{\abs{X_i}\leq C} - \E\sbrc{X_i\one\set{\abs{X_i}\leq C}}}}} \\ 
            &\leq \epsilon + \frac{C}{\sqrt{n}}\to\epsilon
        \end{split}
    \end{equation*}
    as $n\to\infty$. Since $\epsilon$ can be arbitrarily small, we conclude 
    that $\frac{1}{n}S_n\to 0$ in $\L^1$ and hence in probability. 
\end{proof}

