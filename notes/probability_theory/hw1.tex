\documentclass[a4paper, 12pt]{article}
% For fonts
% \usepackage{palatino}
% \usepackage[T1]{fontenc}
% \usepackage{mathpazo}
% \usepackage{textcomp}
% \usepackage{fouriernc}
% \usepackage{mathastext}
\usepackage[scaled = 0.9]{scholax}



\usepackage{setspace}
\usepackage{amsmath}%
\usepackage{amssymb}%
\usepackage{mathtools}
\usepackage{dsfont}
\usepackage[table]{xcolor}%

\setlength{\marginparwidth}{6cm}
\usepackage{todonotes}
\usepackage[in]{fullpage}%
%\usepackage{enumitem}

\usepackage{amsthm}

\usepackage{titlesec}%
\titlelabel{\thetitle.\hspace{0.5em}}%
\usepackage{xcolor}%
\usepackage{mleftright}
\usepackage{xspace}%
\usepackage{graphicx}
\usepackage{hyperref}%
\usepackage{etoolbox}
\usepackage{lipsum}
\usepackage{appendix}
\usepackage{tikz}
\usepackage{mathrsfs}
\usepackage{listings}
\usepackage{enumitem}
% \usepackage[normalem]{ulem}
\usepackage[nameinlink,noabbrev]{cleveref}
% Math font, must be loaded after amsmath
\usepackage[scaled = 0.95,varbb]{newtxmath}


\usepackage{hyperref}%
\hypersetup{%
unicode,
breaklinks,%
colorlinks=true,%
urlcolor=blue,%
linkcolor=[rgb]{0.5,0.0,0.0},%
citecolor=[rgb]{0,0.2,0.445},%
filecolor=[rgb]{0,0,0.4},
anchorcolor=[rgb]={0.0,0.1,0.2}%
}

\newenvironment{thmenum}{%
\begin{enumerate}[label=\small(\alph*\small),topsep=4pt,itemsep=4pt,partopsep=0pt, parsep=0pt]
}{%
\end{enumerate}
}

\newtheoremstyle{break}
  {\topsep}{\topsep}%
  {\itshape}{}%
  {\bfseries}{}%
  {\newline}{}%
\theoremstyle{break}
\newtheorem{theorem}{Theorem}[section]

\theoremstyle{break}
\newtheorem{lemma}[theorem]{Lemma}

\theoremstyle{break}
\newtheorem{corollary}[theorem]{Corollary}

\theoremstyle{break}
\newtheorem{proposition}[theorem]{Proposition}

\theoremstyle{break}
\newtheorem{definition}[theorem]{Definition}

\theoremstyle{break}
\newtheorem*{remark}{Remark}

\theoremstyle{break}
\newtheorem*{example}{Example}

\newtheorem{innercustomthm}{Exercise}
\newenvironment{exercise}[1]{\renewcommand\theinnercustomthm{#1}\innercustomthm}{\endinnercustomthm}

\newenvironment{solution}{\begin{proof}[Solution]}{\end{proof}}

\renewcommand\qedsymbol{\rule{2mm}{2mm}}
%
\newcommand{\Set}[2]{\left\{ #1 \;\middle\vert\; #2 \right\}}

\newcommand{\pth}[1]{\mleft(#1\mright)}%

\renewcommand{\P}{\operatorname{P}}
\newcommand{\E}{\operatorname{E}}
\newcommand{\Var}{\operatorname{Var}}
\newcommand{\Cov}{\operatorname{Cov}}
\newcommand{\diam}[1]{\operatorname{diam}\mleft(#1\mright)}
\newcommand{\osc}[2]{\operatorname{osc}\mleft(#1,\, #2\mright)}
\newcommand{\supp}[1]{\operatorname{supp}\mleft(#1\mright)}
\newcommand{\indicator}{\mathbf{1}}
\newcommand{\argmin}{\operatorname{argmin}}
\newcommand{\argmax}{\operatorname{argmax}}
\newcommand{\deq}{\stackrel{d}{=}}


\newcommand{\Ex}[1]{\E\mleft[ #1 \mright]}

\newcommand{\ceil}[1]{\mleft\lceil {#1} \mright\rceil}
\newcommand{\floor}[1]{\mleft\lfloor {#1} \mright\rfloor}
\newcommand{\sgn}{\operatorname{sgn}}
\newcommand{\card}{\operatorname{card}}

\newcommand{\brc}[1]{\left\{ {#1} \right\}}
\newcommand{\sbrc}[1]{\left[ {#1} \right]}
\newcommand{\set}[1]{\brc{#1}}%

\newcommand{\abs}[1]{\left\lvert {#1} \right\rvert}%
\newcommand{\norm}[1]{\left\lVert {#1} \right\rVert}
\newcommand{\spanby}{\operatorname{span}}
\newcommand{\esssup}{\operatorname{ess\,sup}}
\newcommand{\inp}[2]{\left\langle {#1},\, {#2} \right\rangle}

\newcommand{\mto}{\overset{m}{\to}}
\newcommand{\wto}{\overset{w}{\to}}
\newcommand{\wstarto}{\overset{w^*}{\to}}
\newcommand{\dsubset}{\overset{d}{\subset}}


\newcommand{\ds}{\displaystyle}%

\newcommand{\R}{\mathbb{R}}%
\newcommand{\N}{\mathbb{N}}
\newcommand{\Q}{\mathbb{Q}}
\newcommand{\Z}{\mathbb{Z}}
\newcommand{\C}{\mathbb{C}}
\newcommand{\F}{\mathcal{F}}
\newcommand{\A}{\mathcal{A}}
\newcommand{\B}{\mathcal{B}}
\newcommand{\T}{\mathcal{T}}
\newcommand{\M}{\mathcal{M}}
\renewcommand{\L}{\mathcal{L}}
\renewcommand{\S}{\mathcal{S}}
\newcommand{\I}{\mathcal{I}}
\renewcommand{\H}{\mathcal{H}}
\newcommand{\D}{\mathcal{D}}
\newcommand{\U}{\mathcal{U}}
\newcommand{\G}{\mathcal{G}}

\newcommand{\one}{\mathbf{1}}

\newcommand{\exist}{\exists\:}
\newcommand{\forany}{\forall\:}
\renewcommand{\ae}{\quad\text{a.e.}}
\newcommand{\Res}{\operatorname{Res}}
\newcommand{\cl}{\operatorname{cl}}

\newcommand{\restrict}[1]{\left.\hspace{-0.2em}\right|_{#1}}
\newcommand{\eval}[3]{\left.{#1}\right|_{#2}^{#3}}

% derivatives
\makeatletter
\renewcommand\d[1]{\mspace{6mu}\mathrm{d}#1\@ifnextchar\d{\mspace{-3mu}}{}}
\makeatother
\def\at{
  \left.
  \vphantom{\int}
  \right|
}

\newcommand{\od}[2]{\frac{d{#1}}{d{#2}}}
\newcommand{\odd}[2]{\frac{d^2{#1}}{d{#2}^2}}
\newcommand{\pd}[2]{\frac{\partial{#1}}{\partial{#2}}}
\newcommand{\pdd}[2]{\frac{\partial^2{#1}}{\partial{#2}^2}}
\newcommand{\pdpd}[3]{\frac{\partial^2{#1}}{\partial{#2}\partial{#3}}}

\makeatletter
\newcommand*{\rom}[1]{\expandafter\@slowromancap\romannumeral #1@}
\makeatother

\begin{document} 
\setstretch{1.15}

\begin{center}
    \Large
    \textbf{Probability Theory \rom{1} -- Homework 1}\\
    \large{Kai-Jyun Wang}
\end{center}

\begin{exercise}{1.1}
    Suppose $f$ is a measurable mapping from one measurable space 
    $S$ to another measurable space $U$. If $A$ is a measurable 
    subset of $S$, does it follow that the image $f(A)$ is a 
    measurable subset of $U$? 
\end{exercise}
\begin{solution}
    No. Consider the case where $S = U = \set{0,1}$. Define 
    the $\sigma$-algebra on $S$ to be $\S = 2^S$ and the one 
    on $U$ to be $\U = \set{\varnothing, U}$. $f$ is the identity 
    mapping, which is measurable. However, $f(\set{0}) = \set{0}\notin\U$.  
\end{solution}

\begin{exercise}{1.2}
    Let $\Omega$ be a sample space and let $I$ be an arbitrary 
    index set, which could be uncountable. 
    \begin{thmenum}
        \item For each $i\in I$, let $\F_i$ be a $\sigma$-algebra on 
        $\Omega$. Show that $\bigcap_{i\in I}\F_i$ is a 
        $\sigma$-algebra on $\Omega$. 
        \item Let $\A$ be a collection of subsets of $\Omega$. Show that 
        there is a smallest $\sigma$-algebra containing $\A$. 
    \end{thmenum}
\end{exercise}
\begin{solution}
    For (a), first note that $\varnothing\in\F_i$ for all 
    $i\in I$ and thus $\varnothing\in\cap_{i\in I}\F_i$. 
    Next, if $A\in\cap_{i\in I}\F_i$, then $A\in\F_i$ for 
    all $i\in I$. Since $\F_i$ are $\sigma$-algebras, $A^c\in\F_i$ for 
    all $i$; hence $A^c\in\cap_{i\in I}\F_i$. Finally, if 
    $\set{A_n}\subset\cap_{i\in I}\F_i$ is a countable 
    subcollection, then $\set{A_n}\subset\F_i$ for all $i$. 
    Since $\F_i$ are $\sigma$-algebras, $\cup_n A_n\in\F_i$ for 
    all $i$; hence $\cup_n A_n\in\cap_{i\in I}\F_i$. 
    Therefore, $\cap_{i\in I}\F_i$ is a $\sigma$-algebra.
    
    For (b), set $\F$ to be the intersection of all $\sigma$-algebras 
    containing $\A$. Since the power set of $\Omega$ is a $\sigma$-algebra 
    containing $\A$, $\F$ is non-empty. By (a), we know that $\F$ is 
    a $\sigma$-algebra. By definition, if $\G$ is a $\sigma$-algebra 
    containing $\A$, then $\F\subset\G$ since $\F$ is the intersection 
    of all such $\sigma$-algebras. Hence $\F$ is the smallest 
    $\sigma$-algebra containing $\A$.
\end{solution}

\begin{exercise}{1.3}
    Let $\Omega = \R$ and $\F$ be all subsets of $\R$ such that
    either $A$ or $A^c$ is countable. Define $P(A) = 0$ if $A$
    is countable and $P(A) = 1$ if uncountable. Show that 
    $(\Omega, \F, P)$ is a probability space.
\end{exercise}
\begin{solution}
    I first claim that $\F$ is a $\sigma$-algebra. Since $\varnothing$ 
    is countable, $\varnothing\in\F$. If $A\in\F$, then either $A$ 
    or $A^c$ is countable; either $A^c$ or $A = (A^c)^c$ is countable, respectively. 
    Thus $A^c\in\F$. Suppose that $\set{A_n}\subset\F$ is a countable 
    collection of sets. If all $A_n$ are countable, then $\cup_n A_n$ 
    is countable and thus in $\F$. If there is some $A_n$, say $A_1$, 
    is uncountable, then $A_1^c$ is countable. Now 
    $\pth{\cup_n A_n}^c = \cap_n A_n^c\subset A_1^c$ is countable, so 
    $\cup_n A_n\in\F$. Therefore, $\F$ forms a $\sigma$-algebra. 

    Next, I show that $P$ is a probability measure. Clearly every 
    set in $\F$ is either countable or uncountable, so $P$ is 
    well-defined and $P(A)\geq 0 = P(\varnothing)$ for all $A\in\F$ 
    since $\varnothing$ is countable. Also, $P(\Omega) = 1$ since 
    $\R$ is uncountable. Finally, suppose that $\set{A_n}\subset\F$ 
    is a countable collection of disjoint sets. If all $A_n$ are 
    countable, then $\cup_n A_n$ is countable and 
    \begin{equation*}
        P\pth{\cup_n A_n} = 0 = \sum_n 0 = \sum_n P(A_n).
    \end{equation*}
    If some $A_n$, say $A_1$, is uncountable, then $A_n$ are countable
    for all $n\neq 1$; otherwise, $\cup_{n\neq 1} A_n$ would be uncountable 
    and $A_1^c\supset\cup_{n\neq 1} A_n$ is uncountable, contradiction. Now 
    $\cup_n A_n$ is uncountable and
    \begin{equation*}
        P(\cup_n A_n) = 1 = 1 + \sum_{n\neq 1} 0 = P(A_1) + \sum_{n\neq 1} P(A_n) = \sum_n P(A_n).
    \end{equation*}
    Therefore, $P$ is a probability measure and $(\Omega, \F, P)$ 
    is a probability space.
\end{solution}

\begin{exercise}{1.4}
    Suppose $X$ and $Y$ are random variables on $(\Omega, \F, P)$ and let $A\in\F$. 
    Show that if we let $Z(\omega) = X(\omega)$ for $\omega\in A$ and 
    $Z(\omega) = Y(\omega)$ for $\omega\in A^c$, then $Z$ is a random variable.
\end{exercise}
\begin{solution}
    For any Borel set $B\subset\R$, 
    \begin{equation*}
        Z^{-1}(B) = \pth{A\cap X^{-1}(B)}\cup\pth{A^c\cap Y^{-1}(B)}.
    \end{equation*}
    Since $X$ and $Y$ are random variables, $X^{-1}(B), Y^{-1}(B)\in\F$. 
    The $\sigma$-algebra $\F$ is closed under finite intersections and 
    unions, so $Z^{-1}(B)\in\F$. $Z$ is thus a random variable. 
\end{solution}

\begin{exercise}{1.5}
    Show that if $\S = \sigma(\A)$, then $X^{-1}(\A) = \Set{X^{-1}(A)}{A\in\A}$ 
    generates $\sigma(X) = \Set{X^{-1}(B)}{B\in\S}$.
\end{exercise}
\begin{solution}
    Set
    \begin{equation*}
        \F = \Set{E\in\S}{X^{-1}(E)\in\sigma(X^{-1}(\A))}.
    \end{equation*} 
    Clearly $\varnothing\in\S$ and $X^{-1}(\varnothing) = \varnothing\in\sigma(X^{-1}(\A))$, so 
    $\varnothing\in\F$. If $E\in\F$, then $E^c\in\S$ and $X^{-1}(E^c) = (X^{-1}(E))^c\in\sigma(X^{-1}(\A))$
    since $\sigma(X^{-1}(\A))$ is a $\sigma$-algebra. Thus $E^c\in\F$. Suppose that 
    $\set{E_n}\subset\F$ is a countable collection of sets. Then $\set{E_n}\subset\S$ 
    and $\cup_n E_n\in\S$. Also, 
    \begin{equation*}
        X^{-1}(\cup_n E_n) = \cup_n X^{-1}(E_n)\in\sigma(X^{-1}(\A))
    \end{equation*}
    since $X^{-1}(E_n)\in\sigma(X^{-1}(\A))$ and $\sigma(X^{-1}(\A))$ is a 
    $\sigma$-algebra. Hence $\cup_n E_n\in\F$ and $\F$ is a $\sigma$-algebra.
    
    By the construction, $\A\subset\F\subset\S$ and we must have
    $\S = \sigma(\A)\subset\F\subset\S$; hence $\F = \S$. But by the definition
    of $\F$, $X^{-1}(\F) = \Set{X^{-1}(E)}{E\in\F}$ is a $\sigma$-algebra such 
    that $X^{-1}(\A)\subset X^{-1}(\F)\subset\sigma(X^{-1}(\A))$. Thus 
    $\sigma(X) = X^{-1}(\S) = X^{-1}(\F) = \sigma(X^{-1}(\A))$.
\end{solution}

\begin{exercise}{1.6}
    Conclude that a random variable $Y$ is measurable with respect to $\sigma(X)$ 
    if and only if $Y = f(X)$, where $f:\R\to\R$ is measurable.
\end{exercise}
\begin{solution}
    If $Y = f(X)$ for some measurable function $f:\R\to\R$, then for any Borel set 
    $B\subset\R$, $Y^{-1}(B) = X^{-1}(f^{-1}(B))\in\sigma(X)$ since $f^{-1}(B)$ is 
    also a Borel set and $X$ is $\sigma(X)$-measurable. Thus $Y$ is $\sigma(X)$-measurable. 

    Conversely, suppose that $Y$ is $\sigma(X)$-measurable. If $Y$ is simple, 
    $Y = \sum_{i=1}^n y_i\one_{A_i}$ where $A_i\in\sigma(X)$ are disjoint and 
    $y_i$ are distinct. Now $A_i = X^{-1}(B_i)$ for disjoint Borel sets $B_i$. 
    Set $f = \sum_i y_i\one_{B_i}$, which is measurable. Then 
    \begin{equation*}
        f(X) = \sum_i y_i\one_{X^{-1}(B_i)} = \sum_i y_i\one_{A_i} = Y.
    \end{equation*}

    For general measurable $Y$, there is a sequence of simple measurable functions 
    $Y_k\to Y$ pointwise. For each $k$, there is a corresponding measurable $f_k:\R\to\R$ 
    such that $Y_k = f_k(X)$. Now define $f$ to be the pointwise limit of $f_k$; 
    then $f$ is measurable and
    \begin{equation*}
        Y = \lim_{k\to\infty} Y_k = \lim_{k\to\infty} f_k(X) = f(X).
    \end{equation*}
\end{solution}

\end{document}