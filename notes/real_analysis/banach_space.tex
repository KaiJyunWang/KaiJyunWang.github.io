\begin{definition}
    A space $X$ is called a \textbf{Banach space} if it is a 
    complete normed vector space.
\end{definition}
\begin{remark}
    $\L^1$ is a Banach space with the norm 
    \begin{equation*}
        \norm{f}_{\L^1} = \int \abs{f}d\mu.
    \end{equation*}
    We treat $f = g$ a.e.\ as the same element in $\L^1$.
\end{remark}

\begin{definition}
    Let $V,W$ be vector spaces. A map $T:V\to W$ is \textbf{linear} 
    if for every $c\in\R$, $f,g\in V$, $T(cf+g) = cT(f)+T(g)$.
\end{definition} 

\begin{definition}
    A linear map $T:V\to W$ has \textbf{operator norm} defined 
    by 
    \begin{equation*}
        \norm{T} = \sup_{\norm{f}_V=1}\norm{T(f)}_W.
    \end{equation*} 
    $T$ is \textbf{bounded} if $\norm{T}<\infty$. We denote the 
    set of all bounded linear operators from $V$ to $W$ by 
    $B(V,W)$. 
\end{definition}

\begin{proposition}
    Suppose $W$ is a Banach space. Then $B(V,W)$ is a Banach 
    space with the operator norm.
\end{proposition}
\begin{proof}
    It suffices to show that $B(V,W)$ is complete. Let 
    $\set{T_i}\subset B(V,W)$ be a Cauchy sequence. Then for 
    $f\in V$, 
    \begin{equation*}
        \norm{T_i(f)-T_j(f)}_W \leq \norm{T_i-T_j}\norm{f}_V.
    \end{equation*} 
    Hence $\set{T_i(f)}$ is a Cauchy sequence in $W$. By the 
    completeness of $W$, we may define $Tf$ as the limit of 
    $T_i(f)$ as $i\to\infty$. Now, 
    \begin{equation*}
        \norm{Tf} \leq \sup_i\norm{T_i(f)} \leq \sup_i \norm{T_i}\norm{f}.
    \end{equation*}
    Since Cauchy sequences are bounded, $\norm{Tf} < \infty$ for 
    all $f\in V$ and $T\in B(V,W)$. It remains to show that 
    $T_i$ converges to $T$ in the operator norm. For any $f\in V$, 
    pick $N$ such that $\norm{T_i(f)-T_j(f)}\leq\epsilon$ for all 
    $i,j\geq N$. Then for fixed $i$, 
    \begin{equation*}
        \norm{(T_i-T_j)f} \leq \norm{T_i-T_j}\norm{f} \leq \epsilon\norm{f} 
    \end{equation*}
    for every $f\in V$ and $j\geq N$. Hence $\norm{T_i - T}\leq\epsilon$ 
    for all $i\geq N$ and the proof is complete.
\end{proof} 
\begin{remark}
    Consider $X$, $Y$ are two normed vector space. $\overline{X}$, 
    $\overline{Y}$ are the completion of $X$ and $Y$, respectively. 
    \begin{equation*}
        \overline{X} = \Set{\set{x_n}\subset X}{\set{x_n} \text{ is Cauchy}}.
    \end{equation*}
    Define the equivalence relation $\set{x_n}\sim\set{y_n}$ if 
    $\lim_{n\to\infty}\norm{x_n-y_n} = 0$. It is easy to see 
    that $\overline{X}$ is a Banach space with $\norm{\set{x_n}} = \lim_{n\to\infty}\norm{x_n}$.

    For $L:X\to Y$, a bounded linear operator, its counterpart 
    $\overline{L}:\overline{X}\to\overline{Y}$ is also a bounded 
    linear operator. 
\end{remark}


\begin{definition}
    $T$ is continuous if $f_i\to f$ in $V$ implies that 
    $T(f_i)\to T(f)$ in $W$.
\end{definition} 

\begin{proposition}
    Suppose $T:V\to W$ is linear. Then $T$ is continuous if and 
    only if $T$ is bounded.
\end{proposition}
\begin{proof}
    Suppose $T$ is not bounded. Then there exists $f_i\in V$ with 
    $\norm{f_i} \leq 1$ for all $i$ and $\norm{Tf_i}\to\infty$. 
    Thus 
    \begin{equation*}
        \frac{f_i}{\norm{Tf_i}}\to 0, \quad \text{but}\quad 
        T\pth{\frac{f_i}{\norm{Tf_i}}} = \frac{Tf_i}{\norm{Tf_i}} \not\to 0 
        \quad \text{as}\quad \frac{\norm{Tf_i}}{\norm{Tf_i}} = 1. 
    \end{equation*}
    Hence $T$ is not continuous. 

    Conversely, suppose $T$ is bounded. Let $f_i\to f$ in $V$. 
    Then 
    \begin{equation*}
        \norm{Tf_i-Tf} = \norm{T(f_i-f)} \leq \norm{T}\norm{f_i-f}\to 0.
    \end{equation*}
    Hence $T$ is continuous.
\end{proof}

\begin{definition}
    A \textbf{linear functional} $T$ is a linear map $T:V\to\mathbb{F}$, 
    where $\mathbb{F} = \R$ or $C$ is the scalar field of $V$.
\end{definition} 

\begin{definition}
    Let $V$, $W$ be vector spaces. $T:V\to W$ is linear. 
    The \textbf{kernel} of $T$ is defined as 
    \begin{equation*}
        \ker(T) = \Set{f\in V}{T(f)=0}.
    \end{equation*}
\end{definition}

\begin{proposition}\label{prop:kernel_closed}
    Let $X$ be a normed vector space and $T\in X'$. Then 
    \begin{thmenum}
        \item $\ker(T)$ is a closed subspace of $X$. 
        \item If $T\neq 0$, there exists $x\in X$ such that 
        $T(x)\neq 0$. Then for any $y\in X$, there exists 
        $c\in\R$ and $z\in\ker(T)$ such that $y = cx+z$.
    \end{thmenum}
\end{proposition}
\begin{proof}
    For (a), let $x,y\in\ker(T)$ and $c\in\R$. 
    \begin{equation*}
        T(cx+y) = cT(x)+T(y) = 0. \implies cx+y\in\ker(T).
    \end{equation*}
    Also, let $x_i\to x$ in $X$. Then since $T$ is continuous,  
    \begin{equation*}
        T(x) = \lim_{n\to\infty} T(x_n) = 0. \implies x\in\ker(T).
    \end{equation*}
    Hence $\ker(T)$ is a closed subspace of $X$. 

    For the rest part, fix $x\in X$ and $f(x)\neq 0$. For each 
    $y\in X$, let $\alpha = T(y)/T(x)$ and $z = y - T(y)x/T(x)$. 
    Then 
    \begin{equation*}
        \alpha x + z = \frac{T(y)}{T(x)}x + y - \frac{T(y)}{T(x)}x 
        = y - \frac{T(y)}{T(x)}x = y.
    \end{equation*}
\end{proof}

\begin{definition}
    The \textbf{dual space} of $V$ is defined as $V' = B(V,\mathbb{F})$, 
    where $\mathbb{F} = \R$ or $C$.
\end{definition}
\begin{remark}
    The dual space is a Banach space.
\end{remark}
\begin{remark}
    $T:X\to Y$ is bounded and linear. Then 
    \begin{equation*}
        \norm{T} = \inf\Set{c\in [0,\infty)}{\norm{Tx}_Y \leq c\norm{x}_X \text{ for all } x\in X}.
    \end{equation*}
\end{remark}

\begin{example}
    Let $X = C([0,1])$ with the supremum norm and $Y = \R$ with 
    the usual norm. For $g\in X$, $g(t)\neq 0$ on $[0,1]$,
    define $Tg:X\to\R$ by 
    \begin{equation*}
        Tg(f) = \int_0^1 f(t)g(t)dt.
    \end{equation*}
    Now for $\norm{f}_\infty \leq 1$, 
    \begin{equation*}
        \begin{split}
            \abs{Tg(f)} &= \abs{\int_{0}^{1}f(t)g(t)dt} 
            \leq \int_{0}^{1}\abs{f(t)g(t)}dt 
            \leq \int_{0}^{1}\abs{g(t)}\sup_{[0,1]}\abs{f(t)}dt \\
            &= \norm{f}_\infty\int_{0}^{1}\abs{g(t)}dt 
            \leq \int_{0}^{1}\abs{g(t)}dt.
        \end{split}
    \end{equation*}
    Take $f = g/\abs{g}$, 
    \begin{equation*}
        \abs{Tgf} = \abs{\int_{0}^{1}\frac{g^2(t)}{\abs{g(t)}}dt} 
        = \int_{0}^{1}\abs{g(t)}dt. 
        \implies \norm{Tg} = \int_{0}^{1}\abs{g(t)}dt.
    \end{equation*}
\end{example}

\begin{example}
    Consider $X = Y = C([0,1])$ with the supremum norm. Define 
    $T:C^1([0,1])\to Y$ by $Tf = f'$. Then consider the sequnce 
    $f_n(x) = e^{-n(x-1/2)^2}$, $f_n'(x) = e^{-n(x-1/2)^2}\pth{-2n(x-1/2)}$. 
    Hence $\norm{Tf_n}/\norm{f_n} = \sqrt{2n}e^{-1/2}\to\infty$ 
    as $n\to\infty$. Thus $T$ is not bounded.  
\end{example} 