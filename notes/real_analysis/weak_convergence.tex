\begin{definition}
    Let $(X,\norm{\cdot})$ be a normed space. A sequence $\set{x_n}$ 
    in $X$ is said to \textbf{converge weakly} to $x\in X$, denoted 
    by $x_n\wto x$, if for every $L\in X'$, $L(x_n)\to 
    L(x)$ as $n\to\infty$.
\end{definition}
\begin{remark}
    Strong convergence implies weak convergence. If $x_n\to x$, 
    \begin{equation*}
        \abs{L(x_n)-L(x)} = \abs{L(x_n-x)}\leq\norm{L}\norm{x_n-x}\to 0
    \end{equation*}
    as $n\to\infty$. Thus $x_n\wto x$. However, the 
    converse is not true in general.
\end{remark}

\begin{example}
    Consider $\ell^2$. Note that $(\ell^2)'\cong\ell^2$. For all 
    $L\in(\ell^2)'$, there exists $y\in\ell^2$ such that $L(x) = 
    \sum_{n=1}^\infty x_ny_n$. Let $x_n = e^n$ be the sequence 
    with $1$ at the $n$-th position and $0$ elsewhere. Then 
    $x_n\wto 0$ since for every $L\in(\ell^2)'$, 
    \begin{equation*}
        L(x_n) = \sum_{i} e^n_iy_i = y_n\to 0
    \end{equation*} 
    for $y\in\ell^2$. However, $\norm{x_n}_{\ell^2} = 1$ for 
    every $n$ and thus $x_n\not\to 0$. 
\end{example}

\begin{example}
    Consider $X = C([0,1])$ with the supremum norm. Let 
    \begin{equation*}
        x_n(t) = \begin{cases}
            nt & \text{if } 0\leq t\leq 1/n,\\
            2-nt & \text{if } 1/n\leq t\leq 2/n,\\
            0 & \text{if } 2/n\leq t\leq 1.
        \end{cases}
    \end{equation*}
    Then $\norm{x_n}_\infty = 1$ and thus $x_n\not\to 0$. Instead, 
    we have $x_n\wto 0$. Assume not, then we can find 
    $T\in X'$ and a subsequence $\set{x_{n_k}}$ such that 
    $\abs{T(x_{n_k})} \geq \delta > 0$. For simplicity, we consider 
    the case $T(x_{n_k})\geq\delta$, but the other case is similar. 
    Since $T\in X'$, $\abs{T(x_{n_k})}\leq \norm{T}_{X\to\R}
    \norm{x_{n_k}}_{\infty}$. Let $y_{K} = \sum_{k=1}^{K}x_{n_k}$. 
    Then $T(y_K) = \sum_{k=1}^{K}T(x_{n_k})\geq K\delta$ and 
    $T(y_K)\leq \norm{T}_{X\to\R}\norm{y_K}_\infty$. This implies 
    that $y_K$ cannot be bounded. Now consider $x_{n_k}$ with 
    $n_{k+1}\geq 2n_k$. For $t\in[0,1/n_K]$, $x_{n_k}(t) = n_kt$. 
    \begin{equation*}
        y_K(t) = \sum_{k=1}^{K} n_kt \leq \sum_{k=1}^{K} n_k/n_K 
        \leq 1 + \sum_{k=1}^{K} 2^{K-k} \leq 1 + \sum_k 2^{-k} = 2.
    \end{equation*}
    For $t\in[1/n_K,1/n_{K-1}]$, 
    \begin{equation*}
        y_K(t) = \sum_{k=1}^{K} x_{n_k}(t) 
        \leq 1 + \sum_{k=1}^{K-1} n_kt 
        \leq 1 + \frac{1}{n_{K-1}}\sum_{k=1}^{K-1} n_k 
        \leq 1 + 1 + \sum_k 2^{-k} = 3. 
    \end{equation*}
    On $[1/n_K,1/n_{K-1}]$, we have $\norm{y_K}\leq 3$. 
    Thus $\delta K\leq\norm{T}_{X\to\R}\norm{y_K}_\infty
    \leq 3\norm{T}_{X\to\R}$, which is impossible for 
    sufficiently large $K$. Hence $x_n\wto 0$.
\end{example}

\begin{proposition}
    $(X,\norm{\cdot}_X)$ is a normed space and $x_n\in X$. If 
    $\norm{x_n}_X\leq C$ for all $n\in\N$ and $L(x_n)\to L(x)$ 
    for all $L\in A\subset X'$, where $A$ is dense in $X'$, then 
    $x_n\wto x$ in $X$.
\end{proposition}
\begin{proof}
    Let $\epsilon>0$ be given. $A$ is dense in $X'$. For $T\in X'$, 
    there is an $L\in A$ such that $\norm{T-L}_{X'\to\R}\leq\epsilon$. 
    Also, there exists $N$ such that $\abs{L(x_n)-L(x)}\leq\epsilon$ 
    for all $n\geq N$. Then
    \begin{equation*}
        \begin{split}
            \abs{T(x_n) - T(x)} &\leq \abs{T(x_n) - L(x_n)} + \abs{L(x_n) - L(x)} + \abs{L(x) - T(x)} \\
            &\leq \norm{T-L}_{X'\to\R}(\norm{x_n}_X + \norm{x}_X) + \abs{L(x_n) - L(x)}
            \leq 2C\epsilon + \epsilon
        \end{split}
    \end{equation*}
    for all $n\leq N$. Since $\epsilon$ is arbitrary, $x_n\wto x$.
\end{proof}

\begin{definition}
    A space $X$ is called a \textbf{Baire space} if for any 
    sequence of open dense subsets $\set{E_n}$, $\cap_n E_n$ 
    is dense in $X$.
\end{definition}

\begin{theorem}[Baire Category Theorem]
    A complete metric space is a Baire space.
\end{theorem}
\begin{proof}
    Let $X$ be a complete metric space and $\set{E_n}$ be a 
    sequence of open dense subsets in $X$. Put $E = \cap_n E_n$. 
    We want to show that any nonempty open set $G\subset X$ 
    intersects $E$. 

    $E_1$ is dense in $X$ so $G\cap E_1$ is nonempty. Then 
    there exists $x_1\in E_1\cap G$. Note that $E_1\cap G$ 
    is open; there exists $1>\delta_1>0$ such that 
    $B_{\delta_1}(x_1)\subset E_1\cap G$. By shrinking $\delta_1$, 
    we can have $\overline{B_{\delta_1}(x_1)}\subset E_1\cap G$. 
    Now since $E_2$ is dense in $X$, there exists $x_2\in E_2\cap 
    B_{\delta_1}(x_1)$ and also a $1/2>\delta_2>0$ such that 
    $\overline{B_{\delta_2}(x_2)}\subset E_2\cap B_{\delta_1}(x_1)$. 
    Continue this process, we obtain a sequence $\set{x_n}$ and 
    $\delta_n\leq 1/n$ such that $\overline{B_{\delta_n}(x_n)}\subset 
    E_n\cap B_{\delta_{n-1}}(x_{n-1})$.

    For every $m,n\geq N$, we have $x_n\in B_{\delta_n}(x_n)\subset 
    \cdots\subset B_{\delta_N}(x_N)$ and $x_m\in B_{\delta_m}(x_m) 
    \subset \cdots\subset B_{\delta_N}(x_N)$ by construction. Hence 
    $d(x_n,x_m)\leq 2\delta_N\leq 2/N$ and $\set{x_n}$ is a Cauchy 
    sequence. Since $X$ is complete, $\set{x_n}$ converges to some 
    $x\in X$. We claim that $x\in E\cap G$. Clearly $x\in G$. By 
    construction $x_m\in \overline{B_{\delta_n}(x_n)}$ for all 
    $m\geq n$. Thus $x\in B_{\delta_n}(x_m)\subset E_N$ for $m\geq 
    n\geq N$. We see that $x\in\cap_n E_n$. Notice that $G$ is 
    arbitrary, so $E$ is dense in $X$, proving that $X$ is a Baire 
    space.
\end{proof}

\begin{theorem}[Uniform Boundedness Principle \rom{1}]\label{thm:ubp1}
    $X$ is a complete metric space. $f_\alpha:X\to\R$ is continuous 
    for every $\alpha\in A$, where $A$ is an index set. If for every 
    $x\in X$, there exists $M(x)<\infty$ such that 
    \begin{equation*}
        \sup_{\alpha\in A}\abs{f_\alpha(x)}\leq M(x),
    \end{equation*}
    then there exists an open $G$ and a constant $C<\infty$ such that 
    \begin{equation*}
        \sup_{\alpha\in A}\abs{f_\alpha(x)}\leq C
    \end{equation*}
    for all $x\in G$.
\end{theorem}
\begin{proof}
    By Baire Category Theorem, $X$ is a Baire space. For each $n$, 
    let 
    \begin{equation*}
        F_n = \Set{x\in X}{\sup_{\alpha\in A}\abs{f_\alpha(x)}\leq n}.
    \end{equation*}
    We claim that $F_n$ is closed and $X = \cup_n F_n$. Indeed, 
    set $x_k\to x\in X$, where $x_k\in F_n$ for all $k$. For any 
    $\alpha\in A$, $\abs{f_\alpha(x_k)}\leq n$ for all $k$ and 
    by continuity of $f_\alpha$, 
    \begin{equation*}
        \abs{f_\alpha(x)} = \lim_{k\to\infty}\abs{f_\alpha(x_k)}\leq n.
    \end{equation*}
    Hence $x\in F_n$ and $F_n$ is closed. Next, for any $x\in X$, 
    take $N \geq M(x)$. Then $x\in F_N\subset\cup_n F_n$. This 
    shows that $X = \cup_n F_n$. 

    Finally, observe that $F_n$ cannot have empty interiors for 
    all $n$. Otherwise, $\varnothing = X^c = \pth{\cup_n F_n}^c
    = \cap F_n^c \neq \varnothing$ since $F_n^c$ are open dense 
    subsets of $X$, which is absurd. Hence there is some $n$ such 
    that $F_n$ has nonempty interior, say $G\subset F_n$. Then 
    $\sup_{\alpha\in A}\abs{f_\alpha(x)}\leq n$ for all $x\in G$ 
    as desired.
\end{proof}

\begin{definition}
    A function $f:X\to\R$ is said to be \textbf{sub-additive} if 
    $f(x+y)\leq f(x)+f(y)$ for all $x,y\in X$.
\end{definition}

\begin{theorem}[Uniform Boundedness Principle \rom{2}]\label{thm:ubp2}
    $X$ is a Banach space. $\alpha\in A$ is an arbitrary index set. 
    $f_\alpha:X\to\R$ are continuous, sub-additive and satisfy 
    $f_\alpha(cx) = \abs{c}f_\alpha(x)$ for all $x\in X$ and $c\in\R$. 
    If for every $x\in X$, there exists $M(x)<\infty$ such that 
    \begin{equation*}
        \sup_{\alpha\in A}\abs{f_\alpha(x)}\leq M(x),
    \end{equation*}
    then there exists a constant $C<\infty$ such that
    \begin{equation*}
        \sup_{\alpha\in A}\abs{f_\alpha(x)}\leq C\norm{x}_X
    \end{equation*}
    for all $x\in X$.
\end{theorem}
\begin{proof}
    By \cref{thm:ubp1}, there exists an open $G$ and a constant 
    $C<\infty$ such that 
    \begin{equation*}
        \sup_{\alpha\in A}\abs{f_\alpha(x)}\leq C
    \end{equation*}
    for all $x\in G$. The proof will be complete if we can 
    extend $G$ to $X$. Since $G$ is open, there exists $r>0$ 
    such that $B_r(z)\subset G$ for all $z\in G$. For any 
    $x\in B_r(z)$, $\sup_{\alpha\in A} \abs{f_\alpha(x)}
    \leq C$ and hence $\sup_{\alpha\in A} \abs{f_\alpha(z+y)} 
    \leq C$ for all $y\in B_r(0)$. Take $y$ with $\norm{y}
    \leq r/2$. Then 
    \begin{equation*}
        -2C \leq f_\alpha(y+z) - f_\alpha(z) \leq f_\alpha(y) 
        \leq f_\alpha(y+z) + f_\alpha(-z) 
        = f_\alpha(y+z) + f_\alpha(z) \leq 2C.
    \end{equation*}
    Hence $\abs{f_\alpha(y)}\leq 2C$ for all $y$ with $\norm{y}
    \leq r/2$. Take $x\in X$. 
    \begin{equation*}
        \abs{f_\alpha(x)} 
        = \abs{f_\alpha\pth{\frac{x}{\norm{x}}\frac{r}{2}\frac{2}{r}\norm{x}}} 
        = \frac{2}{r}\norm{x}\abs{f_\alpha(y)} \leq \frac{4C}{r}\norm{x}.
    \end{equation*}
    Thus 
    \begin{equation*}
        \sup_{\alpha\in A}\abs{f_\alpha(x)}\leq \frac{4C}{r}\norm{x}
    \end{equation*}
    for all $x\in X$.
\end{proof}

\begin{corollary}
    $X$ is a Banach space. $L_\alpha\in X'$ and $\alpha\in A$. 
    If for every $x\in X$, there exists $M(x)<\infty$ such that 
    $\sup_{\alpha\in A}\abs{L_\alpha(x)}\leq M(x)$, then there 
    exists a constant $C<\infty$ such that $\sup_{\alpha\in A}
    \norm{L_\alpha}\leq C$.
\end{corollary}
\begin{proof}
    Apply \cref{thm:ubp2} to $f_\alpha(x) = \abs{L_\alpha(x)}$. First,
    $L_\alpha$ is linear and the sub-linearity follows from the triangle 
    inequality. Next, $\abs{L_\alpha(cx)} = \abs{c}\abs{L_\alpha(x)}$ for 
    all $c\in\R$. Also, $L_\alpha\in X'$ implies that $f_\alpha$ is 
    continuous. The conclusion follows from \cref{thm:ubp2}.
\end{proof}

\begin{corollary}\label{cor:ubp}
    $X$ is a normed space. $x_\alpha\in X$ for all $\alpha\in A$ with 
    the property that for every $L\in X'$, there is $M(L)<\infty$ 
    such that $\sup_\alpha \abs{L(x_\alpha)}\leq M(L)$ and 
    $(X',\norm{\cdot}_{X\to\R})$ is a Banach space. Then there exists 
    $C<\infty$ such that $\norm{x_\alpha}_X\leq C$ for all $\alpha\in A$.
\end{corollary}
\begin{proof}
    Apply the \cref{thm:ubp2} to $f_\alpha(L) = \abs{L(x_\alpha)}$. First, 
    for $L,T\in X'$, 
    \begin{equation*}
        f_\alpha(L+T) = \abs{L(x_\alpha)+T(x_\alpha)} 
        \leq \abs{L(x_\alpha)} + \abs{T(x_\alpha)} 
        = f_\alpha(L) + f_\alpha(T).
    \end{equation*}
    Next, for $c\in\R$, 
    \begin{equation*}
        f_\alpha(cL) = \abs{cL(x_\alpha)} = \abs{c}\abs{L(x_\alpha)} 
        = \abs{c}f_\alpha(L).
    \end{equation*}
    Finally, to verify that $f_\alpha$ is continuous, note that for 
    $L_n\to L$ in $X'$, 
    \begin{equation*}
        \abs{f_\alpha(L_n) - f_\alpha(L)} = \abs{L_n(x_\alpha) - L(x_\alpha)} 
        \leq \norm{L_n-L}_{X'\to\R}\norm{x_\alpha}_X\to 0
    \end{equation*}
    for each $\alpha\in A$. The conclusion follows from \cref{thm:ubp2}.
\end{proof}

\begin{corollary}\label{cor:ubp2}
    $X$ is a normed space and $x_n\in X$ with $x_n\wto x$ in $X$. 
    Then there exists $C<\infty$ such that $\norm{x_n}_X\leq C$ for all $n$. 
\end{corollary}
\begin{proof}
    This is a direct consequence of \cref{cor:ubp} with $A = \N$.
\end{proof}

\begin{proposition}
    Let $f_n\in\L^p(X,\mu)$ and $1\leq p<\infty$. Then $f_n\wto f\in\L^p$ if 
    \begin{equation*}
        \lim_{n\to\infty} \int f_ngd\mu = \int fgd\mu
    \end{equation*}
    for all $g\in\L^{p'}(X,\mu)$ and some $f$ in $\L^p$ where $p'$ is the 
    conjugate exponent of $p$.
\end{proposition}
\begin{proof}
    By the assumption and Riesz representation theorem, for every 
    $T\in\pth{\L^p}'$, there exists a unique $g\in\L^{p'}$ such that 
    \begin{equation*}
        T(f_n) = \int f_ngd\mu \to \int fgd\mu = T(f).
    \end{equation*}
    Hence $f_n\wto f$.
\end{proof}

\begin{proposition}\label{prop:weak_conv_bd}
    $f_n\in\L^p(X,\mu)$ and $1\leq p<\infty$. If $f_n\wto f$ in $\L^p$, 
    then $f_n$ is bounded and 
    \begin{equation*}
        \norm{f_n}_p\leq \liminf_{n\to\infty}\norm{f_n}_p.
    \end{equation*}
\end{proposition}
\begin{proof}
    Consider the function 
    \begin{equation*}
        g = \frac{\abs{f}^{p/p'}}{\norm{f}_p^{p/p'}}.
    \end{equation*}
    Note that 
    \begin{equation*}
        \norm{g}_{p'}^{p'} = \int \abs{g}^{p'}d\mu 
        = \int \frac{\abs{f}^p}{\norm{f}_p^p}d\mu = 1.
    \end{equation*}
    Hence $g\in\L^{p'}$ with $\norm{g}_{p'} = 1$. Also notice 
    that $\abs{g} = \abs{f}^{p/p'}/\norm{f}_p^{p/p'} 
    = \abs{f}^{p-1}/\norm{f}_p^{p-1}$. By the weak convergence 
    and Riesz representation theorem, 
    \begin{equation*}
        \begin{split}
            \norm{f}_p = \int \frac{\abs{f}^p}{\norm{f}_p^{p-1}}d\mu 
            = \int \abs{fg}d\mu = \lim_{n\to\infty}\int \abs{f_ng}d\mu
            \leq \liminf_{n\to\infty}\norm{f_n}_p\norm{g}_{p'} 
            = \liminf_{n\to\infty}\norm{f_n}_p
        \end{split}
    \end{equation*}
    by the H\"older inequality. Note that by \cref{cor:ubp2}, $f_n$ 
    is bounded uniformly in $n$.
\end{proof}

\begin{proposition}
    $1\leq p<\infty$ and $1/p+1/p'=1$. Suppose $f_n\to f$ in $\L^p$ 
    and $g_n\to g$ in $\L^{p'}$. Then 
    \begin{equation*}
        \lim_{n\to\infty}\int f_ng_nd\mu = \int fg d\mu.
    \end{equation*}
\end{proposition}
\begin{proof}
    By the H\"older inequality, 
    \begin{equation*}
        \begin{split}
            \abs{\int f_ng_nd\mu - \int fgd\mu} 
            &\leq \abs{\int f_n(g_n-g)d\mu} + \abs{\int (f_n-f)gd\mu} \\
            &\leq \norm{f_n}_p\norm{g_n-g}_{p'} + \norm{f_n-f}_p\norm{g}_{p'}.
        \end{split}
    \end{equation*}
    Note that by \cref{prop:weak_conv_bd}, $f_n$ converges 
    to $f$ strongly and hence weakly. It follows that $\norm{f_n}$ 
    is bounded by some $C<\infty$. Since $g_n$ converges to $g$ 
    and $f_n$ converges to $f$ in their respective norms, the 
    right hand side of the inequality converges to $0$ as $n\to\infty$.
\end{proof}
\begin{remark}
    If we loosen the condition to $f_n\wto f$ in $\L^p$ 
    and $g_n\wto g$ in $\L^{p'}$, then the conclusion 
    fails.
\end{remark}
\begin{example}
    Suppose $p=p'=2$ and $f_n(x) = \sqrt{2/\pi}\sin(nx)$ for 
    $x\in[0,\pi]$. Then $f_n\in\L^2([0,\pi])$ and 
    \begin{equation*}
        \int_0^\pi f_n^2dx = \frac{2}{\pi}\int_0^\pi \sin^2(nx)dx = 1.
    \end{equation*}
    To see that $f_n \wto 0$, let $g\in\L^2([0,\pi])$. 
    For every $\epsilon>0$, there is a step function $\phi$ such 
    that $\norm{g-\phi}_2<\epsilon$. Note that every step function 
    is a finite linear combination of characteristic functions of 
    intervals. Hence it suffices to show that $f_n\chi_I$ can be 
    arbitrary small for $n$ sufficiently large. On every interval, 
    \begin{equation*}
        \abs{\int_I \sin(nx)dx} \leq \int_{0}^{\pi/n}\sin(nx)dx 
        = \frac{2}{n}\to 0
    \end{equation*}
    as $n\to\infty$. Thus $f_n\wto 0$ in $\L^2([0,\pi])$. 
    However, $f_n$ does not converge to $0$ strongly in $\L^2([0,\pi])$ 
    since $\norm{f_n}_2 = 1\neq 0$ for all $n$.
\end{example}

\begin{proposition}\label{prop:weak_conv_on_set}
    $1\leq p<\infty$. Let $f_n\in\L^p(X,\mu)$ be a bounded sequence 
    of functions. Then $f_n\wto f$ in $\L^p$ if and only if 
    \begin{equation*}
        \lim_{n\to\infty}\int_A f_nd\mu = \int_A fd\mu
    \end{equation*}
    for all $A\in\mathcal{A}$ when $p=1$ and for $A$ with finite 
    measure when $p>1$.
\end{proposition}
\begin{proof}
    \begin{equation*}
        \begin{split}
            f_n\wto f 
            &\iff \int f_ngd\mu \to \int fgd\mu \text{ for all } g\in\L^{p'} \\
            &\iff \int_A f_nsd\mu \to \int_A fsd\mu \text{ for all simple } s\in\L^{p'} \\ 
            &\iff \int_A f_nd\mu = \int f_n\chi_Ad\mu \to \int f\chi_Ad\mu = \int_A fd\mu
        \end{split}
    \end{equation*}
    for $A\in\A$ such that $\chi_A\in\L^{p'}$. If $p=1$, then $A$ can be taken 
    to be any $A\in\A$; if $p>1$, then $A$ must have finite measure.
\end{proof}

\begin{proposition}
    $1<p<\infty$. Let $f_n\in\L^p(X,\mu)$ be a sequence with 
    $\norm{f_n}_p\leq M$ and $f_n\to f$ pointwise a.e. Then 
    $f_n\wto f$ in $\L^p$.
\end{proposition}
\begin{proof}
    Since $\norm{f_n}_p\leq M$, 
    \begin{equation*}
        \int \abs{f}^pd\mu = \int \liminf_{n\to\infty}\abs{f_n}^pd\mu 
        \leq \liminf_{n\to\infty}\int \abs{f_n}^pd\mu = M^p
    \end{equation*}
    by Fatou's lemma. Hence $f\in\L^p$. It remains to show that 
    the convergence is weak. By \cref{prop:weak_conv_on_set}, it is equivalent 
    to show that
    \begin{equation*}
        \lim_{n\to\infty}\int_A f_nd\mu = \int_A fd\mu
    \end{equation*}
    for all $A\in\mathcal{A}$ with $\mu(A)<\infty$. Indeed, by Egorov's
    theorem, for every $\epsilon>0$, there exists $F_\epsilon\subset A$ with 
    $\mu(A-F_\epsilon)\leq\epsilon$ and $f_n\to f$ uniformly on $F_\epsilon$. 
    Furthermore, by \cref{prop:int_abs_conti}, we can choose $F_\epsilon$ so 
    that 
    \begin{equation*}
        \int_{A-F_\epsilon} \abs{f_n-f}^pd\mu \leq \epsilon
    \end{equation*}
    since $f_n,f\in\L^p$ and so does $\abs{f_n-f}^p$. Also, let 
    $E = \Set{x\in A-F_\epsilon}{\abs{f_n-f}>1}$. Then for $n$ 
    sufficiently large,
    \begin{equation*}
        \begin{split}
            \int_A \abs{f_n-f}d\mu 
            &\leq \int_{F_\epsilon} \abs{f_n-f}d\mu + \int_{A-F_\epsilon} \abs{f_n-f}d\mu \\
            &\leq \int_A \epsilon d\mu + \int_{A-F_\epsilon-E} \abs{f_n-f}d\mu + \int_E \abs{f_n-f}d\mu \\
            &\leq \epsilon\mu(A) + \mu(A-F_\epsilon) + \int_{A-F_\epsilon} \abs{f_n-f}^pd\mu 
            \leq \epsilon\mu(A) + \epsilon + \epsilon.
        \end{split}
    \end{equation*}
    Hence $f_n\wto f$.
\end{proof}

\begin{remark}
    The proposition fails for $p=1$. Consider $f_n = n\chi_{[0,1/n]}$. 
    Then $\norm{f_n}_1 = 1$ and $f_n\to 0$ pointwise a.e. However, 
    \begin{equation*}
        \int_0^1 f_n(x)dx = 1 \neq 0 = \int_0^1 0dx.
    \end{equation*}
    Thus $f_n$ does not converge weakly to $0$ in $\L^1$.
\end{remark}

\begin{theorem}[Radon-Riesz]
    $1<p<\infty$. Then $f_n\to f$ in $\L^p$ if and only if 
    $\lim_{n\to\infty} \norm{f_n}_p = \norm{f}_p$ and 
    $f_n\wto f$ in $\L^p$.
\end{theorem}
\begin{proof}
    Suppose $f_n\to f$ in $\L^p$. Then the strong convergence 
    immediately implies the weak convergence. Also, note that 
    $\norm{f_n}_p\leq\norm{f_n-f}_p+\norm{f}_p$ and thus 
    \begin{equation*}
        \abs{\norm{f_n}_p-\norm{f}_p} \leq \norm{f_n-f}_p\to 0
    \end{equation*}
    by the strong convergence. Conversely, suppose that 
    $\norm{f_n}_p\to\norm{f}_p$ and $f_n\wto f$ in 
    $\L^p$. 
    
    Assume $p\geq 2$. For any $y\in\R$, notice that $\abs{1+y}^p 
    \geq 1 + py + c\abs{y}^p$ for some $c\in(0,1)$. Let 
    $E = \Set{x\in X}{f(x) = 0}$ and apply $y = (f_n-f)/f$ 
    on $E^c$. Then on $E^c$,
    \begin{equation*}
        \abs{\frac{f_n}{f}}^p
        \geq 1 + p\pth{\frac{f_n-f}{f}} + c\abs{\frac{f_n-f}{f}}^p
    \end{equation*}
    Thus 
    \begin{equation*}
        \abs{f_n}^p\geq \abs{f}^p + p(f_n-f)\abs{f}^{p-1}\sgn(f) + c\abs{f_n-f}^p.
    \end{equation*}
    Rearranging the inequality and integrating both sides on 
    $E^c$ gives 
    \begin{equation*}
        c\int_{E^c} \abs{f_n-f}^pd\mu 
        \leq \int_{E^c}\abs{f_n}^p-\abs{f}^pd\mu - p\int_{E^c}\abs{f}^{p-1}\sgn(f)(f_n-f)d\mu
    \end{equation*}
    Note that as shown in the proof of \cref{prop:weak_conv_bd}, 
    $\abs{f}^{p-1}\sgn(f)\in\L^{p'}$. By the assumptions we see 
    that 
    \begin{equation*}
        \int_{E^c} \abs{f_n-f}^pd\mu\to 0
    \end{equation*}
    as $n\to\infty$. On $E$, we have $f = 0$ and 
    \begin{equation*}
        \int_E \abs{f_n-f}^pd\mu = \int_E \abs{f_n}^pd\mu \to 0
    \end{equation*}
    as $n\to\infty$. Hence $f_n\to f$ in $\L^p$.

    Assume $1<p<2$. Then we have the same inequality for $\abs{z}\geq 1$, i.e., 
    \begin{equation*}
        \abs{1+z}^p \geq 1 + p\abs{z} + c\abs{z}^p
    \end{equation*}
    Also, for $\abs{z}\leq 1$, 
    \begin{equation*}
        \frac{\abs{1+z}^p-1-pz}{z^2}
    \end{equation*}
    is strictly positive. Now let $E_n = \Set{x\in X}{\abs{f_n(x)-f(x)}\leq \abs{f(x)}}$. Then 
    by applying the same argument above on $E_n^c$, we have 
    \begin{equation*}
        \int_{E_n^c} \abs{f_n-f}^pd\mu \leq \frac{1}{c}\int_{E_n^c}\abs{f_n}^p-\abs{f}^pd\mu - \frac{p}{c}\int_{E_n^c}\abs{f}^{p-1}\sgn(f)(f_n-f)d\mu
    \end{equation*} 
    as $n\to\infty$. On $E_n$, replacing $z$ by $(f_n-f)/f$,
    \begin{equation*}
        \abs{\frac{f_n}{f}}^p \geq 1 + p\frac{f_n-f}{f} + c'\pth{\frac{f_n-f}{f}}^2 
        \implies \abs{f_n}^p \geq \abs{f}^p + p(f_n-f)\abs{f}^{p-1}\sgn(f) + c'\abs{f_n-f}^2\abs{f}^{p-2}
    \end{equation*}
    for some $c'>0$. Thus
    \begin{equation*}
        \int_{E_n} \abs{f_n-f}^2\abs{f}^{p-2}d\mu \leq \frac{1}{c'}\int_{E_n}\abs{f_n}^p-\abs{f}^pd\mu - \frac{p}{c'}\int_{E_n}\abs{f}^{p-1}\sgn(f)(f_n-f)d\mu.
    \end{equation*}
    Adding up the two inequalities, we have 
    \begin{equation*}
        \int_{E_n^c} \abs{f_n-f}^pd\mu + \int_{E_n} \abs{f_n-f}^2\abs{f}^{p-2}d\mu \to 0
    \end{equation*}
    as $n\to\infty$ by the assumptions. Note that on $E_n$, $\abs{f}\geq \abs{f_n-f}$ and 
    \begin{equation*}
        \begin{split}
            \int_{E_n} \abs{f_n-f}^pd\mu \leq \int_{E_n} \abs{f_n-f}\abs{f}^{p-1}d\mu 
            &\leq \pth{\int_{E_n} \abs{f_n-f}^2\abs{f}^{p-2}d\mu}^{1/2}\pth{\int_{E_n}\abs{f}^pd\mu}^{1/2} \\
            &\leq \pth{\int_{E_n} \abs{f_n-f}^2\abs{f}^{p-2}d\mu}^{1/2}\norm{f}_p^{p/2}\to 0.
        \end{split}
    \end{equation*}
    Hence $f_n\to f$ in $\L^p$. We conclude that $f_n\to f$ strongly in $\L^p$ if and only if
    $f_n\wto f$ in $\L^p$ and $\norm{f_n}_p\to\norm{f}_p$.
\end{proof}
\begin{remark}
    Radon-Riesz theorem fails for $p=1$. Consider $f_n(x) = 1+\sin(nx)$ on $X=[-\pi,\pi]$. 
    Then for every $g\in\L^\infty$, 
    \begin{equation*}
        \int (f_n - 1)gd\mu \leq \int \sin(nx)gd\mu \to 0
    \end{equation*}
    by the step function approximation argument. Also, $\norm{f_n}_1 = 2\pi$ for all $n$ and 
    hence converges to $\norm{1}_1 = 2\pi$. However, $f_n$ does not converge to $1$ in $\L^1$ 
    since 
    \begin{equation*}
        \int_{-\pi}^{\pi} \abs{f_n-1}d\mu = \int_{-\pi}^{\pi} \abs{\sin(nx)}d\mu  
        = 2n\int_{0}^{\frac{\pi}{2n}} \sin(nx)dx = 2
    \end{equation*}
    for all $n$.
\end{remark}

\begin{definition}
    Let $X$ be a Banach space. A subset $K\subset X$ is \textbf{weakly sequentially compact} 
    if every sequence $\set{x_n}\subset K$ has a subsequence $\set{x_{n_k}}$ such that 
    $x_{n_k}\wto x\in K$.
\end{definition}

\begin{proposition}\label{prop:w_seq_cpt_bd}
    Let $X$ be a Banach space. If $K\subset X$ is weakly sequentially compact, then 
    $K$ is bounded.
\end{proposition}
\begin{proof}
    Suppose $K$ is not bounded. Then we can choose an unbounded sequence 
    $\set{x_n}\subset K$ such that $\norm{x_n}\geq n$ for all $n$. By the 
    weakly sequential compactness of $K$, there exists a weaky convergent 
    subsequence $\set{x_{n_k}}$ such that $x_{n_k}\wto x\in K$ and also 
    $\norm{x_{n_k}}\geq n_k$. However, by \cref{cor:ubp2}, $\norm{x_{n_k}}\leq C$ 
    for some $C<\infty$, which is absurd. Hence $K$ is bounded.
\end{proof}

\begin{theorem}[Kakutani]
    Let $X$ be a reflexive Banach space. Then the closed unit ball 
    \begin{equation*}
        B = \Set{x\in X}{\norm{x}\leq 1}
    \end{equation*}
    is weakly sequentially compact.
\end{theorem}
\begin{proof}
    We consider first the case when $X$ is separable. Reflexivity gives 
    that $X''\cong X$ and hence $X''$ is separable. By \cref{thm:dual_separable}, 
    $X'$ is separable, and there exists a countable dense subset 
    $\set{m_j}\subset X'$. Given $x_n\in X$ with $\norm{x_n}\leq 1$, we need 
    to show that there exists a subsequence $x_{n_k}\wto x\in B$. Since $m_j(x_n)$ 
    is a bounded sequence for each $j$, we can extract a subsequence $x_{n_k}$ such that 
    $m_j(x_{n_{k_j}})\to A(m_{k_j})$ as $j\to\infty$, where 
    \begin{equation*}
        A(m_j) = \lim_{j\to\infty} m_j(x_{n_{k_j}}).
    \end{equation*}

    We claim that for all $m\in X'$, $m(x_{n_k})\to A(m)$ as $k\to\infty$. Indeed, 
    for any $m\in X'$, we can find a sequence $\set{m_j}$ such that $m_j\to m$ as 
    $j\to\infty$. Then 
    \begin{equation*}
        \begin{split}
            \abs{m(x_{n_k}) - m(x_{n_l})} 
            &\leq \abs{m(x_{n_k}) - m_j(x_{n_k})} + \abs{m_j(x_{n_k}) - m_j(x_{n_l})} + \abs{m_j(x_{n_l}) - m(x_{n_l})} \\
            &\leq \norm{m-m_j}\norm{x_{n_k}} + \abs{m_j(x_{n_k}) - m_j(x_{n_l})} + \norm{m-m_j}\norm{x_{n_l}} \\
            &\leq 2\norm{m-m_j} + \abs{m_j(x_{n_k}) - m_j(x_{n_l})} \to 0
        \end{split}
    \end{equation*}
    as $k,l\to\infty$. Hence the sequence $\set{m(x_{n_k})}$ is Cauchy and $A$ 
    is well-defined. Notice that $A$ is also bounded:
    \begin{equation*}
        \abs{A(m)} = \lim_{k\to\infty} \abs{m(x_{n_k})} \leq \lim_{k\to\infty}\norm{m}\norm{x_{n_k}} = \norm{m}.
    \end{equation*}
    We see that $\norm{A}\leq 1$. Because $A$ is bounded, it is continuous and thus 
    $m(x_{n_k})\to m(x)$ for some $x\in X$ by the reflexivity of $X$. Such $x$ belongs 
    to $B$ since $\norm{x} = \norm{A}\leq 1$. Thus $B$ is weakly sequentially compact.

    For the general case where $X$ is not separable, consider the sequence $\set{x_n}\subset B$. 
    Let $Y = \Set{\sum_{n=1}^N \alpha_n x_n}{N\in\N, \alpha_n\in\R}$ be the closed subspace of 
    $X$ spanned by $\set{x_n}$. Since $X$ is reflexive, $Y$ is also reflexive by 
    \cref{thm:subspace_reflexive}. Note that $Y$ is also separable. The established 
    results above show that there exists a subsequence $\set{x_{n_k}}\subset Y$ and $x\in Y$ 
    such that $x_{n_k}\wto x$ in $Y$, i.e., for every $m\in Y'$, $m(x_{n_k})\to m(x)$. 
    Extend the functionals $m\in Y'$ to $\ell\in X'$ by Hahn-Banach theorem. Then 
    $\ell|_Y = m\in Y'$ implies that $\ell(x_{n_k}) = m(x_{n_k})\to m(x) = \ell(x)$. 
    We conclude that $x_{n_k}\wto x\in B$. Thus $B$ is weakly sequentially compact.
\end{proof}

\begin{example}
    Let $p\in(1,\infty)$. Then $\L^p(\Omega,\mu)$ is reflexive. Then for all 
    $\set{f_n}$ with $\norm{f_n}_p\leq 1$, there exists a subsequence $f_{n_k}\wto f$ 
    in $\L^p$ for some $f$ with $\norm{f}_p\leq 1$. By Riesz representation theorem, 
    this is equivalent to saying that for every $g\in\L^{q}(\Omega,\mu)$, 
    \begin{equation*}
        \lim_{k\to\infty} \int f_{n_k}g d\mu = \int fg d\mu,
    \end{equation*}
    where $q$ is the conjugate exponent of $p$. 
\end{example}

\begin{definition}
    Let $M$ be a Banach space. A sequence of bounded linear functionals 
    $\set{x_n}\subset M'$ \textbf{converges $\text{weakly}^*$} to $x$ if 
    for all $m\in M$, $x_n(m)\to x(m)$ as $n\to\infty$. We denote the 
    convergence by $x_n\wstarto x$.
\end{definition}
\begin{remark}
    Since the canonical mapping $M\to M''$ is always injective, $w^*$ 
    convergence is weaker than weak convergence. Allowing for the abuse 
    of notation, we can write $M\subset M''$. Consider now a sequence 
    $x_n\in M'$ with $x_n\wto x$ in $M'$. Then $\ell(x_n)\to\ell(x)$ for 
    any $\ell\in M''$. This implies that $x_n(m)\to x(m)$ for all $m\in M$ 
    and hence $x_n\wstarto x$ in $M'$. Thus weak convergence implies $w^*$ 
    convergence. 

    The converse is true if $M$ is reflexive. However, once we remove the 
    reflexivity condition, the converse fails. Let $X$ be the space of finite 
    signed measures on $[-1,1]$. We have already seen in \cref{thm:dual_C} that 
    $C([-1,1])'\cong X$. Consider the measures $\nu_n(A) = n\mu(A\cap[-1/n,1/n])/2$. 
    We claim that $\nu_n\wstarto \delta_0$, where $\delta_0$ is the Dirac measure 
    at $0$. Indeed, for any $f\in C([-1,1])$, the corresponding functional $\ell_n$ 
    for $\nu_n$ is given by
    \begin{equation*}
        \ell_n(f) = \int_{-1}^1 f d\nu_n = \frac{n}{2}\int_{-1/n}^{1/n} f(x)dx
        \to f(0) = \ell_0(f),
    \end{equation*}
    where $\ell_0$ is the functional defined as $\ell_0:f\mapsto f(0)$. 
    Thus $\ell_n\wstarto\ell_0$.

    However, $\ell_n$ is not weakly convergent to $\ell_0$. To see this, consider 
    the evaluation functional $L_{\set{0}}:\ell\mapsto(\phi\ell)(\set{0})$, where 
    $\phi$ is the isometric isomorphism from $C([-1,1])'$ to $X$. 
    Then $L_{\set{0}}\in X' = M''$. However, 
    \begin{equation*}
        L_{\set{0}}(\ell_n) = \nu_n(\set{0}) = 0 \not\to 1 = \delta_0(\set{0}) = L_{\set{0}}(\ell_0).
    \end{equation*}
    Thus $\nu_n$ does not converge weakly to $\delta_0$.
\end{remark}

\begin{definition}
    Let $M$ be a Banach space. A subset $K\subset M'$ is \textbf{weakly* sequentially compact} 
    if every sequence $\set{x_n}\subset K$ has a subsequence $\set{x_{n_k}}$ such that 
    $x_{n_k}\wstarto x\in K$.
\end{definition}

\begin{theorem}[Banach-Alaoglu]
    Let $M$ be a separable Banach space. Then the closed unit ball 
    \begin{equation*}
        B = \Set{x\in M'}{\norm{x}\leq 1}
    \end{equation*}
    is weakly* sequentially compact.
\end{theorem}
\begin{proof}
    This proof is similar to the proof of Kakutani theorem. Let $\set{x_n}\subset B$ be a 
    sequence. By the separability of $M$, for every $m\in M$, there is a sequence $\set{m_j}\subset M$ 
    such that $m_j\to m$ as $j\to\infty$. For any fixed $j$, $\abs{x_n(m_j)}\leq \norm{m_j}$ 
    is a bounded sequence. Hence we can extract a subsequence $x_{n_k}$ such that 
    \begin{equation*}
        x_{n_k}(m_j)\to A(m_j) \text{ as } k\to\infty
    \end{equation*}
    for some $A(m_j)\in\R$, with $A(m) = \lim_{k\to\infty} x_{n_k}(m)$. $A$ is a 
    bounded linear functional on $M$ since it is clearly linear and 
    \begin{equation*}
        \abs{A(m)} = \lim_{k\to\infty} \abs{x_{n_k}(m)} \leq \lim_{k\to\infty} \norm{x_{n_k}}\norm{m} = \norm{m},
    \end{equation*}
    showing that $\norm{A}\leq 1$. We claim that $A$ is well-defined, i.e., 
    the limit exists. Indeed, for any $m\in M$, we can find a sequence $\set{m_j}$ such that 
    $m_j\to m$ as $j\to\infty$. Then 
    \begin{equation*}
        \begin{split}
            \abs{x_{n_k}(m) - x_{n_l}(m)} &\leq \abs{x_{n_k}(m) - x_{n_k}(m_j)} + \abs{x_{n_k}(m_j) - x_{n_l}(m_j)} + \abs{x_{n_l}(m_j) - x_{n_l}(m)} \\
            &\leq \norm{x_{n_k}-x_{n_l}}\norm{m} + \abs{x_{n_k}(m_j) - x_{n_l}(m_j)} + \norm{x_{n_k}-x_{n_l}}\norm{m} \\
            &\leq 2\norm{x_{n_k}-x_{n_l}} + \abs{x_{n_k}(m_j) - x_{n_l}(m_j)} \to 0
        \end{split}
    \end{equation*}
    as $k,l\to\infty$. Hence the sequence $\set{x_{n_k}(m)}$ is Cauchy and $A$ 
    is well-defined. Because of the boundedness of $A$, it is continuous and thus
    $x_{n_k}(m)\to x(m)$ for some $x\in M'$. Since $\norm{x_{n_k}}\leq 1$, we have 
    $\norm{x}\leq 1$. Thus $x\in B$ and $x_{n_k}\wstarto x$ in $M'$. We conclude that 
    $B$ is weakly* sequentially compact.
\end{proof}

\begin{remark}
    Let $X$, $Y$ be Banach spaces and $B(X, Y)$ be the space of bounded linear 
    operators from $X$ to $Y$. Consider the following topologies on $B(X,Y)$:
    \begin{itemize}
        \item The uniform topology on $B(X,Y)$ is the topology induced by the 
        uniform norm: 
        \begin{equation*}
            \norm{m}_{B(X,Y)} = \sup_{\norm{x}_X\leq 1} \norm{m(x)}_Y.
        \end{equation*}
        This is the coarest among the three topologies.
        \item The strong topology on $B(X,Y)$ is the topology generated by the 
        collection of sets: 
        \begin{equation*}
            \Set{B_{x,\epsilon}(T) = \Set{S\in B(X,Y)}{\norm{Sx - Tx}<\epsilon}}{\epsilon>0, x\in X, T\in B(X,Y)}.
        \end{equation*}
        This is the coarest topology that makes the evaluation map $m\mapsto m(x)$ 
        continuous for all $x\in X$.
        \item The weak topology on $B(X,Y)$ is the topology generated by the collection 
        \begin{equation*}
            \Set{B_{y',x,\epsilon}(T) = \Set{S\in B(X,Y)}{\abs{y'Sx - y'Tx}<\epsilon}}{\epsilon>0, y'\in Y', x\in X, T\in B(X,Y)}.
        \end{equation*}
    \end{itemize}
\end{remark}

\begin{definition}
    Let $\set{M_n}\subset B(X,Y)$ be a sequence of bounded linear operators. $M_n$ 
    \textbf{converges strongly} if for any $x\in X$, $\norm{M_nx - y}_Y\to 0$ for some $y\in Y$.
\end{definition}

\begin{proposition}
    If $\set{M_n}\subset B(X,Y)$ converges strongly, then there is an $M\in B(X,Y)$ such that
    $\norm{M_nx - Mx}_Y\to 0$ for all $x\in X$. 
\end{proposition}
\begin{proof}
    Set $Mx = \lim_{n\to\infty} M_nx$ for all $x\in X$. We check that $M\in B(X,Y)$. 
    Linearity is trivial; we check the boundedness. Let $f_n(x) = \norm{M_nx}_Y$. 
    Then $f_n$ is sub-additive and $f_n(\alpha x) = \abs{\alpha}f_n(x)$ for all $\alpha\in\R$. 
    If $x_k\to x$, 
    \begin{equation*}
        f_n(x_k) = \norm{M_nx_k}_Y\to\norm{Mx_k}_Y = f_n(x)
    \end{equation*}
    for any fixed $n$ as $k\to\infty$; $f_n$ is continuous. Now for any fixed $x\in X$, 
    $\sup_n f_n(x) = \sup_n \norm{M_nx}\leq C(x)$ by the strong convergence. It follows 
    from the uniform boundedness principle that there is $C_0<\infty$ such that 
    $\abs{f_n(x)}\leq C_0\norm{x}_X$ for all $n$. Thus 
    \begin{equation*}
        \norm{Mx}_Y = \lim_{n\to\infty} \norm{M_nx}_Y \leq C_0\norm{x}_X.
    \end{equation*}
    Hence $\norm{M}\leq C_0$ and $M\in B(X,Y)$.
\end{proof}

\begin{definition}
    A sequence $\set{M_n}\subset B(X,Y)$ \textbf{converges weakly} if for all 
    $x\in X$, $M_nx\wto y\in Y$ for some $y$. 
\end{definition}

\begin{proposition}
    If $\set{M_n}\subset B(X,Y)$ converges weakly, then there is an $M\in B(X,Y)$ such that 
    $M_nx\wto Mx$ for all $x\in X$.
\end{proposition}
\begin{proof}
    Set $Mx = \lim_{n\to\infty} M_nx$ for all $x\in X$. We check that $M\in B(X,Y)$. 
    Linearity is trivial; we check the boundedness. Without loss of genrality, we can 
    assume that $\norm{x}_X = 1$. Observe that $\set{M_nx}\subset Y$ 
    is a weakly convergence sequence and hence weakly sequentially compact; by 
    \cref{prop:w_seq_cpt_bd}, it is bounded. Thus there exists $C<\infty$ such that
    $\norm{M_nx}_Y\leq C = C\norm{x}$ for all $n$. Taking the limit, we have 
    $\norm{Mx}_Y = \lim_{n\to\infty} \norm{M_nx}_Y\leq C\norm{x}_X$. Hence $M\in B(X,Y)$.
    
    Lastly, we check the weak convergence. Indeed, for any $\ell\in Y'$, 
    $\ell(M_nx) = \ell(Mx)$ for all $x\in X$ since $M_nx\to Mx$ in $Y$. 
    Thus $M_nx\wto Mx$ in $Y$. 
\end{proof}