\begin{definition}
    Let $\set{M_n}\subset B(X,Y)$ be a sequence of bounded linear operators. $M_n$ 
    \textbf{converges strongly} if for any $x\in X$, $\norm{M_nx - y}_Y\to 0$ for some $y\in Y$.
\end{definition}

\begin{proposition}
    If $\set{M_n}\subset B(X,Y)$ converges strongly, then there is an $M\in B(X,Y)$ such that
    $\norm{M_nx - Mx}_Y\to 0$ for all $x\in X$. 
\end{proposition}
\begin{proof}
    Set $Mx = \lim_{n\to\infty} M_nx$ for all $x\in X$. We check that $M\in B(X,Y)$. 
    Linearity is trivial; we check the boundedness. Let $f_n(x) = \norm{M_nx}_Y$. 
    Then $f_n$ is sub-additive and $f_n(\alpha x) = \abs{\alpha}f_n(x)$ for all $\alpha\in\R$. 
    If $x_k\to x$, 
    \begin{equation*}
        f_n(x_k) = \norm{M_nx_k}_Y\to\norm{Mx_k}_Y = f_n(x)
    \end{equation*}
    for any fixed $n$ as $k\to\infty$; $f_n$ is continuous. Now for any fixed $x\in X$, 
    $\sup_n f_n(x) = \sup_n \norm{M_nx}\leq C(x)$ by the strong convergence. It follows 
    from the uniform boundedness principle that there is $C_0<\infty$ such that 
    $\abs{f_n(x)}\leq C_0\norm{x}_X$ for all $n$. Thus 
    \begin{equation*}
        \norm{Mx}_Y = \lim_{n\to\infty} \norm{M_nx}_Y \leq C_0\norm{x}_X.
    \end{equation*}
    Hence $\norm{M}\leq C_0$ and $M\in B(X,Y)$.
\end{proof}

\begin{definition}
    A sequence $\set{M_n}\subset B(X,Y)$ \textbf{converges weakly} if for all 
    $x\in X$, $M_nx\wto y\in Y$ for some $y$. 
\end{definition}

\begin{proposition}
    If $\set{M_n}\subset B(X,Y)$ converges weakly, then there is an $M\in B(X,Y)$ such that 
    $M_nx\wto Mx$ for all $x\in X$.
\end{proposition}
\begin{proof}
    Set $Mx = \lim_{n\to\infty} M_nx$ for all $x\in X$. We check that $M\in B(X,Y)$. 
    Linearity is trivial; we check the boundedness. Without loss of genrality, we can 
    assume that $\norm{x}_X = 1$. Observe that $\set{M_nx}\subset Y$ 
    is a weakly convergence sequence and hence weakly sequentially compact; by 
    \cref{prop:w_seq_cpt_bd}, it is bounded. Thus there exists $C<\infty$ such that
    $\norm{M_nx}_Y\leq C = C\norm{x}$ for all $n$. Taking the limit, we have 
    $\norm{Mx}_Y = \lim_{n\to\infty} \norm{M_nx}_Y\leq C\norm{x}_X$. Hence $M\in B(X,Y)$.
    
    Lastly, we check the weak convergence. Indeed, for any $\ell\in Y'$, 
    $\ell(M_nx) = \ell(Mx)$ for all $x\in X$ since $M_nx\to Mx$ in $Y$. 
    Thus $M_nx\wto Mx$ in $Y$. 
\end{proof}

\begin{lemma}
    Let $X$ be a reflexive and $\set{T_n}\subset B(X,Y)$. Then 
    $T_n\wto T$ implies $T_n'\wto T'$.
\end{lemma}
\begin{proof}
    For all $x\in X$ and $\ell\in Y'$, $\ell T_nx\to\ell Tx$. We need 
    to show that $T_n'\ell\to T'\ell$ for all $\ell\in Y'$. Note that 
    $T_n'\ell\in X'$ and $X$ is reflexive, so we only need to check 
    $T_n'\ell(x)\to T'\ell(x)$ for all $x\in X$. But this is essentially 
    \begin{equation*}
        T_n'\ell(x) = \ell T_nx\to \ell Tx = T'\ell(x).
    \end{equation*}
\end{proof}
\begin{remark}
    The statement of the lemma fails if we replace weak convergence with 
    strong convergence, i.e., $T_n\to T$ does not imply $T_n'\to T'$ in general. 
    Consider $X = \ell^2(\N)$ and $T_n:\ell^2\to\ell^2$ be the operator 
    \begin{equation*}
        T_n(x_1,\ldots) = (x_n,0,\ldots)
    \end{equation*}
    for $x = (x_1,x_2,\ldots)\in\ell^2$. Since $X$ is a Hilbert space, 
    it is reflexive. Also, $T_n\to 0$ strongly, since 
    \begin{equation*}
        \norm{T_nx - 0}_2^2 = \abs{x_n}^2\to 0 
    \end{equation*}
    as $n\to\infty$ for all $x\in X$. However, $T_n':(\ell^2)'\to(\ell^2)'$ 
    is the operator defined by $\ell\mapsto \ell T_n$. For any 
    $\ell\in(\ell^2)'$, $\ell(x) = \inp{x}{y_\ell}$ for a unique $y_\ell\in\ell^2$. 
    Then $T_n'\ell(x) = \ell T_nx = \inp{y_\ell}{T_nx} = (y_\ell)_1\cdot x_n$. 
    Thus 
    \begin{equation*}
        T_n'\ell = (y_\ell)_1\cdot e_n.
    \end{equation*}
    Pick any $y_\ell\in\ell^2$ such that $(y_\ell)_1\neq 0$ will give 
    \begin{equation*}
        \norm{T_n'\ell - 0} = \abs{(y_\ell)_1} \not\to 0.
    \end{equation*}
\end{remark}

\begin{theorem}
    Let $T_n\in B(X,Y)$ be a sequence of bounded linear operators such that 
    \begin{thmenum}
        \item $\norm{T_n}\leq C<\infty$ for all $n$; 
        \item $T_nx\to Tx$ for all $x\in D\subset X$ where $D$ is a dense 
        subset of $X$. 
    \end{thmenum}
    Then $T_n\to T$ strongly.
\end{theorem}
\begin{proof}
    We claim that for any $z\in X$, the sequence $\set{T_nz}$ is Cauchy in $Y$. 
    Let $\epsilon>0$ be given. Since $D$ is dense in $X$, there exists 
    a $x\in D$ such that $\norm{z - x}_X<\frac{\epsilon}{3C}$. Then 
    \begin{equation*}
        \begin{split}
            \norm{T_nz - T_mz}_Y &\leq \norm{T_n(z - x)}_Y + \norm{T_m(z - x)}_Y + 
            \norm{T_nx - T_mx}_Y \\
            &\leq (\norm{T_n} + \norm{T_m})\cdot \norm{z - x}_X + \norm{T_nx - T_mx}_Y
            \leq 2C\cdot\frac{\epsilon}{3C} + \norm{T_nx - T_mx}_Y.
        \end{split}
    \end{equation*}
    Since $T_nx$ converges, it is Cauchy. Thus there exists $N$ such that for all 
    $n,m\geq N$, $\norm{T_nx - T_mx}_Y<\frac{\epsilon}{3}$. Hence 
    \begin{equation*}
        \norm{T_nz - T_mz}_Y < \frac{2\epsilon}{3} + \frac{\epsilon}{3} = \epsilon.
    \end{equation*}
    Thus $\set{T_nz}$ is Cauchy in $Y$. Since $Y$ is complete, there exists 
    $y_z\in Y$ such that $T_nz\to y_z$. Define $Tz = y_z$. We check that 
    $T\in B(X,Y)$. The linearity is trivial; for any $z\in X$, 
    \begin{equation*}
        \norm{Tz}_Y = \lim_{n\to\infty} \norm{T_nz}_Y\leq C\norm{z}_X.
    \end{equation*}
    Thus $\norm{T}\leq C$ and $T\in B(X,Y)$. Lastly, we check the 
    strong convergence. For any $z\in X$, 
    \begin{equation*}
        \norm{T_nz - Tz}_Y = \norm{T_nz - y_z}_Y\to 0
    \end{equation*}
    as $n\to\infty$. Thus $T_n\to T$ strongly.
\end{proof}

\begin{theorem}[Uniform Boundedness Principle \rom{3}]
    A family of operators $\set{T_\alpha}_{\alpha\in I}\subset B(X,Y)$ satisfies 
    that for all $x\in X$ and $\ell\in Y'$, 
    \begin{equation*}
        \abs{\ell(T_\alpha x)}\leq C(x,\ell) <\infty
    \end{equation*}
    for all $\alpha\in I$. Then there exists $C_0<\infty$ such that  
    \begin{equation*}
        \norm{T_\alpha}\leq C_0<\infty
    \end{equation*}
    for all $\alpha\in I$.
\end{theorem}
\begin{proof}
    Set 
    \begin{equation*}
        f_\alpha(x) = \norm{T_\alpha x}_Y = \sup_{\norm{\ell}=1} \abs{\ell(T_\alpha x)}.
    \end{equation*}
    We verify the conditions of the uniform boundedness principle. Let 
    $x_k\to x$ in $X$. Then $T_\alpha x_k\to T_\alpha x$ in $Y$. Thus 
    $f_\alpha(x_k)\to f_\alpha(x)$ given any $\alpha\in I$. Thus $f_\alpha$ is continuous.
    \begin{equation*}
        f_\alpha(x+y) = \norm{T_\alpha (x+y)}_Y\leq \norm{T_\alpha x}_Y + \norm{T_\alpha y}_Y = f_\alpha(x) + f_\alpha(y).
    \end{equation*}
    $f_\alpha$ is sub-additve. Also, $f_\alpha(cx) = \norm{T_\alpha (cx)}_Y = \abs{c}\norm{T_\alpha x}_Y = \abs{c}f_\alpha(x)$ for all $c\in\R$. 
    Lastly, given $x$, $g_{\alpha}(\ell) = \abs{\ell(T_\alpha x)}$ is clearly continuous, 
    sub-additive and homogeneous. Also, $\abs{g_{\alpha}(\ell)}\leq C(x,\ell)$. Using the 
    boundedness assumption and applying the uniform boundedness principle, we have that 
    \begin{equation*}
        \sup_{\alpha\in I} \abs{g_\alpha(\ell)} \leq C_1(x)\norm{\ell}.
    \end{equation*} 
    Now 
    \begin{equation*}
        \sup_{\alpha\in I} \abs{f_\alpha(x)} = \sup_{\alpha\in I} \sup_{\norm{\ell}=1} \abs{\ell(T_\alpha x)} 
        = \sup_{\norm{\ell}=1} \sup_{\alpha\in I} \abs{\ell(T_\alpha x)} 
        \leq \sup_{\norm{\ell}} \sup_{\alpha\in I} \abs{g_\alpha(\ell)} 
        \leq C_1(x).
    \end{equation*}
    Applying the uniform boundedness principle again on $f_\alpha$, we have that 
    \begin{equation*}
        \sup_{\alpha\in I} f_\alpha(x) \leq C_0\norm{x}_X.
    \end{equation*}
    We conclude that $\norm{T_\alpha}\leq C_0$ for all $\alpha\in I$.
\end{proof}

\begin{proposition}
    Let $T\in B(X,Y)$, $U\in B(Y,Z)$. Then $UT\in B(X,Z)$ and 
    $(UT)' = T'U'$.
\end{proposition}
\begin{proof}
    We first show that $UT\in B(X,Z)$. For any $c\in\R$ and $x,y\in X$, 
    \begin{equation*}
        UT(cx+y) = U(T(cx + y)) = U(cTx + Ty) = cUTx + UTy.
    \end{equation*}
    Thus $UT$ is linear. Now we check the boundedness. For any $x\in X$, 
    \begin{equation*}
        \norm{UTx}_Z \leq \norm{U}\norm{Tx}_Y \leq \norm{U}\norm{T}\norm{x}_X.
    \end{equation*}
    Since $\norm{U},\norm{T}$ are finite, the boundedness follows. 

    Now we check the adjoint. For any $\ell\in Z'$, $(UT)'(\ell) = \ell UT 
    = (U'\ell)T = T'U'\ell$. 
\end{proof} 

\begin{definition}
    $T\in B(X,Y)$ is \textbf{compact} if for any bounded sequence $\set{x_n}\subset X$, 
    $\set{Tx_n}$ has a convergent subsequence in $Y$.
\end{definition}

\begin{definition}
    The \textbf{compact operator space} is denoted by $B_0(X,Y)$. 
\end{definition}

\begin{proposition}
    Let $T\in B(X,Y)$ be a compact operator, $S_1\in B(Y,Z)$ and $S_2\in B(W,X)$. 
    Then $S_1T\in B_0(X,Z)$ and $TS_2\in B_0(W,Y)$.
\end{proposition}
\begin{proof}
    Let $\set{x_n}\subset X$ be a bounded sequence. Then 
    $\set{Tx_n}\subset Y$ has a convergent subsequence $\set{Tx_{n_k}}$. 
    Since $S_1$ is bounded, it is continuous; thus $\set{S_1Tx_{n_k}}$ is 
    convergent in $Z$. Hence $S_1T$ is compact. 

    Now let $\set{w_n}\subset W$ be a bounded sequence. Then 
    $\norm{S_2w_n}_X\leq\norm{S_2}\norm{w_n}_W$ is also bounded 
    in $X$. Thus by the compactness of $T$, $\set{TS_2w_n}$ has a 
    convergent subsequence. We conclude that $TS_2$ is compact.
\end{proof}

\begin{lemma}\label{lem:increasing_separable}
    Let $X$ be a metric space. If $A_n\subset X$ is a sequence of 
    separable subsets of $X$ and $A_n\nearrow A$, then $A$ is 
    separable.
\end{lemma}
\begin{proof}
    Since $A_n$ is separable, there exists a countable dense 
    subset $D_n\subset A_n$. Let $D = \bigcup_{n=1}^\infty D_n$.
    We claim that $D$ is dense in $A$. Let $x\in A$ be given. 
    Since $A_n\nearrow A$, there exists $n_0$ such that $x\in A_{n_0}$. 
    Then for any $\epsilon>0$, there exists $y\in D_{n_0}\subset D$ such 
    that $d(x,y)<\epsilon$. Thus $D$ is dense in $A$.
\end{proof}

\begin{theorem}\label{thm:cpt_op_separable}
    Let $T\in B_0(X,Y)$. Then $T(X)$ is separable.
\end{theorem}
\begin{proof}
    Consider the closed unit ball $B = \Set{x\in X}{\norm{x}_X\leq 1}$ 
    in $X$. Since $T$ is compact, $T(B)$ is sequentially compact. 
    Then $T(B)$ is compact in $Y$. Because every compact metric space 
    is separable, $T(B)$ is separable. Write $X = \bigcup_n nB$. 
    Then $T(X) = \bigcup_n T(nB) = \bigcup_n nT(B)$. By 
    \cref{lem:increasing_separable}, $T(X)$ is separable.
\end{proof}

\begin{theorem}
    Let $T\in B_0(X,Y)$ be a compact operator. Then $T'\in B_0(Y',X')$.
\end{theorem}
\begin{proof}
    Suppose first that $Y$ is separable. Let $g_n\in Y'$ be a bounded 
    sequence. $T(X)\subset Y$ is also separable. There exists a countable 
    dense subset $\set{y_k}\subset T(X)$. For $y_1$, $\set{g_n(y_1)}$ 
    is a bounded sequence in $\R$. By the Bolzano-Weierstrass theorem, 
    there exists a subsequence $\set{g^{(1)}_n}$ such that $g^{(1)}_n(y_1)$ 
    converges. For $y_2$, extract from $\set{g^{(1)}_n}$ to obtain a 
    subsequence $\set{g^{(2)}_n}$ such that $g^{(2)}_n(y_2)$ converges. 
    Continuing this process, we obtain a sequence $\set{g^{(k)}_n}$ such that 
    $g^{(k)}_n(y_j)$ converges for all $j\leq k$. Pick $f_n = g^{(n)}_n$. 
    Then for any $k$, $f_n(y_k)$ converges. Now given any $y\in Y$, we 
    may without loss of generality assume that $y_k\to y$. Then 
    \begin{equation*}
        \begin{split}
            \abs{f_n(y) - f_m(y)} &\leq \abs{f_n(y) - f_n(y_k)} + \abs{f_m(y) - f_m(y_k)} + \abs{f_n(y_k) - f_m(y_k)} \\
            &\leq (\norm{f_n} + \norm{f_m})\norm{y - y_k}_Y + \abs{f_n(y_k) - f_m(y_k)}.
        \end{split}
    \end{equation*}
    Since $f_n(y_k)$ converges for all $k$, it is Cauchy; 
    $\set{f_n}\subset\set{g_n}$ is bounded. Thus taking $m,n\to\infty$ 
    and then $k\to\infty$, we see that $\abs{f_n(y) - f_m(y)}\to 0$. 
    Hence $\set{f_n(y)}$ is Cauchy. 

    Next, we show that $f_n$ is in fact Cauchy in $Y'$. 
    \begin{equation*}
        \norm{f_n - f_m} = \sup_{\norm{y}_Y = 1} \abs{f_n(y) - f_m(y)} 
    \end{equation*}
    For each $m,n$, there exists $y\in Y$ such that $\norm{y}_Y = 1$ and 
    \begin{equation*}
        \abs{f_n(y) - f_m(y)} \geq \frac{1}{2}\norm{f_n - f_m}.
    \end{equation*}
    But $f_n(y)$ is Cauchy and thus $\norm{f_n - f_m}\to 0$ as $n,m\to\infty$.
    Thus $\set{f_n}\subset Y'$ is Cauchy. $Y'$ is complete, so there exists 
    $f\in Y'$ such that $f_n\to f$ in $Y'$. Now $f_n(Tx)\to f(Tx)$ for all 
    $x\in X$. Thus 
    \begin{equation*}
        \norm{T'f_n - T'f} \leq \norm{T'}\norm{f_n - f} \to 0.
    \end{equation*}
    Hence $T'f_n\to T'f$ in $X'$. $\set{T'f_n}$ is a convergent 
    subsequence of $\set{T'g_n}$. $T'$ is compact. 

    In general if $Y$ is not separable, $T(X)$ is a separable subspace 
    of $Y$ (\cref{thm:cpt_op_separable}). The same argument applies and 
    we obtain a sequence $\set{f_n}\subset Y'$ such that 
    \begin{equation*}
        \sup_{\norm{Tx}_Y = 1} \abs{f_n(Tx) - f_m(Tx)} \to 0.
    \end{equation*}
    Thus there exists $f$ on $T(X)$ such that 
    \begin{equation*}
        \sup_{\norm{Tx}_Y = 1} \abs{f_n(Tx) - f(Tx)} \to 0
    \end{equation*}
    by the completeness of $T(X)$. Then 
    \begin{equation*}
        \norm{T'f_n - T'f} = \sup_{\norm{Tx}_Y = 1} \abs{f_n(Tx) - f(Tx)} \to 0.
    \end{equation*}
    Thus $T'f_n\to T'f$ in $X'$. Hence $T'$ is compact.
\end{proof}

\begin{definition}
    Let $X,Y$ be Banach spaces. A linear operator $T$ is said 
    to be \textbf{densely defined} if $D(T) = \Set{x\in X}{Tx\in Y}$ 
    is dense in $X$. We denote it as $T:D(T)\dsubset X\to Y$.
\end{definition} 

\begin{definition}
    A linear operator $T:D(T)\dsubset X\to Y$ is said to be 
    \textbf{bounded} if there is $c < \infty$ such that 
    \begin{equation*}
        \norm{Tx}_Y\leq c\norm{x}_X
    \end{equation*}
    for all $x\in D(T)$; $T$ is \textbf{unbounded} if for all 
    $c>0$, there exists $x\in D(T)$ such that 
    \begin{equation*}
        \norm{Tx}_Y > c\norm{x}_X.
    \end{equation*}
\end{definition}
\begin{remark}
    $T$ is bounded if and only if $T$ is continuous on every 
    point in $D(T)$; $T$ is unbounded if and only if $T$ is not 
    continuous on every point in $D(T)$.
\end{remark}

\begin{definition}
    $T:D(T)\subset X\to Y$ is \textbf{closed} if its graph 
    \begin{equation*}
        G(T) = \Set{(x,y)\in X\times Y}{x\in D(T), Tx = y}
    \end{equation*}
    is closed in the norm $\norm{(x,y)}_{X\times Y} = \norm{x}_X + \norm{y}_Y$.
\end{definition}
\begin{remark}
    $T$ is closed if $x_n\to x$ in $X$, where $x_n\in D(T)$ and 
    $Tx_n\to y$ in $Y$ implies that $x\in D(T)$ and $Tx = y$.
\end{remark}

\begin{definition}
    $T_1:D(T_1)\subset X\to Y$ and $T_2:D(T_2)\subset X\to Y$ are 
    linear unbounded operators. We say that $T_2$ is an \textbf{extension} 
    of $T_1$ if $D(T_1)\subset D(T_2)$ and $T_2x = T_1x$ for all 
    $x\in D(T_1)$. Denote it as $T_1\subset T_2$.
\end{definition}

\begin{definition}
    A linear operator $T:D(T)\subset X\to Y$ is said to be 
    \textbf{closable} if there is a closed extension of $T$. 
\end{definition}
\begin{remark}
    There are three criteria for $T$ being closable. 
    \begin{thmenum}
        \item there is a closed extension of $T$; 
        \item $\overline{G(T)}\subset X\times Y$ is a graph of some operators; 
        \item for any $x_n\to 0$, $x_n\in D(T)$ and $Tx_n\to y\in Y$, we have 
        $y = 0$.
    \end{thmenum}
\end{remark}

\begin{definition}
    Let $T$ be a closable operator. The \textbf{closure} of $T$ 
    is the smallest closed extension of $T$, denoted by $\cl(T)$.
\end{definition}
\begin{remark}
    The closure of $T$ is well-defined. We can consider the 
    closure of the graph $G(T)\in X\times Y$. For $T$ being 
    closable, the closure of $G(T)$ is a graph of some operator 
    $T$. Since the closure of a graph is unique, the closure is 
    well-defined. 
\end{remark}

\begin{example}
    $f:D(T)\in\ell^2\to\R$, where $D(T) = \spanby(\Set{e_n}{n\in\N})$ 
    is defined by $Te_n = n$ and extended by linearity. Then 
    $T$ is unbounded. Since $T$ is unbounded, we may take 
    $x_n\to 0$ in $D(T)$ and $\abs{T(x_n)}\geq \epsilon$ for some 
    $\epsilon>0$. 
    \begin{equation*}
        z_n = \frac{x_n}{Tx_n}\to 0\quad\text{and}\quad 
        Tz_n = 1.
    \end{equation*}
    Thus $T$ is not closable.
\end{example}

\begin{example}
    Here is an example of a closable operator while not closed. 
    Let $X = \ell^2$ and $D(T) = c_{00} = \Set{x\in\ell^2}{x_n = 0\text{ for $n > $ some $N\in\N$}}$. 
    Then the $T:(x_n)\mapsto (n x_n)$ is a closable operator while not closed. 
    Let $x^k = (1,\ldots,\frac{1}{k^2},0,\ldots)\in D(T)$, 
    then $Tx^k = (1,\ldots,\frac{1}{k},0,\ldots)\to (1,1/2,1/3,\ldots)\in\ell^2$ as $k\to\infty$. 
    However, $x^k\to (1,1/4,1/9,\ldots)\notin D(T)$ as $k\to\infty$. 
    So $T$ is not closed. $T$ admits a closure $\cl(T):D(\cl(T))\to\ell^2$ 
    with $D(\cl(T)) = \Set{x\in\ell^2}{(n x_n)\in\ell^2}$ and 
    $\cl(T)x = (n x_n)$ for all $x\in D(\cl(T))$. 
\end{example}

\begin{definition}
    The \textbf{transpose} of a densely defined operator $T:D(T)\dsubset X\to Y$ is 
    $T': D(T')\dsubset Y'\to X'$ with 
    \begin{equation*}
        D(T') = \Set{m\in Y'}{\text{ there is $\ell\in X'$ such that $m(Tx) = \ell(x)$ for all $x\in X$}}.
    \end{equation*}
\end{definition}
\begin{remark}
    The transpose is well-defined since $T$ is densely defined. If $\ell_1,\ell_2$ are 
    two candidates in $X'$ such that $m(Tx) = \ell_1(x) = \ell_2(x)$. Being densely 
    defined implies that $\ell_1 = \ell_2$ on $X$. 
\end{remark}


