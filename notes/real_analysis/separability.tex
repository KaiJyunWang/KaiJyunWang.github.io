\begin{definition}
    A Hilbert space $\H$ is said to be \textbf{separable} if there 
    exists a countable dense subset in $\H$.
\end{definition}

\begin{definition}
    $\Set{x_\alpha}{\alpha\in A}\subset\H$, where $A$ is an arbitrary 
    index set. The \textbf{linear span} of $\Set{x_\alpha}{\alpha\in A}$ 
    is defined as 
    \begin{equation*}
        \spanby\set{x_\alpha}
        = \Set{\sum_{\alpha\in A}c_\alpha x_\alpha}{c_\alpha\in\mathbb{F} = \R \text{ or } \C},
    \end{equation*}
    where the sum is a finite sum.
\end{definition}

\begin{definition}
    $\Set{x_\alpha}{\alpha\in A}\subset\H$, where $A$ is an arbitrary 
    index set. The \textbf{closed linear span} of $\Set{x_\alpha}{\alpha\in A}$ 
    is defined as the smallest closed subspace of $\H$ containing 
    $\Set{x_\alpha}{\alpha\in A}$.
\end{definition}

\begin{proposition}
    Let $Y = \overline{\spanby(\set{x_\alpha})}\subset\H$ be a 
    closed linear span of $\set{x_\alpha}$. Then for any $x\in\H$, 
    $\inp{x}{x_\alpha} = 0$ for all $\alpha\in A$ if and only if 
    $\inp{x}{z} = 0$ for all $z\in Y$.
\end{proposition}
\begin{proof}
    Assume first that for $x\in\H$, $\inp{x}{x_\alpha} = 0$ for all 
    $\alpha\in A$. For each $z\in Y$, write $z = \sum_{\alpha_j\in A}c_{\alpha_j} x_{\alpha_j}$.  
    Then
    \begin{equation*}
        \inp{x}{z} = \lim_{M\to\infty}\inp{x}{\sum_{j = 1}^{M} c_{\alpha_j}x_{\alpha_j}} 
        = \lim_{M\to\infty}\sum_{j = 1}^{M} \overline{c_{\alpha_j}}\inp{x}{x_{\alpha_j}} = 0.
    \end{equation*}
    The converse is trivial since $x_\alpha\in Y$ for all $\alpha\in A$.
\end{proof}

\begin{definition}
    $\Set{x_\alpha}{\alpha\in A}$ is said to be \textbf{orthonormal} 
    if $\inp{x_\alpha}{x_\beta} = \delta_{\alpha\beta}$ for all 
    $\alpha,\beta\in A$.
\end{definition}

\begin{definition}
    $\Set{x_\alpha}{\alpha\in A}$ forms a \textbf{othonormal basis} 
    of $\H$ if it is orthonormal and $\overline{\spanby(\set{x_\alpha})} = \H$. 
\end{definition}
\begin{remark}
    This definition of basis is different from the definition 
    of basis in linear algebra. In linear algebra, one can only 
    express a vector as a finite linear combination of basis vectors; 
    however, in Hilbert space, one can express a vector as a countable 
    linear combination of basis vectors.
\end{remark}

\begin{lemma}[Bessel's Inequality]\label{lem:bessel}
    Let $\Set{x_\alpha}{\alpha\in A}$ be an orthonormal set in $\H$. 
    For any $x\in\H$, let $c_\alpha = \inp{x}{x_\alpha}$. Then 
    \begin{thmenum}
        \item The set $\Set{\alpha}{c_\alpha\neq 0}$ is at most countable. 
        \item $\sum_\alpha \abs{c_\alpha}^2 \leq \norm{x}^2$.
    \end{thmenum}
\end{lemma}
\begin{proof}
    We assume that (a) is established and prove (b) first. 
    Let $J\subset A$ be a countable subset with $J = \Set{\alpha_k}{k\in\N}$. 
    For each $M\in\N$, 
    \begin{equation*}
        \begin{split}
            0 &\leq \norm{\sum_{k=1}^{M} c_{\alpha_k}x_{\alpha_k} - x}^2 
            = \sum_{k=1}^{M}\abs{c_{\alpha_k}}^2 - 2\Re\pth{\inp{x}{\sum_{k=1}^{M}c_{\alpha_k}x_{\alpha_k}}} + \norm{x}^2 \\
            &= \sum_{k=1}^{M}\abs{c_{\alpha_k}}^2 - 2\Re\pth{\sum_{k=1}^{M}\overline{c_{\alpha_k}}\inp{x}{x_{\alpha_k}}} + \norm{x}^2 
            = \sum_{k=1}^{M}\abs{c_{\alpha_k}}^2 - 2\sum_{k=1}^{M}\abs{c_{\alpha_k}}^2 + \norm{x}^2 \\
            &= \norm{x}^2 - \sum_{k=1}^{M}\abs{c_{\alpha_k}}^2 \implies \sum_{k=1}^{M}\abs{c_{\alpha_k}}^2 \leq \norm{x}^2.
        \end{split}
    \end{equation*}
    Taking $M\to\infty$, we have $\sum_{k=1}^{\infty}\abs{c_{\alpha_k}}^2 \leq \norm{x}^2$. 

    Now we turn back to establish (a). For $m\in\N$, let $J_m = \Set{\alpha\in A}{\abs{c_\alpha} \geq 1/m}$. 
    Then $J_m$ is finite or we can find infinitely many $\alpha\in J_m$ such that $\abs{c_\alpha} \geq 1/m$. 
    This implies that $x\in\H$ and 
    \begin{equation*}
        \norm{x}^2 = \sum_{\alpha\in A}\abs{c_\alpha}^2 \geq \sum_{\alpha\in J_m}\abs{c_\alpha}^2 \geq \sum_{\alpha\in J_m} \frac{1}{m^2} = \infty,
    \end{equation*}
    which is absurd. Thus $J_m$ is finite for all $m\in\N$. 
    Observe that $\bigcup_{m\in\N} J_m = \Set{\alpha\in A}{c_\alpha\neq 0}$. 
    It follows that as a countable union of finite sets, $\Set{\alpha\in A}{c_\alpha\neq 0}$ 
    is at most countable. (b) follows from (a) and the previous argument.
\end{proof}
\begin{remark}
    There is a non-separable Hilbert space. Consider an uncountable 
    set $S$. Let 
    \begin{equation*}
        \H = \Set{f:S\to\R}{\sum_{s\in S}f(s)^2 < \infty, f(S)\setminus\set{0}\text{ is at most countable}}.
    \end{equation*}
    Then $\H$ is a Hilbert space with the inner product 
    \begin{equation*}
        \inp{f}{g} = \sum_{s\in S}f(s)g(s).
    \end{equation*}
    To see that $\H$ is a Hilbert space, note that the countable 
    union of countably many non-zero points is countable. Also, 
    it is not separable since the set 
    \begin{equation*}
        \Set{e_s:S\to\R}{e_s(t) = \delta_{st}}
    \end{equation*}
    forms an orthonormal set in $\H$ and for each $s\neq r$,
    $\norm{e_s - e_r} = \sqrt{2}$. This shows that it is nowhere 
    dense in $\H$. Thus $\H$ is not separable.
\end{remark}

\begin{proposition}
    Let $\set{x_\alpha}$ be an orthonormal set and $Y$ be the closed linear span of 
    $\set{x_\alpha}$. Then 
    \begin{equation*}
        Y = \Set{\sum_j c_jx_{\alpha_j}}{\sum_j \abs{c_j}^2 < \infty, \alpha_j\in A}.
    \end{equation*}
\end{proposition}
\begin{proof}
    Let $S = \Set{\sum_j c_jx_{\alpha_j}}{\sum_j \abs{c_j}^2 < \infty, \alpha_j\in A}$. 
    For $x\in S$, $x = \sum_j c_jx_{\alpha_j}$ with $\sum_j \abs{c_j}^2 < \infty$. 
    Then $z_n = \sum_{j=1}^{n} c_jx_{\alpha_j}\to x$ as $n\to\infty$. 
    Each $z_n\in Y$ and thus $x\in Y$. Hence $S\subset Y$. 

    Conversely, we claim that $S$ is a closed subspace of $\H$. 
    Clearly $0\in S$. For $c\in\mathbb{F}$, $x = \sum_j c_jx_{\alpha_j}\in S$ 
    and $y = \sum_j d_jx_{\alpha_j}\in S$, we have 
    \begin{equation*}
        cx + y = c\sum_j c_jx_{\alpha_j} + \sum_j d_jx_{\alpha_j} = \sum_j (cc_j + d_j)x_{\alpha_j}\in S,
    \end{equation*}
    where the summation is over all $j$ such that either $c_j\neq 0$ or $d_j\neq 0$.
    To see that $S$ is closed, let $z_n\in S$ where $z_n = \sum_j c_j^nx_{\alpha_j^n}$ 
    with $\sum_j \abs{c_j^n}^2 < \infty$ for all $n\in\N$. Let 
    $J_n = \Set{\alpha_j^n}{j\in\N}$ and $J = \bigcup_n J_n\subset A$ 
    is at most countable. Consider the transformation $T:S\to\ell^2$ 
    defined by $\sum_j c_jx_{\alpha_j}\mapsto \set{c_j}$. Such 
    definition is well-defined since if $\sum_j c_jx_{\alpha_j} = \sum_j d_jx_{\alpha_j}$, 
    then $\sum_j (c_j - d_j)x_{\alpha_j} = 0$ and thus $c_j = d_j$ for all $j$ since 
    every $x_{\alpha_j}$ is orthogonal and thus linearly independent. 
    Furthermore, $T$ is clearly linear. Also, it is isometric 
    since 
    \begin{equation*}
        \norm{\sum_j c_jx_{\alpha_j}}^2 = \sum_j \abs{c_j}^2 = \norm{\set{c_j}}^2.
    \end{equation*}
    For $z_n\in S$, $z_n\to z$. Since $z_n$ is Cauchy and $T$ is 
    isometric, $\set{c_j^n}$ is Cauchy in $\ell^2$ and thus converges 
    to some $\set{c_j}\in\ell^2$. Define $w = \sum_j c_jx_{\alpha_j}\in S$. 
    It follows that $Tz_n\to Tw$. Hence $z = w\in S$ by the isometry 
    of $T$. Thus $S$ is closed. It follows that by the definition 
    of $Y$, $Y\subset S$. We conclude that $Y = S$.
\end{proof}

\begin{lemma}[Gram-Schmidt]\label{lem:gram_schmidt}
    Suppose $\set{x_\alpha}$ is an orthonormal set in $\H$ with 
    $\overline{\spanby(\set{x_\alpha})}\neq\H$. Then there exists 
    $y\in\H$ such that $\set{x_\alpha}\bigcup\set{y}$ is orthonormal.
\end{lemma}
\begin{proof}
    Pick $z\in\H$ such that $z\notin\overline{\spanby(\set{x_\alpha})}$. 
    By \cref{lem:bessel}, there are at most countably many $\alpha$ 
    such that $\inp{z}{x_\alpha}\neq 0$. Let $\alpha_j$ denumerate all 
    $\alpha$ such that $\inp{z}{x_\alpha}\neq 0$. Set $\hat{z} 
    = \sum_j \inp{z}{x_{\alpha_j}}x_{\alpha_j}$. For each $x_{\alpha_k}$, 
    \begin{equation*}
        \begin{split}
            \inp{z-\hat{z}}{x_{\alpha_k}} 
            &= \lim_{m\to\infty} \inp{z-\sum_{j=1}^{m}\inp{z}{x_{\alpha_j}}x_{\alpha_j}}{x_{\alpha_k}} \\
            &= \lim_{m\to\infty} \inp{z}{x_{\alpha_k}} - \sum_{j=1}^{m} \inp{z}{x_{\alpha_j}}\delta_{jk} \\
            &= \inp{z}{x_{\alpha_k}} - \inp{z}{x_{\alpha_k}} = 0.
        \end{split}
    \end{equation*}
    And for those $x_\alpha$ such that $\inp{z}{x_\alpha} = 0$, 
    $\inp{z-\hat{z}}{x_\alpha} = \inp{z}{x_\alpha} - \inp{\hat{z}}{x_\alpha} = 0$ 
    since $x_{\alpha_j}$ and $x_\alpha$ are orthogonal. Now set 
    $y = (z - \hat{z})/\norm{z - \hat{z}}$. Then $\set{x_\alpha}\bigcup\set{y}$ 
    forms a orthonormal set.
\end{proof}

\begin{theorem}\label{thm:orthonormal_basis_exist}
    Every Hilbert space has an orthonormal basis.
\end{theorem}
\begin{proof}
    We plan to use Zorn's lemma. Denote the space consisting of all orthonormal 
    sets in $\H$ by $\mathcal{O}$. Define a partial order as the inclusion of 
    sets. Let $\mathcal{C}\subset\mathcal{O}$ be a chain. We claim that 
    $B = \bigcup_{\set{x_\alpha}\in\mathcal{C}}\set{x_\alpha}$ is an upper bound 
    of $\mathcal{C}$. By construction we have $\set{x_\alpha}\subset B$ for all 
    $\set{x_\alpha}\in\mathcal{C}$. We need to show that $B\in\mathcal{O}$. For 
    distinct $x_\alpha$, $x_\beta\in B$, they belong to a common set 
    $C\in\mathcal{C}\subset\mathcal{O}$. Hence $C$ is orthonormal and 
    $\inp{x_\alpha}{x_\beta} = 0$. Also, it is clear that for every $x_\alpha\in B$, 
    $x_\alpha$ belongs to some $C\in\mathcal{C}$ and thus $\norm{x_\alpha} = 1$. 
    It follows that $B$ is also orthonormal. By Zorn's lemma, there exists a 
    maximal element in $\mathcal{O}$, say $B$, such that if $C\in\mathcal{O}$ 
    and $B\subset C$, then $B = C$. We claim that $B$ is an orthonormal basis. 
    It suffices to check that $\overline{\spanby(B)} = \H$. Suppose not, then 
    by \cref{lem:gram_schmidt}, there exists $y\in\H$ such that $\set{x_\alpha}\bigcup\set{y}$ 
    forms an orthonormal set. This contradicts the maximality of $B$. We 
    conclude that $B$ is an orthonormal basis.
\end{proof}

\begin{theorem}\label{thm:separable_countable_basis}
    Let $\mathbb{F} = \R$ or $\C$. Suppose that $\H$ is a Hilbert space. Then $\H$ 
    is separable if and only if $\H$ has a countable orthonormal basis.
\end{theorem}
\begin{proof}
    Suppose that $\H$ has a countable orthonormal basis $\set{x_n}$. Then consider 
    the sets
    \begin{equation*}
        A_n = \Set{\sum_{j=1}^n c_jx_j}{c_j\in S},
    \end{equation*}
    where $S = \Q$ if $\mathbb{F} = \R$ and $S = \Q + \Q i$ if $\mathbb{F} = \C$. 
    Since $S$ is countable, each $A_n$ being a finite union of countable sets is 
    countable. Put $A = \bigcup_n A_n$ and let $\epsilon>0$ be given. Since $A$ is 
    a countable union of countable sets, it is also countable. For every $x\in\H$, 
    we can write $x = \sum_j \inp{x}{x_j}x_j$ with 
    \begin{equation*}
        \norm{\sum_{j=N+1}^{\infty}\inp{x}{x_j}x_j} < \frac{\epsilon}{2}
    \end{equation*}
    for some $N\in\N$. Since $S$ is dense in $\mathbb{F}$, we can pick some 
    $c_j\in S$ with $\abs{c_j - \inp{x}{x_j}} < \epsilon/2^{j+1}$. Then 
    \begin{equation*}
        \begin{split}
            \norm{x - \sum_{j=1}^N c_jx_j} 
            &= \norm{\sum_{j=1}^{\infty}\inp{x}{x_j}x_j - \sum_{j=1}^{N} \inp{x}{x_j}x_j + \sum_{j=1}^{N} \inp{x}{x_j}x_j - \sum_{j=1}^N c_jx_j} \\
            &\leq \norm{\sum_{j=1}^{\infty}\inp{x}{x_j}x_j - \sum_{j=1}^{N} \inp{x}{x_j}x_j} + \norm{\sum_{j=1}^{N} \inp{x}{x_j}x_j - \sum_{j=1}^N c_jx_j} \\ 
            &\leq \norm{\sum_{j=N+1}^{\infty}\inp{x}{x_j}x_j} + \sum_{j=1}^{N}\abs{\inp{x}{x_j} - c_j}\norm{x_j} 
            \leq \frac{\epsilon}{2} + \sum_{j=1}^{N}\frac{\epsilon}{2^{j+1}} \leq \epsilon.
        \end{split}
    \end{equation*}
    It follows that $A$ is dense in $\H$ and hence $\H$ is separable. 
    
    Conversely, suppose that $\H$ is separable. Let $S\subset\H$ be a countable 
    subset. Assume that every orthonormal basis of $\H$ is uncountable. Denote an 
    orthonormal basis of $\H$ by $\set{x_\alpha}$. For each distinct $x_\alpha$ 
    $x_\beta\in S$, $\norm{x_\alpha - x_\beta} = \sqrt{2}$. Consider the open 
    balls $B_{1/2}(x_\alpha)$. They are clearly disjoint since if $y$ lies in 
    two such balls, then $\sqrt{2} = \norm{x_\alpha - x_\beta}\leq 
    \norm{x_\alpha - y} + \norm{y - x_\beta} < 1$, which is absurd. Now since $S$ 
    is dense in $\H$, for each $\alpha$ we can find some $s_\alpha\in S$ such that 
    $s_\alpha\in B_{1/2}(x_\alpha)$. It follows that each $s_\alpha$ is distinct 
    and thus $S$ is uncountable. This contradicts to our assumption that $S$ is 
    countable. Thus $\H$ must have a countable orthonormal basis.
\end{proof}

\begin{proposition}
    Let $\H$ be a Hilbert space and $\Set{x_\alpha}{\alpha\in A},\Set{y_\beta}{\beta\in B}$ 
    be two orthonormal bases in $\H$. Then $\card(A) = \card(B)$.
\end{proposition}
\begin{proof}
    Fixed an $\alpha\in A$, $B_\alpha = \Set{\beta\in B}{\inp{y_\beta}{x_\alpha}\neq 0}$ 
    is at most countable by \cref{lem:bessel} and $B_\alpha\subset B$. We 
    claim that $B\subset\bigcup_{\alpha\in A}B_\alpha$. Take $\beta\in B$, 
    we can write $y_\beta = \sum_k \inp{y_\beta}{x_{\alpha_k}}x_{\alpha_k}$ with 
    at least one $\inp{y_\beta}{x_{\alpha_k}} \neq 0$. Hence $\beta\in B_{\alpha_k}$ 
    for some $\alpha_k\in A$. It follows that $B\subset\bigcup_{\alpha\in A}B_\alpha$ 
    and hence $\card(B)\leq \card(A)$. By symmetry, we have $\card(A)\leq \card(B)$ 
    and thus $\card(A) = \card(B)$.
\end{proof}
\begin{remark}
    If $\H$ is separable, then $\H$ has a countable orthonormal basis and hence 
    every orthonormal basis of $\H$ is countable. 
\end{remark}

\begin{proposition}[Parseval's Identity]
    Let $\set{x_\alpha}$ be an orthonormal basis of $\H$. Then 
    \begin{equation*}
        \norm{x}^2 = \sum_j\abs{\inp{x}{x_{\alpha_j}}}^2.
    \end{equation*}
\end{proposition}
\begin{proof}
    Let $x\in\H$. Write $x = \sum_j c_jx_{\alpha_j}$ with $\sum_j \abs{c_j}^2 < \infty$. 
    Then 
    \begin{equation*}
        \inp{x}{x_{\alpha_k}} = \lim_{M\to\infty}\inp{\sum_{j=1}^{M}c_jx_{\alpha_j}}{x_{\alpha_k}} 
        = \lim_{M\to\infty}\sum_{j=1}^{M}c_j\inp{x_{\alpha_j}}{x_{\alpha_k}} = c_k.
    \end{equation*}
    It follows that 
    \begin{equation*}
        \norm{x}^2 = \lim_{M\to\infty}\inp{\sum_{j=1}^{M} c_jx_{\alpha_j}}{\sum_{j=1}^{M} c_jx_{\alpha_j}}
        = \lim_{M\to\infty}\sum_{j=1}^{M}\abs{c_j}^2 = \sum_j\abs{c_j}^2 = \sum_j\abs{\inp{x}{x_{\alpha_j}}}^2.
    \end{equation*}
\end{proof}