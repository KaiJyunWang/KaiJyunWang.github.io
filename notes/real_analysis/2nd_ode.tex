The goal of this section is to present some solution techniques for solving 
the second order ODEs that will be intensively used in the next section. 
The techniques are presnted without proofs. 

We first introduce the variation of parameters method. Consider the 
second order ODE of the form 
\begin{equation*}
    y'' + p(x)y' + q(x)y = f(x),
\end{equation*}
where $a\leq x \leq b$. The first step is to find the solutions of the homogeneous version 
\begin{equation*}
    y'' + p(x)y' + q(x)y = 0.
\end{equation*}
Assume that we can find two linearly independent solutions $y_1(x)$ and $y_2(x)$. 
Then a particular solution of the non-homogeneous equation is 
\begin{equation*}
    y_p(x) = -y_1(x)\int_a^x \frac{y_2(t)f(t)}{W(y_1,y_2)(t)}dt 
    + y_2(x)\int_a^x \frac{y_1(t)f(t)}{W(y_1,y_2)(t)}dt,
\end{equation*}
where the \textbf{Wronskian} $W$ is defined as 
\begin{equation*}
    W(y_1,y_2)(t) = \begin{vmatrix}
        y_1(t) & y_2(t) \\
        y_1'(t) & y_2'(t)
    \end{vmatrix}.
\end{equation*}
The general solution of the non-homogeneous equation is 
\begin{equation*}
    y(x) = c_1y_1(x) + c_2y_2(x) + y_p(x),
\end{equation*}
for some constants $c_1$ and $c_2$ that should be determined by the boundary 
conditions. 

The difficulty of the variation of parameters method is that it is not always easy to find 
two linearly independent solutions of the homogeneous equation. If the coefficients 
are actually constant, we can consider the corresponding characteristic polynomial 
\begin{equation*}
    \lambda^2 + p\lambda + q = 0.
\end{equation*}
If the roots $\lambda_1$ and $\lambda_2$ are distinct, then the two linearly independent 
can be found as 
\begin{equation*}
    y_1(x) = e^{\lambda_1 x}, \quad y_2(x) = e^{\lambda_2 x}.
\end{equation*}
If $\lambda = \lambda_1 = \lambda_2$, the two linearly independent solutions can 
be found as 
\begin{equation*}
    y_1(x) = e^{\lambda x}, \quad y_2(x) = xe^{\lambda x}.
\end{equation*}

The next method is using the Green's function. Consider the second order ODE 
\begin{equation*}
    y'' + p(x)y' + q(x)y = f(x).
\end{equation*}
The differential operator $L$ is defined as 
\begin{equation*}
    L = D^2 + p(x)D + q(x)I,
\end{equation*}
with boundary conditions 
\begin{equation*}
    Ry = 0,
\end{equation*} 
where $D$ is the differential operator and $R$ is a linear operator that 
represents the boundary conditions. Suppose that the solution has the form 
\begin{equation*}
    y(x) = \int_a^b G(x,t)f(t)dt,
\end{equation*}
where $G(x,t)$ is the \textbf{Green's function} characterized by the following 
differential equation 
\begin{equation*}
    \begin{cases}
        L G(x,t) = \delta(x-t), & x\in[a,b], \\ 
        R G(x,t) = 0, \\ 
        G(t^+,t) = G(t^-,t), \\
        G_x(t^+,t) - G_x(t^-,t) = \frac{1}{r(t)},
    \end{cases}
\end{equation*}
where $r$ is the function with 
\begin{equation*}
    L = D(r(x)D) + s(x)
\end{equation*}
being the result of factorization. This form is called the \textbf{Sturm-Liouville form} 
operator. The form of $r$ and $s$ can be found by the following 
\begin{equation*}
    \sbrc{D(rD) + s}y = ry'' + r'y' + sy = 0 \quad\Leftrightarrow\quad 
    y'' + \frac{r'}{r}y' + \frac{s}{r}y = 0.
\end{equation*}
This means that 
\begin{equation*}
    p = \frac{r'}{r}, \quad q = \frac{s}{r}.
\end{equation*}
So 
\begin{equation*}
    r(x) = e^{\int_a^x p(t)dt}.
\end{equation*}
We can rewrite the characterization of the Green's function as 
\begin{equation*}
    \begin{cases}
        L G(x,t) = 0, & x\in[a,b], \\ 
        R G(x,t) = 0, \\ 
        G(t^+,t) = G(t^-,t), \\
        G_x(t^+,t) - G_x(t^-,t) = \exp\pth{-\int_a^t p(s)ds}.
    \end{cases}
\end{equation*}

