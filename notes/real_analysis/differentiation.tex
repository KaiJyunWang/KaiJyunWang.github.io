\begin{definition}
    Let $f\in\L^1(\R^d)$. The \textbf{Hardy-Littlewood maximal function} 
    is defined as 
    \begin{equation*}
        f^*(x) = \sup_{B:x\in B} \frac{1}{\mu(B)}\int_B \abs{f(y)}dy,
    \end{equation*}
    where the supremum is taken over all balls containing $x$.
\end{definition}
\begin{proposition}
    $f^*$ is measurable. 
\end{proposition}
\begin{proof}
    Let $E_\alpha = \Set{x}{f^*(x)>\alpha}$. We claim that it is 
    an open set. Indeed, if $p\in E_\alpha$, there exists a ball 
    $B$ containing $p$ such that 
    \begin{equation*}
        \frac{1}{\mu(B)}\int_B \abs{f(y)}dy > \alpha.
    \end{equation*}
    Now any $x$ close enough to $p$ will be contained in $B$ and 
    hence in $E_\alpha$. Thus $E_\alpha$ is open. Hence $f^*$ is
    measurable.
\end{proof}

\begin{lemma}\label{prop:vitali_cover}[Vitali Covering Lemma]
    Suppose $\set{B_1,\ldots,B_N}$ is a finite collection of open 
    balls in $\R^d$. Then there exists a disjoint subcollection 
    $\set{B_{i_1},\ldots,B_{i_k}}$ such that
    \begin{equation*}
        \mu\pth{\bigcup_{j=1}^N B_j} \leq 3^d\sum_{j=1}^k \mu(B_{i_j}).
    \end{equation*}
\end{lemma}
\begin{proof}
    First we make an observation that if $B$ and $B'$ are balls 
    intersecting with, say, the radius of $B$ is greater than 
    the radius of $B'$, then $B'$ is contained in the ball $\tilde{B}$ 
    that is concentric with $B$ but with $3$ times the radius. 

    The construction of the subcollection is proceeded as follows. 
    First, pick a ball $B_{i_1}$ with the largest radius. Then 
    remove all balls intersecting with $\tilde{B}_{i_1}$, the ball 
    concentric with $B_{i_1}$ but with $3$ times the radius. 
    Among the remaining balls, we repeat the process and pick $B_{i_2}$. 
    The process terminates when no more balls can be picked, 
    after at most $N$ steps and we obtain a disjoint subcollection 
    of balls $\set{B_{i_1},\ldots,B_{i_k}}$. 

    Lastly, we verify the inequality. By the construction, 
    we know that $\cup_{j=1}^N B_j\subset \cup_{j=1}^k\tilde{B}_{i_j}$
    and thus
    \begin{equation*}
        \mu\pth{\bigcup_{j=1}^N B_j} \leq \mu\pth{\bigcup_{j=1}^k\tilde{B}_{i_j}} 
        \leq \sum_{j=1}^{k} \mu(\tilde{B}_{i_j})
        = \sum_{j=1}^{k} 3^d\mu(B_{i_j}).
    \end{equation*}
\end{proof}

\begin{theorem}[Weak-Type Inequality]
    Let $f\in\L^1(\R^d)$. Then for all $\alpha>0$, 
    \begin{equation*}
        \mu\pth{\Set{x\in\R^d}{f^*(x)>\alpha}} 
        \leq \frac{A}{\alpha}\norm{f}_{\L^1(\R^d)},
    \end{equation*}
    where $A=3^d$.
\end{theorem}
\begin{proof}
    Let $E_\alpha = \Set{x}{f^*(x)>\alpha}$. For each $x\in E_\alpha$ 
    there exists a ball $B_x$ containing $x$ such that 
    \begin{equation*}
        \frac{1}{\mu(B_x)}\int_{B_x}\abs{f(y)}dy > \alpha
        \quad\Rightarrow\quad 
        \mu(B_x) < \frac{1}{\alpha}\int_{B_x}\abs{f(y)}dy.
    \end{equation*}
    Now for any fixed compact $K\subset E_\alpha$, $K$ is 
    covered by $\bigcup_{x\in E_\alpha} B_x$, and hence there 
    exists a finite subcover $\set{B_1,\ldots,B_N}$ of $K$. 
    By the Vitali covering lemma, there exists a disjoint 
    subcollection $\set{B_{i_1},\ldots,B_{i_k}}$ with 
    \begin{equation*}
        \mu\pth{\bigcup_{j=1}^N B_j} \leq 3^d\sum_{j=1}^k \mu(B_{i_j}).
    \end{equation*} 
    As a result,
    \begin{equation*}
        \begin{split}
            \mu(K) &\leq \mu\pth{\bigcup_{j=1}^N B_j} 
            \leq 3^d\sum_{j=1}^k \mu(B_{i_j})
            \leq \frac{3^d}{\alpha}\sum_{j=1}^{k}\int_{B_{i_j}}\abs{f(y)}dy \\
            &\leq \frac{3^d}{\alpha}\int_{\cup_{j=1}^kB_{i_j}}\abs{f(y)}dy
            \leq \frac{3^d}{\alpha}\int_{\R^d}\abs{f(y)}dy.
        \end{split}
    \end{equation*}
    Since the inequality holds for all compact subset $K$ of 
    $E_\alpha$, the proof is complete.
\end{proof}
\begin{remark}
    Note that $\Set{x}{f^*(x)=\infty}\subset\Set{x}{f^*(x)>\alpha}$ for 
    every $\alpha>0$. Taking $\alpha\to\infty$ yields 
    \begin{equation*}
        \mu\pth{\Set{x}{f^*(x)=\infty}}=0.
    \end{equation*}
    Hence $f^*(x)<\infty$ a.e.
\end{remark}

\begin{theorem}[Lebesgue Differentiation Theorem]
    Let $f\in\L^1(\R^d)$. Then for almost every $x\in\R^d$, 
    \begin{equation*}
        \lim_{m(B)\to 0, x\in B} \frac{1}{m(B)}\int_B f(y)dy = f(x).
    \end{equation*}
\end{theorem}
\begin{proof}
    Since continuous functions are dense in $\L^1$, we may find 
    a continuous $g$ such that $\norm{f-g}_{\L^1}<\epsilon$. For 
    such $g$, by the continuity, there exists a ball such that 
    $\abs{g(y)-g(x)}<\epsilon$ for all $x,y\in B$. Thus 
    \begin{equation*}
        \begin{split}
            \abs{\frac{1}{m(B)}\int_B f(y)dy - f(x)} 
            &= \abs{\frac{1}{m(B)}\int_B (f(y)-g(y))dy + \frac{1}{m(B)}\int_B g(y)- g(x)dy + g(x) - f(x)} \\ 
            &\leq \frac{1}{m(B)}\int_B \abs{(f(y)-g(y))}dy + \frac{1}{m(B)}\int_B \abs{g(y)- g(x)}dy + \abs{g(x) - f(x)} \\ 
            &\leq (f-g)^*(x) + \epsilon + \abs{g(x) - f(x)}.
        \end{split}
    \end{equation*}
    Since $\epsilon$ can be arbitrary small, we have 
    \begin{equation*}
        \abs{\frac{1}{m(B)}\int_B f(y)dy - f(x)} \leq (f-g)^*(x) + \abs{g(x) - f(x)}.
    \end{equation*}
    Now we let 
    \begin{equation*}
        E_\alpha = \Set{x}{\limsup_{m(B)\to 0, x\in B} \abs{\frac{1}{m(B)}\int_B f(y)dy-f(x)} > 2\alpha}.
    \end{equation*}
    We claim that $E_\alpha$ has measure zero. Set 
    \begin{equation*}
        F_\alpha = \Set{x}{(f-g)^*(x)>\alpha} 
        \quad\text{and}\quad 
        G_\alpha = \Set{x}{\abs{g(x)-f(x)}>\alpha}.
    \end{equation*}
    Then we have $E_\alpha\subset F_\alpha\cup G_\alpha$. By the 
    weak-type inequality and Tchebyshev's inequality, 
    \begin{equation*}
        \mu(F_\alpha) \leq \frac{A}{\alpha}\norm{f-g}_{\L^1}<\frac{A}{\alpha}\epsilon
        \quad\text{and}\quad
        \mu(G_\alpha) \leq \frac{1}{\alpha}\norm{f-g}_{\L^1}<\frac{1}{\alpha}\epsilon.
    \end{equation*}
    Thus $\mu(E_\alpha)\leq \mu(F_\alpha\cup G_\alpha)<\frac{A+1}{\alpha}\epsilon$. 
    Since $\epsilon$ is arbitrary, we have $\mu(E_\alpha)=0$ and 
    the proof is complete.
\end{proof}
\begin{remark}
    For $f\in\L^1(\R)$, and $F(x) = \int_{-\infty}^x f(y)dy$, we 
    have $F'(x) = f(x)$ a.e. Indeed, 
    \begin{equation*}
        \begin{split}
            \abs{\frac{F(x+h)-F(x)}{h}-f(x)} 
            &= \frac{1}{h}\abs{\int_{x}^{x+h}f(y)-f(x)dy} 
            \leq \frac{1}{h}\int_{x}^{x+h}\abs{f(y)-f(x)}dy \\ 
            &\leq \frac{1}{h}\int_{x-h}^{x+h}\abs{f(y)-f(x)}dy 
            \leq 2\frac{1}{2h}\int_{x-h}^{x+h}\abs{f(y)-f(x)}dy \to 0
        \end{split}
    \end{equation*}
    as $h\to 0$ a.e.\ x.
\end{remark}
\begin{remark}
    In fact, the requirement that $f\in\L^1$ can be relaxed to 
    $f\in\L^1_{loc}$, which is defined as the set of all locally 
    integrable functions, i.e., $f\chi_B\in\L^1$ for all finite 
    balls $B$ since the proof only requires $B$ to be a ball 
    near $x$. 
\end{remark} 