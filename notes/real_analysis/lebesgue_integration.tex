\begin{definition}
    For a simple function $s = \sum_{i=1}^n c_i\chi_{E_i}$, its 
    \textbf{Lebesgue integral} is defined as
    \begin{equation*}
        \int sd\mu = \sum_{i=1}^n c_i\mu(E_i).
    \end{equation*}
\end{definition}

\begin{definition}
    For a non-negative measurable function $f$, its \textbf{Lebesgue integral} 
    is defined as
    \begin{equation*}
        \int fd\mu = \sup\Set{\int sd\mu}{s\text{ is simple and }0\leq s\leq f}.
    \end{equation*}
\end{definition}

\begin{definition}
    For a measurable function $f:X\to[-\infty,\infty]$, its \textbf{Lebesgue integral} 
    is defined as
    \begin{equation*}
        \int fd\mu = \int f^+d\mu - \int f^-d\mu,
    \end{equation*}
    where $f^+ = \max\set{f,0}$ and $f^- = \max\set{-f,0}$ provided that 
    \begin{equation*}
        \int \abs{f}d\mu = \int f^+d\mu + \int f^-d\mu < \infty.
    \end{equation*}
    In such a case, we say that $f$ is \textbf{integrable}.
\end{definition}

\begin{proposition}
    For $f,g$ integrable and $c\in\R$,
    \begin{thmenum}
        \item $\int cf+gd\mu = c\int fd\mu + \int gd\mu$.
        \item If $f\leq g$ a.e., then $\int fd\mu \leq \int gd\mu$.
    \end{thmenum}
\end{proposition}
\begin{proof}
    Omitted.
\end{proof}

\begin{theorem}[Lebesgue Monotone Convergence Theorem]
    Let $f_n:X\to[0,\infty]$ be a sequence of measurable functions with 
    $f_n\nearrow f$ a.e. Then
    \begin{equation*}
        \int fd\mu = \lim_{n\to\infty}\int f_nd\mu.
    \end{equation*}
\end{theorem}
\begin{proof}
    By the monotonicity we have 
    \begin{equation*}
        \int f_nd\mu \leq \int fd\mu
    \end{equation*}
    for all $n$ and hence 
    \begin{equation*}
        \lim_{n\to\infty}\int f_nd\mu \leq \int fd\mu.
    \end{equation*}
    To obtain the reverse inequality, note that for any $c\in(0,1)$, 
    there exists $N$ such that $f_n\geq cf$ a.e. for all $n\geq N$. 
    Then 
    \begin{equation*}
        \int f_nd\mu \geq c\int fd\mu
    \end{equation*}
    for all $n\geq N$. Letting $n\to\infty$, 
    \begin{equation*}
        \lim_{n\to\infty}\int f_nd\mu \geq c\int fd\mu.
    \end{equation*}
    Taking $c\to 1^-$ then 
    \begin{equation*}
        \lim_{n\to\infty}\int f_nd\mu \geq \int fd\mu
        \implies \lim_{n\to\infty}\int f_nd\mu = \int fd\mu.
    \end{equation*}
\end{proof}
\begin{remark}
    As a consequence,
    \begin{equation*}
        \int \sum_n f_nd\mu = \sum_n \int f_nd\mu.
    \end{equation*}
\end{remark}

\begin{theorem}[Bounded Covergence Theorem]
    Suppose $\mu(X)<\infty$. Let $f_n:X\to\R_+$ be a sequence 
    measurable functions such that $f_n\leq M$ a.e. for some 
    $M\in\R$. If $f_n\to f$ a.e., then $f$ is integrable and 
    \begin{equation*}
        \int fd\mu = \lim_{n\to\infty}\int f_nd\mu.
    \end{equation*}
\end{theorem}
\begin{proof}
    For any $\epsilon>0$, by Egorov's theorem, there exists 
    $F\subset X$ such that $\mu(X-F)<\epsilon$ and $f_n\to f$ 
    uniformly on $F$. Then there exists $N$ such that 
    $\abs{f_n-f}<\epsilon$ on $F$ for all $n\geq N$. We have 
    \begin{equation*}
        \begin{split}
            \abs{\int f_nd\mu - \int fd\mu} &\leq \int_X \abs{f_n-f}d\mu \\
            &= \int_{F} \abs{f_n-f}d\mu + \int_{X-F}\abs{f_n-f}d\mu \\
            &\leq \epsilon\mu(F)+2M\mu(X-F) = \epsilon(\mu(F)+2M\epsilon).
        \end{split}
    \end{equation*}
    Since $\mu(X)<\infty$ and $\epsilon$ is arbitrary, 
    we may conclude that 
    \begin{equation*}
        \int fd\mu = \lim_{n\to\infty}\int f_nd\mu.
    \end{equation*}
\end{proof}

\begin{lemma}[Fatou]
    $f_n:X\to[0,\infty]$ are measurable. Then
    \begin{equation*}
        \int \liminf_n f_nd\mu \leq \liminf_n \int f_nd\mu.
    \end{equation*}
\end{lemma}
\begin{proof}
    Let $g_n = \inf_{k\geq n} f_k$. Then $g_n\nearrow g = 
    \liminf_n f_n$. By LMCT, 
    \begin{equation*}
        \int g_nd\mu\to \int gd\mu = \int \liminf_n f_nd\mu.
    \end{equation*}
    Note that $f_n\geq g_n$ and thus $\int f_nd\mu\geq\int g_nd\mu$. 
    Hence 
    \begin{equation*}
        \liminf_n \int f_nd\mu \geq\liminf_n\int g_nd\mu 
        = \int gd\mu = \int \liminf_n f_nd\mu.
    \end{equation*}
\end{proof}

\begin{theorem}[Lebesgue Dominated Convergence Theorem]
    Let $f_n:X\to[-\infty,\infty]$ be a sequence of measurable 
    functions such that $f_n\to f$ a.e. and there exists an integrable 
    function $g$ such that $\abs{f_n}\leq g$ a.e. for all $n$. 
    Then
    \begin{equation*}
        \int fd\mu = \lim_{n\to\infty}\int f_nd\mu.
    \end{equation*} 
\end{theorem}
\begin{proof}
    Since $\abs{f_n}\leq g$ a.e., $\abs{f}\leq g$ a.e. Now 
    $\abs{f_n-f}\leq \abs{f_n}+\abs{f}\leq 2g$ a.e. Let $h_n 
    = 2g - \abs{f_n-f}\geq 0$ a.e. By Fatou's lemma,
    \begin{equation*}
        \begin{split}
            \int 2gd\mu &= \int \liminf_n h_nd\mu \leq \liminf_n \int h_nd\mu 
            = \liminf_n \int 2g-\abs{f_n-f}d\mu \\
            &= \int 2gd\mu - \limsup_n \int \abs{f_n-f}d\mu.
        \end{split}
    \end{equation*}
    It follows that 
    \begin{equation*}
        0 \leq \liminf_n \int \abs{f_n-f}d\mu \leq \limsup_n \int \abs{f_n-f}d\mu \leq 0.
    \end{equation*}
    Hence 
    \begin{equation*}
        \lim_{n\to\infty}\int \abs{f_n-f}d\mu = 0.
    \end{equation*}
    By the triangle inequality,
    \begin{equation*}
        \abs{\int fd\mu - \int f_nd\mu} \leq \int \abs{f-f_n}d\mu \to 0.
    \end{equation*}
    So 
    \begin{equation*}
        \int fd\mu = \lim_{n\to\infty}\int f_nd\mu.
    \end{equation*}
\end{proof}
\begin{remark}
    If $\supp{f}$ has finite measure and $f$ is bounded, then 
    \begin{equation*}
        \int f = \inf_{s\geq f} \int sd\mu, 
    \end{equation*}
    where $s$ is simple.
\end{remark}

\begin{definition}
    $\L^1 = \Set{f:X\to\R}{f\text{ is integrable}}$ with the 
    norm $\norm{f}_{\L^1} = \int \abs{f}d\mu$ is called 
    the \textbf{$\L^1$ space}.
\end{definition}
\begin{remark}
    The elements in $\L^1$ are in fact equivalence classes 
    of functions that are equal a.e.
\end{remark}

\begin{proposition}\label{prop:int_abs_conti}
    Let $f\in\L^1$ be a nonegative function. Then for every 
    $\epsilon>0$, there is some $\delta>0$ such that for 
    any measurable $E$ with $\mu(E)\leq\delta$, 
    \begin{equation*}
        \int_E fd\mu \leq \epsilon.
    \end{equation*} 
\end{proposition}
\begin{proof}
    Let $E_n = \Set{x\in X}{f(x)>n}$. Then by Lebesgue 
    dominated convergence theorem, since $f\chi_{E_n}\leq f$,
    \begin{equation*}
        \int_{E_n}fd\mu \to 0.
    \end{equation*}
    For any $\epsilon>0$, there exists $n$ such that 
    \begin{equation*}
        \int_{E_n}fd\mu \leq \frac{\epsilon}{2}.
    \end{equation*}
    Pick $\delta\leq \epsilon/(2n)$. Then for any measurable 
    $E$ with $\mu(E)\leq\delta$, 
    \begin{equation*}
        \int_E fd\mu = \int_{E\cap E_n}fd\mu + \int_{E\cap E_n^c}fd\mu 
        \leq \int_{E_n}fd\mu + n\mu(E) \leq \frac{\epsilon}{2} + \frac{\epsilon}{2} = \epsilon
    \end{equation*}
    since $f\leq n$ on $E_n^c$. This completes the proof.
\end{proof}

\begin{theorem}[Lebesgue-Vitali]
    $f:X\to\R$ is Riemann integrable if and only if the discontinuity 
    set of $f$ has Lebesgue measure zero. Furthermore, if $f$ is Riemann 
    integrable, then the Riemann integral and the Lebesgue integral agrees.
\end{theorem}
\begin{proof}
    Define the oscillation of $f$ at $x$ as 
    \begin{equation*}
        \osc{f}{x} = \inf_{U:x\in U} \diam{f(U)},
    \end{equation*}
    where $U$ is open. 
    
    We first claim that $f$ is continuous at $x$ if and only 
    if $\osc{f}{x}=0$. Indeed, if $f$ is continuous at $x$, 
    then $\forany\epsilon>0$, $\exist\delta>0$ such that 
    $\abs{f(x)-f(y)}<\epsilon$ for all $y\in B_\delta(x)$. 
    Then $\diam{f(B_\delta(x))}\leq 2\epsilon$. Since 
    $\epsilon$ is arbitrary, $\osc{f}{x}=0$. Conversely, if 
    $\osc{f}{x}=0$, then $\forany\epsilon>0$, $\exist$ open 
    $U$ containing $x$ such that $\diam{f(U)}<\epsilon$. 
    This implies that $\abs{f(x)-f(y)}<\epsilon$ for all 
    $y\in U$ and hence $f$ is continuous at $x$. 

    Next, let $D_\epsilon$ collect all points $x$ such that 
    $\osc{f}{x}\geq\epsilon>0$. We claim that $D_\epsilon$ 
    is closed. For any convergent sequence $x_k\in D_\epsilon$, 
    let $x_k\to x$. For any open $U$ containing $x$, $\exist N$ 
    such that $x_k\in U$ for all $k\geq N$. Then $\exist$ an 
    open neighborhood of $x_N$, $U'$, such that $U'\subset U$ 
    and $\diam{f(U')}\geq\epsilon$. Hence $\osc{f}{x}\geq
    \epsilon$ and $x\in D_\epsilon$, showing that $D_\epsilon$ 
    is closed. Observe that $D = \bigcup_{n=1}^{\infty}D_{1/n}$.
    
    Now suppose that $f$ is Riemann integrable. Then for any 
    $\epsilon>0$, $\exist\mathcal{P}$ such that $\mathrm{U}
    (f,\mathcal{P})-\mathrm{L}(f,\mathcal{P})<\frac{1}{n}$ and 
    $\norm{\mathcal{P}}<\frac{1}{n}$. Then 
    \begin{equation*}
        \begin{split}
            &\quad\sum_{\substack{Q\in\mathcal{P},\\ Q\bigcap 
            D_\frac{1}{n}\neq\varnothing}} (\sup_Q f -\inf_Q f)\abs{Q} 
            + \sum_{\substack{Q\in\mathcal{P}, \\ Q\bigcap 
            D_\frac{1}{n} = \varnothing}} (\sup_Q f -\inf_Q f)\abs{Q}\\
            &= \sum_{Q\in\mathcal{P}}(\sup_Q f -\inf_Q f)\abs{Q} 
            = \mathrm{U}(f,\mathcal{P})-\mathrm{L}(f,\mathcal{P}) 
            < \epsilon.
        \end{split}
    \end{equation*}
    Note that $\sup_Q f -\inf_Q f = \diam{f(Q)}$. This gives 
    that $2M\mu^*(D_\frac{1}{n})<\epsilon$ for every $n$. Since 
    $\epsilon$ is arbitrary, we conclude that $\mu^*(D_\frac{1}{n}) 
    = 0$ for each $n$. Thus $D$ is an union of sets of measure 
    zero and hence also has measure zero. 

    For the converse, suppose that $m(D)=0$. Then $D_\epsilon$ 
    also has measure zero. Let $\mathcal{P}$ be a partition 
    on $E$ with $\norm{\mathcal{P}}<\delta$ for some 
    $\delta>0$, which will be determined later. Then 
    \begin{equation*}
        \begin{split}
            \mathrm{U}(f,\mathcal{P})-\mathrm{L}(f,\mathcal{P}) 
            &= \sum_{Q\in\mathcal{P}}(\sup_Q f -\inf_Q f)\abs{Q}\\ 
            &= \sum_{\substack{Q\in\mathcal{P},\\ Q\bigcap 
            D_\epsilon = \varnothing}}(\sup_Q f -\inf_Q f)\abs{Q} 
            + \sum_{\substack{Q\in\mathcal{P},\\ Q\bigcap 
            D_\epsilon\neq\varnothing}}(\sup_Q f -\inf_Q f)\abs{Q}
        \end{split}
    \end{equation*}
    For the first term, $\sup_Q f -\inf_Q f < \epsilon$ for 
    $\norm{\mathcal{P}}<\delta_1$ for some $\delta_1>0$. And 
    thus the first term is bounded by $\epsilon m(E)$. For the 
    second term, $\sup_Q f -\inf_Q f < 2M$ and since 
    $D_\epsilon$ has measure zero, $\exist Q_k$ cubic cover 
    of $D_\epsilon$ such that $\sum_k \abs{Q_k}<\epsilon$. 
    Now if $\diam{Q}<\delta_2$ for some $\delta_2>0$, then 
    those $Q$ intersecting $D_\epsilon$ nonempty are subset 
    of $\bigcup_k Q_k$. Thus the second term is bounded by 
    $2M\epsilon$. Choosing $\delta = \min\set{\delta_1, 
    \delta_2}$ yields that 
    \begin{equation*}
        \mathrm{U}(f,\mathcal{P})-\mathrm{L}(f,\mathcal{P}) 
        < \epsilon m(E) + 2M\epsilon
    \end{equation*}
    whenever $\norm{\mathcal{P}}<\delta$. Since $\epsilon$ is 
    arbitrary, $f$ is Riemann integrable.
\end{proof}

\begin{proposition}\label{prop:conti_approx}\ \vspace{-1em}
    \begin{thmenum}
        \item Step functions are dense in $\L^1$.
        \item Continuous functions with compact support are dense 
        in $\L^1$.
    \end{thmenum}
\end{proposition}
\begin{proof}
    Let $f\in\L^1$. By \cref{lem:simple_approx}, we already know that 
    simple functions are dense in $\L^1$. It now remains to 
    show that step functions can approximate simple functions. 
    Since simple functions are linear combinations of finitely 
    many characteristic functions, it suffices to show that 
    characteristic functions can be approximated by step functions. 
    Now for any measurable $E$, there is a family of almost disjoint 
    cubes $Q_i$ such that $\mu(E\triangle \cup_{i=1}^MQ_i)\leq 
    2\epsilon$, and thus we may set the step function to be 
    $\phi = \sum_{i=1}^M\chi_{Q_i}$, with $\norm{\chi_E-\phi}_{\L^1}
    \leq 2\epsilon$. 

    For the second part, let it now suffices to show that 
    continuous functions with compact support can approximate
    characteristic functions of a rectangle, say $[a,b]$. 
    Then set 
    \begin{equation*}
        g(x) = \begin{cases}
            0 & x\leq a-\epsilon, \\
            \frac{x-a+\epsilon}{\epsilon} & a-\epsilon\leq x\leq a, \\ 
            1 & a\leq x\leq b, \\
            1-\frac{x-b}{\epsilon} & b\leq x\leq b+\epsilon, \\
            0 & x\geq b+\epsilon.
        \end{cases}
    \end{equation*}
    Then $g$ is continuous with compact support and $\norm{\chi_{[a,b]}-g}_{\L^1} 
    \leq \epsilon/2 + \epsilon/2 = \epsilon$.
\end{proof}