\begin{definition}
    Let $\S,\T$ be two $\sigma$-algebra on $X$ and $Y$ respectively. 
    The smallest $\sigma$-algebra on $X\times Y$ containting the collection 
    $\Set{S\times T}{S\in\S, T\in\T}$ is called the \textbf{product $\sigma$-algebra} 
    of $\S$ and $\T$, denoted by $\S\otimes\T$.
\end{definition}

\begin{definition}
    Suppose $X$ is an arbitrary set and $\M$ is a collection of subsets of $X$. 
    We say that $\M$ is a \textbf{monotone class} if 
    \begin{thmenum}
        \item If $E_i\subset E_{i+1}$ for countably many $E_i\in\M$, then $\bigcup_{i=1}^\infty E_i\in\M$.
        \item If $E_i\supset E_{i+1}$ for countably many $E_i\in\M$, then $\bigcap_{i=1}^\infty E_i\in\M$.
    \end{thmenum}
\end{definition}

\begin{definition}
    A collection $\A$ of subsets in $X$ is called an \textbf{algebra} on $X$ if 
    \begin{thmenum}
        \item $\varnothing\in\A$.
        \item If $A\in\A$, then $A^c\in\A$.
        \item If $A_1,A_2\in\A$, then $A_1\cup A_2\in\A$.
    \end{thmenum}
\end{definition}
\begin{remark}
    The condition (c) implies that for finitly many $A_i\in\A$, $\cup_{i=1}^n A_i\in\A$.
\end{remark}

\begin{theorem}[Monotone Class Theorem]
    Suppose $\A$ is an algebra on $X$. Then the smallest $\sigma$-algebra 
    containing $\A$ is the smallest monotone class containing $\A$.
\end{theorem}
\begin{proof}
    Let $\M$ be the smallest monotone class containing $\A$. The theorem 
    can be written as $\sigma(\A) = \M$. First we show that $\M\subset\sigma(\A)$. 
    To see this, we claim first that a $\sigma$-algebra is automatically a monotone 
    class. Indeed, let $\S$ be a $\sigma$-algebra. Then for any countably 
    many $E_i\in\S$ with $E_i\nearrow E$, we have $E = \bigcup_{i=1}^\infty E_i\in\S$. 
    Also, for any countably many $E_i\in\S$ with $E_i\searrow E$, we have 
    $E_i^c\nearrow E^c$. Thus $E^c\in\S$ and $E\in\S$. Therefore $\S$ is a monotone class. 
    It follows that $\sigma(\A)$ is a monotone class and hence $\M\subset\sigma(\A)$ 
    by the minimality of $\M$.

    Next, we claim that $\M$ is a $\sigma$-algebra. By definition, we already 
    have $\varnothing\in\M$. Let $E\in\M$. Then there is a sequence 
    of sets $E_i\in\A$ such that either $E_i\nearrow E$ or $E_i\searrow E$. In 
    the former case, we have $E^c = \bigcap_{i=1}^\infty E_i^c\in\M$; in the latter 
    case, we have $E^c = \bigcup_{i=1}^\infty E_i^c\in\M$. Thus $E^c\in\M$. Lastly, 
    we need to show that $\M$ is closed under countable unions. We start by showing 
    that it is closed under finite unions. Consider $A\in\A$. Define $\D_1 
    = \Set{D\in\M}{D\cup A \in\M}$. It is clear that $\D_1$ is a monotone class and 
    $\A\subset\D_1$. Consider also $\D_2 = \Set{D\in\M}{D\cup E\in\M \text{ for all } E\in\M}$. 
    Then $\D_2$ is also a monotone class and $\A\subset\D_2$. By the minimality of 
    $\M$, we have $\M\subset\D_1\cap\D_2$ and hence $\M$ is closed under finite unions. 
    Now let $E_i\in\M$ be countably many sets. Put $F_n = \cup_{i=1}^n E_i$. 
    Then $F_n\nearrow E = \bigcup_i E_i$. By the closure of $\M$ under countable unions, 
    $F_n\in\M$; by the definition of $\M$, $E\in\M$. We conclude that $\M$ is closed under 
    countable unions. Thus $\M$ forms a $\sigma$-algebra. It now follows by the minimality 
    of $\sigma(\A)$ that $\sigma(\A)\subset\M$. We conclude that $\sigma(\A) = \M$.
\end{proof}

\begin{lemma}\label{lem:product_sigma_algebra}
    Suppose $(X,\S,\mu)$ and $(Y,\T,\nu)$ are two finite measure spaces. Let 
    \begin{equation*}
        \F = \Set{E\subset X\times Y}{\int\int \chi_E(x,y)d\nu(y)d\mu(x) = \int\int \chi_E(x,y)d\mu(x)d\nu(y)}.
    \end{equation*}
    Then $\S\otimes\T\subset\F$.
\end{lemma}
\begin{proof}
    Since $\varnothing\in\F$, $\F$ is non-empty. Let $E = A\times B$ for some 
    $A\in\S$ and $B\in\T$. Then 
    \begin{equation*}
        \begin{split}
            \int\int \chi_E(x,y)d\nu(y)d\mu(x) 
            &= \int_A\int_B d\nu(y)d\mu(x) = \nu(B)\int_A d\mu(x)\\
            &= \nu(B)\mu(A) = \int_B\mu(A)d\nu(y)\\
            &= \int_B\int_A d\mu(x)d\nu(y) = \int\int \chi_E(x,y)d\mu(x)d\nu(y).
        \end{split}
    \end{equation*}

    Now let $\mathcal{R}$ be the collection of all rectangles on $X\times Y$, i.e., 
    $\mathcal{R} = \Set{A\times B}{A\in\S, B\in\T}$. For $R_1,R_2\in\mathcal{R}$, 
    $R_1\cap R_2 = \varnothing$ implies $\chi_{R_1\cup R_2} = \chi_{R_1} + \chi_{R_2}$. 
    By the above calculation, we know that $\mathcal{R}\subset\F$. Consider 
    a sequence of sets $E_i\in\F$. If $E_i\nearrow E$, then
    \begin{equation*}
        \begin{split}
            \int\int \chi_E(x,y)d\nu(y)d\mu(x) 
            &= \lim_{i\to\infty}\int\int \chi_{E_i}(x,y)d\nu(y)d\mu(x)\\
            &= \lim_{i\to\infty}\int\int \chi_{E_i}(x,y)d\mu(x)d\nu(y)
            = \int\int \chi_E(x,y)d\mu(x)d\nu(y).
        \end{split}
    \end{equation*}
    Also, if $E_i\searrow E$, then 
    \begin{equation*}
        \begin{split}
            \int\int \chi_E(x,y)d\nu(y)d\mu(x) 
            &= \lim_{i\to\infty}\int\int \chi_{E_i}(x,y)d\nu(y)d\mu(x)\\
            &= \lim_{i\to\infty}\int\int \chi_{E_i}(x,y)d\mu(x)d\nu(y)
            = \int\int \chi_E(x,y)d\mu(x)d\nu(y).
        \end{split}
    \end{equation*}
    Hence $\F$ is a monotone class containing $\mathcal{R}$. By the monotone 
    class theorem, $\S\otimes\T\subset\F$.
\end{proof}

\begin{theorem}[Existence and Uniqueness of Product Measure]\label{thm:product_measure}
    Let $(X,\S,\mu)$ and $(Y,\T,\nu)$ be two $\sigma$-finite measure spaces. 
    Let $\omega$ be a set function on $\S\otimes\T$. For $A\in\S$ and 
    $B\in\T$, define 
    \begin{equation*}
        \omega(A\times B) = \mu(A)\nu(B).
    \end{equation*}
    Then, $\omega$ extends uniquely to a measure on $(X\times Y,\S\otimes\T)$ 
    such that for every $E\in\S\otimes\T$, 
    \begin{equation*}
        \omega(E) = (\mu\times\nu)(E) = \int\int \chi_E(x,y)d\nu(y)d\mu(x) = \int\int \chi_E(x,y)d\mu(x)d\nu(y).
    \end{equation*}
\end{theorem}
\begin{proof}
    If we consider $\mu$ and $\nu$ to be $\sigma$-finite measures, 
    \cref{lem:product_sigma_algebra} gives us that $\omega(A\times B) = \mu(A)\nu(B)$. 
    We want to extend $\omega$ to a set function 
    \begin{equation*}
        \omega(E) = \int\int \chi_E(x,y)d\nu(y)d\mu(x) = \int\int \chi_E(x,y)d\mu(x)d\nu(y).
    \end{equation*}
    Since integrals are linear, $\omega$ is finitely additive. Applying the 
    monotone class theorem, $\omega$ becomes $\sigma$-additive. Hence $\omega$ 
    becomes a measure on $(X\times Y,\S\otimes\T)$. To see the uniqueness, 
    let $\rho$ be another measure on $(X\times Y,\S\otimes\T)$ such that 
    $\rho(A\times B) = \mu(A)\nu(B)$. Let $\M = \Set{E\subset X\times Y}{\omega(E) = \rho(E)}$. 
    For countably many $E_i\in\M$ with $E_i\nearrow E$, we can write $E = \cup_{i=1}^\infty D_i$ 
    where $D_i = E_{i+1}-E_i$ and $E_0 = \varnothing$ are disjoint. The $\sigma$-additivity 
    gives $\omega(E) = \rho(E)$. Thus $E\in\M$. A similar argument gives us that 
    $E_i\searrow E$ implies $E\in\M$. Hence $\M$ is a monotone class. By the 
    monotone class theorem, $\S\otimes\T\subset\M$. Thus $\omega = \rho$. 

    For the case $\mu,\nu$ being $\sigma$-finite, consider $\set{A_i}$ and $\set{B_i}$ 
    to be two disjoint partitions of $X$ and $Y$ respectively with $\mu(A_i) < \infty$ 
    and $\nu(B_i) < \infty$ for all $i$. Let $E_{ij} = E\cap(A_i\times B_j)$. 
    By the established result for finite measures, 
    \begin{equation*}
        \int\int \chi_{E_{ij}}d\mu d\nu = \int\int \chi_{E_{ij}}d\nu d\mu.
    \end{equation*}
    Taking the sum over $i,j$ and applying Lebesgue monotone convergence 
    theorem gives us 
    \begin{equation*}
        \omega(E) = \int\int \chi_E d\nu d\mu = \int\int \chi_E d\mu d\nu
    \end{equation*}
    for any $E\in\S\otimes\T$. Applying Lebesgue monotone convergence theorem again 
    results in that $\omega$ is $\sigma$-additive. Hence $\omega$ is a measure on 
    $(X\times Y,\S\otimes\T)$. To see the uniqueness, let $\rho$ be another measure on 
    $(X\times Y,\S\otimes\T)$ such that $\rho(A\times B) = \mu(A)\nu(B)$. By 
    the $\sigma$-additivity and the uniqueness of the finite measure case, 
    \begin{equation*}
        \omega(E) = \sum_{i,j}\omega(E_{ij}) = \sum_{i,j}\rho(E_{ij}) = \rho(E)
    \end{equation*}
    for all $E\in\S\otimes\T$. Thus $\omega = \rho$.
\end{proof}

\begin{theorem}[Fubini-Tonelli]
    Let $(X,\S,\mu)$ and $(Y,\T,\nu)$ be two $\sigma$-finite measure spaces. 
    Let $F:X\times Y\to\R$ be a $\S\otimes\T$-measurable function such that 
    one of the following conditions holds: 
    \begin{thmenum}
        \item $F\geq 0$ a.e. (Tonelli);
        \item $F$ is integrable (Fubini).
    \end{thmenum}
    Then 
    \begin{equation*}
        \int F(x,y)d(\mu\times\nu) = \int\int Fd\mu d\nu = \int\int Fd\nu d\mu
    \end{equation*}
    and furthermore, 
    \begin{equation*}
        \begin{cases}
            y\mapsto\int F(x,y)d\mu(x) &\text{is $\T$-measurable},\\
            x\mapsto\int F(x,y)d\nu(y) &\text{is $\S$-measurable}.
        \end{cases}
    \end{equation*}
\end{theorem}
\begin{proof}
    By \cref{thm:product_measure}, the statement holds for indicator functions
    and hence for simple functions. By Lebesgue monotone convergence theorem, 
    the non-negative case (Tonelli) is proved. For the integrable case (Fubini), 
    write $F = F^+ - F^-$. We also have that $y\mapsto\int F^\pm(x,y)d\mu(x)$ 
    and $x\mapsto\int F^\pm(x,y)d\nu(y)$ are $\S$-measurable and $\T$-measurable 
    by the \cref{thm:product_measure}. Furthermore, $y\mapsto\int F^\pm(x,y)d\mu(x)$ 
    and $x\mapsto\int F^\pm(x,y)d\nu(y)$ are integrable a.e.\ or the condition 
    (b) is violated. Thus 
    \begin{equation*}
        \begin{split}
            \int F d(\mu\times\nu) 
            &= \int\int F^+ d(\mu\times\nu) - \int\int F^- d(\mu\times\nu)\\
            &= \int\int F^+ d\mu d\nu - \int\int F^- d\mu d\nu\\
            &= \int\int F^+ d\nu d\mu - \int\int F^- d\nu d\mu.
        \end{split}
    \end{equation*}
    The proof is complete.
\end{proof}
\begin{remark}
    By induction, one can extend the Lebesgue measure to any $\R^d$, $d\in\N$. 
    $\B_m\otimes\B_n = \B_{m+n}$, where $\B_n$ is the Borel $\sigma$-algebra on $\R^n$. 
    Extension to $\R^{\infty}$ is also possible.
\end{remark}