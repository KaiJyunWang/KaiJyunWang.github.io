\begin{definition}
    For $f\in\L^1(\R)$, its \textbf{Fourier transform} is defined as 
    \begin{equation}
        \hat{f}(t) = \int_{\R} f(x)e^{-2\pi itx}dx.
    \end{equation}
\end{definition}
\begin{remark}
    The Fourier series coefficients can be viewed as discrete Fourier transform 
    $f\mapsto \set{a_n}_{n\in\Z}$, with 
    \begin{equation}
        a_n = \int_{-1}^{1}f(x)e^{-2\pi inx}dx.
    \end{equation}
    The inverse discrete Fourier transform is then given by 
    \begin{equation}
        f(x) = \sum_{n\in\Z}a_ne^{2\pi inx}.
    \end{equation}
\end{remark}

\begin{example}
    \begin{equation*}
        \hat{\chi}_{[a,b]}(t) = \int_{a}^{b}e^{-2\pi itx}dx 
        = \begin{cases}
            b-a & \text{if }t=0, \\
            \frac{-1}{2\pi it}\pth{e^{-2\pi itb} - e^{-2\pi ita}} & \text{if }t\neq 0.
        \end{cases} 
    \end{equation*}    
\end{example} 

\begin{lemma}[Riemann-Lebesgue \rom{2}]
    Let $f\in\L^1(\R)$. Then $\hat{f}$ is uniformly continuous on $\R$, 
    satisfying $\norm{\hat{f}}_\infty\leq\norm{f}_1$, and 
    \begin{equation*}
        \lim_{\abs{t}\to\infty}\hat{f}(t) = 0.
    \end{equation*} 
\end{lemma}