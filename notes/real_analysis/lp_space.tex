\begin{definition}
    $\ell^p = \Set{\set{x_i}_{i\in\I}}{\norm{x}_p < \infty}$, where 
    $\I$ is an countable index set and $\norm{x}_p = 
    \pth{\sum_i \abs{x_i}^p}^{1/p}$, $1\leq p <\infty$,  
    is called the \textbf{$\ell^p$ space}. For $p=\infty$, 
    the norm is defined as $\norm{x}_\infty = \sup_i\abs{x_i}$.
\end{definition}

\begin{definition}
    $f:X\to Y$ is called a \textbf{homomophism} if it preserves 
    the algebraic structure. In particular, for $X,Y$ being 
    vector spaces, $f$ is a homomorphism if $f(cx+y) = cf(x)+f(y)$.
\end{definition}

\begin{definition}
    $f:X\to Y$ is called an \textbf{isomorphism} if it is a 
    bijective homomorphism.
\end{definition}

\begin{definition}
    $f:X\to Y$ is called an \textbf{isometry} if $\norm{f(x)}_Y 
    = \norm{x}_X$ for all $x\in X$.
\end{definition}

\begin{example}
    A rightward shift operator $S_R:\ell^p(\N)\to\ell^p(\N)$ is 
    not an isomorphism, but $S_R:\ell^p(\Z)\to\ell^p(\Z)$ is.
\end{example}

\begin{lemma}[Young's Inequality]
    Let $1< p,p'<\infty$ with $\frac{1}{p}+\frac{1}{p'}=1$. 
    Then for all $a,b\geq 0$, 
    \begin{equation*}
        ab \leq \frac{a^p}{p} + \frac{b^{p'}}{p'}.
    \end{equation*}
    Furthermore, the equality holds if and only if $a^p = b^{p'}$.
\end{lemma}
\begin{proof}
    If $a = 0$ or $b = 0$, the inequality is trivial. Suppose 
    $a,b>0$. Let $t = 1/p$ and we can write 
    \begin{equation*}
        \log(ab) = \log(a) + \log(b) = t\log(a^p) + (1-t)\log(b^{p'}) 
        \leq \log\pth{ta^p + (1-t)b^{p'}}
    \end{equation*} 
    by the concavity of logarithm and Jensen's inequality.
    Exponentiating both sides yields the desired inequality. 
    The equality holds if and only if $a^p = b^{p'}$ by the 
    Jensen's inequality.
\end{proof}

\begin{theorem}[H\"older's Inequality in $\ell^p$]
    Let $1\leq p,p'\leq\infty$ with $\frac{1}{p}+\frac{1}{p'}=1$. 
    Then for all $f\in\ell^p$ and $g\in\ell^{p'}$, 
    \begin{equation*}
        \norm{fg}_1 \leq \norm{f}_p\norm{g}_{p'}.
    \end{equation*}
    Moreover, the equality holds if and only if $f = cg$ for some 
    constant $c$.
\end{theorem}
\begin{proof}
    If one of $f$ or $g$ is zero, the inequality is trivial.
    If $p = 1$ and $p' = \infty$, $\abs{f_ig_i} \leq \norm{g}_{\infty}\abs{f_i}$. 
    Summing over $i$ yields the desired inequality. For the case 
    $p = \infty$ and $p' = 1$ the proof is similar. Now suppose
    $1<p<\infty$ and $1<p'<\infty$. Without loss of generality, 
    we may assume that $\norm{f}_p = \norm{g}_{p'} = 1$.
    By Young's inequality, 
    \begin{equation*}
        \abs{f_ig_i} \leq \frac{\abs{f_i}^p}{p} + \frac{\abs{g_i}^{p'}}{p'}.
    \end{equation*}
    Thus 
    \begin{equation*}
        \norm{fg}_1 = \sum_i \abs{f_ig_i} \leq \sum_i \frac{\abs{f_i}^p}{p} + \sum_i\frac{\abs{g_i}^{p'}}{p'} 
        = \frac{1}{p}\norm{f}_p^p + \frac{1}{p'}\norm{g}_{p'}^{p'} = 1.
    \end{equation*}
    Hence we obtain the desired inequality. The equality holds if 
    and only if $\abs{f_i}^p = \abs{g_i}^{p'}$ for all $i$ by the 
    Young's inequality. In general, the equality holds if and 
    only if $f = cg$ for some constant $c$ after scaling the 
    both sides of the inequality by $c$. 
\end{proof}
\begin{remark}
    We call $p'$ the \textbf{conjugate exponent} of $p$ for 
    $1/p + 1/p' = 1$.
\end{remark}

\begin{theorem}[Minkowski's Inequality in $\ell^p$]
    Let $1\leq p\leq\infty$. Then for all $f,g\in\ell^p$, 
    \begin{equation*}
        \norm{f+g}_p \leq \norm{f}_p + \norm{g}_p.
    \end{equation*}
\end{theorem}
\begin{proof}
    If $p = 1$, the inequality comes from the triangle inequality. 
    For $1<p<\infty$, 
    \begin{equation*}
        \begin{split}
            \norm{f+g}_p^p &= \sum_i \abs{f_i+g_i}\abs{f_i+g_i}^{p-1} \\
            &\leq \sum_i \abs{f_i}\abs{f_i+g_i}^{p-1} + \sum_i \abs{g_i}\abs{f_i+g_i}^{p-1} \\
            &\leq \norm{f}_p\pth{\sum_i \abs{f_i+g_i}^{(p-1)p'}}^{1/p'} + \norm{g}_p\pth{\sum_i \abs{f_i+g_i}^{(p-1)p'}}^{1/p'} \\ 
            &= \norm{f}_p\norm{f+g}_p^{p/p'} + \norm{g}_p\norm{f+g}_p^{p/p'}
        \end{split}
    \end{equation*}
    by the H\"older's inequality. Rearranging the inequality yields 
    \begin{equation*}
        \norm{f+g}_p = \norm{f+g}_p^{p-p/p'} \leq \norm{f}_p + \norm{g}_p.
    \end{equation*}
    For $p = \infty$, 
    \begin{equation*}
        \norm{f+g}_\infty = \sup_i\abs{f_i+g_i} \leq \sup_i\abs{f_i} + \sup_i\abs{g_i} = \norm{f}_\infty + \norm{g}_\infty.
    \end{equation*}
    The proof is complete.
\end{proof}
\begin{remark}
    The Minkowski's inequality is exactly the triangle inequality 
    in $\ell^p$ spaces. We can thus confirm that $\ell^p$ norms 
    are indeed norms.
\end{remark}

\begin{theorem}[Dualities of $\ell^p$ Spaces]
    Let $1< p < \infty$. Then $(\ell^p)' \cong \ell^{p'}$, where 
    $p'$ is the conjugate exponent of $p$.
\end{theorem}
\begin{proof}
    We need to prove that there exists an isometric isomorphism 
    $\psi:\ell^{p'}\to\pth{\ell^{p}}'$ such that $\psi gf = \sum_i f_ig_i$ 
    for all $g\in\ell^{p'}$ and $f\in\ell^p$. We show that $\psi$ 
    is well-defined, linear, bounded, bijective, and isometric. 

    First, we show that $\psi$ is well-defined. For $f\in\ell^p$ and 
    $g\in\ell^{p'}$, 
    \begin{equation*}
        \abs{\psi gf} \leq \sum_i \abs{f_ig_i} \leq \norm{f}_p\norm{g}_{p'} < \infty
    \end{equation*}
    by the H\"older's inequality. Thus $\psi g\in\pth{\ell^{p}}'$ is 
    well-defined. 

    Next, $\psi$ is linear since for $g_1,g_2\in\ell^{p'}$ and $c\in\R$, 
    \begin{equation*}
        \psi(cg_1 + g_2)(f) = \sum_i f_i(cg_{1i}+g_{2i}) = c\sum_i f_ig_{1i} + \sum_i f_ig_{2i} = c\psi g_1(f) + \psi g_2(f)
    \end{equation*}
    for all $f\in\ell^p$. Hence $\psi(cg_1 + g_2) = c\psi g_1 + \psi g_2$. 

    Now, to show that $\psi$ is bounded,
    \begin{equation*}
        \begin{split}
            \norm{\psi g} &= \sup\Set{\abs{\psi gf}}{\norm{f}_p = 1} 
            = \sup\Set{\abs{\sum_i f_ig_i}}{\norm{f}_p = 1} \\
            &\leq \sup_{\norm{f}_p = 1}\set{\norm{g}_{p'}} \leq \norm{g}_{p'}.
        \end{split}
    \end{equation*}
    We see that $\norm{\psi}\leq 1$. Next, let 
    $h\in\pth{\ell^p}'$ and define $g$ by $g_i = h(e_i)$. Then 
    \begin{equation*}
        \begin{split}
            \norm{g}_{p'} &= \pth{\sum_i \abs{g_i}^{p'}}^{1/p'} 
            = \pth{\sum_i \abs{h(e_i)}^{p'}}^{1/p'} 
            \leq \pth{\sum_i \norm{h}^{p'}}^{1/p'} 
            = \norm{h}.
        \end{split}
    \end{equation*}
    Then $g\in\ell^{p'}$. Furthermore, for such $g$,
    \begin{equation*}
        \psi g(f) = \sum_i f_ig_i = \sum_i f_ih(e_i) = h\pth{\sum_i f_ie_i} = h(f)
    \end{equation*} 
    for every $f\in\ell^p$. Hence $\psi$ is surjective and $\norm{\psi g} 
    = \norm{h}$. The isometry of $\psi$ is immediate from that 
    \begin{equation*}
        \norm{\psi g} \leq \norm{g}_{p'} \leq \norm{h} = \norm{\psi g}.
    \end{equation*} 
    Finally, $\psi$ is injective since otherwise there exists 
    $g\neq 0$ such that $\psi g = 0$. Then $\norm{g}_{p'} = 0$ 
    by the isometry of $\psi$, which implies that $g = 0$, 
    a contradiction. We conclude that $\psi$ is an isometric 
    isomorphism and the proof is complete.
\end{proof}