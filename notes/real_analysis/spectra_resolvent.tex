\begin{definition}
    Let $T:D(T)\subset X\to X$ be a closed linear operator. The 
    \textbf{resolvent set} of $T$ is defined as 
    \begin{equation*}
        \rho(T) = \Set{\xi\in\C}{(T-\xi I) \text{ has bounded inverse on $X$}}.
    \end{equation*}
    The \textbf{spectrum} of $T$ is defined as $\sigma(T) = \C\setminus\rho(T)$. 
    $R_T(\xi) = (T-\xi I)^{-1}$ is called the \textbf{resolvent operator} of $T$.
\end{definition}
\begin{remark}
    $\xi\in\rho(T)$ if and only if $T-\xi I$ has the bounded inverse on $X$.
\end{remark}
\begin{remark}
    $\xi\in\sigma(T) = \C\setminus\rho(T)$ if either $T-\xi I$ is not invertible 
    or $T-\xi I$ is invertible but has range smaller than $X$. If $\dim X<\infty$, 
    $\sigma(T) = \Set{\lambda\in\C}{Tx = \lambda x \text{ for some } x\in X\setminus \set{0}}$.
\end{remark}

\begin{example}
    $X = C[a,b]$. $Tu = u'$ and $D(T) = C^1[a,b]$. $T$ is not invertible since 
    $T(u) = 0$ for every constant function $u$. Consider the following domains 
    \begin{itemize}
        \item $D_1 = \Set{u\in D(T)}{u(a) = 0}$, 
        \item $D_2 = \Set{u\in D(T)}{u(b) = 0}$, 
        \item $D_3 = \Set{u\in D(T)}{u(a) = ku(b)}$, 
        \item $D_0 = \Set{u\in D(T)}{u(a) = u(b) = 0}$.
    \end{itemize}
    $T_i = T|_{D_i}$ are invertible on $D_i$ for $i = 0,1,2,3$, but the 
    inverses are different. For example, 
    \begin{equation*}
        (T^{-1}_1 v)(x) = \int_a^x v(t)dt.
    \end{equation*}
\end{example}

\begin{theorem}[Neumann Series]
    Let $T:X\to X$ be a bounded lienar operator. If $\norm{T}<1$, then $I-T$ is invertible and
    \begin{equation*}
        (I-T)^{-1} = \sum_{n=0}^\infty T^n.
    \end{equation*}
\end{theorem}
\begin{proof}
    Denote $S_n = \sum_{k=0}^n T^k$. Compute that 
    \begin{equation*}
        (I-T)S_n = S_n - S_nT = \sum_{k=0}^n T^k - T^{k+1} = I - T^{n+1}.
    \end{equation*}
    Take the limit as $n\to\infty$: 
    \begin{equation*}
        (I-T)S = I - \lim_{n\to\infty} T^{n+1} = I
    \end{equation*}
    since $\norm{T}<1$ implies that $\lim_{n\to\infty} T^{n+1} = 0$. 
    Thus $(I-T)S = I$. By a similar argument, $S(I-T) = I$. Hence 
    $I-T$ is invertible and $(I-T)^{-1} = S = \sum_{n=0}^\infty T^n$.
\end{proof}

\begin{proposition}[First Resolvent Identity]
    Let $T:D(T)\to X$ be a closed linear operator. The followings are true. 
    \begin{thmenum}
        \item For all $\xi_1,\xi_2\in\rho(T)$, 
        \begin{equation*}
            R_T(\xi_1) - R_T(\xi_2) = (\xi_1-\xi_2)R_T(\xi_1)R_T(\xi_2).
        \end{equation*}
        \item For all $\xi\to\xi_0\in\rho(T)$, 
        \begin{equation*}
            \lim_{\xi\to\xi_0} \frac{R_T(\xi) - R_T(\xi_0)}{\xi - \xi_0} = R_T(\xi_0)^2.
        \end{equation*}
        \item If $\abs{\xi - \xi_0}<\norm{R_T(\xi)}^{-1}$, then
        \begin{equation*}
            R_T(\xi) = \sbrc{I - (\xi - \xi_0)R_T(\xi_0)}^{-1}R_T(\xi_0) = \sum_{n=0}^\infty (\xi - \xi_0)^n R_T(\xi_0)^{n+1}.
        \end{equation*}
    \end{thmenum} 
\end{proposition}
\begin{proof}
    For (a), write 
    \begin{equation*}
        \begin{split}
            \sbrc{R_T(\xi_1) - R_T(\xi_2)}(T - \xi_2I) 
            &= (T - \xi_1I)^{-1}(T - \xi_2I) - I \\ 
            &= (T - \xi_1I)^{-1}(T - \xi_1I) + (T - \xi_1I)^{-1}(\xi_1 - \xi_2) - I \\
            &= (T - \xi_1I)^{-1}(\xi_1 - \xi_2).
        \end{split}
    \end{equation*}
    Rearranging the equation gives 
    \begin{equation*}
        R_T(\xi_1) - R_T(\xi_2) = (\xi_1 - \xi_2)R_T(\xi_1)R_T(\xi_2).
    \end{equation*}

    For (b), using (a), 
    \begin{equation*}
        \lim_{\xi\to\xi_0} \frac{R_T(\xi) - R_T(\xi_0)}{\xi - \xi_0} 
        = \lim_{\xi\to\xi_0} R_T(\xi_0)R_T(\xi) = R_T(\xi_0)^2.
    \end{equation*}

    For (c), (a) implies 
    \begin{equation*}
        R_T(\xi) = \sbrc{I - (\xi - \xi_0)R_T(\xi_0)}^{-1}R_T(\xi_0) 
        = \sum_{n=0}^\infty (\xi - \xi_0)^n R_T(\xi_0)^{n+1}
    \end{equation*}
    since $\abs{\xi - \xi_0}<\norm{R_T(\xi)}^{-1}$ by the von Neumann series.
\end{proof}

\begin{example}
    $Tu = u'$ on $X = C[a,b]$ with $D(T) = C^1[a,b]$. 
    \begin{equation*}
        (T-\xi I)u = 0 \Leftrightarrow u' = \xi u \Leftrightarrow u(x) = Ce^{\xi x}
    \end{equation*}
    for all $C\in\R$. Thus $(T-\xi I)^{-1}$ does not exists for all $\xi\in\C$. 
    Hence $\rho(T) = \varnothing$ and $\sigma(T) = \C$.
\end{example}

\begin{example}
    Consider $Tu = u'$ on $X = C[0,1]$ and $D(T) = \Set{u\in C^1[0,1]}{u(0) = u(1) = 0}$. 
    Then 
    \begin{equation*}
        \begin{cases}
            (T-\xi I)u = v, & v\in C[0,1], \\
            u(0) = u(1) = 0.
        \end{cases}\quad\Rightarrow\quad 
        \begin{cases}
            u(x) = e^{-\xi x}\int_0^x e^{-\xi t}v(t)dt \\ 
            u(1) = 0.
        \end{cases}
    \end{equation*}
    Clearly, this is impossible for all $v\in C[0,1]$. Hence 
    \begin{equation*}
        \rho(T) = \varnothing \quad\text{and}\quad \sigma(T) = \C.
    \end{equation*}
\end{example}

\begin{example}
    Consider $Tu = u'$ on $X = C[0,1]$ and $D(T) = \Set{u\in C^1[0,1]}{u(0) = ku(1)}$. 
    Solving 
    \begin{equation*}
        \begin{cases}
            u' - \xi u = v, & v\in C[0,1], \\
            u(0) = ku(1).
        \end{cases}\quad\Rightarrow\quad
        \begin{cases}
            (e^{-\xi x}u)' = e^{-\xi x}v, \\
            u(0) = ku(1).
        \end{cases}
    \end{equation*}
    So 
    \begin{equation*}
        u(x) = c_1 e^{\xi x}\int_0^x e^{-\xi t}v(t)dt + c_2 e^{\xi x}\int_x^1 e^{-\xi t}v(t)dt,
    \end{equation*}
    for some $c_1,c_2\in\R$ that should be determined. The 
    boundary condition gives $c_2 = ke^\xi c_1$ and 
    \begin{equation*}
        u(x) = c_1e^{\xi x}\sbrc{\int_0^x e^{-\xi t}v(t)dt + ke^\xi\int_x^1 e^{-\xi t}v(t)dt}.
    \end{equation*}
    Then 
    \begin{equation*}
        (e^{-\xi x}u(x))' = c_1\sbrc{e^{-\xi x}v(x) - ke^\xi e^{-\xi x}v(x)} = e^{-\xi x}v(x) 
        \quad\Rightarrow\quad 
        c_1 = \frac{1}{1-ke^\xi}.  
    \end{equation*}
    Thus 
    \begin{equation*}
        R_T(\xi)v(x) = \frac{e^{\xi x}}{1-ke^\xi}\sbrc{\int_0^x e^{-\xi t}v(t)dt + ke^\xi\int_x^1 e^{-\xi t}v(t)dt}.
    \end{equation*}
    We see that 
    \begin{equation*}
        \sigma(T) = \Set{\xi\in\C}{1 - ke^\xi = 0} = \Set{\xi\in\C}{\xi = -\log k + 2\pi i n, n\in\Z}
        \quad\text{and}\quad 
        \rho(T) = \C\setminus\sigma(T).
    \end{equation*}
\end{example}

\begin{definition}
    An operator is said to be with \textbf{compact resolvent} if there exists 
    $\xi\in\rho(T)$ such that $R_T(\xi)$ is compact.
\end{definition}
\begin{remark}
    If $T$ has compact resolvent, then for any $\xi\in\rho(T)$, 
    $R_T(\xi)$ is compact. This is because of the first resolvent identity. 
    If $R_T(\xi)$ is compact, then 
    \begin{equation*}
        R_T(\xi) = \sbrc{I + (\xi - \xi_0)R_T(\xi)}R_T(\xi_0)
    \end{equation*}
    is also compact. 
\end{remark}

\begin{theorem}\label{thm:bd_cpt_spectrum}
    $T\in B(X)$. Then $\sigma(T)$ is compact and 
    \begin{equation*}
        \sup_{\xi\in\sigma(T)}\abs{\xi} \leq \norm{T} < \infty. 
    \end{equation*}
\end{theorem}
\begin{proof}
    $\sigma(T)$ is closed if and only if $\rho(T)$ is open. 
    Take $\xi_0\in\rho(T)$. Consider the ball 
    \begin{equation*}
        B = \Set{\lambda\in\C}{\abs{\lambda - \xi_0} < \norm{R_T(\xi_0)}^{-1}}. 
    \end{equation*}
    For $\lambda\in B$, 
    \begin{equation*}
        T - \lambda I = (T - \xi_0 I) + (\xi_0 - \lambda)I 
        = (T - \xi_0 I)\sbrc{I + (\xi_0 - \lambda)R_T(\xi_0)}.
    \end{equation*}
    Using the Neumann series, $I + (\xi_0 - \lambda)R_T(\xi_0)$ 
    is invertible since $\abs{\xi_0 - \lambda}\norm{R_T(\xi_0)} < 1$. 
    Hence $(T - \lambda I)^{-1}$ exists and bounded by the bounded inverse 
    theorem. Then $\lambda\in\rho(T)$. $\rho(T)$ is open and hence $\sigma(T)$ 
    is closed. For arbitrary $\lambda>\norm{T}$, the Neumann series 
    shows that $T - \lambda I$ is boundedly invertible. Hence $\lambda\notin\sigma(T)$. 
    Thus 
    \begin{equation*}
        \sup_{\xi\in\sigma(T)}\abs{\xi} \leq \norm{T} < \infty.
    \end{equation*}
    Using Heine-Borel theorem, $\sigma(T)$ is compact.
\end{proof}

\begin{theorem}\label{thm:transpose_inverse_commute}
    Let $T:D(T)\dsubset X\to Y$ be a closed linear operator. 
    \begin{thmenum}
        \item If $T^{-1}$ exists and is bounded, then $(T')^{-1}$ 
        exists and is bounded, and $(T')^{-1} = (T^{-1})'$. 
        \item If $(T')^{-1}$ exists and is bounded, then $T^{-1}$ exists and is bounded, 
        and $T^{-1} = (T')^{-1}$. 
    \end{thmenum}
\end{theorem}
\begin{proof}
    (a) Assume first that $T^{-1}$ exists and is bounded. We first check 
    the identity $(T')^{-1} = (T^{-1})'$. For $g\in D(T')$, 
    \begin{equation*}
        (T^{-1})'T'g = (T'g)T^{-1} = gTT^{-1} = g\quad\Rightarrow\quad 
        (T^{-1})'T' = I.
    \end{equation*}
    For the other side, let $f\in X'$.
    \begin{equation*}
        T'(T^{-1})'f = ((T^{-1})'f)T = f(T^{-1}T) = fI = f.
        \quad\Rightarrow\quad 
        T'T^{-1} = I.
    \end{equation*}
    Hence $(T')^{-1} = (T^{-1})'$. Now we show that $(T')^{-1}$ is bounded. 
    \begin{equation*}
        \begin{split}
            \norm{(T')^{-1}} &= \sup_{\norm{f} = 1} \norm{(T')^{-1}f}
            = \sup_{\norm{f} = 1}\sup_{\norm{y} = 1} \abs{(T')^{-1}f(y)} 
            = \sup_{\norm{f} = 1}\sup_{\norm{y} = 1} \abs{f(T^{-1}y)} \\
            &\leq \sup_{\norm{f} = 1}\sup_{\norm{y} = 1} \norm{f}\norm{T^{-1}}\norm{y} 
            = \norm{T^{-1}}.
        \end{split}
    \end{equation*}
    (b) can be shown in a similar way.
\end{proof}

\begin{theorem}
    Let $T:D(T)\dsubset X\to X$ be closed linear operator. Then 
    \begin{thmenum}
        \item $R_{T'}(\overline{\xi}) = R_T(\xi)'$ for all $\xi\in\rho(T)$. 
        \item $\rho(T') = \Set{\overline{\lambda}}{\lambda\in\rho(T)}$ and 
        $\sigma(T') = \Set{\overline{\lambda}}{\lambda\in\sigma(T)}$.
    \end{thmenum}
\end{theorem}
\begin{proof}
    We first prove (a). Let $\lambda\in\rho(T)$. For all $f\in X'$, 
    \begin{equation*}
        \inp{\lambda f}{x} = \inp{f}{\overline{\lambda}Ix}\quad\forall x\in X. 
        \quad\Rightarrow\quad 
        f(\lambda I) = \overline{\lambda}f = \overline{\lambda}If.
    \end{equation*}
    Thus 
    \begin{equation*}
        (T-\lambda I)'f = f(T-\lambda I) = fT - f(\lambda I) 
        = fT - \overline{\lambda}If = (T'-\overline{\lambda}I)f.
    \end{equation*}
    Hence $(T-\lambda I)' = (T'-\overline{\lambda}I)$. Let $x_n\to x$ 
    in $X$ and $(T-\lambda I)x_n\to y$ in $X$. $x$ lies in $D(T-\lambda I) 
    = D(T)$. By the closedness of $T$, 
    \begin{equation*}
        (T-\lambda I)x_n = Tx_n - \lambda x_n \to Tx - \lambda x = (T-\lambda I)x.
    \end{equation*}
    On the other hand, $(T-\lambda I)x_n\to y$ so $(T-\lambda I)x = y$ and 
    $T-\lambda I$ is closed. Since $T-\lambda I$ is closed, densely defined 
    and invertible, 
    \begin{equation*}
        R_{T'}(\overline{\lambda}) = (T'-\overline{\lambda}I)^{-1} 
        = \pth{(T-\lambda I)'}^{-1} = ((T-\lambda I)^{-1})' = R_T(\lambda)'.
    \end{equation*}

    For (b), let $\lambda\in\rho(T)$. Then $R_T(\lambda)$ exists 
    and is bounded. Then $R_T(\lambda)':X'\to X'$ defined by 
    $R_T(\lambda)'f = fR_T(\lambda)$ also exists and 
    \begin{equation*}
        \norm{R_T(\lambda)'f} = \sup_{\norm{x}=1}\abs{fR_T(\lambda)x} 
        \leq \sup_{\norm{x}=1}\norm{f}\norm{R_T(\lambda)}\norm{x} 
        = \norm{f}\norm{R_T(\lambda)},
    \end{equation*} 
    so $R_T(\lambda)'$ is bounded. It now follows from (b) that 
    $R_{T'}(\overline{\lambda}) = R_T(\lambda)'$ exists and is bounded. 
    Thus $\overline{\lambda}\in\rho(T')$. 

    Now let $\lambda\in\sigma(T)$. If $\lambda$ is an eigenvalue, then 
    $T-\lambda I$ is not invertible. Thus from \cref{thm:transpose_inverse_commute}, 
    $T' - \overline{\lambda}I = (T-\lambda I)'$ is not invertible. 
    Hence $\overline{\lambda}\in\sigma(T')$. If $\lambda$ is such that 
    $R_T(\lambda)$ exists but is not bounded, then $R_T(\lambda)'$ exists 
    but is not bounded either, since 
    \begin{equation*}
        \infty = \norm{R_T(\lambda)x} = \sup_{\norm{f}=1}\abs{fR_T(\lambda)x} 
        \leq \sup_{\norm{f}=1} \abs{(R_T(\lambda)'f)x} 
        \leq \sup_{\norm{f}=1}\norm{R_T(\lambda)'f}\norm{x} 
        = \norm{R_T(\lambda)'}\norm{x}.
    \end{equation*}
    From the proof of (b), we have seen that $(T-\lambda I)' = (T'-\overline{\lambda}I)$. 
    Thus by (b), 
    \begin{equation*}
        R_{T'}(\overline{\lambda}) = (T'-\overline{\lambda}I)^{-1} 
        = \pth{(T-\lambda I)'}^{-1} = ((T-\lambda I)^{-1})' = R_T(\lambda)'
    \end{equation*}
    is not bounded either. Hence $\overline{\lambda}\in\sigma(T')$. 
    It follows that $\rho(T')$ contains the mirror image of $\rho(T)$ 
    and also $\sigma(T')$ contains the mirror image of $\sigma(T)$. Since 
    $\rho(T)\cap\sigma(T) = \varnothing$ and $\rho(T)\cup\sigma(T) = \C$, 
    we conclude that $\rho(T')$ and $\sigma(T')$ are exactly the mirror 
    images of $\rho(T)$ and $\sigma(T)$ with respect to the real axis.
\end{proof}
\begin{remark}
    If $X = \H$, then $T' = T^*$, and if $\lambda\in\sigma(T)$, 
    then $\overline{\lambda}\in\sigma(T^*) = \sigma(T')$.
\end{remark}

\begin{lemma}[Riesz]
    Let $X$ be a normed vector space with $\dim X = \infty$. 
    Let $Y$ be a proper closed subspace of $X$. Then for all 
    $\alpha\in(0,1)$, there exists $x\in X$ with $\norm{x} = 1$ 
    such that $\norm{x-y}\geq \alpha$ for all $y\in Y$.
\end{lemma}
\begin{proof}
    Fix $v\in X\setminus Y$. Let $\beta = \inf_{y\in Y} \norm{v - y}$. 
    Since $y$ is closed, $\beta > 0$. For all $\alpha\in (0,1)$, 
    there is a $y_0\in Y$ such that $\beta\leq\norm{v - y_0}\leq \beta/\alpha$. 
    Let $z = \frac{v-y_0}{\norm{v-y_0}}$ so $\norm{z} = 1$. We claim 
    that $\norm{z-y}\geq \alpha$ for all $y\in Y$. Indeed, 
    \begin{equation*}
        \norm{z-y} = \frac{1}{\norm{v-y_0}}\norm{v - y_0 - \norm{v-y_0}y} 
        = \frac{1}{\norm{v-y_0}}\norm{v - (y_0 + \norm{v-y_0}y)} 
        \geq \frac{1}{\norm{v-y_0}}\beta
    \end{equation*}
    by the definition of $\beta$. Hence, 
    \begin{equation*}
        \norm{z-y} \geq \frac{\beta}{\norm{v-y_0}} 
        \geq \frac{\beta}{\beta/\alpha} = \alpha. 
    \end{equation*}
    Since $y$ is arbitrary, $z$ is the desired vector.
\end{proof}

\begin{proposition}\label{prop:compact_image_closed}
    Let $T\in B(X)$ be a compact operator. Then $(T-I)(X)$ is closed. 
\end{proposition}
\begin{proof}
    Let $x_n\in X$ be a sequence such that $(T-I)x_n\to y$. We first 
    show that $d(x_n, \ker(T-I))$ is bounded. Suppose not. We can 
    find a divergent subsequence, say $x_n$, and define 
    $z_n = x_n/\norm{x_n + \ker(T-I)}_{X/\ker(T-I)}$. Now 
    \begin{equation*}
        \norm{x_n + \ker(T-I)}_{X/\ker(T-I)} = d(x_n, \ker(T-I))
    \end{equation*}
    is unbounded. Then 
    \begin{equation*}
        (T-I)z_n = \frac{(T-I)x_n}{\norm{x_n + \ker(T-I)}_{X/\ker(T-I)}} 
        \to 0.
    \end{equation*}
    Notice that $z_n = Tz_n - (T-I)z_n$. By the compactness of $T$,
    we may choose a subsequence $z_{n_k}$ such that $Tz_{n_k}\to z\in X$ 
    and thus $z_{n_k}\to z$. It follows that $(T-I)z = 0$ and 
    $z\in\ker(T-I)$, so $z+\ker(T-I)$ is a zero vector in 
    $X/\ker(T-I)$. On the other hand, $z_n$ is a sequence of unit vectors 
    in $X/\ker(T-I)$, a contradiction. Hence $d(x_n, \ker(T-I))$ is bounded. 

    Now for $x_n$, since $d(x_n, \ker(T-I))$ is bounded, we can 
    find a sequence $y_n\in\ker(T-I)$ such that $x_n-y_n$ is bounded. 
    Since $T$ is compact, we can find a subsequence $x_{n_k} - y_{n_k}$ such that 
    \begin{equation*}
        x_{n_k} - y_{n_k} = T(x_{n_k} - y_{n_k}) - (T-I)(x_{n_k} - y_{n_k}) 
        = T(x_{n_k} - y_{n_k}) - (T-I)x_{n_k}
    \end{equation*}
    is convergent, say to $x$. Then
    \begin{equation*}
        (T-I)x = \lim_{k\to\infty} (T-I)x_{n_k} = y
    \end{equation*}
    lies in $(T-I)(X)$. Hence $(T-I)(X)$ is closed.
\end{proof}

\begin{theorem}[Spectral Theorem for Compact Operators]
    Let $T\in B(X)$ be a compact operator. Then 
    \begin{thmenum}
        \item Every non-zero $\lambda\in\sigma(T)$ is an eigenvalue 
        of $T$. 
        \item For each non-zero $\lambda\in\sigma(T)$, $\dim(E_\lambda)<\infty$. 
        \item $\sigma(T)$ has no limit point except possibly $0$.
        \item $\sigma(T)$ is at most countable.  
        \item If $\dim(X) = \infty$, then $0\in\sigma(T)$.
    \end{thmenum}
\end{theorem}
\begin{proof}
    For (a), let $\lambda\in\sigma(T)$ be non-zero. $T-\lambda I = \lambda(\lambda^{-1}T - I)$. 
    $T$ is compact if and only if $\lambda^{-1}T$ is compact and 
    hence the case reduced to the case where $\lambda = 1$. 

    Now suppose that $\lambda = 1$. If $1$ is not an eigenvalue, then 
    $T-I$ is injective and has no bounded inverse. It follows from the 
    bounded inverse theorem and \cref{prop:compact_image_closed} that 
    $(T-I)(X)$ is a proper closed subspace of $X$. 

    Put $Y_1 = (T-I)(X)$ and $Y_2 = (T-I)^2(X)$. Since $T-I$ is injective, 
    $Y_2$ is a proper closed subspace of $Y_1$. Define $Y_n = (T-I)^n(X)$ for $n\geq 1$. 
    We obtain a sequence of proper closed subspaces 
    \begin{equation*}
        Y_1\supset Y_2\supset Y_3\supset \cdots.
    \end{equation*}
    By the Riesz lemma, we can choose a sequence of unit vectors 
    $y_n\in Y_n$ such that $d(y_n, Y_{n+1})\geq 1/2$. For $m>n$, 
    \begin{equation*}
        \norm{Ty_m - Ty_n} = \norm{(T-I)y_m + y_m - (T-I)y_n - y_n} 
        \geq d(y_n, Y_{n+1}) \geq \frac{1}{2}.
    \end{equation*}
    On the other hand, $y_n$ is a bounded sequence and hence $Ty_n$ has 
    a Cauchy subsequence, which is absurd. Hence $1$ is an eigenvalue 
    of $T$. Thus every non-zero $\lambda\in\sigma(T)$ is an eigenvalue of $T$. 

    For (b), let $\lambda\in\sigma(T)$ be a non-zero eigenvalue. 
    Suppose that $\dim(E_\lambda) = \infty$. Then we can find a 
    sequence of unit vectors $x_n\in E_\lambda$ such that $Tx_n 
    = \lambda x_n$ for all $n$ and $\norm{x_n - x_m}\geq \epsilon > 0$ 
    for all distinct $m,n$. Then 
    \begin{equation*}
        \norm{Tx_n - Tx_m} = \norm{\lambda x_n - \lambda x_m} 
        = \abs{\lambda}\norm{x_n - x_m} \geq \abs{\lambda}\epsilon > 0.
    \end{equation*}
    This shows that $Tx_n$ cannot have a Cauchy subsequence, 
    and $T$ is not compact, a contradiction. Hence $\dim(E_\lambda)<\infty$.  

    For (c), suppose that $\lambda_n\in\sigma(T)$ is a sequence of 
    distinct eigenvalues of $T$ such that $\abs{\lambda_n}\to\abs{\lambda} > 0$. 
    By (a), each $\lambda_n$ corresponds to an eigenvector $x_n$. 
    Set $Y_n = \spanby\set{x_1,\ldots,x_n}$. Then $Y_n$ forms a 
    strictly increasing sequence of closed subspaces of $X$. 
    For each $Y_n$, we can pick a unit vector $y_n\in Y$ 
    such that $d(y_n, Y_{n-1})\geq 1/2$. For $m>n$,
    \begin{equation*}
        \norm{Ty_m - Ty_n} = \norm{(T-\lambda_n I)y_m + \lambda_my_m - (T-\lambda_n I)y_n + \lambda_ny_n} 
    \end{equation*}
    Since 
    \begin{equation*}
        (T-\lambda_mI)y_m\in Y_{m-1},\quad  
        (T-\lambda_nI)y_n\in Y_{n-1}\subset Y_{m-1},\quad\text{and}\quad 
        \lambda_ny_n\in Y_n\subset Y_{m-1},
    \end{equation*}
    we have
    \begin{equation*}
        \norm{Ty_m - Ty_n} \geq d(y_n, Y_{n-1}) \geq \frac{1}{2}.
    \end{equation*}
    This contradicts to the compactness of $T$. Hence $\sigma(T)$ has 
    no limit point except possibly $0$. 

    For (d), we claim a fact that every uncountable set in a separable 
    metric space must have infinitely many limit points. Suppose not. 
    Let $A\subset X$ be an uncountable set. Since $X$ is separable, 
    we can consider a countable dense subset $D\subset X$. Let 
    \begin{equation*}
        \B = \Set{B_r(d)}{r\in\Q^+, d\in D}. 
    \end{equation*} 
    Consider $A_0 = \Set{a\in A}{a\not\in A'}$, the set of isolated points of $A$.
    For each $a\in A_0$, there is a ball $B_a\in\B$ such that 
    $B_a\cap A = \set{a}$. Since $\B$ is countable and each $B_a$ corresponds to 
    only one $a\in A_0$, we conclude that $A = A'\cup A_0$ is at most countable, 
    a contradiction. The fact follows. Now suppose that $\sigma(T)$ is 
    uncountable, it must have infinitely many limit points, contradicting 
    (c). Hence $\sigma(T)$ is at most countable.

    For (e), if $\sigma(T)$ contains no $0$, then $T^{-1}$ exists and 
    is bounded and $I = TT^{-1}$ is compact. This only happens if 
    $\dim(X) < \infty$. 
\end{proof}

\begin{example}
    $X = \ell^2(\N)$, $T:X\to X$ defined by $T(x_n) = (nx_n)$. 
    Then $T$ is compact. Consider $T_N:(x_n)\mapsto (x_1,\ldots,x_N/N,0,\ldots)$. 
    \begin{equation*}
        \norm{T_Nx - Tx} = \sum_{j=N+1}^\infty \frac{\abs{a_j}^2}{j^2} 
        \leq \frac{\norm{x}^2}{(N+1)^2}. 
    \end{equation*}
    Then $\norm{T_N - T}\leq \frac{1}{N+1}\to 0$ as $N\to\infty$. 
    Let $x_n\in X$ with $\norm{x_n} \leq 1$. Since $R(T_N)$ is 
    finite dimensional, we can find a subsequence $x_{n_k}$ such that 
    $T_Nx_{n_k}$ is Cauchy. Then 
    \begin{equation*}
        \begin{split}
            \norm{Tx_{n_k} - Tx_{n_l}} 
            &\leq \norm{Tx_{n_k} - T_Nx_{n_k}} + \norm{T_Nx_{n_k} - T_Nx_{n_l}} 
            + \norm{T_Nx_{n_l} - Tx_{n_l}} \\
            &\leq \norm{T_N - T}\pth{\norm{x_{n_k}} + \norm{x_{n_l}}} + \norm{T_Nx_{n_k} - T_Nx_{n_l}}\to 0. 
        \end{split}
    \end{equation*}
\end{example}

\begin{theorem}\label{thm:compact_resolvent}
    Let $T$ be a closed operator on $X$ with compact resolvent. 
    Then
    \begin{thmenum}
        \item $\sigma(T)$ consists entirely of eigenvalues of $T$, 
        \item $\dim(E_\lambda)<\infty$ for all eigenvalues $\lambda$ of $T$, 
        \item $R_T(\xi)$ is compact for all $\xi\in\rho(T)$.
    \end{thmenum}
\end{theorem}
\begin{proof}
    We first show (c). Let $\xi_0\in\rho(T)$ such that 
    $R_T(\xi_0)$ is compact. By the resolvent equation, 
    \begin{equation*}
        R_T(\xi) - R_T(\xi_0) = (\xi - \xi_0)R_T(\xi)R_T(\xi_0)
        \quad\Rightarrow\quad 
        R_T(\xi) = (I + (\xi - \xi_0)R_T(\xi))R_T(\xi_0)
    \end{equation*}
    for any $\xi\in\rho(T)$. Since $R_T(\xi_0)$ is compact and 
    \begin{equation*}
        \norm{I+(\xi - \xi_0)R_T(\xi)} \leq \norm{I} + \abs{\xi - \xi_0}\norm{R_T(\xi)} < \infty
    \end{equation*}
    for all $\xi\in\rho(T)$, we conclude that $R_T(\xi)$ is also 
    compact. 

    Next, we claim that $\sigma(R_T(\xi_0)) = f(\sigma(T))$, where $f(\xi) = \frac{1}{\xi-\xi_0}$.
    Let $\xi_0\in\rho(T)$ be such that $R_T(\xi_0)$ is compact. Observe that 
    \begin{equation*}
        T-\xi I = (T-\xi_0 I) - (\xi - \xi_0)I = (T-\xi_0 I)(I - (\xi - \xi_0)R_T(\xi_0)) 
        = -(\xi - \xi_0)(T - \xi_0 I)\pth{R_T(\xi_0) - \frac{1}{\xi-\xi_0}I}.
    \end{equation*}
    Then $T-\xi I$ has bounded inverse if and only if $R_T(\xi_0) - \frac{1}{\xi-\xi_0}I$ 
    has bounded inverse. Thus $\xi\in\rho(T)$ if and only if $(\xi - \xi_0)^{-1} \in\rho(R_T(\xi_0))$. 
    The claim follows. 

    Now let $f(\lambda) = \frac{1}{\lambda - \xi_0}\in\sigma(R_T(\xi_0))$. 
    Then if $f(\lambda) = 0$, $\lambda$ cannot be finite. We only need to 
    deal with the case where $f(\lambda)\neq 0$. We claim that if $\mu\in\sigma(R_T(\xi_0))$ 
    is non-zero, then $\mu$ is an eigenvalue of $R_T(\xi_0)$. 

    Suppose not. Then $\ker(R_T(\xi_0) - \mu I) = \set{0}$. We shall 
    deduce that $(R_T(\xi_0) - \mu I)(X) = X$. Assume again that 
    this is not the case. Then $X_1 = (R_T(\xi_0) - \mu I)(X)\subset X$ is 
    a proper closed subspace since $R_T(\xi_0)$ is compact. Also, 
    $R_T(\xi_0)(X_1) \subset X_1$ since if $x\in X_1$, then there is $y\in X_1$ 
    and $z\in X$ such that 
    \begin{equation*}
        x = R_T(\xi_0)y = (R_T(\xi_0) - \mu I)y + \mu y = (R_T(\xi_0) - \mu I)y + \mu(R_T(\xi_0) - \mu I)z \in X_1.   
    \end{equation*}
    So $R_T(\xi_0)(X_1) \subset X_1$. Put $X_2 = (R_T(\xi_0) - \mu I)(X_1)$. 
    Then $X_2$ is a subspace of $X_1$ since if $x\in (R_T(\xi_0) - \mu I)(X_1)$, 
    there is $y\in X_1$ such that 
    \begin{equation*}
        x = (R_T(\xi_0) - \mu I)y = R_T(\xi_0)y - \mu y \in X_1.
    \end{equation*} 
    It is also a proper subspace of $X_1$ since if not, then we may 
    pick $y\in X\setminus X_1$ and there is $z\in X_1$ such that 
    \begin{equation*}
        (R_T(\xi_0) - \mu I)z = (R_T(\xi_0) - \mu I)y\quad\Rightarrow\quad 
        y = z\in X_1
    \end{equation*}
    since $R_T(\xi_0) - \mu I$ is injective. This is a contradiction 
    and $X_2\subset X_1$ is a proper closed subspace. Continue this 
    process, we can find a sequence of strictly decreasing closed subsapces 
    $X_1\supset X_2\supset\cdots$ such that $X_{n+1} = (R_T(\xi_0) - \mu I)(X_n)$. 
    Applying the Riesz lemma, we can construct a sequence $x_n\in X_n$, 
    $\norm{x_n} = 1$ and $d(\mu x_n, X_{n+1}) \geq 1/2$. 
    \begin{equation*}
        R_T(\xi_0)x_n - R_T(\xi_0)x_m 
        = (R_T(\xi_0)x_n - \mu x_n) + \mu(x_n - x_m) - (R_T(\xi_0)x_m - \mu x_m)
    \end{equation*}
    Suppose $n> m$. Then $X_{n+1}\subset X_n\subset X_{m+1}$ and 
    we conclude that 
    \begin{equation*}
        \norm{R_T(\xi_0)x_n - R_T(\xi_0)x_m} \geq d(\mu x_m, X_{m+1}) 
        \geq 1/2. 
    \end{equation*}
    This contradicts the fact that $R_T(\xi_0)$ is compact. Thus 
    $(R_T(\xi_0) - \mu I)(X) = X$ and by the bounded inverse theorem, 
    $R_T(\xi_0) - \mu I$ has bounded inverse and $\mu\in\rho(R_T(\xi_0))$, 
    a contradiction. Thus $\mu$ must be an eigenvalue of $R_T(\xi_0)$. 
    Thus for $\lambda\in\sigma(T)$, 
    \begin{equation*}
        (T - \xi_0 I)^{-1}x = \frac{1}{\lambda - \xi_0}x
        \quad\Rightarrow\quad 
        x = \frac{1}{\lambda - \xi_0}(Tx - \xi_0 x)
        \quad\Rightarrow\quad
        Tx - \xi_0 x = \lambda x - \xi_0 x
        \quad\Rightarrow\quad
        Tx = \lambda x.
    \end{equation*}
    Hence $\lambda$ is an eigenvalue of $T$ and (a) follows. 

    For any eigenvalue $\lambda$ of $T$ lies in $\sigma(T)$, 
    $(\lambda - \xi_0)^{-1}\in\sigma(R_T(\xi_0))$. 
    \begin{equation*}
        \begin{split}
            T-\lambda I = (T - \xi_0 I) - (\lambda - \xi_0)I 
            &\quad\Rightarrow\quad 
            R_T(\xi_0)(T-\lambda I) = I - (\lambda - \xi_0)R_T(\xi_0) \\
            &\quad\Rightarrow\quad
            \frac{1}{\xi_0 - \lambda}R_T(\xi_0)(T-\lambda I) = R_T(\xi_0) - \frac{1}{\lambda - \xi_0}I.
        \end{split}
    \end{equation*}
    $R_T(\xi_0)$ is invertible and the right hand side has finite nullity 
    since $R_T(\xi_0)$ is compact. Thus $\dim(\ker(T-\lambda I)) 
    = \dim(\ker(R_T(\xi_0) - \frac{1}{\lambda - \xi_0}I)) < \infty$. 
\end{proof}
\begin{remark}
    If $T$ is a bounded closed operator with compact resolvent, 
    then $\dim(X) < \infty$. Indeed, suppose that $\dim(X) = \infty$. 
    Let $\xi\in\rho(T)$. Since $\xi\in\rho(T)$, we shall write 
    $T = \xi I + (T-\xi I)$. Then $R_T(\xi)T = \xi R_T(\xi) + I$. 
    Hence $I = R_T(\xi)(T-\xi I)$ is compact. Thus $\dim(X) < \infty$, 
    a contradiction. Hence $\dim(X) < \infty$.
\end{remark}

\begin{theorem}[Riesz Projection]
    Let $T$ be a closed operator on $X$ and $\lambda\in\sigma(T)$ 
    is an isolated point of $\sigma(T)$. Then there is an associated 
    eigen-projection 
    \begin{equation*}
        Pv = -\frac{1}{2\pi i}\oint_{\Gamma} R_T(\zeta)vd\zeta,
    \end{equation*}
    where $\Gamma$ is a simple closed curve enclose only $\lambda$.
\end{theorem}
\begin{proof}
    We verify that $P$ is indeed a projection, i.e. $P^2 = P$. 
    \begin{equation*}
        P^2 = \pth{\frac{1}{2\pi i}}^2\oint_{\Gamma}\oint_{\Gamma} R_T(\zeta)R_T(\eta)d\zeta d\eta 
        = \pth{\frac{1}{2\pi i}}^2\oint_{\Gamma}\oint_{\Gamma} \frac{R_T(\zeta)-R_T(\eta)}{\zeta - \eta}d\zeta d\eta.
    \end{equation*}
    We can shrink one of the contours, say $\Gamma_1$ enclosed by $\Gamma_2$, and 
    \begin{equation*}
        \begin{split}
            P^2 &= \pth{\frac{1}{2\pi i}}^2\oint_{\Gamma_2}R_T(\zeta)\oint_{\Gamma_1} \frac{1}{\zeta - \eta}d\eta d\zeta 
            - \pth{\frac{1}{2\pi i}}^2\oint_{\Gamma_1}R_T(\eta)\oint_{\Gamma_2} \frac{1}{\zeta - \eta}d\zeta d\eta \\
            &= - \pth{\frac{1}{2\pi i}}^2\oint_{\Gamma_1}R_T(\eta)(2\pi i) d\eta = P.
        \end{split}
    \end{equation*}
    Hence $P^2 = P$ and $P$ must be some projection on some subspace. 
\end{proof}
\begin{remark}
    With the eigen-projection $P$, we may write $X = M_1\oplus M_2$ 
    and decompose $T = T_1 + T_2$, where $M_1 = P(X)$, 
    and $M_2 = (I-P)(X)$, and $T_1 = TP$, $T_2 = T(I-P)$. 
    Now expanding $R_{T_1}(\lambda)$ by the Neumann series, 
    \begin{equation*}
        R_{T_1}(\xi) = R_T(\xi)P = \frac{-P}{\xi - \lambda} - \sum_{n=1}^\infty \frac{D^n}{(\xi - \lambda)^{n+1}},
    \end{equation*}
    where 
    \begin{equation*}
        D = (T-\lambda I)P = -\frac{1}{2\pi i}\oint_{\Gamma}(\xi - \lambda)R_T(\xi)d\xi.
    \end{equation*}
    To see this expression, we can expand $R_T(\xi)$ by the Laurent series 
    \begin{equation*}
        R_T(\xi) = \sum_{n=-k}^\infty C_n(\xi - \lambda)^n, 
        \quad\text{where } C_n = \frac{1}{2\pi i}\oint_{\Gamma} (\xi - \lambda)^{-n-1}R_T(\xi)d\xi. 
    \end{equation*}
    Note that $C_{-1} = -P$. Thus 
    \begin{equation*}
        R_T(\xi)P = \frac{C_{-k}P}{(\xi - \lambda)^{k}} + \cdots + \frac{C_{-2}P}{(\xi-\lambda)^2} + \frac{C_{-1}P}{\xi - \lambda} + C_0P + \cdots
    \end{equation*}
    Since $C_{-1}P = -P$, matching the $C_{-2}P$ term, 
    \begin{equation*}
        D = -C_{-2} = -\frac{1}{2\pi i}\oint_{\Gamma}(\xi - \lambda)R_T(\xi)d\xi.
    \end{equation*}
\end{remark}
\begin{remark}
    For the case where we have several isolated eigenvalues $\lambda_1,\ldots,\lambda_K$, 
    \begin{equation*}
        TP = \sum_{k=1}^K \lambda_k P_k + D_k, 
    \end{equation*}
    with $P_k$ and $D_k$ defined with respect to $\lambda_k$. We also have 
    \begin{thmenum}
        \item $P_kD_k = D_kP_k = D_k$.
        \item $P_kP_j = \delta_{kj}P_k$.
    \end{thmenum}
\end{remark}

\begin{example}
    Let $X = C[0,\pi]$ and $D(T) = \Set{u\in C^2[0,\pi]}{u'(0) = u'(\pi) = 0}$.
    Define $T:D(T)\to X$ by $Tu = -u''$. We solve the differential equation 
    \begin{equation*}
        \begin{cases}
            -u'' - \lambda u = v,\\
            u'(0) = u'(\pi) = 0.
        \end{cases}
    \end{equation*}
    i.e. 
    \begin{equation*}
        u = R_T(\lambda)v = (T-\lambda I)^{-1}v.
    \end{equation*}
    For suitable $\lambda$, we seek to write 
    \begin{equation*}
        R_T(\lambda)v = (T-\lambda I)^{-1}v = \int_0^\pi G(x,t)v(t)dt,
    \end{equation*}
    where $G(x,t)$ is the Green's function. We characterize 
    $G$ by the following differential equation
    \begin{equation*}
        \begin{cases}
            -G_{xx}(x,t)-\lambda G(x,t) = \delta(x-t) \\ 
            G_x(0,t) = G_x(\pi,t) = 0 \\ 
            G_x(t^+,t) - G_x(t^-,t) = -1 \\
            G(t^+,t) = G(t^-,t).
        \end{cases}
    \end{equation*}
    For $x\neq t$, we have 
    \begin{equation*}
        -G_{xx}(x,t)-\lambda G(x,t) = 0\quad\Rightarrow\quad 
        G(x,t) = \begin{cases}
            A(t)\cos(\sqrt{\lambda}x) + B(t)\sin(\sqrt{\lambda}x) & x<t,\\
            C(t)\cos(\sqrt{\lambda}x) + D(t)\sin(\sqrt{\lambda}x) & x>t.
        \end{cases}
    \end{equation*}
    Taking the derivative and using the boundary conditions,
    \begin{equation*}
        \begin{cases}
            G_x(0,t) = -A(t)\sqrt{\lambda}\sin(\sqrt{\lambda}0) + B(t)\sqrt{\lambda}\cos(\sqrt{\lambda}0) = 0 \quad\Rightarrow\quad B(t) = 0,\\ 
            G_x(\pi,t) = -C(t)\sqrt{\lambda}\sin(\sqrt{\lambda}\pi) + D(t)\sqrt{\lambda}\cos(\sqrt{\lambda}\pi) = 0 \quad\Rightarrow\quad D(t) = C(t)\tan(\sqrt{\lambda}\pi).\\
        \end{cases}
    \end{equation*}
    Also, 
    \begin{equation*}
        G_x(t^+,t) - G_x(t^-,t) = -C(t)\sqrt{\lambda}\sin(\sqrt{\lambda}t) + C(t)\tan(\sqrt{\lambda}\pi)\sqrt{\lambda}\cos(\sqrt{\lambda}t) 
        + A(t)\sqrt{\lambda}\sin(\sqrt{\lambda}t)  = -1.
    \end{equation*}
    Thus 
    \begin{equation*}
        \sqrt{\lambda}\sin(\sqrt{\lambda}t)A(t) = C(t)\pth{\sqrt{\lambda}\sin(\sqrt{\lambda}t) - \sqrt{\lambda}\tan(\sqrt{\lambda}\pi)\cos(\sqrt{\lambda}t)} - 1
    \end{equation*}
    The last condition gives 
    \begin{equation*}
        A(t)\cos(\sqrt{\lambda}t) = C(t)\cos(\sqrt{\lambda}t) + D(t)\sin(\sqrt{\lambda}t).
    \end{equation*}
    \begin{equation*}
        A(t) = C(t) + D(t)\tan(\sqrt{\lambda}t)
        = C(t)(1 + \tan(\sqrt{\lambda}\pi)\tan(\sqrt{\lambda}t))
    \end{equation*}
    Thus 
    \begin{equation*}
        C(t) = \frac{-\cos(\sqrt{\lambda}t)}{\sqrt{\lambda}\tan(\sqrt{\lambda}\pi)}.
    \end{equation*}
    \begin{equation*}
        A(t) = \frac{-1}{\sqrt{\lambda}}\pth{\frac{\cos(\sqrt{\lambda}t)}{\tan(\sqrt{\lambda}\pi)} + \sin(\sqrt{\lambda}t)}. 
    \end{equation*}
    \begin{equation*}
        D(t) = \frac{-\cos(\sqrt{\lambda}t)}{\sqrt{\lambda}}
    \end{equation*}
    Plugging the solution back into the Green's function, 
    \begin{equation*}
        G(x,t) = \begin{cases}
            -\frac{\cos(\sqrt{\lambda}(\pi-t))\cos(\sqrt{\lambda}x)}{\sqrt{\lambda}\sin(\sqrt{\lambda}\pi)}, & x\leq t,\\
            -\frac{\cos(\sqrt{\lambda}t)\cos(\sqrt{\lambda}(\pi - x))}{\sqrt{\lambda}\sin(\sqrt{\lambda}\pi)}, & x\geq t.
        \end{cases}
    \end{equation*}
    Hence 
    \begin{equation*}
        u(x) = \int_0^\pi G(x,t)v(t)dt.
    \end{equation*}
    The resolvent exists and is bounded if and only if $\lambda \neq k^2$ 
    for $k\in\Z$. Thus 
    \begin{equation*}
        \rho(T) = \C\setminus\Set{k^2}{k\in\Z}\quad\text{and}\quad 
        \sigma(T) = \Set{k^2}{k\in\Z}.
    \end{equation*}
    Now notice that for any bounded sequence $v_n\in C^2[0,\pi]$, $\norm{v_n}_\infty\leq M$, 
    \begin{equation*}
        \begin{split}
            \sup_{x\in[0,\pi]}\abs{\int_0^\pi G(x,t)v_n(t)dt}
            &\leq M\int_0^\pi \sup_{x\in[0,\pi]}\abs{G(x,t)}dt \\
        \end{split}
    \end{equation*}
    by the Cauchy-Schwarz inequality. Pick $\lambda = 1/4$. Then $\abs{G(x,t)} \leq 2$ 
    and 
    \begin{equation*}
        \norm{R_T(\lambda)v_n}_\infty \leq 2\pi M.
    \end{equation*}
    Thus $R_T(\lambda)v_n$ is bounded in $\norm{\cdot}_\infty$. In order 
    to apply the Arzelà-Ascoli theorem, we need to show that $R_T(\lambda)v_n$ 
    is equicontinuous. For any $x\in[0,\pi]$ and $x_k\to x$, 
    \begin{equation*}
        \abs{G(x_k,t) - G(x,t)} \leq 2\sup_{(x,t)\in[0,\pi]^2}\abs{G(x,t)} < \infty.
    \end{equation*}
    The right hand side is integrable on $[0,\pi]$. LDCT gives 
    \begin{equation*}
        \int_0^\pi \abs{G(x_k,t) - G(x,t)}dt \to 0\quad\text{as } k\to\infty
    \end{equation*}
    since $G(x,t)$ is continuous in $x$ by construction. Thus 
    \begin{equation*}
        \begin{split}
            \abs{\int_0^\pi G(x_k,t)v_n(t)dt - \int_0^\pi G(x,t)v_n(t)dt} 
            &\leq \int_0^\pi \abs{G(x_k,t) - G(x,t)}\abs{v_n(t)}dt \\ 
            &\leq M\int_0^\pi \abs{G(x_k,t) - G(x,t)}dt\to 0
        \end{split}
    \end{equation*}
    as $k\to\infty$. Since $x\mapsto \int_0^\pi G(x,t)v_n(t)dt$ is continuous 
    on a compact set, we obtain the uniform equicontinuity of $R_T(\lambda)v_n$. 
    From the Arzelà-Ascoli theorem, $R_T(\lambda)v_n$ has a subsequence Cauchy 
    in $\norm{\cdot}_\infty$, and thus $R_T(\lambda)$ is compact. 
    
    We conclude that $T$ is a closed operator with compact resolvent. 
    let $\lambda = k^2$ be an isolated eigenvalue. The 
    eigen-projection associated to $\lambda$ is given by 
    \begin{equation*}
        P_\lambda v = -\frac{1}{2\pi i}\oint_{\Gamma_\lambda} R_T(z)vdz
        = -\frac{1}{2\pi i}\oint_{\Gamma_\lambda} \int_0^\pi G_z(x,t)v(t)dt dz
        = \int_0^\pi -\frac{1}{2\pi i}\oint_{\Gamma_\lambda} G_z(x,t)dz v(t)dt.
    \end{equation*}
    \begin{equation*}
        -\frac{1}{2\pi i}\oint_{\Gamma_\lambda} G_z(x,t)dz = -\Res(G_z(x,t); z=k^2) 
        = -\lim_{z\to k^2} (z-k^2)G_z(x,t).
    \end{equation*}
    For $x\leq t$, using the L'Hospital's rule, we have
    \begin{equation*}
        \begin{split}
            &\lim_{z\to k^2} (z-k^2)G_z(x,t) 
            = \lim_{z\to k^2} (z-k^2)\frac{-\cos(\sqrt{z}(\pi-t))\cos(\sqrt{z}x)}{\sqrt{z}\sin(\sqrt{z}\pi)} \\ 
            &= \lim_{z\to k^2} \frac{-\cos(\sqrt{z}(\pi - t))\cos(\sqrt{z}x) + (z - k^2)\sbrc{\sin(\sqrt{z}(\pi - t))\cos(\sqrt{z}x)\frac{(\pi - t)}{2\sqrt{z}} + \cos(\sqrt{z}(\pi - t))\sin(\sqrt{z}x)\frac{x}{2\sqrt{z}}}}{\frac{\sin(\sqrt{z}\pi)}{2\sqrt{z}} + \sqrt{z}\cos(\sqrt{z}\pi)\frac{\pi}{2\sqrt{z}}} \\ 
            &= \begin{cases}
                (-1)^{k+1}\frac{2}{\pi}\cos(k(\pi - t))\cos(kx) & k\neq 0, x\leq t,\\
                -\frac{1}{\pi} & k = 0, x\leq t.
            \end{cases} \\ 
            &= \begin{cases}
                -\frac{2}{\pi}\cos(kt)\cos(kx) & k\neq 0, x\leq t,\\
                -\frac{1}{\pi} & k = 0, x\leq t.
            \end{cases}
        \end{split}
    \end{equation*}
    For $x\geq t$, by similar arguments, 
    \begin{equation*}
        \lim_{z\to k^2} (z-k^2)G_z(x,t) 
        = \begin{cases}
            -\frac{2}{\pi}\cos(kt)\cos(kx) & k\neq 0, x\geq t,\\
            -\frac{1}{\pi} & k = 0, x\geq t.
        \end{cases}
    \end{equation*}
    Hence 
    \begin{equation*}
        -\frac{1}{2\pi i}\oint_{\Gamma_\lambda} G_z(x,t)dz 
        = \begin{cases}
            \frac{2}{\pi}\cos(kx)\cos(kt) & k\neq 0, \\ 
            \frac{1}{\pi} & k = 0.
        \end{cases}
    \end{equation*}
    So 
    \begin{equation*}
        P_\lambda v = \begin{cases}
            \frac{2}{\pi}\int_0^\pi \cos(kt)v(t)dt\cos(kx) & k\neq 0, \\ 
            \frac{1}{\pi}\int_0^\pi v(t)dt & k = 0,
        \end{cases}
    \end{equation*}
    where $\lambda = k^2$. 
\end{example}
