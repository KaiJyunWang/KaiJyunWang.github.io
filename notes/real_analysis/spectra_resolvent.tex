\begin{definition}
    Let $T:D(T)\subset X\to X$ be a closed linear operator. The 
    \textbf{resolvent set} of $T$ is defined as 
    \begin{equation*}
        \rho(T) = \Set{\xi\in\C}{(T-\xi I) \text{ has bounded inverse on $X$}}.
    \end{equation*}
    The \textbf{spectrum} of $T$ is defined as $\sigma(T) = \C\setminus\rho(T)$. 
    $R_T(\xi) = (T-\xi I)^{-1}$ is called the \textbf{resolvent operator} of $T$.
\end{definition}
\begin{remark}
    $\xi\in\rho(T)$ if and only if $T-\xi I$ has the bounded inverse on $X$.
\end{remark}
\begin{remark}
    $\xi\in\sigma(T) = \C\setminus\rho(T)$ if either $T-\xi I$ is not invertible 
    or $T-\xi I$ is invertible but has range smaller than $X$. If $\dim X<\infty$, 
    $\sigma(T) = \Set{\lambda\in\C}{Tx = \lambda x \text{ for some } x\in X\setminus \set{0}}$.
\end{remark}

\begin{example}
    $X = C[a,b]$. $Tu = u'$ and $D(T) = C^1[a,b]$. $T$ is not invertible since 
    $T(u) = 0$ for every constant function $u$. Consider the following domains 
    \begin{itemize}
        \item $D_1 = \Set{u\in D(T)}{u(a) = 0}$, 
        \item $D_2 = \Set{u\in D(T)}{u(b) = 0}$, 
        \item $D_3 = \Set{u\in D(T)}{u(a) = ku(b)}$, 
        \item $D_0 = \Set{u\in D(T)}{u(a) = u(b) = 0}$.
    \end{itemize}
    $T_i = T|_{D_i}$ are invertible on $D_i$ for $i = 0,1,2,3$, but the 
    inverses are different. For example, 
    \begin{equation*}
        (T^{-1}_1 v)(x) = \int_a^x v(t)dt.
    \end{equation*}
\end{example}

\begin{theorem}[Neumann Series]
    Let $T:X\to X$ be a bounded lienar operator. If $\norm{T}<1$, then $I-T$ is invertible and
    \begin{equation*}
        (I-T)^{-1} = \sum_{n=0}^\infty T^n.
    \end{equation*}
\end{theorem}
\begin{proof}
    Denote $S_n = \sum_{k=0}^n T^k$. Compute that 
    \begin{equation*}
        (I-T)S_n = S_n - S_nT = \sum_{k=0}^n T^k - T^{k+1} = I - T^{n+1}.
    \end{equation*}
    Take the limit as $n\to\infty$: 
    \begin{equation*}
        (I-T)S = I - \lim_{n\to\infty} T^{n+1} = I
    \end{equation*}
    since $\norm{T}<1$ implies that $\lim_{n\to\infty} T^{n+1} = 0$. 
    Thus $(I-T)S = I$. By a similar argument, $S(I-T) = I$. Hence 
    $I-T$ is invertible and $(I-T)^{-1} = S = \sum_{n=0}^\infty T^n$.
\end{proof}

\begin{proposition}[First Resolvent Identity]
    Let $T:D(T)\to X$ be a closed linear operator. The followings are true. 
    \begin{thmenum}
        \item For all $\xi_1,\xi_2\in\rho(T)$, 
        \begin{equation*}
            R_T(\xi_1) - R_T(\xi_2) = (\xi_1-\xi_2)R_T(\xi_1)R_T(\xi_2).
        \end{equation*}
        \item For all $\xi\to\xi_0\in\rho(T)$, 
        \begin{equation*}
            \lim_{\xi\to\xi_0} \frac{R_T(\xi) - R_T(\xi_0)}{\xi - \xi_0} = R_T(\xi_0)^2.
        \end{equation*}
        \item If $\abs{\xi - \xi_0}<\norm{R_T(\xi)}^{-1}$, then
        \begin{equation*}
            R_T(\xi) = \sbrc{I - (\xi - \xi_0)R_T(\xi_0)}^{-1}R_T(\xi_0) = \sum_{n=0}^\infty (\xi - \xi_0)^n R_T(\xi_0)^{n+1}.
        \end{equation*}
    \end{thmenum} 
\end{proposition}
\begin{proof}
    For (a), write 
    \begin{equation*}
        \begin{split}
            \sbrc{R_T(\xi_1) - R_T(\xi_2)}(T - \xi_2I) 
            &= (T - \xi_1I)^{-1}(T - \xi_2I) - I \\ 
            &= (T - \xi_1I)^{-1}(T - \xi_1I) + (T - \xi_1I)^{-1}(\xi_1 - \xi_2) - I \\
            &= (T - \xi_1I)^{-1}(\xi_1 - \xi_2).
        \end{split}
    \end{equation*}
    Rearranging the equation gives 
    \begin{equation*}
        R_T(\xi_1) - R_T(\xi_2) = (\xi_1 - \xi_2)R_T(\xi_1)R_T(\xi_2).
    \end{equation*}

    For (b), using (a), 
    \begin{equation*}
        \lim_{\xi\to\xi_0} \frac{R_T(\xi) - R_T(\xi_0)}{\xi - \xi_0} 
        = \lim_{\xi\to\xi_0} R_T(\xi_0)R_T(\xi) = R_T(\xi_0)^2.
    \end{equation*}

    For (c), (a) implies 
    \begin{equation*}
        R_T(\xi) = \sbrc{I - (\xi - \xi_0)R_T(\xi_0)}^{-1}R_T(\xi_0) 
        = \sum_{n=0}^\infty (\xi - \xi_0)^n R_T(\xi_0)^{n+1}
    \end{equation*}
    since $\abs{\xi - \xi_0}<\norm{R_T(\xi)}^{-1}$ by the von Neumann series.
\end{proof}

\begin{example}
    $Tu = u'$ on $X = C[a,b]$ with $D(T) = C^1[a,b]$. 
    \begin{equation*}
        (T-\xi I)u = 0 \Leftrightarrow u' = \xi u \Leftrightarrow u(x) = Ce^{\xi x}
    \end{equation*}
    for all $C\in\R$. Thus $(T-\xi I)^{-1}$ does not exists for all $\xi\in\C$. 
    Hence $\rho(T) = \varnothing$ and $\sigma(T) = \C$.
\end{example}

\begin{example}
    Consider $Tu = u'$ on $X = C[0,1]$ and $D(T) = \Set{u\in C^1[0,1]}{u(0) = u(1) = 0}$. 
    Then 
    \begin{equation*}
        \begin{cases}
            (T-\xi I)u = v, & v\in C[0,1], \\
            u(0) = u(1) = 0.
        \end{cases}\quad\Rightarrow\quad 
        \begin{cases}
            u(x) = e^{-\xi x}\int_0^x e^{-\xi t}v(t)dt \\ 
            u(1) = 0.
        \end{cases}
    \end{equation*}
    Clearly, this is impossible for all $v\in C[0,1]$. Hence 
    \begin{equation*}
        \rho(T) = \varnothing \quad\text{and}\quad \sigma(T) = \C.
    \end{equation*}
\end{example}

\begin{example}
    Consider $Tu = u'$ on $X = C[0,1]$ and $D(T) = \Set{u\in C^1[0,1]}{u(0) = ku(1)}$. 
    Solving 
    \begin{equation*}
        \begin{cases}
            u' - \xi u = v, & v\in C[0,1], \\
            u(0) = ku(1).
        \end{cases}\quad\Rightarrow\quad
        \begin{cases}
            (e^{-\xi x}u)' = e^{-\xi x}v, \\
            u(0) = ku(1).
        \end{cases}
    \end{equation*}
    So 
    \begin{equation*}
        u(x) = c_1 e^{\xi x}\int_0^x e^{-\xi t}v(t)dt + c_2 e^{\xi x}\int_x^1 e^{-\xi t}v(t)dt,
    \end{equation*}
    for some $c_1,c_2\in\R$ that should be determined. The 
    boundary condition gives $c_2 = ke^\xi c_1$ and 
    \begin{equation*}
        u(x) = c_1e^{\xi x}\sbrc{\int_0^x e^{-\xi t}v(t)dt + ke^\xi\int_x^1 e^{-\xi t}v(t)dt}.
    \end{equation*}
    Then 
    \begin{equation*}
        (e^{-\xi x}u(x))' = c_1\sbrc{e^{-\xi x}v(x) - ke^\xi e^{-\xi x}v(x)} = e^{-\xi x}v(x) 
        \quad\Rightarrow\quad 
        c_1 = \frac{1}{1-ke^\xi}.  
    \end{equation*}
    Thus 
    \begin{equation*}
        R_T(\xi)v(x) = \frac{e^{\xi x}}{1-ke^\xi}\sbrc{\int_0^x e^{-\xi t}v(t)dt + ke^\xi\int_x^1 e^{-\xi t}v(t)dt}.
    \end{equation*}
    We see that 
    \begin{equation*}
        \sigma(T) = \Set{\xi\in\C}{1 - ke^\xi = 0} = \Set{\xi\in\C}{\xi = -\log k + 2\pi i n, n\in\Z}
        \quad\text{and}\quad 
        \rho(T) = \C\setminus\sigma(T).
    \end{equation*}
\end{example}

\begin{definition}
    An operator is said to be with \textbf{compact resolvent} if there exists 
    $\xi\in\rho(T)$ such that $R_T(\xi)$ is compact.
\end{definition}
\begin{remark}
    If $T$ has compact resolvent, then for any $\xi\in\rho(T)$, 
    $R_T(\xi)$ is compact. This is because of the first resolvent identity. 
    If $R_T(\xi)$ is compact, then 
    \begin{equation*}
        R_T(\xi) = \sbrc{I + (\xi - \xi_0)R_T(\xi)}R_T(\xi_0)
    \end{equation*}
    is also compact. 
\end{remark}

\begin{theorem}
    $T\in B(X)$. Then $\sigma(T)$ is compact and 
    \begin{equation*}
        \sup_{\xi\in\sigma(T)}\abs{\xi} \leq \norm{T} < \infty. 
    \end{equation*}
\end{theorem}
\begin{proof}
    $\sigma(T)$ is closed if and only if $\rho(T)$ is open. 
    Take $\xi_0\in\rho(T)$. Consider the ball 
    \begin{equation*}
        B = \Set{\lambda\in\C}{\abs{\lambda - \xi_0} < \norm{R_T(\xi_0)}^{-1}}. 
    \end{equation*}
    For $\lambda\in B$, 
    \begin{equation*}
        T - \lambda I = (T - \xi_0 I) + (\xi_0 - \lambda)I 
        = (T - \xi_0 I)\sbrc{I + (\xi_0 - \lambda)R_T(\xi_0)}.
    \end{equation*}
    Using the Neumann series, $I + (\xi_0 - \lambda)R_T(\xi_0)$ 
    is invertible since $\abs{\xi_0 - \lambda}\norm{R_T(\xi_0)} < 1$. 
    Hence $(T - \lambda I)^{-1}$ exists and bounded by the bounded inverse 
    theorem. Then $\lambda\in\rho(T)$. $\rho(T)$ is open and hence $\sigma(T)$ 
    is closed. For arbitrary $\lambda>\norm{T}$, the Neumann series 
    shows that $T - \lambda I$ is boundedly invertible. Hence $\lambda\notin\sigma(T)$. 
    Thus 
    \begin{equation*}
        \sup_{\xi\in\sigma(T)}\abs{\xi} \leq \norm{T} < \infty.
    \end{equation*}
    Using Heine-Borel theorem, $\sigma(T)$ is compact.
\end{proof}

\begin{theorem}\label{thm:transpose_inverse_commute}
    Let $T:D(T)\dsubset X\to Y$ be a closed linear operator. 
    \begin{thmenum}
        \item If $T^{-1}$ exists and is bounded, then $(T')^{-1}$ 
        exists and is bounded, and $(T')^{-1} = (T^{-1})'$. 
        \item If $(T')^{-1}$ exists and is bounded, then $T^{-1}$ exists and is bounded, 
        and $T^{-1} = (T')^{-1}$. 
    \end{thmenum}
\end{theorem}
\begin{proof}
    (a) Assume first that $T^{-1}$ exists and is bounded. We first check 
    the identity $(T')^{-1} = (T^{-1})'$. For $g\in D(T')$, 
    \begin{equation*}
        (T^{-1})'T'g = (T'g)T^{-1} = gTT^{-1} = g\quad\Rightarrow\quad 
        (T^{-1})'T' = I.
    \end{equation*}
    For the other side, let $f\in X'$.
    \begin{equation*}
        T'(T^{-1})'f = ((T^{-1})'f)T = f(T^{-1}T) = fI = f.
        \quad\Rightarrow\quad 
        T'T^{-1} = I.
    \end{equation*}
    Hence $(T')^{-1} = (T^{-1})'$. Now we show that $(T')^{-1}$ is bounded. 
    \begin{equation*}
        \begin{split}
            \norm{(T')^{-1}} &= \sup_{\norm{f} = 1} \norm{(T')^{-1}f}
            = \sup_{\norm{f} = 1}\sup_{\norm{y} = 1} \abs{(T')^{-1}f(y)} 
            = \sup_{\norm{f} = 1}\sup_{\norm{y} = 1} \abs{f(T^{-1}y)} \\
            &\leq \sup_{\norm{f} = 1}\sup_{\norm{y} = 1} \norm{f}\norm{T^{-1}}\norm{y} 
            = \norm{T^{-1}}.
        \end{split}
    \end{equation*}
    (b) can be shown in a similar way.
\end{proof}

\begin{theorem}
    Let $T:D(T)\dsubset X\to X$ be closed linear operator. Then 
    \begin{thmenum}
        \item $R_{T'}(\overline{\xi}) = R_T(\xi)'$ for all $\xi\in\rho(T)$. 
        \item $\rho(T') = \Set{\overline{\lambda}}{\lambda\in\rho(T)}$ and 
        $\sigma(T') = \Set{\overline{\lambda}}{\lambda\in\sigma(T)}$.
    \end{thmenum}
\end{theorem}
\begin{proof}
    We first prove (a). Let $\lambda\in\rho(T)$. For all $f\in X'$, 
    \begin{equation*}
        \inp{\lambda f}{x} = \inp{f}{\overline{\lambda}Ix}\quad\forall x\in X. 
        \quad\Rightarrow\quad 
        f(\lambda I) = \overline{\lambda}f = \overline{\lambda}If.
    \end{equation*}
    Thus 
    \begin{equation*}
        (T-\lambda I)'f = f(T-\lambda I) = fT - f(\lambda I) 
        = fT - \overline{\lambda}If = (T'-\overline{\lambda}I)f.
    \end{equation*}
    Hence $(T-\lambda I)' = (T'-\overline{\lambda}I)$. Let $x_n\to x$ 
    in $X$ and $(T-\lambda I)x_n\to y$ in $X$. $x$ lies in $D(T-\lambda I) 
    = D(T)$. By the closedness of $T$, 
    \begin{equation*}
        (T-\lambda I)x_n = Tx_n - \lambda x_n \to Tx - \lambda x = (T-\lambda I)x.
    \end{equation*}
    On the other hand, $(T-\lambda I)x_n\to y$ so $(T-\lambda I)x = y$ and 
    $T-\lambda I$ is closed. Since $T-\lambda I$ is closed, densely defined 
    and invertible, 
    \begin{equation*}
        R_{T'}(\overline{\lambda}) = (T'-\overline{\lambda}I)^{-1} 
        = \pth{(T-\lambda I)'}^{-1} = ((T-\lambda I)^{-1})' = R_T(\lambda)'.
    \end{equation*}

    For (b), let $\lambda\in\rho(T)$. Then $R_T(\lambda)$ exists 
    and is bounded. Then $R_T(\lambda)':X'\to X'$ defined by 
    $R_T(\lambda)'f = fR_T(\lambda)$ also exists and 
    \begin{equation*}
        \norm{R_T(\lambda)'f} = \sup_{\norm{x}=1}\abs{fR_T(\lambda)x} 
        \leq \sup_{\norm{x}=1}\norm{f}\norm{R_T(\lambda)}\norm{x} 
        = \norm{f}\norm{R_T(\lambda)},
    \end{equation*} 
    so $R_T(\lambda)'$ is bounded. It now follows from (b) that 
    $R_{T'}(\overline{\lambda}) = R_T(\lambda)'$ exists and is bounded. 
    Thus $\overline{\lambda}\in\rho(T')$. 

    Now let $\lambda\in\sigma(T)$. If $\lambda$ is an eigenvalue, then 
    $T-\lambda I$ is not invertible. Thus from \cref{thm:transpose_inverse_commute}, 
    $T' - \overline{\lambda}I = (T-\lambda I)'$ is not invertible. 
    Hence $\overline{\lambda}\in\sigma(T')$. If $\lambda$ is such that 
    $R_T(\lambda)$ exists but is not bounded, then $R_T(\lambda)'$ exists 
    but is not bounded either, since 
    \begin{equation*}
        \infty = \norm{R_T(\lambda)x} = \sup_{\norm{f}=1}\abs{fR_T(\lambda)x} 
        \leq \sup_{\norm{f}=1} \abs{(R_T(\lambda)'f)x} 
        \leq \sup_{\norm{f}=1}\norm{R_T(\lambda)'f}\norm{x} 
        = \norm{R_T(\lambda)'}\norm{x}.
    \end{equation*}
    From the proof of (b), we have seen that $(T-\lambda I)' = (T'-\overline{\lambda}I)$. 
    Thus by exercise 5.1 (b), 
    \begin{equation*}
        R_{T'}(\overline{\lambda}) = (T'-\overline{\lambda}I)^{-1} 
        = \pth{(T-\lambda I)'}^{-1} = ((T-\lambda I)^{-1})' = R_T(\lambda)'
    \end{equation*}
    is not bounded either. Hence $\overline{\lambda}\in\sigma(T')$. 
    It follows that $\rho(T')$ contains the mirror image of $\rho(T)$ 
    and also $\sigma(T')$ contains the mirror image of $\sigma(T)$. Since 
    $\rho(T)\cap\sigma(T) = \varnothing$ and $\rho(T)\cup\sigma(T) = \C$, 
    we conclude that $\rho(T')$ and $\sigma(T')$ are exactly the mirror 
    images of $\rho(T)$ and $\sigma(T)$ with respect to the real axis.
\end{proof}
\begin{remark}
    If $X = \H$, then $T' = T^*$, and if $\lambda\in\sigma(T)$, 
    then $\overline{\lambda}\in\sigma(T^*) = \sigma(T')$.
\end{remark}

\begin{lemma}[Riesz]
    Let $X$ be a normed vector space with $\dim X = \infty$. 
    Let $Y$ be a proper closed subspace of $X$. Then for all 
    $\alpha\in(0,1)$, there exists $x\in X$ with $\norm{x} = 1$ 
    such that $\norm{x-y}\geq \alpha$ for all $y\in Y$.
\end{lemma}
\begin{proof}
    Fix $v\in X\setminus Y$. Let $\beta = \inf_{y\in Y} \norm{v - y}$. 
    Since $y$ is closed, $\beta > 0$. For all $\alpha\in (0,1)$, 
    there is a $y_0\in Y$ such that $\beta\leq\norm{v - y_0}\leq \beta/\alpha$. 
    Let $z = \frac{v-y_0}{\norm{v-y_0}}$ so $\norm{z} = 1$. We claim 
    that $\norm{z-y}\geq \alpha$ for all $y\in Y$. Indeed, 
    \begin{equation*}
        \norm{z-y} = \frac{1}{\norm{v-y_0}}\norm{v - y_0 - \norm{v-y_0}y} 
        = \frac{1}{\norm{v-y_0}}\norm{v - (y_0 + \norm{v-y_0}y)} 
        \geq \frac{1}{\norm{v-y_0}}\beta
    \end{equation*}
    by the definition of $\beta$. Hence, 
    \begin{equation*}
        \norm{z-y} \geq \frac{\beta}{\norm{v-y_0}} 
        \geq \frac{\beta}{\beta/\alpha} = \alpha. 
    \end{equation*}
    Since $y$ is arbitrary, $z$ is the desired vector.
\end{proof}

\begin{proposition}\label{prop:compact_image_closed}
    Let $T\in B(X)$ be a compact operator. Then $(T-I)(X)$ is closed. 
\end{proposition}
\begin{proof}
    Let $x_n\in X$ be a sequence such that $(T-I)x_n\to y$. We first 
    show that $d(x_n, \ker(T-I))$ is bounded. Suppose not. We can 
    find a divergent subsequence, say $x_n$, and define 
    $z_n = x_n/\norm{x_n + \ker(T-I)}_{X/\ker(T-I)}$. Now 
    \begin{equation*}
        \norm{x_n + \ker(T-I)}_{X/\ker(T-I)} = d(x_n, \ker(T-I))
    \end{equation*}
    is unbounded. Then 
    \begin{equation*}
        (T-I)z_n = \frac{(T-I)x_n}{\norm{x_n + \ker(T-I)}_{X/\ker(T-I)}} 
        \to 0.
    \end{equation*}
    Notice that $z_n = Tz_n - (T-I)z_n$. By the compactness of $T$,
    we may choose a subsequence $z_{n_k}$ such that $Tz_{n_k}\to z\in X$ 
    and thus $z_{n_k}\to z$. It follows that $(T-I)z = 0$ and 
    $z\in\ker(T-I)$, so $z+\ker(T-I)$ is a zero vector in 
    $X/\ker(T-I)$. On the other hand, $z_n$ is a sequence of unit vectors 
    in $X/\ker(T-I)$, a contradiction. Hence $d(x_n, \ker(T-I))$ is bounded. 

    Now for $x_n$, since $d(x_n, \ker(T-I))$ is bounded, we can 
    find a sequence $y_n\in\ker(T-I)$ such that $x_n-y_n$ is bounded. 
    Since $T$ is compact, we can find a subsequence $x_{n_k} - y_{n_k}$ such that 
    \begin{equation*}
        x_{n_k} - y_{n_k} = T(x_{n_k} - y_{n_k}) - (T-I)(x_{n_k} - y_{n_k}) 
        = T(x_{n_k} - y_{n_k}) - (T-I)x_{n_k}
    \end{equation*}
    is convergent, say to $x$. Then
    \begin{equation*}
        (T-I)x = \lim_{k\to\infty} (T-I)x_{n_k} = y
    \end{equation*}
    lies in $(T-I)(X)$. Hence $(T-I)(X)$ is closed.
\end{proof}

\begin{theorem}[Spectral Theorem for Compact Operators]
    Let $T\in B(X)$ be a compact operator. Then 
    \begin{thmenum}
        \item Every non-zero $\lambda\in\sigma(T)$ is an eigenvalue 
        of $T$. 
        \item For each non-zero $\lambda\in\sigma(T)$, $\dim(E_\lambda)<\infty$. 
        \item $\sigma(T)$ is at most countable. 
        \item $\sigma(T)$ has no limit point except possibly $0$. 
        \item If $\dim(X) = \infty$, then $0\in\sigma(T)$.
    \end{thmenum}
\end{theorem}
\begin{proof}
    For (a), let $\lambda\in\sigma(T)$ be non-zero. $T-\lambda I = \lambda(\lambda^{-1}T - I)$. 
    $T$ is compact if and only if $\lambda^{-1}T$ is compact and 
    hence the case reduced to the case where $\lambda = 1$. 

    Now suppose that $\lambda = 1$. If $1$ is not an eigenvalue, then 
    $T-I$ is injective and has no bounded inverse. It follows from the 
    bounded inverse theorem and \cref{prop:compact_image_closed} that 
    $(T-I)(X)$ is a proper closed subspace of $X$. 

    Put $Y_1 = (T-I)(X)$ and $Y_2 = (T-I)^2(X)$. Since $T-I$ is injective, 
    $Y_2$ is a proper closed subspace of $Y_1$. Define $Y_n = (T-I)^n(X)$ for $n\geq 1$. 
    We obtain a sequence of proper closed subspaces 
    \begin{equation*}
        Y_1\supset Y_2\supset Y_3\supset \cdots.
    \end{equation*}
    By the Riesz lemma, we can choose a sequence of unit vectors 
    $y_n\in Y_n$ such that $d(y_n, Y_{n+1})\geq 1/2$. For $m>n$, 
    \begin{equation*}
        \norm{Ty_m - Ty_n} = \norm{(T-I)y_m + y_m - (T-I)y_n - y_n} 
        \geq d(y_n, Y_{n+1}) \geq \frac{1}{2}.
    \end{equation*}
    On the other hand, $y_n$ is a bounded sequence and hence $Ty_n$ has 
    a Cauchy subsequence, which is absurd. Hence $1$ is an eigenvalue 
    of $T$. Thus every non-zero $\lambda\in\sigma(T)$ is an eigenvalue of $T$. 

    For (b), let $\lambda\in\sigma(T)$ be a non-zero eigenvalue. 
\end{proof}