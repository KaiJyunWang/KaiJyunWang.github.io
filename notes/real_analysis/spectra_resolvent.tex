\begin{definition}
    Let $T:D(T)\subset X\to X$ be a closed linear operator. The 
    \textbf{resolvent set} of $T$ is defined as 
    \begin{equation*}
        \rho(T) = \Set{\xi\in\C}{(T-\xi I) \text{ has bounded inverse on $X$}}.
    \end{equation*}
    The \textbf{spectrum} of $T$ is defined as $\sigma(T) = \C\setminus\rho(T)$. 
    $R_T(\xi) = (T-\xi I)^{-1}$ is called the \textbf{resolvent operator} of $T$.
\end{definition}
\begin{remark}
    $\xi\in\rho(T)$ if and only if $T-\xi I$ has the bounded inverse on $X$.
\end{remark}
\begin{remark}
    $\xi\in\sigma(T) = \C\setminus\rho(T)$ if either $T-\xi I$ is not invertible 
    or $T-\xi I$ is invertible but has range smaller than $X$. If $\dim X<\infty$, 
    $\sigma(T) = \Set{\lambda\in\C}{Tx = \lambda x \text{ for some } x\in X\setminus \set{0}}$.
\end{remark}

\begin{example}
    $X = C[a,b]$. $Tu = u'$ and $D(T) = C^1[a,b]$. $T$ is not invertible since 
    $T(u) = 0$ for every constant function $u$. Consider the following domains 
    \begin{itemize}
        \item $D_1 = \Set{u\in D(T)}{u(a) = 0}$, 
        \item $D_2 = \Set{u\in D(T)}{u(b) = 0}$, 
        \item $D_3 = \Set{u\in D(T)}{u(a) = ku(b)}$, 
        \item $D_0 = \Set{u\in D(T)}{u(a) = u(b) = 0}$.
    \end{itemize}
    $T_i = T|_{D_i}$ are invertible on $D_i$ for $i = 0,1,2,3$, but the 
    inverses are different. For example, 
    \begin{equation*}
        (T^{-1}_1 v)(x) = \int_a^x v(t)dt.
    \end{equation*}
\end{example}

\begin{theorem}[von Neumann Series]
    Let $T:X\to X$ be a bounded lienar operator. If $\norm{T}<1$, then $I-T$ is invertible and
    \begin{equation*}
        (I-T)^{-1} = \sum_{n=0}^\infty T^n.
    \end{equation*}
\end{theorem}
\begin{proof}
    Denote $S_n = \sum_{k=0}^n T^k$. Compute that 
    \begin{equation*}
        (I-T)S_n = S_n - S_nT = \sum_{k=0}^n T^k - T^{k+1} = I - T^{n+1}.
    \end{equation*}
    Take the limit as $n\to\infty$: 
    \begin{equation*}
        (I-T)S = I - \lim_{n\to\infty} T^{n+1} = I
    \end{equation*}
    since $\norm{T}<1$ implies that $\lim_{n\to\infty} T^{n+1} = 0$. 
    Thus $(I-T)S = I$. By a similar argument, $S(I-T) = I$. Hence 
    $I-T$ is invertible and $(I-T)^{-1} = S = \sum_{n=0}^\infty T^n$.
\end{proof}

\begin{proposition}[First Resolvent Identity]
    Let $T:D(T)\to X$ be a closed linear operator. The followings are true. 
    \begin{thmenum}
        \item For all $\xi_1,\xi_2\in\rho(T)$, 
        \begin{equation*}
            R_T(\xi_1) - R_T(\xi_2) = (\xi_1-\xi_2)R_T(\xi_1)R_T(\xi_2).
        \end{equation*}
        \item For all $\xi\to\xi_0\in\rho(T)$, 
        \begin{equation*}
            \lim_{\xi\to\xi_0} \frac{R_T(\xi) - R_T(\xi_0)}{\xi - \xi_0} = R_T(\xi_0)^2.
        \end{equation*}
        \item If $\abs{\xi - \xi_0}<\norm{R_T(\xi)}^{-1}$, then
        \begin{equation*}
            R_T(\xi) = \sbrc{I - (\xi - \xi_0)R_T(\xi_0)}^{-1}R_T(\xi_0) = \sum_{n=0}^\infty (\xi - \xi_0)^n R_T(\xi_0)^{n+1}.
        \end{equation*}
    \end{thmenum} 
\end{proposition}
\begin{proof}
    For (a), write 
    \begin{equation*}
        \begin{split}
            \sbrc{R_T(\xi_1) - R_T(\xi_2)}(T - \xi_2I) 
            &= (T - \xi_1I)^{-1}(T - \xi_2I) - I \\ 
            &= (T - \xi_1I)^{-1}(T - \xi_1I) + (T - \xi_1I)^{-1}(\xi_1 - \xi_2) - I \\
            &= (T - \xi_1I)^{-1}(\xi_1 - \xi_2).
        \end{split}
    \end{equation*}
    Rearranging the equation gives 
    \begin{equation*}
        R_T(\xi_1) - R_T(\xi_2) = (\xi_1 - \xi_2)R_T(\xi_1)R_T(\xi_2).
    \end{equation*}

    For (b), using (a), 
    \begin{equation*}
        \lim_{\xi\to\xi_0} \frac{R_T(\xi) - R_T(\xi_0)}{\xi - \xi_0} 
        = \lim_{\xi\to\xi_0} R_T(\xi_0)R_T(\xi) = R_T(\xi_0)^2.
    \end{equation*}

    For (c), (a) implies 
    \begin{equation*}
        R_T(\xi) = \sbrc{I - (\xi - \xi_0)R_T(\xi_0)}^{-1}R_T(\xi_0) 
        = \sum_{n=0}^\infty (\xi - \xi_0)^n R_T(\xi_0)^{n+1}
    \end{equation*}
    since $\abs{\xi - \xi_0}<\norm{R_T(\xi)}^{-1}$ by the von Neumann series.
\end{proof}

\begin{definition}
    An operator is said to be with \textbf{compact resolvent} if there exists 
    $\xi\in\rho(T)$ such that $R_T(\xi)$ is compact.
\end{definition}

\begin{example}
    $Tu = u'$ on $X = C[a,b]$ with $D(T) = C^1[a,b]$. 
    \begin{equation*}
        (T-\xi I)u = 0 \Leftrightarrow u' = \xi u \Leftrightarrow u(x) = Ce^{\xi x}
    \end{equation*}
    for all $C\in\R$. Thus $(T-\xi I)$ does not exists for all $\xi\in\C$. 
    Hence $\rho(T) = \varnothing$ and $\sigma(T) = \C$.
\end{example}