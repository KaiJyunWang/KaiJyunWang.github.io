\begin{definition}
    The \textbf{Fourier series} of a function $f$ is given by 
    \begin{equation*}
        Sf(x) = \frac{a_0}{2} + \sum_{k=1}^\infty a_k \cos(kx) + b_k \sin(kx),
    \end{equation*}
    where the \textbf{Fourier coefficients} $a_k$ and $b_k$ are 
    given by 
    \begin{equation*}
        a_k = \frac{1}{\pi} \int_{-\pi}^\pi f(x) \cos(kx)dx, \quad
        b_k = \frac{1}{\pi} \int_{-\pi}^\pi f(x) \sin(kx)dx.
    \end{equation*}
    Or, alternatively, 
    \begin{equation*}
        Sf(x) = \sum_{k=-\infty}^\infty c_k e^{ikx},
    \end{equation*}
    where the \textbf{Fourier coefficients} $c_k$ are given by
    \begin{equation*}
        c_k = \frac{1}{2\pi} \int_{-\pi}^\pi f(x) e^{-ikx}dx.
    \end{equation*}
\end{definition}

\begin{definition}
    The \textbf{Truncated Fourier series} of a function $f$ is denoted by 
    \begin{equation*}
        S_N f(x) = \frac{a_0}{2} + \sum_{k=1}^N a_k \cos(kx) + b_k \sin(kx) = \sum_{k=-N}^N c_k e^{ikx}.
    \end{equation*}
\end{definition}

\begin{proposition}
    Let $a_k$ and $b_k$ be the Fourier coefficients of a function $f$. Then 
    \begin{thmenum}
        \item If $f\in\L^1$, $\abs{a_k}, \abs{b_k}\leq C\norm{f}_1$ for some constant $C>0$.
        \item If $f\in\L^\infty$, $\abs{a_k}, \abs{b_k}\leq C\norm{f}_\infty$ for some constant $C>0$.
    \end{thmenum}
\end{proposition}
\begin{proof}
    To see (a), compute that
    \begin{equation*}
        \abs{a_k} \leq \frac{1}{\pi}\int_{-\pi}^\pi \abs{f(x)}\abs{\cos(kx)}dx \leq \frac{1}{\pi}\int_{-\pi}^\pi \abs{f(x)}dx = C\norm{f}_1
    \end{equation*}
    and similarly for $b_k$. 
    For (b), 
    \begin{equation*}
        \abs{a_k} \leq \frac{1}{\pi}\int_{-\pi}^\pi \abs{f(x)}\abs{\cos(kx)}dx \leq \frac{1}{\pi}\int_{-\pi}^\pi \norm{f}_\infty dx = 2\norm{f}_\infty = C\norm{f}_\infty.
    \end{equation*}
    The proof for $b_k$ is analogous.
\end{proof}

\begin{lemma}[Riemann-Lebesgue \rom{1}]
    Let $f\in \L^1[a,b]$. Then 
    \begin{equation*}
        \lim_{n\to\infty} \int_a^b f(x)e^{-inx}dx = 0.
    \end{equation*}
\end{lemma}
\begin{proof}
    Let $\epsilon > 0$. Since $f\in\L^1[a,b]$, there is a step function 
    $g$ such that $\norm{f-g}_1 < \epsilon$. For any interval $E$, 
    \begin{equation*}
        \abs{\int_a^b \chi_E(x)e^{-inx}dx} \leq \abs{\int_E \cos(nx)dx} + \abs{\int_E \sin(nx)dx} 
        \leq \frac{2\pi}{n} \to 0
    \end{equation*}
    as $n\to\infty$. A step function is a linear combination of characteristic 
    functions of intervals, and thus $\abs{\int_a^b g(x)e^{-inx}dx} \to 0$ as $n\to\infty$.
    Therefore, 
    \begin{equation*}
        \begin{split}
            \abs{\int_a^b f(x)e^{-inx}dx} &\leq \abs{\int_a^b (f(x)-g(x))e^{-inx}dx} + \abs{\int_a^b g(x)e^{-inx}dx} \\
            &\leq \norm{f-g}_1 + \abs{\int_a^b g(x)e^{-inx}dx} \to 0
        \end{split}
    \end{equation*}
    as $n\to\infty$.
\end{proof}

\begin{definition}
    The space of piecewise continuous functions on $[a,b]$ is denoted by $PC[a,b]$. 
    The symbol $PC^n[a,b]$ denotes the space of functions having continuous derivatives up to order $n-1$, 
    with the $n$-th derivative being piecewise continuous.
\end{definition}

\begin{proposition}
    Let $f\in PC^1[-\pi,\pi]$ and 
    \begin{equation*}
        f(x) = \frac{a_0}{2} + \sum_{k=1}^\infty a_k \cos(kx) + b_k \sin(kx).
    \end{equation*}
    Then 
    \begin{equation*}
        f'(x) = \sum_{k=1}^\infty -ka_k \sin(kx) + kb_k \cos(kx).
    \end{equation*}
\end{proposition}
\begin{proof}
    Differentiation term by term gives the desired result.
\end{proof}
\begin{remark}
    If $f\in PC^n[-\pi,\pi]$, then
    \begin{equation*}
        \abs{a_k}, \abs{b_k} \leq \frac{\norm{f}_{PC^n}}{k^n},
    \end{equation*}
    where $\norm{f}_{PC^n} = \sum_{j=0}^n \norm{f^{(j)}}_\infty$.
\end{remark}

\begin{definition}
    Let $f$ be a function on $\R$. The \textbf{right-limit} and 
    the \textbf{left-limit} of $f$ at $x$ are defined by
    \begin{equation*}
        f(x^+) = \lim_{h\to 0^+} f(x+h), \quad f(x^-) = \lim_{h\to 0^+} f(x-h).
    \end{equation*}
\end{definition}

\begin{definition}
    A \textbf{kernel} is a function $k:X\times X\to\R$ such that 
    \begin{thmenum}
        \item $k(x,y) = k(y,x)$ for all $x,y\in X$,
        \item For finitely many points $x_1,\ldots,x_n\in X$ and scalars $a_1,\ldots,a_n\in\R$, 
        \begin{equation*}
            \sum_{i=1}^n \sum_{j=1}^n a_ia_jk(x_i,x_j) \geq 0.
        \end{equation*}
    \end{thmenum}
\end{definition}

\begin{definition}
    The \textbf{Dirichlet kernel} is defined by
    \begin{equation*}
        D_N(x) = \frac{1}{2\pi}\sum_{k=-N}^N e^{ikx}.
    \end{equation*}
\end{definition}
\begin{remark}
    The Dirichlet kernel can be simplified to 
    \begin{equation*}
        D_N(x) = \frac{\sin\pth{(N+1/2)x}}{2\pi\sin(x/2)}.
    \end{equation*}
    To see this, note that 
    \begin{equation*}
        2\pi D_N(x)(e^{ix}-1) = e^{i(N+1)x} - e^{-iNx} 
        = \frac{e^{i(N+1/2)x} - e^{-i(N+1/2)x}}{e^{ix/2} - e^{-ix/2}}.
    \end{equation*}
    And thus,
    \begin{equation*}
        D_N(x) = \frac{\sin\pth{(N+1/2)x}}{2\pi\sin(x/2)}.
    \end{equation*}
    Some other properties of the Dirichlet kernel include $D_N(-x) = D_N(x)$ 
    and $\int_{-\pi}^\pi D_N(x)dx = 1$.
\end{remark}

\begin{definition}
    Let $f,g:X\to\R$. The \textbf{convolution} of $f$ and $g$ is defined by
    \begin{equation*}
        (f*g)(x) = \int_X f(x-y)g(y)dy.
    \end{equation*}
\end{definition}

\begin{proposition}
    For any $2\pi$-periodic function $f\in PC$,
    \begin{equation*}
        S_Nf = D_N*f.
    \end{equation*}
\end{proposition}
\begin{proof}
    Compute that 
    \begin{equation*}
        \begin{split}
            S_Nf(x) &= \sum_{k=-N}^N c_k e^{ikx} = \sum_{k=-N}^N \frac{1}{2\pi} \int_{-\pi}^\pi f(y)e^{-iky}dy e^{ikx} \\
            &= \int_{-\pi}^\pi f(y) \frac{1}{2\pi}\sum_{k=-N}^N e^{ik(x-y)}dy = \int_{-\pi}^\pi f(y)D_N(x-y)dy = (D_N*f)(x).
        \end{split}
    \end{equation*}
\end{proof}

\begin{theorem}[Dirichlet-Jordan]
    Let $f$ be a $2\pi$-periodic function and piecewise Lipschitz. Then 
    \begin{equation*}
        \lim_{N\to\infty} S_Nf(x) = \frac{f(x^+)+f(x^-)}{2}.
    \end{equation*}
    In particular, if $f$ is continuous at $x$, then 
    \begin{equation*}
        \lim_{N\to\infty} S_Nf(x) = f(x).
    \end{equation*}
\end{theorem}
\begin{proof}
    Since $f$ is $2\pi$-periodic,
    \begin{equation*}
        \begin{split}
            S_Nf(x) &= \int_{-\pi}^\pi D_N(x-y)f(y)dy = \int_{-\pi}^\pi D_N(y)f(x-y)dy \\
            &= \int_0^\pi D_N(y)f(x-y)dy + \int_{-\pi}^0 D_N(y)f(x-y)dy \\
            &= \int_0^\pi D_N(y)f(x-y)dy + \int_0^\pi D_N(-y)f(x+y)dy \\
            &= \int_0^\pi D_N(y)\pth{f(x-y)+f(x+y)}dy.
        \end{split}
    \end{equation*}
    Notice that 
    \begin{equation*}
        \frac{1}{2}\pth{f(x^+)+f(x^-)} = \int_0^\pi D_N(y)\pth{f(x^+)+f(x^-)}dy.
    \end{equation*}
    Thus for given $x$, we have 
    \begin{equation*}
        \begin{split}
            \abs{S_Nf(x) - \frac{f(x^+)+f(x^-)}{2}} &\leq \abs{\int_0^\pi D_N(y)\pth{f(x+y)-f(x^+)}dy} \\
            &\quad + \abs{\int_0^\pi D_N(y)\pth{f(x-y)-f(x^-)}dy}.
        \end{split}
    \end{equation*}
    We claim that 
    \begin{equation*}
        \int_0^\pi D_N(y)\abs{f(x+y)-f(x^+)}dy \to 0
    \end{equation*}
    as $N\to\infty$ and the other integral is similar. There is a $\delta>0$ such that 
    $f$ is continuous on $[x,x+\delta]$. Thus $f$ is uniformly continuous on 
    $[x,x+\delta]$, and $\abs{f(x+y)-f(x^+)} < Cy$ for some constant $C>0$ and 
    $y\in[0,\delta]$. Then 
    \begin{equation*}
        \int_0^\delta \abs{D_N(y)}\abs{f(x+y)-f(x^+)}dy \leq C\int_0^\delta y\abs{D_N(y)}dy
        \leq C\int_0^\delta dy = C\delta, 
    \end{equation*}
    because $\abs{D_N(t)}\leq 1/\abs{t}$. On the other hand, 
    \begin{equation*}
        \begin{split}
            \int_\delta^\pi \abs{D_N(y)}\abs{f(x+y)-f(x^+)}dy 
            &\leq \frac{1}{2\pi}\int_\delta^\pi \frac{\sin((N+1/2)y)}{\sin(y/2)}\abs{f(x+y)-f(x^+)}dy \\
            &\leq \frac{1}{2\pi\sin(\delta/2)}\int_\delta^\pi \sin((N+1/2)y)g(y)dy \to 0,
        \end{split}
    \end{equation*}
    as $N\to\infty$ by the Riemann-Lebesgue lemma, where $g(y) = \abs{f(x+y)-f(x^+)}$ 
    is a continuous function on $[\delta,\pi]$. Hence we have
    \begin{equation*}
        \abs{\int_0^\pi D_N(y)\pth{f(x+y)-f(x^+)}dy} \to 0 \quad \text{as } N\to\infty.
    \end{equation*}
    We now see that 
    \begin{equation*}
        \abs{S_Nf(x) - \frac{f(x^+)+f(x^-)}{2}} \to 0 \quad \text{as } N\to\infty.
    \end{equation*}
    The pointwise convergence is achieved whenever $f$ is continuous since 
    $f(x^+)=f(x^-)=f(x)$.
\end{proof}

\begin{definition}
    The series $\sigma_Nf$ is defined by 
    \begin{equation*}
        \sigma_Nf(x) = \frac{1}{N+1}\sum_{n=0}^{N} S_nf(x).
    \end{equation*}
\end{definition}
\begin{remark}
    The series $\sigma_Nf$ is the Cesaro's mean of the Fourier series of $f$.
\end{remark}

\begin{definition}
    The \textbf{Fejer kernel} is defined by 
    \begin{equation*}
        F_N(x) = \frac{1}{N+1}\sum_{k=0}^N D_k(x).
    \end{equation*}
\end{definition}
\begin{remark}
    The Fejer kernel can be simplified to 
    \begin{equation*}
        F_N(x) = \frac{\sin^2\pth{\frac{N+1}{2}x}}{2\pi(N+1)\sin^2(x/2)}.
    \end{equation*}
    To see this, note that 
    \begin{equation*}
        \begin{split}
            F_N(t) &= \frac{1}{N+1}\sum_{k=0}^{N} D_k(t) 
            = \frac{1}{2\pi(N+1)}\sum_{k=0}^{N} \frac{\sin((k+1/2)t)}{\sin(t/2)} \\
            &= \frac{1}{2\pi(N+1)\sin^2(t/2)}\sum_{k=0}^{N} \sin((k+1/2)t)\sin(t/2) \\
            &= \frac{1}{2\pi(N+1)\sin^2(t/2)}\sum_{k=0}^{N} \cos(kt) - \cos((k+1)t) \\ 
            &= \frac{1}{4\pi(N+1)\sin^2(t/2)}\pth{1 - \cos((N+1)t)}
            = \frac{\sin^2\pth{\frac{N+1}{2}t}}{2\pi(N+1)\sin^2(t/2)}.
        \end{split}
    \end{equation*}
    Some other properties of the Fejer kernel include that if $f = 1$, then 
    \begin{equation*}
        \sigma_Nf = \int_{-\pi}^{\pi} F_N(x)dx = 1,
    \end{equation*}
    and that $F_N(-x) = F_N(x), F_N \geq 0$. Also, 
    \begin{equation*}
        \sigma_Nf(x) = \frac{1}{N+1}\sum_{k=0}^{N} \int_{-\pi}^\pi D_k(x-y)f(y)dy
        = \int_{-\pi}^\pi \pth{\frac{1}{N+1}\sum_{k=0}^{N}D_k(x-y)}f(y)dy
        = F_N*f(x).
    \end{equation*}
\end{remark}

\begin{definition}
    $C_{2\pi}$ denote the space of $2\pi$-periodic continuous functions. 
    $C^k_{2\pi}$ denotes the space of $2\pi$-periodic functions having continuous 
    derivatives up to order $k$.
\end{definition}

\begin{theorem}[Fejer]
    Let $f\in C_{2\pi}$. Then $\sigma_Nf \to f$ uniformly.
\end{theorem}
\begin{proof}
    Let $\epsilon>0$. There is a $\delta>0$ such that $\abs{f(x)-f(y)}<\epsilon$ 
    whenever $\abs{x-y}<\delta$. Observe that if $\abs{t}\geq\delta$, then 
    \begin{equation*}
        F_N(t) \leq \frac{1}{2\pi(N+1)\sin^2(\delta/2)} \to 0
    \end{equation*}
    as $N\to\infty$. Hence 
    \begin{equation*}
        \begin{split}
            \abs{\sigma_Nf(x) - f(x)} &= \abs{\int_{-\pi}^\pi F_N(x-y)f(y)dy - f(x)}  
            \leq \int_{-\pi}^\pi F_N(x-y)\abs{f(y)-f(x)}dy \\
            &\leq \int_{\abs{x-y}\leq \delta} F_N(x-y)\abs{f(y)-f(x)}dy + \int_{\delta\leq\abs{x-y}\leq\pi} F_N(x-y)\abs{f(y)-f(x)}dy \\
            &\leq \epsilon\int_{-\pi}^\pi F_N(x-y)dy + 2\norm{f}_\infty\int_{\delta\leq\abs{x-y}\leq\pi} F_N(x-y)dy \\
            &= \epsilon + 2\norm{f}_\infty\frac{\pi-\delta}{2\pi(N+1)\sin^2(\delta/2)} \to 0
        \end{split}
    \end{equation*}
    as $N\to\infty$.
\end{proof}

\begin{definition}
    The \textbf{trigonometric polynomial} of degree $N$ is a function of the form 
    \begin{equation*}
        TP_N(x) = \sum_{k=0}^N a_k\cos(kx) + b_k\sin(kx). 
    \end{equation*}
    The \textbf{trigonometric polynomial space} is denoted by $TP = \cup_N TP_N$.
\end{definition}

\begin{theorem}
    Under $\L^2[-\pi,\pi]$, $PC_{2\pi}\subset\overline{TP}$.
\end{theorem}
\begin{proof}
    Since continuous functions are dense in $PC_{2\pi}$, it suffices to show 
    that continuous functions can be approximated by trigonometric polynomials.
    Let $f\in C_{2\pi}$. By the Fejer theorem, $\sigma_Nf \to f$ uniformly.
    Since $\sigma_Nf$ is a trigonometric polynomial, $f$ can be approximated
    by trigonometric polynomials. 
\end{proof}

\begin{definition}
    The \textbf{best approximation error} of a function $f$ by a trigonometric 
    polynomial is defined by $\tilde{E}_N(f) = \inf_{p\in TP_N}\norm{p-f}_\infty$. 
\end{definition}

\begin{definition}
    For $f,g > 0$, $f\lesssim g$ if there is some constant $c>0$ such that $f\leq cg$.
\end{definition}

\begin{theorem}\label{thm:fourier_series_approx_loss}
    For $f\in C_{2\pi}$, 
    \begin{equation*}
        \norm{S_Nf - f}_\infty \lesssim (1+\log N)\tilde{E}_N(f).
    \end{equation*}
\end{theorem}
\begin{proof}
    Recall that $S_Nf = D_N*f$. Then
    \begin{equation*}
        \abs{S_Nf(x)} = \abs{\int_{-\pi}^\pi D_N(x-y)f(y)dy} \leq \int_{-\pi}^\pi \abs{D_N(x-y)}\abs{f(y)}dy
        \leq \norm{f}_\infty\int_{-\pi}^\pi \abs{D_N(t)}dt.
    \end{equation*}
    Observe that 
    \begin{equation*}
        \abs{D_N(t)} = \frac{1}{2\pi}\abs{\frac{\sin\pth{(N+1/2)t}}{\sin(t/2)}} \leq \min\set{\frac{2N+1}{2\pi}, \frac{1}{2\abs{t}}}.
    \end{equation*}
    Thus 
    \begin{equation*}
        \begin{split}
            \int_{-\pi}^{\pi} \abs{D_N(t)}dt &\leq \int_{\abs{t}\leq \pi/(2N+1)} \frac{2N+1}{2\pi}dt + \int_{\pi/(2N+1)\leq\abs{t}\leq\pi} \frac{1}{2\abs{t}}dt \\
            &\leq \frac{2N+1}{2\pi}\frac{2\pi}{2N+1} + 2\cdot\frac{1}{2}\log(2N+1) \lesssim (1 + \log N).
        \end{split}
    \end{equation*}
    Now let $q^*$ be the best approximation to $f$ in $TP_N$. 
    Notice that $S_Nq^* = q^*$ and $S_N$ is a linear operator. Then 
    \begin{equation*}
        \norm{S_Nf - q^*}_\infty = \norm{S_Nf - S_Nq^*}_\infty 
        \leq \norm{S_N}\norm{f-q^*}_\infty \lesssim (1+\log N)\tilde{E}_N(f)
    \end{equation*}
    as desired.
\end{proof}

\begin{theorem}
    If $f\in C_{2\pi}$ is $L$-Lipschitz, then
    \begin{thmenum}
        \item $\norm{\sigma_Nf - f}_\infty \lesssim \frac{1+\log N}{N}L$,
        \item $\norm{S_Nf - f}_\infty \lesssim \frac{(1+\log N)^2}{N}L$.
    \end{thmenum}
\end{theorem}
\begin{proof}
    For (a) we have 
    \begin{equation*}
        \sigma_Nf(x) = (F_N*f)(x) = \int_{-\pi}^\pi F_N(x-y)f(y)dy.
    \end{equation*}
    And thus
    \begin{equation*}
        \begin{split}
            \abs{\sigma_Nf(x) - f(x)} &\leq \int_{-\pi}^\pi F_N(x-y)\abs{f(y)-f(x)}dy \\
            &= \int_{-\pi}^\pi F_N(u)\abs{f(x-u)-f(x)}du \leq L\int_{-\pi}^\pi F_N(u)\abs{u}du.
        \end{split}
    \end{equation*}
    Observe that 
    \begin{equation*}
        \abs{F_N(u)} = \frac{1}{2\pi(N+1)}\abs{\frac{\sin^2\pth{\frac{N+1}{2}u}}{\sin^2(u/2)}} 
        \leq \min\set{\frac{N+1}{2\pi}, \frac{\pi}{2(N+1)\abs{u}^2}}.
    \end{equation*}
    Then 
    \begin{equation*}
        \begin{split}
            L\int_{-\pi}^\pi F_N(u)\abs{u}du &\leq L\int_{\abs{u}\leq \pi/(N+1)} \frac{N+1}{2\pi}\abs{u}du + L\int_{\pi/(N+1)\leq\abs{u}\leq\pi} \frac{\pi}{2(N+1)\abs{u}^2}\abs{u}du \\
            &\leq L\int_{\abs{u}\leq \pi/(N+1)} \frac{N+1}{2\pi}\frac{\pi}{N+1}du + \frac{L\pi}{2(N+1)}\cdot 2\log(N+1) \\
            &= \frac{L\pi}{N+1} + L\pi\frac{\log(N+1)}{N+1} \lesssim \frac{1+\log N}{N}L,
        \end{split}
    \end{equation*}
    proving (a).
    
    For (b), from \cref{thm:fourier_series_approx_loss} we have
    \begin{equation*}
        \norm{S_Nf - f}_\infty \lesssim (1+\log N)\tilde{E}_N(f) 
        \lesssim \norm{\sigma_Nf - f}_\infty \lesssim \frac{(1+\log N)^2}{N}L,
    \end{equation*}
    proving (b).
\end{proof}

\begin{definition}
    The \textbf{Chebyshev polynomials} are defined by $T_n(x) = \cos(n\cos^{-1}(x))$ 
    for $n = 0,1,\ldots$ on $[-1,1]$.
\end{definition}
\begin{remark}
    The Chebyshev polynomials have the following recurrence property, 
    \begin{equation*}
        \begin{split}
            T_{n+1}(x) &= \cos((n+1)\cos^{-1}(x)) = \cos(n\cos^{-1}(x))\cos(\cos^{-1}(x)) - \sin(n\cos^{-1}(x))\sin(\cos^{-1}(x)) \\
            &= xT_n(x) - \frac{1}{2}\cos((n-1)\cos^{-1}(x)) + \frac{1}{2}\cos((n+1)\cos^{-1}(x)) \\
            &= xT_n(x) - \frac{1}{2}T_{n-1}(x) + \frac{1}{2}T_{n+1}(x).
        \end{split}
    \end{equation*}
    Then 
    \begin{equation*}
        T_{n+1}(x) = 2xT_n(x) - T_{n-1}(x).
    \end{equation*}
    Immediately we see that $T_n(x)\in P_n[-1,1]$.
\end{remark}

\begin{proposition}
    $\set{T_n}_{n=0}^\infty$ forms an orthogonal set with respect to 
    the inner product 
    \begin{equation*}
        \inp{f}{g}_T = \frac{1}{\pi}\int_{-1}^{1} f(x)g(x)\frac{dx}{\sqrt{1-x^2}}.
    \end{equation*}
\end{proposition}
\begin{proof}
    Using the change of variable $x = \cos(\theta)$, a direct 
    computation gives 
    \begin{equation*}
        \begin{split}
            \inp{T_m}{T_n}_T &= \frac{1}{\pi}\int_{-1}^{1} \cos(m\cos^{-1}(x))\cos(n\cos^{-1}(x))\frac{dx}{\sqrt{1-x^2}} 
            = \frac{1}{\pi}\int_0^\pi \cos(m\theta)\cos(n\theta)d\theta \\
            &= \frac{1}{2\pi}\int_0^\pi \cos((m+n)\theta) + \cos((m-n)\theta)d\theta 
            = 0
        \end{split}
    \end{equation*}
    for $m\neq n$.
\end{proof}

\begin{proposition}\label{prop:chebyshev_isomorphism}
    Let $E[-\pi,\pi]$ be the subspace of $C[-1,1]$ consisting of 
    all even continuous function on $[-\pi,\pi]$. Consider the mapping 
    $\Phi:C[-1,1]\to E[-\pi,\pi]$ defined by $\Phi:f\to f\circ\cos$. 
    Then the followings are true. 
    \begin{thmenum}
        \item $\Phi$ is well-defined and is an isomorphism. 
        \item $(\Phi T_n)(\theta) = \cos(n\theta)$. 
        \item $\Phi(P_n) = E[-\pi,\pi]\cap TP_n$. 
        \item $\Phi$ is isometric. 
        \item $E_n(f) = \tilde{E}_n(\Phi f)$ for all $f\in C[-1,1]$.
        \item $\inp{f}{g}_T = \inp{\Phi f}{\Phi g}$ for all $f,g\in C[-1,1]$.
    \end{thmenum}
\end{proposition}
\begin{proof}
    For (a), since $(\Phi f)(-x) = f(\cos(-x)) = f(\cos(x)) =( \Phi f)(x)$ and 
    both $f$ and $\cos$ are continuous, $f\circ \cos$ is also continuous, 
    we conclude that $\Phi f\in E[-\pi,\pi]$ and $\Phi$ is well-defined. 
    Now if $\Phi f = 0$, then $\norm{f}_\infty = \norm{\Phi f}_\infty = 0$ 
    and thus $f = 0$. Hence $\Phi$ is injective. For the sujectivity, let 
    $g\in E[-\pi,\pi]$. $\Phi (g\circ\cos^{-1}) = g$ and $g\circ\cos^{-1}$ 
    is continuous. Thus $\Phi$ is surjective. 

    (b) is immediate from the definition of $\Phi$. $(\Phi T_n)(\theta) 
    = \cos(n\cos^{-1}(\cos(\theta))) = \cos(n\theta)$.

    We now prove (c). From (a) we have $\Phi$ is an isomorphism. Also, 
    from (b) we have that $\Phi(P_n)\subset TP_n$. For any even trigonometric 
    polynomial $p\in TP_n$, $p(x) = \sum_{k=0}^n a_k\cos(kx)$. Then 
    consider $g = \sum_{k=0}^{n} a_kT_k$. Then 
    \begin{equation*}
        (\Phi g)(x) = \sum_{k=0}^{n} a_k(T_k\circ\cos)(x) = \sum_{k=0}^{n} a_k\cos(k\cos^{-1}(\cos(x))) = \sum_{k=0}^{n} a_k\cos(kx) = p(x).
    \end{equation*}
    Hence $TP_n\subset \Phi(P_n)$ and (c) is proven. 

    For (d), 
    \begin{equation*}
        \norm{\Phi f}_\infty = \sup_{x\in[-\pi,\pi]} \abs{f(\cos(x))} = \sup_{x\in[-1,1]}\abs{f(x)} = \norm{f}_\infty.
    \end{equation*} 

    (e) is an immediate consequence of (d). 

    Finally, for (f), by changing the variable $x = \cos\theta$ and 
    the fact that $f(\cos(\theta))g(\cos(\theta))$ is even, we have
    \begin{equation*}
        \inp{f}{g}_T = \frac{1}{\pi}\int_{-1}^{1} f(x)g(x)\frac{dx}{\sqrt{1-x^2}} = \frac{1}{2\pi}\int_{-\pi}^{\pi} f(\cos(\theta))g(\cos(\theta))d\theta = \inp{\Phi f}{\Phi g},
    \end{equation*}
    showing (f).
\end{proof}

\begin{theorem}
    If $f\in C[-1,1]$ admits a Chebyshev series
    \begin{equation*}
        f(x) = \sum_{n=0}^\infty a_nT_n(x),
    \end{equation*}
    then 
    \begin{equation*}
        \tau_Nf = \sum_{k=0}^{N}(1-\frac{k}{N+1})a_kT_k \to f 
    \end{equation*}
    uniformly on $[-1,1]$. If $f$ is Lipschitz and $P_N^Cf$ is 
    the truncated Chebyshev series to $f$, then 
    \begin{equation*}
        \norm{P_N^Cf - f}_\infty \lesssim (1+\log N)E_N(f).
    \end{equation*}
\end{theorem}
\begin{proof}
    Consider the transformation $\Phi:C[-1,1]\to E[-\pi,\pi]$ defined by 
    $\Phi:f\to f\circ\cos$. Then from \cref{prop:chebyshev_isomorphism} we 
    have 
    \begin{equation*}
        (\Phi\tau_Nf)(\theta) = \sum_{k=0}^{N}\pth{1-\frac{k}{N+1}}a_k\cos(k\theta) 
        = \frac{1}{N+1}\sum_{j=0}^N\sum_{k=0}^j a_j\cos(j\theta) = \sigma_N(\Phi f)(\theta). 
    \end{equation*}
    By the Fejer theorem, 
    \begin{equation*}
        \norm{\tau_Nf - f}_\infty = \norm{\Phi\tau_Nf - \Phi f}_\infty = \norm{\sigma_N(\Phi f) - \Phi f}_\infty \to 0
    \end{equation*}
    as $N\to\infty$. 

    To see the second part, suppose that $f$ is $L$-Lipschitz. Then
    \begin{equation*}
        \abs{\Phi f(\alpha) - \Phi f(\beta)} \leq L\abs{\cos(\alpha) - \cos(\beta)} \leq L\abs{\alpha - \beta}
    \end{equation*}
    since the derivative of $\cos$ is bounded by $1$. Thus $\Phi f$ 
    is $L$-Lipschitz and 
    \begin{equation*}
        \norm{P_N^Cf - f}_\infty = \norm{\Phi P_N^Cf - \Phi f}_\infty 
        = \norm{S_N(\Phi f) - \Phi f}_\infty \lesssim (1+\log N)\tilde{E}_N(\Phi f) 
        = (1+\log N)E_N(f)
    \end{equation*}
    by \cref{thm:fourier_series_approx_loss} and part (e) of \cref{prop:chebyshev_isomorphism}.
\end{proof}

\begin{theorem}[Jackson]
    For $f\in C[-1,1]$, 
    \begin{equation*}
        E_N(f) \lesssim \omega(f,\frac{1}{N}).
    \end{equation*}
\end{theorem}
\begin{proof}
    Let $\phi(\theta) = \sum_{k=0}^{N} c_k\cos(k\theta)\in TP_N$, 
    where $c_k\in\R$ such that $\phi\geq 0$. For any $2\pi$-periodic 
    $f$, define $\Psi$ by 
    \begin{equation*}
        \Psi f(\theta) = \frac{1}{2\pi}\int_{-\pi}^\pi f(\theta - t)\phi(t)dt = \frac{1}{2\pi}(f*\phi)(\theta).
    \end{equation*}

    Next we make the following observations. First, $\Psi\mathbf{1} = \mathbf{1}$ 
    where $\mathbf{1}$ is the constant function $\mathbf{1}(\theta) = 1$. Second, 
    $\Psi$ is linear and positive. To see this, note that 
    \begin{equation*}
        \Psi(cf+g) = \frac{1}{2\pi}\int_{-\pi}^\pi (cf+g)(\theta - t)\phi(t)dt = c\Psi f + \Psi g
    \end{equation*}
    for any $c\in\R$ and $f,g\in C[-1,1]$. Also, if $f\geq 0$, then 
    \begin{equation*}
        \Psi f(\theta) = \frac{1}{2\pi}\int_{-\pi}^\pi f(\theta - t)\phi(t)dt \geq 0.
    \end{equation*}
    Third, $\Psi f\in TP_N$ for any $f\in C[-1,1]$. Indeed, 
    \begin{equation*}
        \begin{split}
            \Psi f(\theta) &= \frac{1}{2\pi}\int_{-\pi}^{\pi} f(t)\phi(\theta - t)dt 
            = \frac{1}{2\pi}\sum_{k=0}^{N}\int_{-\pi}^{\pi} f(t)c_k\cos(k(\theta - t))dt \\ 
            &= \frac{1}{2\pi}\sum_{k=0}^{N} B_k\cos(k\theta) + D_k\sin(k\theta) \in TP_N
        \end{split}
    \end{equation*}
    with 
    \begin{equation*}
        B_k = \frac{c_k}{2\pi}\int_{-\pi}^{\pi} f(t)\cos(kt)dt, 
        \quad \text{and} \quad
        D_k = \frac{c_k}{2\pi}\int_{-\pi}^{\pi} f(t)\sin(kt)dt.
    \end{equation*}

    Now we have the last claim that 
    \begin{equation*}
        \norm{\Psi f - f}_\infty \leq \omega(f,\frac{1}{N})\pth{1 + \frac{N\pi}{2}\sqrt{2 - C}} 
    \end{equation*}
    for some constant $C>0$, which will be determined later. Since $f$ is uniformly continuous, 
    \begin{equation*}
        \abs{f(\theta - t) - f(\theta)} \leq \omega\pth{f,\abs{t}} \leq (1 + N\abs{t})\omega\pth{f,\frac{1}{N}}.
    \end{equation*}
    Using the first observation $\Psi \mathbf{1} = \mathbf{1}$, we have
    \begin{equation*}
        \abs{\Psi f(\theta) - f(\theta)} = \abs{\frac{1}{2\pi}\int_{-\pi}^{\pi}\pth{f(\theta - t) - f(\theta)}\phi(t)dt}
        \leq \frac{1}{2\pi}\omega\pth{f,\frac{1}{N}}\int_{-\pi}^{\pi}\pth{1 + N\abs{t}}\phi(t)dt.
    \end{equation*}
    Also, 
    \begin{equation*}
        \begin{split}
            \frac{1}{2\pi}\int_{-\pi}^{\pi} \pth{1 + N\abs{t}}\phi(t)dt &= 1 + \frac{N}{2\pi}\int_{-\pi}^{\pi} \abs{t}\phi(t)dt \\
            &\leq 1 + N\pth{\frac{1}{2\pi}\int_{-\pi}^{\pi} \abs{t}^2\phi(t)dt}^{1/2}\pth{\frac{1}{2\pi}\int_{-\pi}^{\pi} \phi(t)dt}^{1/2} \\
            &= 1 + N\pth{\frac{1}{2\pi}\int_{-\pi}^{\pi} \abs{t}^2\phi(t)dt}^{1/2}
        \end{split}
    \end{equation*}
    by the Cauchy-Schwarz inequality. Notice that 
    \begin{equation*}
        1 - \cos t = 2\sin^2\frac{t}{2} \geq 2\frac{4}{\pi^2}\frac{t^2}{4} = \frac{2t^2}{\pi^2} 
        \Rightarrow \abs{t}^2 \leq \frac{\pi^2}{2}(1-\cos t).
    \end{equation*}
    Then 
    \begin{equation*}
        \frac{1}{2\pi}\int_{-\pi}^{\pi} \pth{1 + N\abs{t}}\phi(t)dt \leq 1 + N\pth{\frac{1}{2\pi}\frac{\pi^2}{2}\int_{-\pi}^{\pi} (1-\cos t)\phi(t)dt}^{1/2} 
        = 1 + \frac{N\pi}{2}\sqrt{2 - C}.
    \end{equation*}
    Thus 
    \begin{equation*}
        \norm{\Psi f - f}_\infty \leq \omega(f,\frac{1}{N})\pth{1 + \frac{N\pi}{2}\sqrt{2 - C}}.
    \end{equation*}

    Finally, we want to pin down our constant $C$ so that $\sqrt{2 - C}$ is minimized 
    and $\Psi \mathbf{1} = \mathbf{1}$. We conjecture that $\phi(\theta) = C_1\abs{p(\theta)}^2$, 
    where $p(\theta) = \sum_{k=0}^{N} a_ke^{ik\theta}$, $a_k = \sin\pth{\frac{k+1}{N+2}\pi}$. 
    Compute that 
    \begin{equation*}
        \begin{split}
            C_1\abs{p(\theta)}^2 &= C_1p(\theta)\overline{p(\theta)} = C_1\sum_{k=0}^{N}a_ke^{ik\theta}\sum_{j=0}^{N}a_je^{-ij\theta}  \\
            &= C_1\sum_{k=0}^{N}\sum_{j=0}^{N}a_ka_j e^{i(k-j)\theta} = C_1\sum_{k=0}^{N}a_k^2 + C_1\sum_{s=1}^{N}\sum_{k=0}^{N-s}a_ka_{k+s}\pth{e^{is\theta} + e^{-is\theta}} \\ 
            &= C_1\sum_{k=0}^{N}a_k^2 + 2C_1\sum_{s=1}^{N}\sum_{k=0}^{N-s}a_ka_{k+s}\cos(s\theta).
        \end{split}
    \end{equation*}
    Take $C = \pth{\sum_{k=0}^N a_k^2}^{-1}$, then
    \begin{equation*}
        \phi(\theta) = 1 + \sum_{s=1}^{N}2C_1b_s\cos(s\theta), \quad \text{where } b_s = \sum_{k=0}^{N-s}a_ka_{k+s}.
    \end{equation*}
    Now 
    \begin{equation*}
        \begin{split}
            2b_1 &= \sum_{k=0}^{N-1}2\sin\pth{\frac{k+1}{N+2}\pi}\sin\pth{\frac{k+2}{N+2}\pi} 
            = \sum_{k=1}^{N}2\sin\pth{\frac{k}{N+2}\pi}\sin\pth{\frac{k+1}{N+2}\pi} \\
            &= \sum_{k=0}^{N-1}2\sin\pth{\frac{k}{N+2}\pi}\sin\pth{\frac{k+1}{N+2}\pi}.
        \end{split}
    \end{equation*}
    Combining the first expression and the third one allows us to write 
    \begin{equation*}
        \begin{split}
            2b_1 &= \sum_{k=0}^{N-1}\sin\pth{\frac{k+1}{N+2}\pi}\pth{\sin\pth{\frac{k}{N+2}\pi} + \sin\pth{\frac{k+2}{N+2}\pi}} \\
            &= \sum_{k=0}^{N-1}\sin^2\pth{\frac{k+1}{N+2}\pi}\cos\frac{2\pi}{N+2} 
            = \cos\pth{\frac{2\pi}{N+2}}\sum_{k=0}^{N-1}a_k^2 = C_1^{-1}\cos\pth{\frac{2\pi}{N+2}}.
        \end{split}
    \end{equation*}
    Now 
    \begin{equation*}
        C = 2C_1b_1 = 2\cos(\frac{2\pi}{N+2}) \Rightarrow 2 - C \lesssim \frac{1}{N^2}
    \end{equation*}
    by the Taylor expansion of $\cos$. It now follows from the last claim that 
    \begin{equation*}
        \norm{\Psi f - f}_\infty \leq \omega(f,\frac{1}{N})\pth{1 + \frac{N\pi}{2}\sqrt{2 - C}} \lesssim \omega(f,\frac{1}{N}).
    \end{equation*}
    The proof is complete.
\end{proof}