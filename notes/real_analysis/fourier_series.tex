\begin{definition}
    The \textbf{Fourier series} of a function $f$ is given by 
    \begin{equation*}
        Sf(x) = \frac{a_0}{2} + \sum_{k=1}^\infty a_k \cos(kx) + b_k \sin(kx),
    \end{equation*}
    where the \textbf{Fourier coefficients} $a_k$ and $b_k$ are 
    given by 
    \begin{equation*}
        a_k = \frac{1}{\pi} \int_{-\pi}^\pi f(x) \cos(kx)dx, \quad
        b_k = \frac{1}{\pi} \int_{-\pi}^\pi f(x) \sin(kx)dx.
    \end{equation*}
    Or, alternatively, 
    \begin{equation*}
        Sf(x) = \sum_{k=-\infty}^\infty c_k e^{ikx},
    \end{equation*}
    where the \textbf{Fourier coefficients} $c_k$ are given by
    \begin{equation*}
        c_k = \frac{1}{2\pi} \int_{-\pi}^\pi f(x) e^{-ikx}dx.
    \end{equation*}
\end{definition}

\begin{definition}
    The \textbf{Truncated Fourier series} of a function $f$ is denoted by 
    \begin{equation*}
        S_N f(x) = \frac{a_0}{2} + \sum_{k=1}^N a_k \cos(kx) + b_k \sin(kx) = \sum_{k=-N}^N c_k e^{ikx}.
    \end{equation*}
\end{definition}

\begin{proposition}
    Let $a_k$ and $b_k$ be the Fourier coefficients of a function $f$. Then 
    \begin{thmenum}
        \item If $f\in\L^1$, $\abs{a_k}, \abs{b_k}\leq C\norm{f}_1$ for some constant $C>0$.
        \item If $f\in\L^\infty$, $\abs{a_k}, \abs{b_k}\leq C\norm{f}_\infty$ for some constant $C>0$.
    \end{thmenum}
\end{proposition}
\begin{proof}
    To see (a), compute that
    \begin{equation*}
        \abs{a_k} \leq \frac{1}{\pi}\int_{-\pi}^\pi \abs{f(x)}\abs{\cos(kx)}dx \leq \frac{1}{\pi}\int_{-\pi}^\pi \abs{f(x)}dx = C\norm{f}_1
    \end{equation*}
    and similarly for $b_k$. 
    For (b), 
    \begin{equation*}
        \abs{a_k} \leq \frac{1}{\pi}\int_{-\pi}^\pi \abs{f(x)}\abs{\cos(kx)}dx \leq \frac{1}{\pi}\int_{-\pi}^\pi \norm{f}_\infty dx = 2\norm{f}_\infty = C\norm{f}_\infty.
    \end{equation*}
    The proof for $b_k$ is analogous.
\end{proof}

\begin{lemma}[Riemann-Lebesgue]
    Let $f\in \L^1[a,b]$. Then 
    \begin{equation*}
        \lim_{n\to\infty} \int_a^b f(x)e^{-inx}dx = 0.
    \end{equation*}
\end{lemma}
\begin{proof}
    Let $\epsilon > 0$. Since $f\in\L^1[a,b]$, there is a step function 
    $g$ such that $\norm{f-g}_1 < \epsilon$. For any interval $E$, 
    \begin{equation*}
        \abs{\int_a^b \chi_E(x)e^{-inx}dx} \leq \abs{\int_E \cos(nx)dx} + \abs{\int_E \sin(nx)dx} 
        \leq \frac{2\pi}{n} \to 0
    \end{equation*}
    as $n\to\infty$. A step function is a linear combination of characteristic 
    functions of intervals, and thus $\abs{\int_a^b g(x)e^{-inx}dx} \to 0$ as $n\to\infty$.
    Therefore, 
    \begin{equation*}
        \begin{split}
            \abs{\int_a^b f(x)e^{-inx}dx} &\leq \abs{\int_a^b (f(x)-g(x))e^{-inx}dx} + \abs{\int_a^b g(x)e^{-inx}dx} \\
            &\leq \norm{f-g}_1 + \abs{\int_a^b g(x)e^{-inx}dx} \to 0
        \end{split}
    \end{equation*}
    as $n\to\infty$.
\end{proof}

\begin{definition}
    The space of piecewise continuous functions on $[a,b]$ is denoted by $PC[a,b]$. 
    The symbol $PC^n[a,b]$ denotes the space of functions having continuous derivatives up to order $n-1$, 
    with the $n$-th derivative being piecewise continuous.
\end{definition}

\begin{proposition}
    Let $f\in PC^1[-\pi,\pi]$ and 
    \begin{equation*}
        f(x) = \frac{a_0}{2} + \sum_{k=1}^\infty a_k \cos(kx) + b_k \sin(kx).
    \end{equation*}
    Then 
    \begin{equation*}
        f'(x) = \sum_{k=1}^\infty -ka_k \sin(kx) + kb_k \cos(kx).
    \end{equation*}
\end{proposition}
\begin{proof}
    Differentiation term by term gives the desired result.
\end{proof}
\begin{remark}
    If $f\in PC^n[-\pi,\pi]$, then
    \begin{equation*}
        \abs{a_k}, \abs{b_k} \leq \frac{\norm{f}_{PC^n}}{k^n},
    \end{equation*}
    where $\norm{f}_{PC^n} = \sum_{j=0}^n \norm{f^{(j)}}_\infty$.
\end{remark}

\begin{definition}
    Let $f$ be a function on $\R$. The \textbf{right-limit} and 
    the \textbf{left-limit} of $f$ at $x$ are defined by
    \begin{equation*}
        f(x^+) = \lim_{h\to 0^+} f(x+h), \quad f(x^-) = \lim_{h\to 0^+} f(x-h).
    \end{equation*}
\end{definition}

\begin{definition}
    A \textbf{kernel} is a function $k:X\times X\to\R$ such that 
    \begin{thmenum}
        \item $k(x,y) = k(y,x)$ for all $x,y\in X$,
        \item For finitely many points $x_1,\ldots,x_n\in X$ and scalars $a_1,\ldots,a_n\in\R$, 
        \begin{equation*}
            \sum_{i=1}^n \sum_{j=1}^n a_ia_jk(x_i,x_j) \geq 0.
        \end{equation*}
    \end{thmenum}
\end{definition}

\begin{definition}
    The \textbf{Dirichlet kernel} is defined by
    \begin{equation*}
        D_N(x) = \frac{1}{2\pi}\sum_{k=-N}^N e^{ikx}.
    \end{equation*}
\end{definition}
\begin{remark}
    The Dirichlet kernel can be simplified to 
    \begin{equation*}
        D_N(x) = \frac{\sin\pth{(N+1/2)x}}{2\pi\sin(x/2)}.
    \end{equation*}
    To see this, note that 
    \begin{equation*}
        2\pi D_N(x)(e^{ix}-1) = e^{i(N+1)x} - e^{-iNx} 
        = \frac{e^{i(N+1/2)x} - e^{-i(N+1/2)x}}{e^{ix/2} - e^{-ix/2}}.
    \end{equation*}
    And thus,
    \begin{equation*}
        D_N(x) = \frac{\sin\pth{(N+1/2)x}}{2\pi\sin(x/2)}.
    \end{equation*}
    Some other properties of the Dirichlet kernel include $D_N(-x) = D_N(x)$ 
    and $\int_{-\pi}^\pi D_N(x)dx = 1$.
\end{remark}

\begin{definition}
    Let $f,g:X\to\R$. The \textbf{convolution} of $f$ and $g$ is defined by
    \begin{equation*}
        (f*g)(x) = \int_X f(x-y)g(y)dy.
    \end{equation*}
\end{definition}

\begin{proposition}
    For any $2\pi$-periodic function $f\in PC$,
    \begin{equation*}
        S_Nf = D_N*f.
    \end{equation*}
\end{proposition}
\begin{proof}
    Compute that 
    \begin{equation*}
        \begin{split}
            S_Nf(x) &= \sum_{k=-N}^N c_k e^{ikx} = \sum_{k=-N}^N \frac{1}{2\pi} \int_{-\pi}^\pi f(y)e^{-iky}dy e^{ikx} \\
            &= \int_{-\pi}^\pi f(y) \frac{1}{2\pi}\sum_{k=-N}^N e^{ik(x-y)}dy = \int_{-\pi}^\pi f(y)D_N(x-y)dy = (D_N*f)(x).
        \end{split}
    \end{equation*}
\end{proof}

\begin{theorem}[Dirichlet-Jordan]
    Let $f$ be a $2\pi$-periodic function and piecewise Lipschitz. Then 
    \begin{equation*}
        \lim_{N\to\infty} S_Nf(x) = \frac{f(x^+)+f(x^-)}{2}.
    \end{equation*}
    In particular, if $f$ is continuous at $x$, then 
    \begin{equation*}
        \lim_{N\to\infty} S_Nf(x) = f(x).
    \end{equation*}
\end{theorem}